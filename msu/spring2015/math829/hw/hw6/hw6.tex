
\documentclass{article}

\usepackage{amsmath}
\usepackage{amsfonts}
\usepackage{amssymb}
\usepackage{enumitem}
\usepackage{mathtools}
\usepackage{xfrac}
\usepackage[lastexercise]{exercise}

\newcommand\dif{\mathop{}\!\mathrm{d}}
\DeclarePairedDelimiter\floor{\lfloor}{\rfloor}

\newcounter{Problem}
\newenvironment{Problem}{\begin{Exercise}[name={Problem},
                                          counter={Problem}]}
                        {\end{Exercise}}
\title{MATH 829 Homework \#6}
\date{February 25, 2015}
\author{Sam Boling}

\begin{document}

\begin{titlepage}
\maketitle
%TODO -- \S 2.4(c), last part
%         additional problem
\end{titlepage}

\section{Additional Problems}
\begin{Problem}
Prove that Cauchy's theorem for Jordan curves follows from
Green's theorem if we assume that $f^\prime$ is continuous.
\end{Problem}

\begin{Answer}
Write $z = x + i y$ so that
$$
f(z) = f(x + iy) = u(x, y) + i v(x, y).
$$
Let $\gamma$ be a closed curve parameterized by
$$
\gamma(t) = x(t) + i y(t), \quad t \in [a, b]
$$
for some $a, b \in \mathbb{R}$ and some
$x, y : [a, b] \to \mathbb{R}$.
Then by definition
\begin{align*}
\oint_\gamma f
&=
\int_a^b
  f(\gamma(t)) \gamma^\prime(t)
  \dif t \\
&=
\int_a^b
  [u(x(t), y(t)) + i v(x(t), y(t))]
  (x^\prime(t) + i y^\prime(t))
  \dif t \\
&=
\int_a^b
  [u(x(t), y(t)) x^\prime(t) - v(x(t), y(t)) y^\prime(t)]
  \dif t
+
i
\int_a^b
  [u(x(t), y(t)) y^\prime(t) - v(x(t), y(t)) x^\prime(t)]
  \dif t.
\end{align*}
Since $\dif x = x^\prime(t) \dif t$ and
$\dif y = y^\prime(t) \dif t$ we perform substitutions to get
\begin{align*}
\oint_\gamma f
&=
    \int_{x(a)}^{x(b)} u \dif x
-   \int_{y(a)}^{y(b)} v \dif y
+ i \int_{y(a)}^{y(b)} u \dif y
+ i \int_{x(a)}^{x(b)} v \dif x,
\end{align*}
where $x(a) = x(b)$ and $y(a) = y(b)$ by assumption since
$\gamma$ is closed. Therefore
\begin{align*}
\oint_\gamma f
&=
    \oint_\gamma (u \dif x - v \dif y)
+ i \oint_\gamma (v \dif x + u \dif y).
\end{align*}
Since $f^\prime$ is continuous, $u$ and $v$ have continuous
partial derivatives, so applying Green's theorem this means
\begin{align*}
\oint_\gamma f
&=
\iint_{\mathrm{Int}(\gamma)}
  \left(
    \frac{\partial(-v)}{\partial x}
  - \frac{\partial u}{\partial y}
  \right)
  \dif x \dif y
+
i
\iint_{\mathrm{Int}(\gamma)}
  \left(
    \frac{\partial u}{\partial x}
  - \frac{\partial v}{\partial y}
  \right)
  \dif x \dif y \\
&= 0
\end{align*}
since $f$ is holomorphic and so the
Cauchy-Riemann equations are satisfied.
\end{Answer}

\begin{Problem}
Let $\gamma$ be a positively oriented Jordan curve. Use
Green's theorem to compute $\int_\gamma \bar{z} \dif z$,
noting that the conjugate map is not holomorphic.
\end{Problem}

\begin{Answer}
We have
$$
f(x + i y) = \overline{x + iy} = x - iy
$$
so that $u(x, y) = x$, $v(x, y) = -y$,
$u_x = 1$, $u_y = 0$, $v_x = 0$, $v_y = -1$. Then
\begin{align*}
\oint_\gamma f
&=
-
\iint_{\mathrm{Int}(\gamma)}
 (v_x + u_y) \dif x \dif y
+ i
\iint_{\mathrm{Int}(\gamma)}
  (u_x - v_y) \dif x \dif y \\
&=
2i
\iint_{\mathrm{Int}(\gamma)} \dif x \dif y,
\end{align*}
i.e. $2i$ times the area bounded by $\gamma$.
\end{Answer}

\begin{Problem}
Compute
$$
\int_\gamma \frac{e^{3z}}{(z - 2)^3} \dif z
$$
for
\begin{enumerate}[label=(\alph*)]
  \item{
    $\gamma = \{ |z| = 3 \}$,
  }
  \item{
    $\gamma = \{ |z| = 1 \}$.
  }
\end{enumerate}
\end{Problem}

\begin{Answer}
\begin{enumerate}[label=(\alph*)]
  \item{
    Here $2 \in \mathrm{Int}(\gamma)$. Let $f(z) = e^{3z}$,
    so that $f$ is entire. Then we may use the Cauchy formula to write
    $$
    f^{(2)}(2)
    =
    \frac{2!}{2 \pi i}
    \int_{\gamma}
      \frac{f(z)}{(z - 2)^3}
      \dif z.
    $$
    But $f^{(2)}(z) = 9 e^{3z}$, so
    $$
    \int_{\gamma}
      \frac{e^{3z}}{(z - 2)^3}
    = 9 \pi i e^{3z}.
    $$
  }
  \item{
    Here $\gamma$ is the unit circle, so $2 \in \mathrm{Ext}(\gamma)$. Therefore
    $\frac{e^{3z}}{(z - 2)^3}$ is holomorphic on $\mathrm{Int}(\gamma) \cup \gamma$,
    so $\int_\gamma \frac{e^{3z}}{(z-2)^3} = 0$.
  }
\end{enumerate}
\end{Answer}

\end{document}
