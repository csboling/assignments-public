
\documentclass{article}

\usepackage{amsmath}
\usepackage{amsfonts}
\usepackage{amssymb}
\usepackage{enumitem}
\usepackage[lastexercise]{exercise}

\newcommand\dif{\mathop{}\!\mathrm{d}}

\newcounter{Problem}
\newenvironment{Problem}{\begin{Exercise}[name={Problem},
                                          counter={Problem}]}
                        {\end{Exercise}}
\title{MATH 829 Homework \#5}
\date{February 18, 2015}
\author{Sam Boling}

\begin{document}

\begin{titlepage}
\maketitle
%TODO -- \S 2.4(c), last part
%         additional problem
\end{titlepage}

\section{Textbook Problems}

\subsection*{Ch III, \S 2.4}
\begin{itemize}
  \item[(b)]{
    \begin{enumerate}[label=(\alph*)]
      \item{
        \begin{align*}
           \int_\gamma \bar{z} \dif z
        &= \int_0^1 \overline{\gamma(t)} \gamma^\prime(t) \dif t \\
        &= \int_0^1 (1 - it) i \dif t
         = \int_0^1 t \dif t + i \int_0^1 \dif t \\
        &= \left.\frac{t^2}{2}\right|_0^1 + i \left.t\right|_0^1\\
        &= \frac{1}{2} + i.
        \end{align*}
      }
      \item{
        \begin{align*}
           \int_\gamma \bar{z} \dif z
        &= \int_0^1 \overline{\gamma(t)} \gamma^\prime(t) \dif t \\
        &= \int_0^1 e^{\pi i t} (-\pi i e^{-\pi i t}) \dif t \\
        &= -\pi i \int_0^1 \dif t = - \pi i.
        \end{align*}
      }
      \item{
        \begin{align*}
           \int_\gamma \bar{z} \dif z
        &= \int_0^1 \overline{\gamma(t)} \gamma^\prime(t) \dif t \\
        &= \int_0^1 e^{-\pi i t} (\pi i e^{\pi i t}) \dif t \\
        &= \pi i.
        \end{align*}
      }
      \item{
        \begin{align*}
           \int_\gamma \bar{z} \dif z
        &= \int_0^1 \overline{\gamma(t)} \gamma^\prime(t) \dif t \\
        &= \int_0^1 (1 + t^2 - it)(2t + i) \dif t \\
        &= \int_0^1 [(3t + 2t^3) + i(1 - t^2)] \dif t \\
        &= \left[\frac{3}{2}t^2 + \frac{t^4}{2}\right]_0^1
         + i\left[t - \frac{t^3}{3}\right]_0^1 \\
        &= 2 + \frac{2}{3}i.
        \end{align*}
      }
    \end{enumerate}
  }
  \item[(c)]{
    \begin{enumerate}[label=(\alph*)]
      \item{
        \begin{align*}
           \int_\gamma \frac{1}{z} \dif z
        &= \int_0^1 \frac{1}{\gamma(t)} \gamma^\prime(t) \dif t \\
        &= \int_0^1 \frac{i}{1 + it} \dif t \\
        &= \int_0^1 \frac{t}{1 + t^2} \dif t
         + i\int_0^1 \frac{1}{1 + t^2} \dif t \\
        &= \left.\frac{1}{2} \log |1 + t^2|\right|_0^1
         + i\left.\arctan t\right|_0^1 \\
        &= \frac{\log 2}{2} + i\frac{\pi}{4}.
        \end{align*}
        Alternatively since $\gamma$ does not lie on
        the branch cut of the principal logarithm,
        $\log z$ is a primitive for $\frac{1}{z}$ on
        an open set containing $\gamma$ so
        \begin{align*}
           \int_{\gamma} \frac{1}{z} \dif z
        &= \log (\gamma(1)) - \log(\gamma(0))
         = \log (1 + i) \\
        &= \frac{\log 2}{2} + i\frac{\pi}{4}.
        \end{align*}
      }
      \item{
        \begin{align*}
           \int_\gamma \frac{1}{z} \dif z
        &= \int_0^1 \frac{1}{\gamma(t)} \gamma^\prime(t) \dif t \\
        &= \int_0^1 \overline{\gamma(t)} \gamma^\prime(t) \dif t \\
        &= -\pi i
        \end{align*}
        in this case since $\gamma$ lies in the unit circle so
        $\frac{1}{z} = \bar{z}$.
      }
      \item{
        \begin{align*}
           \int_\gamma \frac{1}{z} \dif z
        &= \int_0^1 \frac{1}{\gamma(t)} \gamma^\prime(t) \dif t \\
        &= \int_0^1 \overline{\gamma(t)} \gamma^\prime(t) \dif t \\
        &= \pi i
        \end{align*}
        in this case since $\gamma$ lies in the unit circle so
        $\frac{1}{z} = \bar{z}$.
      }
      \item{
        Consider that $\frac{1}{z}$ is continuous on the open set
        $$
        U = \mathbb{C} - \{ z \in \mathbb{R} : \mathrm{Re} z \leq 0 \}
        $$
        and that $\gamma \in U$, since
        $$
        \mathrm{Re} (\gamma(t)) = 1 + t^2 > 0
        $$
        for $t \in [0, 1]$. Furthermore the principal logarithm
        $\mathrm{Log} z \in (-\pi, \pi]$ is a primitive for
        $\frac{1}{z}$ on $U$ (as seen in Homework 2, additional problem 1).
        Therefore
        \begin{align*}
        \int_\gamma \frac{1}{z} \dif z
        &=
        \mathrm{Log}(\gamma(1)) - \mathrm{Log}(\gamma(0))
         =
        \mathrm{Log}(2 + i) \\
        &= (\log \sqrt{5})\left(\arctan \frac{1}{2}\right).
        \end{align*}
      }
    \end{enumerate}
  }
\end{itemize}

\subsection*{Ch III, \S 2.6}
Because $\sin z$ has primitive $-\cos z$ on $\mathbb{C}$,
we have
\begin{align*}
   \int_\gamma \sin z \dif z
&= -\cos(1 + i) - (-\cos 0)
 = 1 - \cos (1 + i).
\end{align*}

\section{Additional Problems}

\begin{Problem}
Let $a, b \in \mathbb{C}$ and $c \in [a, b]$. Let $f$ be continuous
on $[a, b]$. Use the definition to show that
$$
\int_{[a, b]} f = \int_{[a, c]} f + \int_{[c, b]} f.
$$
\end{Problem}

\begin{Answer}
The curve $[a, b]$ can be parameterized as
$$
\gamma(t) = a + t (b - a), \quad t \in [0, 1]
$$
so to say that $c \in [a, b]$ means that $c \in \gamma([0, 1])$.
Therefore $c = a + t_c (b - a)$ for some $t_c \in [0, 1]$.

Next we have that
$$
\gamma_{[a,c]}(t) = a + t(c - a), \quad
\gamma_{[c,b]}(t) = c + t(b - c), \quad
t \in [0, 1].
$$
Define reparameterizations
$$
\psi_{[a,c]}(t) = \frac{t}{t_c}, \quad t \in [0, t_c]
$$
and
$$
\psi_{[c,b]}(t) = \frac{t - t_c}{1 - t_c}, \quad t \in [t_c, 1]
$$
so that

\begin{align*}
   \int_{[a,c]} f
&= \int_{\gamma_{[a, c]} \circ \psi_{[a,c]}} f \\
&= \int_0^{t_c}
     f(\gamma_{[a, c]}(\psi_{[a,c]}(t)))
     (\gamma_{[a, c]} \circ \psi_{[a,c]})^\prime (t)
     \dif t \\
&= \int_0^{t_c}
     f\left(a + \frac{t}{t_c}(c - a)\right)
     \frac{1}{t_c}(c - a)
     \dif t, \\
   \int_{[c, b]} f
&= \int_{\gamma_{[c, b]} \circ \psi_{[c, b]}} f \\
&= \int_0^{t_c}
     f(\gamma_{[c, b]}(\psi_{[c, b]}(t)))
     (\gamma_{[c, b]} \circ \psi_{[c, b]})^\prime (t)
     \dif t \\
&= \int_{t_c}^{1}
     f\left(c + \frac{t - t_c}{1 - t_c}(b - c)\right)
     \frac{1}{1 - t_c}(b - c)
     \dif t
\end{align*}
But from $c = a + t_c(b-a)$ we have
\begin{align*}
c - a &= t_c (b - a)
\end{align*}
and
\begin{align*}
   b - c
&= b - (a + t_c (b - a)) \\
&= b - a - t_c(b - a) \\
&= (1 - t_c)(b - a)
\end{align*}
so that
\begin{align*}
\int_0^{t_c}
  f\left(a + \frac{t}{t_c}(c - a)\right)
  \frac{1}{t_c}(c - a)
  \dif t
&=
\int_0^{t_c}
  f(a + t(b - a))
  (b - a)
  \dif t
\end{align*}
and
\begin{align*}
\int_{t_c}^{1}
  f\left(c + \frac{t - t_c}{1 - t_c}(b - c)\right)
  \frac{1}{1 - t_c}(b - c) \dif t
&=
\int_{t_c}^{1}
  f(c + (t - t_c)(b - a))
  (b - a)
  \dif t \\
&=
\int_{t_c}^{1}
  f(a + t(b - a))
  (b - a)
  \dif t
\end{align*}
so that
\begin{align*}
\int_{[a,c]} f + \int_{[c,b]} f
&=
\int_0^{t_c}
  f(a + t(b - a))
  (b - a)
  \dif t
+
\int_{t_c}^{1}
  f(a + t(b - a))
  (b - a)
  \dif t \\
&=
\int_0^1
  f(a + t(b - a))
  (b - a)
  \dif t \\
&=
\int_{[a, b]} f.
\end{align*}

\end{Answer}

\end{document}
