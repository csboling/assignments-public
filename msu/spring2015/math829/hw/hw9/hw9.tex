
\documentclass{article}

\usepackage{amsmath}
\usepackage{amsfonts}
\usepackage{amssymb}
\usepackage{enumerate}
\usepackage[lastexercise]{exercise}

\newcommand\dif{\mathop{}\!\mathrm{d}}
\newcommand\horline{\noindent\makebox[\linewidth]{\rule{\textwidth}{0.4pt}}}

\newcounter{Problem}
\newenvironment{Problem}{\begin{Exercise}[name={Problem},
                                          counter={Problem}]}
                        {\end{Exercise}}
\title{MATH 829 Homework \#9}
\date{March 25, 2015}
\author{Sam Boling}

\begin{document}

\begin{titlepage}
\maketitle
\end{titlepage}

\section{Textbook Problems}
\subsection{Chapter V, \S 2.8}
First note that
$$
  \frac{1}{(z - 1)(z-2)}
= \frac{-1}{z - 1}
+ \frac{1}{z - 2}
$$
\begin{enumerate}[(a)]
  \item{
    In the disc $|z| < 1$ we have also $\left|\frac{z}{2}\right| < 1$ so we can
    use the geometric series
    $$
      -\frac{1}{z - 1}
    =  \frac{1}{1 - z}
    =  \sum_{n=0}^\infty z^n, \quad
      \frac{1}{z - 2}
    = \frac{1}{-2(1 - \frac{z}{2})}
    = -\frac{1}{2}
       \sum_{n=0}^\infty
         \left(\frac{z}{2}\right)^n
    = -\sum_{n=0}^\infty
         \frac{1}{2^{n+1}} z^n
    $$
    so that
    $$
      \frac{1}{(z-1)(z-2)}
    = \sum_{n=0}^\infty
        \left(
          1 - \frac{1}{2^{n+1}}
        \right)
        z^n.
    $$
  }
  \item{
    In this annulus we have $1 < |z| < 2$, so
    $\left|\frac{1}{z}\right| < 1$ and
    $\left|\frac{z}{2}\right| < 1$. Then
    \begin{align*}
       \frac{1}{z - 1}
    &= \frac{\frac{1}{z}}
           {1 - \frac{1}{z}} \\
    &= \frac{1}{z}
       \sum_{n=0}^\infty
         \left(\frac{1}{z}\right)^n
     = \sum_{n=0}^\infty z^{-(n+1)}
     = \sum_{n=1}^\infty z^{-n} \\
    &= \sum_{n=-\infty}^{-1} z^n
    \end{align*}
    and
    $$
      \frac{1}{z - 2}
    = \frac{\frac{1}{2}}
           {\frac{z}{2} - 1}
    = -\frac{1}{2}
       \frac{1}{1 - \frac{z}{2}}
    = -\frac{1}{2}
       \sum_{n=0}^\infty
         \left(\frac{z}{2}\right)^n
    $$
    so that
    \begin{align*}
       \frac{1}{(z-1)(z-2)}
    &= \sum_{n=-\infty}^{-1}
        (-1) z^n
     + \sum_{n=0}^\infty
        \left(-\frac{1}{2^{n+1}}\right) z^n \\
    &= -\left[
          \sum_{n=0}^\infty
            \frac{1}{2^{n+1}}
            z^n
        + \sum_{n=-\infty}^{-1}
            z^n
        \right]
    \end{align*}
    on $1 < |z| < 2$.
  }
  \item{
    Here we have $1 < |z|$ and $1 < \left|\frac{z}{2}\right|$, so
    $$
      \frac{1}{z - 1}
    = \sum_{n=-\infty}^{-1}
        z^{n}
    $$
    as in part (b)
    and $\left|\frac{2}{z}\right| < 1$ so that
    $$
      \frac{1}{z - 2}
    = \frac{\frac{1}{z}}{1 - \frac{2}{z}}
    = \frac{1}{z}
      \sum_{n=0}^\infty
        \left(\frac{2}{z}\right)^n
    = \sum_{n=1}^\infty
        2^{n-1} z^{-n}
    = \sum_{n=-\infty}^{-1}
        2^{1-n} z^n
    $$
    so that
    $$
      \frac{1}{(z-1)(z-2)}
    = \sum_{n=-\infty}^{-1}
        (2^{1-n} - 1) z^n.
    $$
  }
\end{enumerate}

\section{Additional Problems}

\begin{Problem}
Suppose that $f$ is holomorphic on
$A = \{ r < |z| < R \}$, where $0 \leq r < R \leq \infty$.
Suppose that there are two series of complex numbers
$(a_n)_{n \in \mathbb{Z}}$ and $(b_n)_{n \in \mathbb{Z}}$ such that
$$
f(z) = \sum_{n=-\infty}^\infty a_n z^n = \sum_{n=-\infty}^\infty b_n z^n
$$
for $z \in A$. Show that $a_n = b_n$ for all $n \in \mathbb{Z}$.
\end{Problem}

\begin{Answer}
We have that $\sum_{n=0}^\infty c_n z^n$ converges in $\{ |z| < R \}$
and $\sum_{n=-\infty}^{-1} c_n z^n$ converges in $\{ |z| > r \}$.
Let
$$
  f(z)
= \left\{\begin{array}{c c}
           \sum_{n=0}^\infty c_n z^n, & \quad |z| < R \\
           \sum_{n=-\infty}^{-1} c_n z^n, & \quad |z| > R
         \end{array}\right.
$$
Then $f$ is well-defined and
$$
  \lim_{|z| \to \infty} f(z)
= \lim_{w \to 0} f\left(\frac{1}{w}\right)
= \lim_{w \to 0} -\sum_{n=1}^\infty c_n z^n
$$
which has no constant terms and is thus 0. We can then
show that $f$ is bounded.
\end{Answer}

% \begin{Answer}
% Since under these assumptions
% $$
%   0
% = f(z) - f(z)
% = \sum_{n=-\infty}^\infty a_n z^n - \sum_{n=-\infty}^\infty b_n z^n
% = \sum_{n=-\infty}^\infty (a_n - b_n) z^n,
% $$
% it suffices to show that the constant function 0 has all
% Laurent coefficients 0, for in this case $a_n - b_n = 0$ and
% so $a_n = b_n$.

% Let $Z$ be the entire function such that
% $Z(z) = 0$ for all $z \in A$. Consider the Laurent expansion for $Z$ around 0, i.e.
% $$
%   0
% = \sum_{n=-\infty}^\infty a_n z^n
% = \sum_{n=-\infty}^{-1} a_n z^n
% + \sum_{n=0}^\infty a_n z^n
% $$
% so that
% $$
%    \sum_{n=0}^\infty a_n z^n
% = -\sum_{n=-\infty}^{-1} a_n z^n
% = -\sum_{n=1}^\infty a_{-n} z^{-n}.
% $$
% Since $Z$ is entire, this Laurent expansion holds everywhere, so
% $\sum_{n=0}^\infty a_n z^n$ and $\sum_{n=-\infty}^{-1} a_n z^n$ converge everywhere.
% Since $\sum_{n=0}^\infty a_n z^n$ is convergent, we have an analytic
% function
% $$
% g(z) = \sum_{n=0}^\infty a_n z^n = -\sum_{n=1}^\infty a_{-n} z^{-n}.
% $$
% \begin{itemize}
%   \item{
%     Suppose $|z| \leq 1$. Then
%     $$
%          |g(z)|
%     =    \left|\sum_{n=0}^\infty a_n z^n\right|
%     \leq \sum_{n=0}^\infty |a_n| |z|^n
%     \leq \sum_{n=0}^\infty |a_n|.
%     $$

%     But since the power series
%     $\sum_{n=0}^\infty a_n z^n$ has an infinite radius of convergence,
%     it converges absolutely, so we have a function
%     $\tilde{g} : \mathbb{C} \to \mathbb{R}$ given by
%     $$
%     \tilde{g}(z) = \sum_{n=0}^\infty |a_n||z|^n,
%     $$
%     and since $\sum_{n=0}^\infty |a_n| = \tilde{g}(1) < \infty$
%     it follows that $g$ is bounded.
%   }
%   \item{
%     Suppose $|z| > 1$. Then $\left|\frac{1}{z}\right| < 1$ so
%     $$
%          |g(z)|
%     =    \left|-\sum_{n=1}^\infty a_{-n} z^{-n}\right|
%     \leq \sum_{n=1}^\infty
%            |a_{-n}|
%            \left|\frac{1}{z}\right|^n
%     \leq \sum_{n=1}^\infty |a_{-n}|.
%     $$

%     But since the power series
%     $\sum_{n=1}^\infty a_{-n} \left(\frac{1}{z}\right)^n$ has an infinite
%     radius of convergence, it converges absolutely, so we have a function
%     $\tilde{g} : \mathbb{C} \to \mathbb{R}$ given by
%     $$
%     \tilde{g}(z) = \sum_{n=0}^\infty |a_{-n}|\left|\frac{1}{z}\right|^n,
%     $$
%     and since $\sum_{n=0}^\infty |a_{-n}| = \tilde{g}(1) < \infty$ it follows
%     that $g$ is bounded.
%   }
% \end{itemize}
% Therefore $g$ is a bounded entire function and thus constant.
% Since $g$ is entire the principal part of the series
% $g(z) = -\sum_{n=-\infty}^{-1} a_n z^n$ must vanish, so that
% $a_{n} = 0$ for $n < 0$.
% Since $g$ is constant we have $a_n = \frac{g^{(n)}(0)}{n!} = 0$ for
% $n > 0$. But then since
% $$
%   \sum_{n=0}^\infty a_n z^n
% = a_0 + \sum_{n=1}^\infty a_n z^n
% = -\sum_{n=-\infty}^{-1} a_n z^n
% = 0
% $$
% we have $a_0 = 0$ as well.
% \end{Answer}

% \begin{Problem}
% Suppose $f$ is holomorphic on $r < |z| < R$. Prove that for any
% $s \in (r, R)$ and $n \in \mathbb{Z}$,
% $$
%    \int_{|z| = s} \frac{f^\prime(z)}{z^n} \dif z
% = n\int_{|z| = s} \frac{f(z)}{z^{n+1}} \dif z
% $$
% \end{Problem}

\begin{Answer}
Since $f$ is holomorphic on this annulus, its Laurent
coefficients are given by
$$
  a_n
= \frac{1}{2 \pi i}
  \int_{|z| = s}
    \frac{f(z)}{z^{n+1}}
    \dif z
$$
so that
$$
\int_{|z| = s}
    \frac{f(z)}{z^{n+1}}
    \dif z
= 2 \pi i a_n.
$$
Furthermore, we have
$$
   f^\prime(z)
= \sum_{n=-\infty}^\infty n a_n z^{n-1} \\
= \frac{1}{z} \sum_{n=-\infty}^{\infty} n a_n z^n
$$
which means that
$$
z f^\prime(z) = \sum_{n=-\infty}^\infty n a_n z^n.
$$

Since $f$ is holomorphic on $A$, so must $f^\prime$ be, and the
map $z \mapsto z$ is entire, so the function $z \mapsto z f^\prime(z)$
is holomorphic on $A$. Therefore $z f^\prime(z)$ has a Laurent expansion on $A$
given by
$z f^\prime(z) = \sum_{n=-\infty}^\infty b_n z^n$, where
$$
  b_n
= \frac{1}{2 \pi i}
  \int_{|z| = s}
    \frac{z f^\prime(z)}{z^{n+1}}
    \dif z
= \frac{1}{2 \pi i}
  \int_{|z| = s}
    \frac{f^\prime(z)}{z^n}
    \dif z.
$$
But the Laurent expansion is unique, so we must have $b_n = n a_n$, and then
$$
  \int_{|z| = s}
    \frac{f^\prime(z)}{z^n}
    \dif z
= n \cdot 2 \pi i a_n
= n \int_{|z| = s}
      \frac{f(z)}{z^{n+1}}
      \dif z.
$$

\end{Answer}

\end{document}
