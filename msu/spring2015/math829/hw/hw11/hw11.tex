\documentclass{article}

\usepackage{amsmath}
\usepackage{amsfonts}
\usepackage{amssymb}
\usepackage{enumerate}
\usepackage[lastexercise]{exercise}

\newcommand\id{\mathrm{id}}
\newcommand\ord{\mathrm{ord}}
\newcommand\Res{\mathrm{Res}}
\renewcommand\Im{\mathrm{Im}}
\renewcommand\Re{\mathrm{Re}}
\newcommand\dif{\mathop{}\!\mathrm{d}}
\newcommand\horline{\noindent\makebox[\linewidth]{\rule{\textwidth}{0.4pt}}}

\newcounter{Problem}
\newenvironment{Problem}{\begin{Exercise}[name={Problem},
                                          counter={Problem}]}
                        {\end{Exercise}}
\title{MATH 829 Homework \#11}
\date{Apr 8, 2015}
\author{Sam Boling}

\begin{document}

\begin{titlepage}
\maketitle
\end{titlepage}

\section{Textbook Problems}

\subsection*{Chapter V, \S 3.7}
Since $f$ has order $n$ at $w_0$, we may write
$$
f(w) = g(w) (w - w_0)^n
$$
for an analytic function $g$ such that $g(w_0) \neq 0$.
Then
\begin{align*}
  (f \circ \varphi)(z)
&= f(\varphi(z))
 = g (\varphi(z)) \cdot (\varphi(z) - \varphi(z_0))^n \\
&= (g \circ \varphi) (z) \cdot (\varphi(z) - \varphi(z_0))^n
\end{align*}
so that
\begin{align*}
   \ord_{z_0} (f \circ \varphi)
&= \ord_{z_0} (g \circ \varphi)
 + \ord_{z_0} (\varphi - \varphi(z_0))^n \\
&= \ord_{z_0} (g \circ \varphi)
 + n \cdot \ord_{z_0} (\varphi - \varphi(z_0)).
\end{align*}
But $\varphi$ is an analytic isomorphism, so
$$
  (\varphi - \varphi(z_0))^\prime(z_0)
= \varphi^\prime(z_0)
\neq 0
$$
since $\varphi^\prime$ is nonzero on all of its domain $V$.
Since $(\varphi - \varphi(z_0))(z_0) = 0$, this means that
$\ord_{z_0} (\varphi - \varphi(z_0)) = 1$ since
$\varphi - \varphi(z_0)$ is analytic. Furthermore,
since $g$ is analytic and
$g(\varphi(z_0)) = g(w_0) \neq 0$, we have
$\ord_{z_0} (g \circ \varphi) = 0$. Hence
$\ord_{z_0} (f \circ \varphi) = n$.

\subsection*{Chapter VI, \S 2.2}
\begin{enumerate}[(a)]
  \item{
    Let $f(z) = \frac{z^2}{z^4 + 1}$.
    First observe that
    $$
      f(-z)
    = \frac{(-z)^2}{(-z)^4 + 1}
    = \frac{z^2}{z^4 + 1}
    = f(z)
    $$
    so that this is an even function, the poles of which are
    at $\pm \frac{\sqrt{2}}{2} \pm i \frac{\sqrt{2}}{2}$. Consider
    $\Gamma = [-R, R] \oplus S_R$ where $S_R$ is a positively oriented
    semicircle of radius $R$ in the upper half-plane. Then
    \begin{align*}
          \left|
            \int_{S_R} f
          \right|
    &\leq \left\|
            \frac{z^2}{z^4 + 1}
          \right\|_{S_R}
          L(S_R)
     \leq \frac{|z|^2}{|z|^4 - 1} \pi R
     =    \frac{\pi R^3}{R^4 - 1}
     \to 0
    \end{align*}
    as $R \to \infty$, since $|z^4 + 1| \geq |z|^4 - 1$.

    Then
    \begin{align*}
       \lim_{R \to \infty}
       \int_\Gamma
         f
    &= \lim_{R \to \infty}
       \left[
         \int_{[-R, R]}
           \frac{z^2}{z^4 + 1}
           \dif z
       + \int_{S_R}
           f
       \right]
     = \mathrm{P.V.}
         \int_{-\infty}^\infty
           \frac{x^2}{x^4 + 1}
           \dif x \\
    &= 2 \pi i
       ( \Res_{\frac{\sqrt{2}}{2}(1 + i)} f
       + \Res_{\frac{\sqrt{2}}{2}(-1 + i)} f).
    \end{align*}
    We have also
    $$
      \Res_{z_k} \frac{z^2}{z^4 + 1}
    = \frac{z_k^2}
           {(z^4 + 1)^\prime(z_k)}
    = \frac{z_k^2}
           {4z_k^3}
    = \frac{1}{4 z_k}
    $$
    so that
    $$
      \mathrm{P.V.}
      \int_{-\infty}^\infty
        \frac{x^2}{x^4 + 1}
        \dif x
    = \frac{2 \pi i}{4}
      \frac{2}{\sqrt{2}}
      \left(
        \frac{1}
             {1 + i}
      + \frac{1}
             {-1 + i}
      \right)
    = \frac{\pi}{\sqrt{2}}
    = \frac{\pi \sqrt{2}}{2}.
    $$
    Since the integrand is an even function,
    the integral is equal to its principal value,
    so
    $$
      \int_{-\infty}^\infty
        \frac{x^2}
             {x^4 + 1} \dif x
    = \frac{\pi \sqrt{2}}{2}
    $$
    as desired.
  }
  \item{
    Let
    $$
    f(z) = \frac{z^2}{z^6 + 1}
    $$
    and observe that
    $$
      f(-z)
    = \frac{(-z)^2}{(-z)^6 + 1}
    = \frac{z^2}{z^6 + 1}
    = f(z)
    $$
    so that $f$ is an even function with poles at the
    sixth roots of $-1$, of which three lie in the
    upper half-plane, namely
    $z_1 = e^{i\frac{\pi}{6}}$, $z_2 = i$, and $z_3 = e^{i\frac{\pi}{3}}$.

    Next observe that on the semicircle $S_R$,
    since $|z^6 + 1| \geq |z|^6 - 1$ we have
    $$
         \left|
           \int_{S_R}
             \frac{z^2}{z^6 + 1}
         \right|
    \leq \frac{|R|^2}{R^6 - 1}  L(S_R)
    =    \frac{\pi R^3}{R^6}
    \to  0
    $$
    as $R \to \infty$. Therefore since the integrand
    is even we have
    \begin{align*}
        \int_{-\infty}^\infty
          \frac{x^2}{x^6 + 1}
          \dif x
     &= \int_{[-R, R] \oplus S_R}
          \frac{z^2}{z^6 + 1}
          \dif z
      = 2 \pi i
        \sum_{k=1}^3
          \Res_{z_k} \frac{z^2}{z^6 + 1} \\
     &= 2 \pi i
         \sum_{k=1}^3
           \frac{z_k^2}{6z_k^5}
      = \frac{\pi i}{3}
          \sum_{k=1}^3
            \frac{1}{z_k^{3}} \\
     &= \frac{\pi i}{3}
          ( e^{-i \frac{\pi}{2}}
          + e^{-i 2 \pi}
          + e^{-i \pi})
      = \frac{\pi}{3}
    \end{align*}
    so that
    $$
      \int_{0}^\infty \frac{x^2}{x^6 + 1} \dif x
    = \frac{1}{2}\int_{-\infty}^\infty \frac{x^2}{x^6 + 1} \dif x
    = \frac{\pi}{6}.
    $$
  }
\end{enumerate}

\subsection*{Chapter VI, \S 2.8(a)}
The function
$$
f(z) = \frac{e^{iz}}{(z^2 + a^2)^2}
$$
has poles at $\pm i a$, each of order 2, and
$$
  \frac{e^{iz}}{(z^2 + a^2)^2}
= \frac{\cos z}{(z^2 + a^2)^2}
+ i \frac{\sin z}{(z^2 + a^2)^2}.
$$
Since
$$
  i\frac{\sin (-z)}{((-z)^2 + a^2)^2}
= i\frac{-\sin z}{(z^2 + a^2)^2},
$$
this term is an odd function, so that
$$
  \int_{-R}^R
    \frac{e^{ix}}{(x^2 + a^2)^2}
    \dif x
= \int_{-R}^R
    \frac{\cos x}{(x^2 + a^2)^2}
    \dif x
+ i \int_{-R}^R
    \frac{\sin x}{(x^2 + a^2)^2}
    \dif x
= \int_{-R}^R
    \frac{\cos x}{(x^2 + a^2)^2}
    \dif x.
$$
On the semicircle $S_R$, we have
$$
     \left|
       \int_{S_R} f
     \right|
\leq \left\|
       \frac{e^{iz}}
            {(a^2 + z^2)^2}
     \right\|
     L(S_R).
$$
Since $|e^{iz}| = e^{\Re(iz)}$ and
$\Im(z) \geq 0$ on $S_R$,
$\Re(iz) \leq 0$ so that $|e^{iz}| \leq 1$.
Since $|z^2 + a^2| \geq |z|^2 - a^2$, whenever
$R > |a|$ this means that
$$
     \left|
       \frac{e^{iz}}
            {(z^2 + a^2)^2}
     \right| L(S_R)
\leq \frac{\pi R}{(R^2 - a^2)^2}
\to  0.
$$
Therefore since $\frac{\cos x}{(x^2 + a^2)^2}$ is an
even function,
\begin{align*}
    \int_{-\infty}^\infty
      \frac{\cos x}
           {(x^2 + a^2)^2}
&= 2 \pi i \cdot
   \Res_{ia} \frac{e^{iz}}{(z^2 + a^2)^2}.
\end{align*}
To compute this residue, we expand $e^{iz}$ around $z = ia$
by first computing
$$
  \left.\frac{d^k}{dz^k} e^{iz}\right|_{z = ia}
= i^k e^{-a}
$$
so that
\begin{align*}
  e^{iz}
&= \sum_{k=0}^\infty
     \frac{i^k e^{-a}}{k!}
     (z - ia)^k
\end{align*}

We also see that
$$
   z^2 + a^2
 = (z - ia)(z + ia)
$$
so
\begin{align*}
   \frac{e^{iz}}{(z^2 + a^2)^2}
&= \frac{1}{(z + ia)^2}
   \sum_{k=0}^\infty
     \frac{i^k e^{-a}}{k!}
     (z - ia)^{k-2} \\
&= \left(
     \frac{1}{4}
     \sum_{k=0}^\infty
       \frac{i^{k+2}}{a^{k+2} k!}
       (z - ia)^k
   \right)
   \left(
     \sum_{k=0}^\infty
       \frac{i^k e^{-a}}{k!}
       (z - ia)^{k-2}
   \right) \\
&= \frac{-e^{-a}}{4a^2 (z - ia)^2}
   \left(
     \sum_{k=0}^\infty
       \frac{i^k}{a^k k!}
       (z - ia)^k
   \right)
   \left(
     \sum_{k=0}^\infty
       \frac{i^k}{k!}
       (z - ia)^{k}
   \right) \\
&= \frac{-e^{-a}}{4a^2 (z - ia)^2}
   \sum_{k=0}^\infty
     \left(
       \sum_{j=0}^k
         \frac{i^j}
              {a^j j!}
         \frac{i^{k-j}}
              {(k - j)!}
     \right)
     (z - ia)^k,
\end{align*}
using the expansion
$$
  \frac{1}{(z + ia)^2}
= \frac{1}{4}
     \sum_{k=0}^\infty
       \frac{i^{k+2}}{a^{k+2} k!}
       (z - ia)^k.
$$

Then by definition
\begin{align*}
   \Res_{ia} f(z)
&= \frac{-e^{-a}}{4a^2}
   i^1
   \sum_{j=0}^1
     \frac{1}
          {a^j j! (k - j)!}
 = -\frac{i e^{-a}}{4a^2}
   \left(
     \frac{1}{a} + 1
   \right) \\
&= -\frac{i}{4}
   \frac{1}{a^3 e^a}
    (1 + a)
\end{align*}
and therefore
$$
  \int_{-\infty}^\infty
    \frac{\cos x}
         {(x^2 + a^2)^2}
    \dif x
= 2 \pi i \cdot \Res_{ia} f(z)
= \frac{\pi (1 + a)}
       {2 a^3 e^a}
$$
as desired.

\subsection*{Chapter VI, \S 2.9}
First we see that
\begin{align*}
   1 - e^{2 i z}
&= 1 - (e^{iz})^2
 = 1 - (\cos z + i \sin z)^2 \\
&= 1 - (\cos^2 z + 2 i \sin z \cos z - \sin^2 z) \\
&= 1 - \cos^2 z + i \sin 2z + \sin^2 z
 = 2 \sin^2 z + i \sin 2z,
\end{align*}
and $i \sin 2z$ is an even function. Therefore since
$f(z) = \frac{\sin^2 z}{z^2}$ is an even function,
$$
  \int_{-\infty}^\infty
    \frac{\sin^2 x}{x^2}
    \dif x
= \int_{-\infty}^\infty
    \frac{1 - e^{2ix}}{x^2}
    \dif x.
$$
This integrand $f(z) = \frac{1 - e^{2 i z}}{z^2}$
has a pole of order 2 at 0.
Using the positively oriented semicircle $S_R$ of radius $R$
and the negatively oriented semicircle $S_\varepsilon^{-}$ of radius
$\varepsilon$, we can write
$$
  \Gamma
= [\varepsilon, R] \oplus
  S_R \oplus
  [-R, -\varepsilon] \oplus
  S_\varepsilon^{-}
$$
so that $\int_\Gamma f = 0$. We note that
\begin{align*}
   \int_{[\varepsilon, R]} f
&= \int_\varepsilon^R
     \left(
       \frac{1 - e^{2ix}}
            {x^2}
     \right)
     \dif x
  = \int_\varepsilon^R
      \left(
        \frac{1 - \cos 2x - i \sin 2x}
             {x^2}
      \right)
      \dif x, \\
   \int_{[-R, -\varepsilon]} f
&= \int_{-R}^{-\varepsilon}
     \left(
       \frac{1 - e^{2ix}}
            {x^2}
     \right)
     \dif x
  = \int_{-R}^{-\varepsilon}
      \left(
        \frac{1 - \cos 2x - i \sin 2x}
             {x^2}
      \right)
      \dif x,
\end{align*}
and we can observe that $\frac{1}{x^2}$ and
$\frac{\cos 2x}{x^2}$ are even while $\frac{\sin 2x}{x^2}$ is
odd so that
$$
  \int_{[-R, -\varepsilon]} f
+ \int_{[\varepsilon, R]} f
= 2
  \int_{\varepsilon}^{R}
    \left(
      \frac{1 - \cos 2x}{x^2}
    \right)
    \dif x
= 4
  \int_\varepsilon^R
    \frac{\sin^2 x}{x^2}
    \dif x,
$$
which in the limit as $\varepsilon \to 0$ and $R \to \infty$ is
the integral we wish to evaluate. Therefore
$$
  \int_0^\infty
    \frac{\sin^2 x}{x^2}
    \dif x
= \frac{1}{4}
  \lim_{\substack{R \to \infty \\ \varepsilon \to 0}}
  \left(
    -\int_{S_R} f
    -\int_{S_\varepsilon^{-}} f
  \right)
= \frac{1}{4}
  \lim_{\substack{R \to \infty \\ \varepsilon \to 0}}
  \left(
    \int_{S_\varepsilon} f
  - \int_{S_R} f.
  \right)
$$

Next we see that
$$
   \int_{S_R} f
=  \int_{S_R}
     \frac{1}{z^2}
     \dif z
-  \int_{S_R}
     \frac{e^{iz}}{z^2}
     \dif z
$$
and
$$
     \left|
       \int_{S_R}
         \frac{\dif z}{z^2}
     \right|
\leq \left\|
       \frac{1}{z^2}
     \right\|_{S_R}
     L(S_R)
=    \frac{\pi R}{R^2}
\to  0,
$$
\begin{align*}
      \left|
        \int_{S_R}
          \frac{e^{iz}}{z^2}
          \dif z
      \right|
&\leq \left\|
        \frac{e^{iz}}{z^2}
      \right\|_{S_R}
      L(S_R)
 =    \|e^{iz}\|_{S_R} \frac{\pi R}{R^2}.
\end{align*}
But $|e^{iz}| = e^{-\Im(z)}$, and on $S_R$ we have
$\Im(z) > 0$ so that $e^{-\Im(z)} < 1$. Therefore
$\int_{S_R} f \to 0$ as $R \to \infty$.

Next we observe that on the circle
$$
\gamma(t) = \varepsilon e^{i t}, \quad
0 \leq t \leq 2 \pi
$$
we have
\begin{align*}
   \int_{S_\varepsilon} f
&= \frac{1}{2}
   \int_\gamma f
 = \frac{2 \pi i}{2}
   \left(
     \Res_0 \frac{1}{z^2}
   + \Res_0 \left(
       -\frac{e^{2 i z}}{z^2}
     \right)
   \right).
\end{align*}
$\frac{1}{z^2}$ is its own Laurent series around 0 and so
$\Res_0 \frac{1}{z^2} = 0$. Meanwhile
$$
\frac{d}{dz} e^{2 i z} = 2 i e^{2 i z}
$$
so that
$$
e^{2 i z} = 1 + 2 i z + \cdots
$$
and so $\Res_0 \frac{e^{2 i z}}{z^2} = 2 i$.
Therefore $\int_\gamma f = 4 \pi$ and it follows that
$$
  \int_0^\infty
    \frac{\sin^2 x}{x^2}
    \dif x
= \frac{1}{4}
  \left(
    \frac{1}{2}
    \int_\gamma f
  \right)
= \frac{\pi}{2}.
$$

\section{Additional Problems}
\begin{Problem}
Let $f_1, f_2$ be analytic isomorphisms. Show that
$f_1^{-1}$ and $f_1 \circ f_2$ are analytic isomorphisms.
\end{Problem}

\begin{Answer}
For the composition $f_1 \circ f_2$ to be well-defined,
we have analytic functions $f_2 : U \to f_2(U)$ and
$f_1 : f_2(U) \to f_1(f_2(U))$ for some open set $U$.

By assumption, $f_1$'s inverse $f_1^{-1}$ is analytic.
Furthermore $f_1 \circ f_1^{-1} = \id_{f_2(U)} = f_1^{-1} \circ f_1$,
so $f_1$ is an analytic inverse of $f_1^{-1}$, hence
$f_1^{-1}$ is an analytic isomorphism.

Since $f_1, f_2$ are analytic, so is $f_1 \circ f_2$.
Furthermore we have analytic functions $f_1^{-1}$, $f_2^{-1}$,
so $f_2^{-1} \circ f_1^{-1}$ is analytic and so
\begin{align*}
  (f_2^{-1} \circ f_1^{-1}) \circ (f_1 \circ f_2)
&= f_2^{-1} \circ (f_1^{-1} \circ f_1) \circ f_2)
 = f_2^{-1} \circ \id_{f_2(U)} \circ f_2 \\
&= f_2^{-1} \circ f_2
 = \id_U
\end{align*}
since function composition is associative. Therefore
$f_2^{-1} \circ f_1^{-1}$ is an analytic inverse of $f_1 \circ f_2$,
so $f_1 \circ f_2$ is an analytic isomorphism.
\end{Answer}

\begin{Problem}
Suppose $f$ is analytic on $D(z_0, R) \setminus \{ z_0 \}$
and $z_0$ is a pole of $f$. Prove that for any $r \in (0, R)$
there is $M \in (0, \infty)$ such that
$f(D(z_0, r) \setminus \{z_0\})
 \supset
 \{ z \in \mathbb{C} : |z| > M \}$.
\end{Problem}

\begin{Answer}
Let $U = D(z_0, r) \setminus \{ z_0 \}$.
Showing that
$f(U) \supset \{ |z| > M \}$
is equivalent to showing that
$w \notin f(U)$ implies
$|w| \leq M$, or that
$\mathbb{C} \setminus f(U) = f(U)^c$ is bounded.

Since $|f(z)|$ can be made arbitrarily large by choosing
$z$ sufficiently close to $z_0$, $f$ is not constant on
$U = D(z_0, r) \setminus \{ z_0 \}$. Therefore $f(U)$ is open,
and so its complement $f(U)^c = \mathbb{C} \setminus f(U)$ is closed.

Let $\varepsilon > 0$, and let $\bar{A}_\varepsilon$ denote the closed annulus
$\{ z \in \mathbb{C} : \varepsilon \leq |z - z_0| \leq r \}$, which is closed and
bounded and therefore compact. Then $f(\bar{A}_\varepsilon)$ is compact since $f$
is holomorphic on $\bar{A}$ and thus continuous. Therefore
$\min \{ |w - f(z)| : w \in f(U)^c, z \in \bar{A}_\varepsilon \}$ exists, and if
it is equal to 0 then $f(U)^c \cap f(\bar{A}_\varepsilon) \neq \varnothing$. But
$f(U)^c$ has no intersection with $f(U)$ by definition, so
$f(U)^c \cap f(\bar{A})$ must lie in $f(\partial \bar{A}_\varepsilon)$, which is
the continuous image of a compact set and thus compact. It follows
that $f(U)^c$ is bounded.

It remains to show that
$\inf \{ |w - f(z)| : w \in f(U)^c, z \in \bar{A}_\varepsilon \} = 0$
for some choice of $\varepsilon$. Roughly,
$f(z)$ can be made arbitrarily close to an arbitrarily large number.
\end{Answer}

\begin{Problem}
Let $f$ be analytic on a domain $U$, $z_0 \in U$, and $w_0 = f(z_0)$.
Suppose that $\ord_{z_0} (f - w_0) = m \in \mathbb{N}$. Prove that
there exists an open set $U_0$ with $z_0 \in U_0 \subset U$ such that
$f^{-1}(w_0) \cap U_0 = \{ z_0 \}$, and $f^{-1}(w) \cap U_0$ contains
exactly $m$ elements (without repetition) for all
$w \in f(U_0) \backslash \{ w_0 \}$. This means that $f|_{U_0}$ is
$m$-to-1 except at $z_0$.
\end{Problem}

\begin{Answer}
By assumption we can write
$$
f(z) - f(z_0) = (z - z_0)^m h(z)
$$
for an analytic function $h$ with $h(z_0) \neq 0$. Then since $h$ is
continuous, there is a $\delta > 0$ such that
$|h(z) - h(z_0)| < \frac{|h(z_0)|}{2}$ when $|z - z_0| < \delta$, i.e.
$h$ is nonzero on $D(z_0, \delta)$. Therefore
$\frac{f(z) - f(z_0)}{(z - z_0)^m}$ is nonzero on
$D(z_0, \delta) \setminus \{ z_0 \}$, so $f - f(z_0) = f - w_0$
has no zero but $z_0$, i.e.
$f^{-1}(w_0) \cap D(z_0, \delta) = \{ z_0 \}$.

As in the proof of the open mapping theorem,
where $g(z) = w_0 + a_m (z - z_0)^m$,
for any choice of $w \in f(D(z_0, \delta) \setminus \{ z_0 \})$
the function $g_w = g - w$ has $m$ zeros in
$D(z_0, r) \subset D(z_0, \delta)$
for an appropriate choice of $r$, so we may choose
$U_0 = D(z_0, r) \subset D(z_0, \delta)$ so that we still have
$U_0 \cap f^{-1}(w_0) = \{ z_0 \}$. Then
$|g_w(z) - f_w(z)| < |g_w(z)|$, where $f_w(z) = f(z) - w$, so that
$f - w$ has $m$ zeros in $U_0$.

To show that these
zeros of $f - w$ are not repeated, and thus that
$f^{-1}(w) \cap U_0$ has exactly $m$ elements, it suffices to show
that each zero $z_k$ of $f - w$ has order 1, i.e. that
$f(z) - f(z_k) = (z - z_k)\varphi(z)$ for some analytic function $\varphi$ with
$\varphi(z_k) \neq 0$. But
\begin{align*}
   f(z) - f(z_k)
&= f(z) - (f(z_k) - f(z_0)) - f(z_0)
 = (f - w_0)(z) - (f - w_0)(z_k) \\
&= \sum_{n=m}^\infty a_n (z - z_0)^n
 - \sum_{n=m}^\infty a_n (z_k - z_0)^n \\
&= \sum_{n=m}^\infty
     a_n
     \left[
       (z - z_0)^n - (z_k - z_0)^n
     \right] \\
&= \sum_{n=m}^\infty
      a_n
      ((z - z_0) - (z_k - z_0))
      \sum_{j=0}^{n-1}
        (z_k - z_0)^{n - j - 1}
        (z - z_0)^{j} \\
&= (z - z_k)
   \sum_{n=m}^\infty
     a_n
     \sum_{j=0}^{n-1}
       (z_k - z_0)^{n - j - 1}
       (z - z_0)^{j} \\
&= (z - z_k) \varphi(z)
\end{align*}
so that
\begin{align*}
   \varphi(z_k)
&= \sum_{n=m}^\infty
     a_n
     \sum_{j=0}^{n-1}
       (z_k - z_0)^j
       (z_k - z_0)^{n - j - i} \\
&= \sum_{n=m}^\infty
     a_n
     \sum_{j=0}^{n-1}
       (z_k - z_0)^{n-1} \\
&= \sum_{n=m}^\infty
     a_n \cdot n (z_k - z_0)^{n-1}
 = f^\prime(z_k) \neq 0,
\end{align*}
since $f - w_0$ is nonzero except at $z_0$. This completes the proof.
\end{Answer}

\end{document}
