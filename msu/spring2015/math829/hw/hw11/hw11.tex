\documentclass{article}

\usepackage{amsmath}
\usepackage{amsfonts}
\usepackage{amssymb}
\usepackage{enumerate}
\usepackage[lastexercise]{exercise}

\newcommand\id{\mathrm{id}}
\newcommand\ord{\mathrm{ord}}
\newcommand\Res{\mathrm{Res}}
\renewcommand\Im{\mathrm{Im}}
\renewcommand\Re{\mathrm{Re}}
\newcommand\dif{\mathop{}\!\mathrm{d}}
\newcommand\horline{\noindent\makebox[\linewidth]{\rule{\textwidth}{0.4pt}}}

\newcounter{Problem}
\newenvironment{Problem}{\begin{Exercise}[name={Problem},
                                          counter={Problem}]}
                        {\end{Exercise}}
\title{MATH 829 Homework \#11}
\date{Apr 8, 2015}
\author{Sam Boling}

\begin{document}

\begin{titlepage}
\maketitle
\end{titlepage}

\section{Textbook Problems}
\subsection*{Chapter V, \S 3.7}
First we observe that for any $w \in U$,
$$
  (\varphi \circ \varphi^{-1})^\prime(w)
= (\varphi^\prime \circ \varphi^{-1})(w)
  (\varphi^{-1})^\prime(w).
$$
But since $\varphi$ is an analytic isomorphism,
$\varphi^{-1}$ is as well, so
$(\varphi^{-1})^\prime(w) \neq 0$ for any $w \in U$.
Furthermore
$(\varphi \circ \varphi^{-1})(w) = w$ so that
$(\varphi \circ \varphi^{-1})^\prime \equiv 1$.
Therefore
$$
  \frac{1}
       {(\varphi^{-1})^\prime(w)}
= \varphi^\prime(z)
$$
so that since
$$
  (f \circ \varphi)^\prime(z)
= (f^\prime \circ \varphi)(z)
  \varphi^\prime(z)
$$
we have
$$
  (f^\prime \circ \varphi)(z)
= \frac{(f \circ \varphi)^\prime(z)}
       {\varphi^\prime(z)}
= (f \circ \varphi)^\prime(z)
  (\varphi^{-1})^\prime(w)
$$
for any $f : U \to \mathbb{C}$.

Since $f$ has order $n$ at $w_0$, we have an analytic function
$g$ such that
$$
f(w) = \frac{g(w)}{(w - w_0)^n}
$$
for $w \in U - \{ w_0 \}$, where $g(w_0) \neq 0$.
Then
$$
  (f \circ \varphi)(z)
= f(\varphi(z))
= \frac{(g(\varphi(z))}
       {(\varphi(z) - \varphi(z_0))^n}
$$
and then
$$
  (\varphi(z) - \varphi(z_0))^n
  (f \circ \varphi)(z)
= (g \circ \varphi)(z).
$$
Taking a derivative on both sides gives
\begin{align*}
   (g \circ \varphi)^\prime(z)
&= \left[
     \frac{d}{dz}
     (\varphi(z) - \varphi(z_0))^n
   \right]
   (f \circ \varphi)(z) \\
&+ (\varphi(z) - \varphi(z_0))^n
   (f \circ \varphi)^\prime(z) \\
&= n(\varphi(z) - \varphi(z_0))^{n-1}
   \varphi^\prime(z)
   (f \circ \varphi)(z) \\
&+ (\varphi(z) - \varphi(z_0))^n
   (f^\prime \circ \varphi)(z)
   \varphi^\prime(z)
\end{align*}
or
\begin{align*}
   \varphi^\prime(z)(g^\prime \circ \varphi)(z)
&= \varphi^\prime(z)
   n(\varphi(z) - \varphi(z_0))^{n-1}
   (f \circ \varphi)(z) \\
&+ \varphi^\prime(z)
   (\varphi(z) - \varphi(z_0))^n
   (f^\prime \circ \varphi)(z)
\end{align*}
so that
\begin{align*}
   (g^\prime \circ \varphi)(z)
&= n(\varphi(z) - \varphi(z_0))^{n-1}
   (f \circ \varphi)(z) \\
&+ (\varphi(z) - \varphi(z_0))^n
   (f^\prime \circ \varphi)(z).
\end{align*}
Then
$$
  \frac{(g^\prime \circ \varphi)(z)}
       {(\varphi(z) - \varphi(z_0))^{n-1}}
- (\varphi(z) - \varphi(z_0))
  (f^\prime \circ \varphi)(z)
= n (f \circ \varphi)(z).
$$
Since $\varphi$ is bijective, we may
write $w = \varphi(z)$ for a unique $z \in U$ for each
$w \in V$. This gives
$$
  n (f \circ \varphi)(z)
= \frac{g^\prime(w)}
       {(w - w_0)^{n-1}}
- (w - w_0)(f^\prime(w))
$$


\subsection*{Chapter VI, \S 2.2}
\begin{enumerate}[(a)]
  \item{
    Let $f(z) = \frac{z^2}{z^4 + 1}$.
    First observe that
    $$
      f(-z)
    = \frac{(-z)^2}{(-z)^4 + 1}
    = \frac{z^2}{z^4 + 1}
    = f(z)
    $$
    so that this is an even function, the poles of which are
    at $\pm \frac{\sqrt{2}}{2} \pm i \frac{\sqrt{2}}{2}$. Consider
    $\Gamma = [-R, R] \oplus S_R$ where $S_R$ is a positively oriented
    semicircle of radius $R$ in the upper half-plane. Then
    \begin{align*}
          \left|
            \int_{S_R} f
          \right|
    &\leq \left\|
            \frac{z^2}{z^4 + 1}
          \right\|_{S_R}
          L(S_R)
     \leq \frac{|z|^2}{|z|^4 - 1} \pi R
     =    \frac{\pi R^3}{R^4 - 1}
     \to 0
    \end{align*}
    as $R \to \infty$, since $|z^4 + 1| \geq |z|^4 - 1$.

    Then
    \begin{align*}
       \lim_{R \to \infty}
       \int_\Gamma
         f
    &= \lim_{R \to \infty}
       \left[
         \int_{[-R, R]}
           \frac{z^2}{z^4 + 1}
           \dif z
       + \int_{S_R}
           f
       \right]
     = \mathrm{P.V.}
         \int_{-\infty}^\infty
           \frac{x^2}{x^4 + 1}
           \dif x \\
    &= 2 \pi i
       ( \Res_{\frac{\sqrt{2}}{2}(1 + i)} f
       + \Res_{\frac{\sqrt{2}}{2}(-1 + i)} f).
    \end{align*}
    We have also
    $$
      \Res_{z_k} \frac{z^2}{z^4 + 1}
    = \frac{z_k^2}
           {(z^4 + 1)^\prime(z_k)}
    = \frac{z_k^2}
           {4z_k^3}
    = \frac{1}{4 z_k}
    $$
    so that
    $$
      \mathrm{P.V.}
      \int_{-\infty}^\infty
        \frac{x^2}{x^4 + 1}
        \dif x
    = \frac{2 \pi i}{4}
      \frac{2}{\sqrt{2}}
      \left(
        \frac{1}
             {1 + i}
      + \frac{1}
             {-1 + i}
      \right)
    = \frac{\pi}{\sqrt{2}}
    = \frac{\pi \sqrt{2}}{2}.
    $$
    Since the integrand is an even function,
    the integral is equal to its principal value,
    so
    $$
      \int_{-\infty}^\infty
        \frac{x^2}
             {x^4 + 1} \dif x
    = \frac{\pi \sqrt{2}}{2}
    $$
    as desired.
  }
  \item{
    Let
    $$
    f(z) = \frac{z^2}{z^6 + 1}
    $$
    and observe that
    $$
      f(-z)
    = \frac{(-z)^2}{(-z)^6 + 1}
    = \frac{z^2}{z^6 + 1}
    = f(z)
    $$
    so that $f$ is an even function with poles at the
    sixth roots of $-1$, of which three lie in the
    upper half-plane, namely
    $z_1 = e^{i\frac{\pi}{6}}$, $z_2 = i$, and $z_3 = e^{i\frac{\pi}{3}}$.

    Next observe that on the semicircle $S_R$,
    since $|z^6 + 1| \geq |z|^6 - 1$ we have
    $$
         \left|
           \int_{S_R}
             \frac{z^2}{z^6 + 1}
         \right|
    \leq \frac{|R|^2}{R^6 - 1}  L(S_R)
    =    \frac{\pi R^3}{R^6}
    \to  0
    $$
    as $R \to \infty$. Therefore since the integrand
    is even we have
    \begin{align*}
        \int_{-\infty}^\infty
          \frac{x^2}{x^6 + 1}
          \dif x
     &= \int_{[-R, R] \oplus S_R}
          \frac{z^2}{z^6 + 1}
          \dif z
      = 2 \pi i
        \sum_{k=1}^3
          \Res_{z_k} \frac{z^2}{z^6 + 1} \\
     &= 2 \pi i
         \sum_{k=1}^3
           \frac{z_k^2}{6z_k^5}
      = \frac{\pi i}{3}
          \sum_{k=1}^3
            \frac{1}{z_k^{3}} \\
      = \frac{\pi i}{3}
          ( e^{-i \frac{\pi}{2}}
          + e^{-i 2 \pi}
          + e^{-i \pi})
      = \frac{\pi}{3}
    \end{align*}
    so that
    $ \int_{0}^\infty \frac{x^2}{x^6 + 1} \dif x
    = \frac{1}{2}\int_{-\infty}^\infty \frac{x^2}{x^6 + 1} \dif x
    = \frac{\pi}{6}$.
  }
\end{enumerate}

\subsection*{Chapter VI, \S 2.8(a)}
The function
$$
f(z) = \frac{e^{iz}}{(z^2 + a^2)^2}
$$
has poles at $\pm i a$, each of order 2, and
$$
  \frac{e^{iz}}{(z^2 + a^2)^2}
= \frac{\cos z}{(z^2 + a^2)^2}
+ i \frac{\sin z}{(z^2 + a^2)^2}.
$$
Since
$$
  i\frac{\sin (-z)}{((-z)^2 + a^2)^2}
= i\frac{-\sin z}{(z^2 + a^2)^2},
$$
this term is an odd function, so that
$$
  \int_{-R}^R
    \frac{e^{ix}}{(x^2 + a^2)^2}
    \dif x
= \int_{-R}^R
    \frac{\cos x}{(x^2 + a^2)^2}
    \dif x
+ i \int_{-R}^R
    \frac{\sin x}{(x^2 + a^2)^2}
    \dif x
= \int_{-R}^R
    \frac{\cos x}{(x^2 + a^2)^2}
    \dif x.
$$
On the semicircle $S_R$, we have
$$
     \left|
       \int_{S_R} f
     \right|
\leq \left\|
       \frac{e^{iz}}
            {(a^2 + z^2)^2}
     \right\|
     L(S_R).
$$
Since $|e^{iz}| = e^{\Re(iz)}$ and
$\Im(z) \geq 0$ on $S_R$,
$\Re(iz) \leq 0$ so that $|e^{iz}| \leq 1$.
Since $|z^2 + a^2| \geq |z|^2 - a^2$, whenever
$R > |a|$ this means that
$$
     \left|
       \frac{e^{iz}}
            {(z^2 + a^2)^2}
     \right| L(S_R)
\leq \frac{\pi R}{(R^2 - a^2)^2}
\to  0.
$$
Therefore since $\frac{\cos x}{(x^2 + a^2)^2}$ is an
even function,
\begin{align*}
    \int_{-\infty}^\infty
      \frac{\cos x}
           {(x^2 + a^2)^2}
&= 2 \pi i \cdot
   \Res_{ia} \frac{e^{iz}}{(z^2 + a^2)^2}.
\end{align*}
To compute this residue, we expand $e^{iz}$ around $z = ia$
by first computing
$$
  \left.\frac{d^k}{dz^k} e^{iz}\right|_{z = ia}
= i^k e^{-a}
$$
so that
\begin{align*}
  e^{iz}
&= \sum_{k=0}^\infty
     \frac{i^k e^{-a}}{k!}
     (z - ia)^k
\end{align*}

We also see that
$$
   z^2 + a^2
 = (z - ia)(z + ia)
$$
so
\begin{align*}
   \frac{e^{iz}}{(z^2 + a^2)^2}
&= \frac{1}{(z + ia)^2}
   \sum_{k=0}^\infty
     \frac{i^k e^{-a}}{k!}
     (z - ia)^{k-2} \\
&= \left(
     \frac{1}{4}
     \sum_{k=0}^\infty
       \frac{i^{k+2}}{a^{k+2} k!}
       (z - ia)^k
   \right)
   \left(
     \sum_{k=0}^\infty
       \frac{i^k e^{-a}}{k!}
       (z - ia)^{k-2}
   \right) \\
&= \frac{-e^{-a}}{4a^2 (z - ia)^2}
   \left(
     \sum_{k=0}^\infty
       \frac{i^k}{a^k k!}
       (z - ia)^k
   \right)
   \left(
     \sum_{k=0}^\infty
       \frac{i^k}{k!}
       (z - ia)^{k}
   \right) \\
&= \frac{-e^{-a}}{4a^2 (z - ia)^2}
   \sum_{k=0}^\infty
     \left(
       \sum_{j=0}^k
         \frac{i^j}
              {a^j j!}
         \frac{i^{k-j}}
              {(k - j)!}
     \right)
     (z - ia)^k,
\end{align*}
using the expansion
$$
  \frac{1}{(z + ia)^2}
= \frac{1}{4}
     \sum_{k=0}^\infty
       \frac{i^{k+2}}{a^{k+2} k!}
       (z - ia)^k.
$$

Then by definition
\begin{align*}
   \Res_{ia} f(z)
&= \frac{-e^{-a}}{4a^2}
   i^1
   \sum_{j=0}^1
     \frac{1}
          {a^j j! (k - j)!}
 = -\frac{i e^{-a}}{4a^2}
   \left(
     \frac{1}{a} + 1
   \right) \\
&= -\frac{i}{4}
   \frac{1}{a^3 e^a}
    (1 + a)
\end{align*}
and therefore
$$
  \int_{-\infty}^\infty
    \frac{\cos x}
         {(x^2 + a^2)^2}
    \dif x
= 2 \pi i \cdot \Res_{ia} f(z)
= \frac{\pi (1 + a)}
       {2 a^3 e^a}
$$
as desired.

\subsection*{Chapter VI, \S 2.9}
First we see that
\begin{align*}
   1 - e^{2 i z}
&= 1 - (e^{iz})^2
 = 1 - (\cos z + i \sin z)^2 \\
&= 1 - (\cos^2 z + 2 i \sin z \cos z - \sin^2 z) \\
&= 1 - \cos^2 z + i \sin 2z + \sin^2 z
 = 2 \sin^2 z + i \sin 2z,
\end{align*}
and $i \sin 2z$ is an even function. Therefore since
$f(z) = \frac{\sin^2 z}{z^2}$ is an even function,
$$
  \int_{-\infty}^\infty
    \frac{\sin^2 x}{x^2}
    \dif x
= \int_{-\infty}^\infty
    \frac{1 - e^{2ix}}{x^2}
    \dif x.
$$
This integrand $f(z) = \frac{1 - e^{2 i z}}{z^2}$
has a pole of order 2 at 0.
Using the positively oriented semicircle $S_R$ of radius $R$
and the negatively oriented semicircle $S_\varepsilon^{-}$ of radius
$\varepsilon$, we can write
$$
  0
= \int_{S_R} f
- \int_{S_\varepsilon^{-}} f
+ \int_{[\varepsilon, R]} f(z) \dif z
+ \int_{[-R, -\varepsilon]} f(z) \dif z
$$
\end{align*}

\section{Additional Problems}
\begin{Problem}
Let $f_1, f_2$ be analytic isomorphisms. Show that
$f_1^{-1}$ and $f_1 \circ f_2$ are analytic isomorphisms.
\end{Problem}

\begin{Answer}
For the composition $f_1 \circ f_2$ to be well-defined,
we have analytic functions $f_2 : U \to f_2(U)$ and
$f_1 : f_2(U) \to f_1(f_2(U))$ for some open set $U$.

By assumption, $f_1$'s inverse $f_1^{-1}$ is analytic.
Furthermore $f_1 \circ f_1^{-1} = \id_{f_2(U)} = f_1^{-1} \circ f_1$,
so $f_1$ is an analytic inverse of $f_1^{-1}$, hence
$f_1^{-1}$ is an analytic isomorphism.

Since $f_1, f_2$ are analytic, so is $f_1 \circ f_2$.
Furthermore we have analytic functions $f_1^{-1}$, $f_2^{-1}$,
so $f_2^{-1} \circ f_1^{-1}$ is analytic and so
\begin{align*}
  (f_2^{-1} \circ f_1^{-1}) \circ (f_1 \circ f_2)
&= f_2^{-1} \circ (f_1^{-1} \circ f_1) \circ f_2)
 = f_2^{-1} \circ \id_{f_2(U)} \circ f_2 \\
&= f_2^{-1} \circ f_2
 = \id_U
\end{align*}
since function composition is associative. Therefore
$f_2^{-1} \circ f_1^{-1}$ is an analytic inverse of $f_1 \circ f_2$,
so $f_1 \circ f_2$ is an analytic isomorphism.
\end{Answer}

\begin{Problem}
Suppose $f$ is analytic on $D(z_0, R) \backslash \{ z_0 \}$
and $z_0$ is a pole of $f$. Prove that for any $r \in (0, R)$
there is $M \in (0, \infty)$ such that
$f(D(z_0, r) \backslash \{z_0\})
 \supset
 \{ z \in \mathbb{C} : |z| > M \}$.
\end{Problem}

\begin{Problem}
Let $f$ be analytic on a domain $U$, $z_0 \in U$, and $w_0 = f(z_0)$.
Suppose that $\ord_{z_0} (f - w_0) = m \in \mathbb{N}$. Prove that
there exists an open set $U_0$ with $z_0 \in U_0 \subset U$ such that
$f^{-1}(w_0) \cap U_0 = \{ z_0 \}$, and $f^{-1}(w) \cap U_0$ contains
exactly $m$ elements (without repetition) for all
$w \in f(U_0) \backslash \{ w_0 \}$. This means that $f|_{U_0}$ is
$m-to1$ except at $z_0$.
\end{Problem}

\end{document}
