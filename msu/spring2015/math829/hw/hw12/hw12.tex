\documentclass{article}

\usepackage{amsmath}
\usepackage{amsfonts}
\usepackage{amssymb}
\usepackage{enumerate}
\usepackage[lastexercise]{exercise}

\newcommand\id{\mathrm{id}}
\newcommand\ord{\mathrm{ord}}
\newcommand\Res{\mathrm{Res}}
\newcommand\Log{\mathrm{Log}}
\renewcommand\Im{\mathrm{Im}}
\renewcommand\Re{\mathrm{Re}}
\newcommand\dif{\mathop{}\!\mathrm{d}}
\newcommand\horline{\noindent\makebox[\linewidth]{\rule{\textwidth}{0.4pt}}}

\newcounter{Problem}
\newenvironment{Problem}{\begin{Exercise}[name={Problem},
                                          counter={Problem}]}
                        {\end{Exercise}}
\title{MATH 829 Homework \#12}
\date{Apr 15, 2015}
\author{Sam Boling}

\begin{document}

\begin{titlepage}
\maketitle
% TODO: 6.2.14(b), 7.2.1
\end{titlepage}

\section{Textbook Problems}
\subsection{Ch. 6 \S 2.14(b)}
Let $\theta \in [0, 2 \pi)$, $R, \varepsilon > 0$.
Define curves by
\begin{align*}
   S_R(t)
&= R e^{i t},
&  \quad \theta \leq t \leq 2 \pi - \theta, \\
   S_\varepsilon(t)
&= \varepsilon e^{i t},
&  \quad \theta \leq t \leq 2 \pi - \theta, \\
   L_+(t)
&= t e^{i \theta}, \quad \varepsilon \leq t \leq R \\
   L_-(t)
&= t e^{i(2 \pi - \theta)}, \varepsilon \leq t \leq R
\end{align*}
and a contour $\Gamma$ by
\begin{align*}
       \Gamma
&=      S_R
 \oplus [R e^{i(2 \pi - \theta)}, \varepsilon e^{i(2 \pi - \theta)}]
 \oplus S_\varepsilon^{-}
 \oplus [\varepsilon e^{i \theta}, R e^{i \theta}] \\
&=      S_R
 \oplus L_-^{-}
 \oplus S_\varepsilon^{-}
 \oplus L_+.
\end{align*}
We observe that with
$$
  f(z)
= \frac{\Log(z)}{(z^2 + 1)^2}
$$

The poles of the function
$$
  f(z)
= \frac{1}{(z^2 + 1)^2}
$$
that lie in $\mathrm{Int}(\Gamma)$ are at $\pm i$. Here we choose
$\Log(z)$ to be the branch of the logarithm holomorphic on
$\mathbb{C} \setminus \{ \Re(z) \geq 0 \}$ with $\arg z \in [0, 2\pi)$
so that $f$ is holomorphic except for at these poles. We note
that
\begin{align*}
      \left|
        \int_{S_R}
          f
      \right|
&\leq \left\|
        \frac{\Log z}
             {(z^2 + 1)^2}
      \right\|
      L(S_R)
 \leq \left|
        \frac{\Log (Re^{it})}
             {(R^2e^{2it} + 1)^2}
      \right|
      2(\pi - \theta)R \\
&=    \left|
        \frac{\log R + it}
             {(R^2e^{2it} + 1)^2}
      \right|
      2(\pi - \theta)R \\
&\leq \frac{2 \pi R \log R}
           {(R^2 + 1)^2}
 \to  0
\end{align*}
since $R \log R < (R^2 + 1)^2$ and
\begin{align*}
      \left|
        \int_{S_\varepsilon}
          f
        \right|
&\leq 2 (\pi - \theta) \varepsilon
      \frac{\log \varepsilon + it}
           {(1 + \varepsilon^2 e^{it})^2}
 \to  \frac{2 \pi \varepsilon i t}
           {1}
 \to  0.
\end{align*}
% We also see that
% \begin{align*}
%    \int_{L_+}
%      f \cdot \Log
%  + \int_{L_-^-}
%      f \cdot \Log
% &= \int_{L_+}
%      f \cdot \Log
%  - \int_{L_-}
%      f \cdot \Log \\
% &= \int_\varepsilon^R
%      f(t e^{i\theta})
%      \Log(t e^{i\theta})
%      e^{i\theta}
%      \dif t \\
% &- \int_\varepsilon^R
%      f(t e^{i(2\pi - \theta)})
%      \Log(t e^{i (2 \pi - \theta)})
%      e^{i(2\pi - \theta)}
%      \dif t \\
% &= \int_\varepsilon^R
%      f(x)
%      [ \Log (t e^{i \theta})
%        e^{i \theta}
%      - \Log (t e^{i (2 \pi - \theta)})
%        e^{i (2\pi - \theta)}
%      ] \\
% &=   \int_\varepsilon^R
%        f(x)
%        [ e^{i \theta} (\log x + i \theta)
%        - e^{i (2\pi - \theta)} (\log x + i (2\pi - \theta)) ]
%        \dif x \\
% &\to \int_\varepsilon^R
%        f(x) [ \log x (1 - e^{i 2 \pi}) - 2 \pi i e^{i 2 \pi} ]
%        \dif x \\
% &=   e^{i \pi}
%      \int_\varepsilon^R
%        f(x) \log x (e^{-i \pi} - e^{i \pi})
%        \dif x
%    - 2 \pi i
%      \int_\varepsilon^R
%        f(x) \dif x
% % &= \int_\varepsilon^R
% %      \frac{\Log (t e^{i\theta})}
% %           {(t^2 e^{2 i \theta} + 1)^2}
% %      e^{i \theta}
% %      \dif t
% %  - \int_\varepsilon^R
% %      \frac{\Log (t e^{i(2\pi - \theta)})}
% %           {(t^2 e^{2i(2\pi - \theta)} + 1)^2}
% %      \dif t \\
% % &= \int_\varepsilon^R
% %      \left[
% %        \frac{\log t + i \theta}
% %             {t^2 e^{2 i \theta} + 1}
% %        e^{i \theta}
% %      - \frac{\log t + i (2 \pi - \theta)}
% %             {(t^2 e^{2i(2\pi - \theta)})
% %      \right]
% %     \dif t
% \end{align*}

On the other hand,
\begin{align*}
   \int_\Gamma
     f
&= \int_{\gamma_R}
     f
 + \int_{[R e^{i(2 \pi - \theta)},
          \varepsilon e^{i(2 \pi - \theta)}]}
     f
 - \int_{\gamma_\varepsilon}
     f
 + \int_{[\varepsilon e^{i \theta},
          R e^{i \theta}]}
     f \\
&= 2 \pi i
   \left(
     \Res_{-i} f
   + \Res_i f
   \right).
\end{align*}

We see that
\begin{align*}
   \frac{\dif}{\dif z}
   \frac{\Log~z}{(z + i)^2}
&= \frac{(z + i)^2 z^{-1} - 2(z + i)\Log~z}
        {(z + i)^4}, \\
   \frac{\dif}{\dif z}
   \frac{\Log~z}{(z - i)^2}
&= \frac{(z - i)^2 z^{-1} - 2(z - i)\Log~z}
        {(z - i)^4}
\end{align*}
so that
\begin{align*}
   \frac{\Log~z}{(z + i)^2}
&= \frac{\Log(i)}{(2i)^2}
 + \frac{(2i)^2 i^{-1} - 2(2i)\Log(i)}
        {(2i)^4}
   (z - i)
 + \cdots \\
&= -i\frac{\pi}{8}
 + \frac{4i - 4i \cdot i \frac{\pi}{2}}
        {16}
   (z - i)
 + \cdots \\
&= -\frac{\pi}{8}
 + \frac{4i + 2 \pi}{16} (z - i)
 + \cdots \\
   \frac{\Log~z}{(z - i)^2}
&= \frac{\Log(-i)}{(-2i)^2}
 + \frac{(-2i)^2 (-i)^{-1} - 2(-2i) \Log(-i)}
        {(-2i)^4}
   (z + i) \\
&= -i\frac{3 \pi}{8}
 + \frac{-4i - 4i \cdot i \frac{3 \pi}{2}}
        {16}
   (z + i) \\
&= -i\frac{3 \pi}{8}
 + \frac{-4i + 6 \pi}{16} (z + i)
\end{align*}
so that
\begin{align*}
   \Res_i \frac{\Log~z}{(z - i)^2 (z + i)^2}
&= \frac{4i + 2 \pi}{16}
 = \frac{\pi}{8} + \frac{i}{4}, \\
   \Res_i \frac{\Log~z}{(z - i)^2 (z + i)^2}
&= \frac{-4i + 6 \pi}{16}
 = \frac{3 \pi}{8} - \frac{i}{4}
\end{align*}
and hence
$$
  \int_\Gamma f
= 2 \pi i \cdot \frac{\pi}{2}
= i \pi^2.
$$
Thus we anticipate that
$$
  \lim_{\substack{\varepsilon \to 0 \\ R \to \infty}}
  \left[
    \int_\varepsilon^R
      \frac{\Log (t e^{i \theta})}
           {(t^2 e^{2 i \theta} + 1)^2}
      \dif t
  - \int_\varepsilon^R
      \frac{\Log (t e^{i (2\pi - \theta)})}
           {(t^2 e^{i (2 \pi - \theta)} + 1)^2}
      \dif t
  \right]
= -4 \pi i
  \int_0^\infty
    \frac{\log x}
         {(x^2 + 1)^2}
    \dif x.
$$


\subsection{Ch. 6 \S 2.19}
First observe that
$$
  Q(\cos \theta, \sin \theta)
= \frac{1}{3 + 2 \cos \theta}
$$
is even and so
$$
  \int_0^\pi
    Q(\cos \theta, \sin \theta)
    \dif \theta
= \frac{1}{2}
  \int_{-\pi}^{\pi}
    Q(\cos \theta, \sin \theta)
    \dif \theta.
$$
If we consider
\begin{align*}
   f(z)
&= \frac{
     Q
     \left(
       \frac{1}{2}
       \left(
         z + \frac{1}{z}
       \right),
       \frac{1}{2i}
       \left(
         z - \frac{1}{z}
       \right)
     \right)
   }
   {i z}
 = \frac{1}{iz}
   \frac{1}{3 + z + \frac{1}{z}} \\
&= \frac{1}{i}
   \frac{1}{z^2 + 3z + 1}.
\end{align*}
The polynomial $z^2 + 3z + 1$ has zeros at
$$
z_1 = \frac{1}{2}(\sqrt{5} - 3), \quad
z_2 = \frac{1}{2}(-\sqrt{5} - 3),
$$
where $|z_1| < 1$ and $|z_2| > 1$.
We can write
$$
  \frac{1}{z^2 + 3z + 1}
= \frac{1}{\sqrt{5}}
  \frac{1}{z - z_1}
- \frac{1}{\sqrt{5}}
  \frac{1}{z - z_2},
$$
and since $\frac{1}{z - z_2}$ is holomorphic
at $z_1$, the Laurent expansion of this term around
$z_1$ has no negative terms, so
$$
  \Res_{z_1}
    \frac{1}{z^2 + 3z + 1}
= \frac{1}{\sqrt{5}}.
$$
Therefore
\begin{align*}
   \int_0^\pi
     Q(\cos \theta, \sin \theta)
     \dif \theta
&= \frac{1}{2}
   \int_{|z| = 1}
     \frac{1}{i}
     \frac{1}{z^2 + 3z + 1}
     \dif z \\
&= \frac{1}{2i} \cdot
   2 \pi i \cdot
   \Res_{z_1}
     \frac{1}{z^2 + 3z + 1}
 = \frac{\pi}{\sqrt{5}}
\end{align*}
as desired.

\subsection{Ch. 6\S 24}
First let $n \geq 0$. We have
\begin{align*}
   \int_0^{2\pi}
     (\cos \theta)^n
     \dif \theta
&= \int_{|z| = 1}
     \frac{1}{iz}
     \left[
       \frac{1}{2}
       \left(
         z + \frac{1}{z}
       \right)
     \right]^n
     \dif z \\
&= \frac{1}{i 2^n}
   \int_{|z| = 1}
     \frac{1}{z}
     \sum_{k = 0}^{n}
       {n \choose k}
       z^{n - k}
       (z^{-1})^k
     \dif z \\
&= \frac{1}{i 2^n}
   \sum_{k = 0}^n
     {n \choose k}
     \int_{|z| = 1}
       z^{n - 2k - 1}
       \dif z \\
&= \frac{\pi}{2^{n-1}}
   \sum_{k = 0}^n
     {n \choose k}
     \Res_0 z^{-2k}.
\end{align*}
Since $n$ is even we can write $n = 2m$ for some nonnegative
integer $m$, and observe that
$2m - 2k - 1 = 2(m - k) - 1 > 0$ for $m > k$, and
$2(m - k) - 1 = -1$ for $m = k$. But nonnegative powers of
$z$ are entire, so
$$
  \Res_0 z^{n - 2k - 1}
= \left\{
    \begin{array}{c c}
      0, & \quad k \neq \frac{n}{2} \\
      1, & \quad k = \frac{n}{2}
    \end{array}
  \right..
$$
Then
\begin{align*}
   \int_0^{2\pi}
     (\cos \theta)^n
     \dif \theta
&= \frac{\pi}{2^{n-1}}
     {n \choose \frac{n}{2}}
 = \frac{\pi}{2^{n-1}}
 = \frac{\pi}{2^{n-1}}
     \frac{n!}
          {\left(\frac{n}{2}\right)!
           \left(n - \frac{n}{2}\right)!} \\
&= \frac{\pi}{2^{n-1}}
     \frac{n!}
          {\left[
             \left(\frac{n}{2}\right)!
           \right]^2}
, \quad n \geq 0.
\end{align*}
If $n < 0$ then $(\cos \theta)^n = (\sec \theta)^{-n} > 0$, since
$\sec \theta$ is strictly positive and $-n > 0$. But $\sec \theta$
is unbounded on any open neighborhood of $\frac{\pi}{2}$, so in this
case the integral $\int_0^{2\pi} (\cos \theta)^n \dif \theta$ does not
exist.

\subsection{Ch. 7 \S 2.1}
Let $g_a$ denote the automorphism
$$
g_a(z) = \frac{a - z}{1 - \bar{a} z}.
$$
Then we note that $g_a(a) = 0$ and $g_a(0) = a$, so
$$
  (g_a \circ f \circ g_a)(0)
= g_a(f(a))
= g_a(a)
= 0.
$$
But $g_a^{-1} = g_a$, so
$$
  (g_a \circ f \circ g_a)(g_a(b))
= (g_a \circ f)(b)
= g_a(b)
$$
and thus both 0 and $g_a(b)$ are fixed points of
$h = g_a \circ f \circ g_a$. Since $f$ has distinct
fixed points, i.e. $a \neq b$, we
have
$$
g_a(b) = \frac{a - b}{1 - \bar{a} b} \neq 0,
$$
we conclude from the Schwarz lemma that
$h = \id$. But then
$$
f = g_a \circ \id \circ g_a = \id.
$$

\subsection{Ch. 7 \S 2.2}
We define
$$
g(z) = \frac{a - z}
            {1 - \bar{a} z}
$$
and
$$
h(z) = \frac{f(a) - z}
            {1 - \overline{f(a)} z}
$$
so that
\begin{align*}
   F(z)
&= (h \circ f \circ g)(z)
 = \frac{f(a) - f(g(z))}
        {1 - \overline{f(a)} f(g(z))} \\
&= \frac{ f(a)
        - f\left(
             \frac{a - z}
                  {1 - \bar{a} z}
           \right)}
        { 1
        - \overline{f(a)}
          f\left(
            \frac{a - z}
                 {1 - \bar{a} z}
          \right)}
\end{align*}
and in particular $F(0) = 0$. From the Schwarz lemma,
$|F^\prime(z)| \leq 1$, and we have
\begin{align*}
   F^\prime
&= (h \circ (f \circ g))^\prime
 = (h^\prime \circ (f \circ g)) \cdot (f \circ g)^\prime \\
&= (h^\prime \circ (f \circ g)) \cdot
   (f^\prime \circ g) \cdot
   g^\prime
\end{align*}
and since
\begin{align*}
  g^\prime(z)
&= \frac{(1 - \bar{a} z)(-1) - (a - z)(-\bar{a})}
        {(1 - \bar{a} z)^2}
 = \frac{-1 + \bar{a} z + \bar{a} a - \bar{a} z}
        {(1 - \bar{a} z)^2} \\
&= \frac{|a|^2 - 1}
        {(1 - \bar{a} z)^2},
\end{align*}
\begin{align*}
   (f^\prime \circ g)(z)
&= f^\prime\left(\frac{a - z}{1 - \bar{a} z}\right),
\end{align*}
\begin{align*}
  (h^\prime \circ f \circ g)(z)
= h^\prime\left(f\left(\frac{a - z}{1 - \bar{a}z}\right)\right)
= \frac{ -|f(a)|^2
       + 2 \overline{f(a)}
           f\left(\frac{a - z}{1 - \bar{a}z}\right)
       - 1}
       {\left(
          1
        - \overline{f(a)}
          f\left(\frac{a - z}{1 - \bar{a} z}\right)
        \right)^2},
\end{align*}
we have
\begin{align*}
   F^\prime(0)
&= \left[
     \frac{-|f(a)|^2 + 2|f(a)|^2 - 1}
          {(1 - \overline{f(a)} f(a))^2}
   \right]
   \cdot
   f^\prime(a)
   \cdot
   (|a|^2 - 1) \\
&= \frac{(|f(a)|^2 - 1)(|a|^2 - 1)}
         {(1 - |f(a)|^2)^2}
    f^\prime(a) \\
&= \frac{|a|^2 - 1}
        {1 - |f(a)|^2}
   f^\prime(a).
\end{align*}
But from the Schwarz lemma, $|F^\prime(0)| \leq 1$,
and since the image of $f$ lies in the unit disk we have
$|f(a)| < 1$, whence $|f(a)|^2 < 1$, so $1 - |f(a)|^2 > 0$
and thus $|1 - |f(a)|^2| = 1 - |f(a)|^2$. Similarly since
$a \in \mathbb{D}$, $|1 - |a|^2| = 1 - |a|^2$. It follows
that
$$
     \frac{|f^\prime(a)|}
          {1 - |f(a)|^2}
\leq \frac{1}{1 - |a|^2}.
$$

\section{Additional Problems}

\begin{Problem}
  Find the coefficients of the Laurent series of
  $\cot z$ centered at $0$ up to $z^3$.
\end{Problem}

\begin{Answer}
Since $\cot z = \frac{\cos z}{\sin z}$, we first expand
these functions around 0:
\begin{align*}
   \cos z
&= \sum_{n=0}^\infty
     \frac{(-1)^{n}}
          {(2n)!}
     z^{2n}
 = \sum_{n=0}^\infty
     a_n z^n, \\
   \sin z
&= \sum_{n=0}^\infty
     \frac{(-1)^{n}}
          {(2n + 1)!}
     z^{2n + 1}
 = \sum_{n=0}^\infty
     b_n z^n
\end{align*}
where
\begin{align*}
   a_n
&= \left\{
     \begin{array}{c c}
       \frac{(-1)^{\frac{n}{2}}}{n!}, & \quad n \text{ even} \\
       0,                          & \quad n \text{ odd}
     \end{array}
   \right., \\
   b_n
&= \left\{
     \begin{array}{c c}
       \frac{(-1)^\frac{n-1}{2}}{n!}, & \quad n \text{ odd} \\
       0,                            & \quad n \text{ even}
     \end{array}
   \right.. \\
\end{align*}
Since $\cos(0) = 1$, $\sin(0) = 0$, and
$\sin^\prime(0) = 1$,
the pole of $\cot z$ at 0 has order 1, so
$\ord_0 \cot = -1$, and we can write
$\cot z = \sum_{n=-1}^\infty c_n z^n$ for some
coefficients $c_n$. Since
$\cot z \cdot \sin z = \cos z$, we have
\begin{align*}
   \cos z
&= \left(
     \sum_{n=0}^\infty
       b_n
       z^n
   \right)
   \cdot
   \left(
     \sum_{n=-1}^\infty
       c_n
       z^n
   \right) \\
&= \left(
     \sum_{n=0}^\infty
       b_n
       z^n
   \right)
   \cdot
   \left(
     \sum_{n=0}^\infty
       c_{n-1}
       z^{n-1}
   \right) \\
&= \frac{1}{z}
   \sum_{n=0}^\infty
     \left(
       \sum_{k=0}^n
         b_{n-k}
         c_{k-1}
     \right)
     z^n \\
\end{align*}
so that
\begin{align*}
   \sum_{n=0}^\infty
     \left(
       \sum_{k=0}^n
         b_{n-k}
         c_{k-1}
     \right)
     z^n
&= z \cos z
 = \sum_{n=0}^\infty
     a_n z^{n+1} \\
&= \sum_{n=1}^\infty
     a_{n-1} z^n.
\end{align*}
Then we have
\begin{align*}
  b_0 c_{-1}
&= 0 \cdot c_{-1}
 =  a_{-1}
 =  0, \\
   b_1 c_{-1}
 + b_0 c_0
&= c_{-1}
 =  a_0  = 1, \\
   b_2 c_{-1}
 + b_1 c_0
 + b_0 c_1
&= c_0
 =  a_1 =  0, \\
   b_3 c_{-1}
 + b_2 c_0
 + b_1 c_1
 + b_0 c_2
&= b_3 c_{-1}
 + b_1 c_1 \\
&= -\frac{1}{6} c_{-1}
   + c_1
 =  a_2 = -\frac{1}{2}, \\
   b_4 c_{-1}
 + b_3 c_0
 + b_2 c_1
 + b_1 c_2
 + b_0 c_3
&= b_3 c_0
 + b_1 c_2 \\
&=  -\frac{1}{6} c_0
 + c_2
 =  a_3 = 0, \\
   b_5 c_{-1}
 + b_4 c_0
 + b_3 c_1
 + b_2 c_2
 + b_1 c_3
 + b_0 c_4
&= b_5 c_{-1}
 + b_3 c_1
 + b_1 c_3 \\
&= \frac{1}{5!} c_{-1}
 - \frac{1}{6} c_1
 + c_3
 =  a_4 = \frac{1}{4!}
\end{align*}
which can be recursively solved to find
\begin{align*}
c_{-1} &= 1, \\
c_0   &= 0, \\
c_1   &= -\frac{1}{3}, \\
c_2   &= 0, \\
c_3   &= \frac{1}{4!} - \frac{1}{18} - \frac{1}{5!}
       = -\frac{1}{45}.
\end{align*}
\end{Answer}

\begin{Problem}
  \begin{enumerate}[(i)]
    \item{
       Compute
       $$
       \sum_{n=0}^\infty \frac{1}{n^2}
       $$
       by integrating
       $$
       f(z) = \frac{\cot z}{z^2}
       $$
       along the boundary of a square with vertices
       $\{ \pm C_N \pm i C_N \}$, where $C_N = (N + \frac{1}{2})\pi$
       and $N \in \mathbb{N}$, and letting $N \to \infty$.
    }
    \item{
      Sketch the computation of
      $\sum_{n=1}^\infty \frac{1}{n^4}$. Note: You need to prove
      that $\int_{\partial S} f \to 0$ as $N \to \infty$, and give an answer.
    }
  \end{enumerate}
\end{Problem}

\begin{Answer}
  \begin{enumerate}[(i)]
    \item{
      The function $\frac{\cot z}{z^2} = \frac{\cos z}{z^2 \sin z}$
      has poles at $\pi n$ for all $n \in \mathbb{Z}$. We note that
      $$
        \frac{\dif}{\dif z}
        z^2 \sin z
      = 2z \sin z + z^2 \cos z
      $$
      so that $(z^2 \sin z)^\prime(0) = 0$ but
      $$
        (z^2 \sin z)^\prime(\pi n)
      = (\pi n)^2 \cos \pi n
      \neq 0
      $$
      for $n \neq 0$, and thus
      $$
        \Res_{\pi n}
          \frac{\cot z}{z^2}
      = \frac{\cos \pi n}
             {(\pi n)^2 \cos \pi n}
      = \frac{1}{\pi^2 n^2}, \quad n \neq 0.
      $$

      In the previous problem we saw that the Laurent expansion of
      $\cot z$ at 0 has $c_1 = -\frac{1}{3}$, so the residue of
      $\frac{\cot z}{z^2}$ at 0 is $-\frac{1}{3}$.

      The curve $\partial S_N$ given by the boundary of the square $S_N$
      with vertices
      $\{ C_N + i C_N, -C_n + i C_N, -C_N - i C_N, C_N - i C_N \}$,
      with $C_N = \left(N + \frac{1}{2}\right)\pi$,
      encloses the point $0$ as well as $\pm \pi k$ for
      $k = 1, 2, \dots, N$. Therefore
      \begin{align*}
         \int_{\partial S_N}
           \frac{\cot z}{z^2}
           \dif z
      &= 2 \pi i
         \left(
           \Res_0
             \frac{\cot z}{z^2}
         + \sum_{k=1}^N
             \Res_{\pi k}
               \frac{\cot z}{z^2}
         + \sum_{k=1}^N
             \Res_{-\pi k}
               \frac{\cot z}{z^2}
         \right) \\
      &= 2 \pi i
         \left(
           -\frac{1}{3}
         + \frac{2}{\pi^2}
           \sum_{k=1}^N
             \frac{1}{n^2}
         \right)
      \end{align*}
      so that
      \begin{align*}
         \sum_{k=1}^{\infty}
           \frac{1}{n^2}
      &= \lim_{N \to \infty}
           \sum_{k=1}^N
             \frac{1}{n^2}
       = \lim_{N \to \infty}
         \frac{\pi^2}{2}
         \left[
           \frac{1}{3}
        +  \frac{1}{2 \pi i}
             \int_{\partial S_N}
               \frac{\cot z}{z^2}
               \dif z
          \right].
      \end{align*}
      Next we see that
      \begin{align*}
            \left|
              \int_{\partial S_N}
                \frac{\cot z}{z^2}
                \dif z
            \right|
      &\leq \left\|
              \frac{\cot z}{z^2}
            \right\|_{\partial S_N}
            L(\partial S_N)
       =    \left\|
              \frac{\cot z}{z^2}
            \right\|_{\partial S_N}
            \cdot 4 \cdot 2 C_N.
      \end{align*}
      On any given side of the square $\partial S_N$,
      we have $C_N \leq |z| \leq \sqrt{2} C_N$ so that
      $C_N^2 \leq |z|^2 \leq \sqrt{2} C_N^2$ and so
      $\frac{1}{|z|^2} \leq \frac{1}{C_N^2}$ on $\partial S_N$.
      Furthermore,
      \begin{align*}
         |\cot (x + iy)|
      &= \left|
           \frac{\cos x \cosh y - i \sin x \sinh y}
                {\sin x \cosh y + i \cos x \sinh y}
         \right| \\
      &= \frac{\sqrt{
                 \cos^2 x \cosh^2 y
               + \sin^2 x \sinh^2 y
               }}
              {\sqrt{
                 \sin^2 x \cosh^2 y
               + \cos^2 x \sinh^2 y
              }},
      \end{align*}
      and on the sides of the square $S_N^{\pm 1}$ we have
      $x = \pi\left(N + \frac{1}{2}\right)$ so that $\cos x = 0$. Then
      on these segments
      $$
        |\cot (x + i y)|
      = \frac{\sqrt{\sin^2 x \sinh^2 y}}
             {\sqrt{\sin^2 x \cosh^2 y}}
      = \tanh y,
      $$
      and $|\tanh y| < 1$, so on the left and right sides
      $\partial S_N^{\pm 1}$ of the square we have
      $$
           \left\|\frac{\cot z}{z^2}\right\|_{\partial S_N^{\pm 1}}
           L(\partial S_N^{\pm 1})
      \leq \frac{\tanh y}{C_N^2} 2 C_N
      <    \frac{2}{C_N} \to 0.
      $$

      On the top of the square $\partial S_N^{+i}$
      we have $y \geq \pi(0 + 1/2) = \frac{\pi}{2}$, so
      \begin{align*}
            \left| \cot(x + i y) \right|
      &=    \left|
              \frac{e^{ix} e^{-y} + e^{-ix} e^y}
                   {e^{ix} e^{-y} - e^{-ix} e^y}
            \right|
       =    \left|
              \frac{e^{2ix} e^{-2y} + 1}
                   {e^{2ix} e^{-2y} - 1}
            \right| \\
      &\leq \frac{1 + |e^{2ix}||e^{-2y}|}
                 {1 - |e^{2ix}||e^{-2y}|}
       =    \frac{1 + e^{-2y}}
                 {1 - e^{-2y}} \\
      &=    \coth y
       =    \coth \left(\pi\left(N + \frac{1}{2}\right)\right)
      \end{align*}
      so that on the top of the square we have
      $$
           \left\|
             \frac{\cot z}{z^2}
           \right\|_{\partial S_N^{+i}}
           L(\partial S_N^{+i})
      \leq \frac{\coth \left(\pi\left(N + \frac{1}{2}\right)\right)}
                {C_N^2}
           2C_N
      \leq \frac{2 \coth \frac{3 \pi}{2}}
                {C_N} \to 0.
      $$

      On the bottom of the square $\partial S_N^{-i}$ we have
      $y \leq -\pi(1 + 1/2) = -\frac{\pi}{2}$, so
      \begin{align*}
            \left| \cot(x + i y) \right|
      &=    \left|
              \frac{e^{ix} e^{-y} + e^{-ix} e^y}
                   {e^{ix} e^{-y} - e^{-ix} e^y}
            \right|
       =    \left|
              \frac{1 + e^{-2ix} e^{2y}}
                   {1 - e^{-2ix} e^{2y}}
            \right| \\
      &\leq \frac{1 + |e^{-2ix}||e^{2y}|}
                 {1 - |e^{-2ix}||e^{2y}|}
       =    \frac{1 + e^{2y}}
                 {1 - e^{2y}} \\
      &= -\cot y
       = -\cot \left( -\pi \left(N + \frac{1}{2}\right) \right)
       = \cot \left( \pi \left(N + \frac{1}{2}\right) \right)
      \end{align*}
      so that
      $$
           \left\|
             \frac{\cot z}{z^2}
           \right\|_{\partial S_N^{-i}}
           L(\partial S_N^{-i})
      \to  0
      $$
      as well.

      It follows that
      $
      \left|
        \int_{\partial S_N}
          \frac{\cot z}{z^2}
          \dif z
      \right| \to 0$ as $N \to \infty$, so we conclude that
      $$
      \sum_{n=1}^\infty \frac{1}{n^2} = \frac{\pi^2}{6}.
      $$
    }
    \item{
      The function $\frac{\cot z}{z^4}$ has poles at $\pi n$,
      and as computed above the coefficient $c_3$ of $\cot z$
      is $-\frac{1}{45}$, so
      $\Res_0 \frac{\cot z}{z^4} = -\frac{1}{45}$. All other poles
      have order 1, so
      $$
        \Res_{\pi n} \frac{\cot z}{z^4}
      = \frac{\cos (\pi n)}
             { 4(\pi n)^3 \sin (\pi n)
             + (\pi n)^4 \cos (\pi n)}
      = \frac{1}{\pi^4 n^4}.
      $$
      Using the same contour as before, we have
      \begin{align*}
         \int_{\partial S_N}
           \frac{\cot z}{z^4}
           \dif z
      &= 2 \pi i
         \left(
           \Res_0 \frac{\cot z}{z^2}
         + \sum_{n=1}^N
             \Res_{\pi n}
               \frac{\cot z}{z^2}
         + \sum_{n=1}^N
             \Res_{-\pi n}
               \frac{\cot z}{z^2}
         \right) \\
      &= 2 \pi i
         \left(
         - \frac{1}{45}
         + \frac{2}{\pi^4}
           \sum_{n=1}^N
             \frac{1}{n^4}
         \right)
      \end{align*}
      so that
      $$
        \sum_{n=1}^\infty
          \frac{1}{n^4}
      = \lim_{N \to \infty}
          \frac{\pi^4}{2}
          \left[
            \frac{1}{45}
          + \frac{1}{2 \pi i}
            \int_{\partial S_N}
              \frac{\cot z}{z^4}
              \dif z
          \right],
      $$
      so we expect to find
      $$
        \sum_{n=1}^\infty
          \frac{1}{n^4}
      = \frac{\pi^2}{90}
      $$
      and a computer algebra system confirms this result.
    }
  \end{enumerate}
\end{Answer}

\begin{Problem}
Let $f \in \mathrm{Iso}(\mathbb{D}, U)$. Show that if
$D(f(0), R) \subset U$, then $R \leq |f^\prime(0)|$.
Hint: Consider the restriction of $f^{-1}$ to $D(f(0), R)$.
\end{Problem}

\begin{Answer}
Note that the function
$$
g(z) = R z + f(0)
$$
is holomorphic and satisfies
$$
  |g(z) - f(0)|
= |Rz|
< R
$$
whenever $|z| < 1$, and thus this function maps $\mathbb{D}$ to
$D_f = D(f(0), R)$. Thus we have a holomorphic function
$$
h = f^{-1} \big|_{D_f} \circ g : \mathbb{D} \to \mathbb{D}
$$
with
$$
  h(0)
= f^{-1}(0 + f(0))
= 0
$$
so that $|h^\prime(z)| \leq 1$. But
\begin{align*}
   h^\prime(z)
&= (f^{-1}\big|_{D_f} \circ g)^\prime(z)
 = [ ((f^{-1}\big|_{D_f})^\prime
     \circ g)
     \cdot g^\prime](z) \\
&= (f^{-1}\big|_{D_f})^\prime(Rz + f(0)) \cdot R.
\end{align*}
In particular,
$$
     |h^\prime(0)|
=    R |(f^{-1})^\prime(f(0))|
\leq 1,
$$
or $R \leq \frac{1}{|(f^{-1})^\prime(f(0))|}$.
But $f$ is an isomorphism, so
$$
  f^\prime(0)
= \frac{1}{(f^{-1})^\prime(f(0))}
$$
and therefore $R \leq |f^\prime(0)|$.
\end{Answer}

\end{document}
