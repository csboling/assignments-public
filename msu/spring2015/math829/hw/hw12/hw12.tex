\documentclass{article}

\usepackage{amsmath}
\usepackage{amsfonts}
\usepackage{amssymb}
\usepackage{enumerate}
\usepackage[lastexercise]{exercise}

\newcommand\id{\mathrm{id}}
\newcommand\ord{\mathrm{ord}}
\newcommand\Res{\mathrm{Res}}
\renewcommand\Im{\mathrm{Im}}
\renewcommand\Re{\mathrm{Re}}
\newcommand\dif{\mathop{}\!\mathrm{d}}
\newcommand\horline{\noindent\makebox[\linewidth]{\rule{\textwidth}{0.4pt}}}

\newcounter{Problem}
\newenvironment{Problem}{\begin{Exercise}[name={Problem},
                                          counter={Problem}]}
                        {\end{Exercise}}
\title{MATH 829 Homework \#12}
\date{Apr 15, 2015}
\author{Sam Boling}

\begin{document}

\begin{titlepage}
\maketitle
% TODO: 6.2.14(b), 7.2.1
\end{titlepage}

\section{Textbook Problems}
\subsection{Ch. 6 \S 2.14(b)}

\subsection{Ch. 6 \S 2.19}
First observe that
$$
  Q(\cos \theta, \sin \theta)
= \frac{1}{3 + 2 \cos \theta}
$$
is even and so
$$
  \int_0^\pi
    Q(\cos \theta, \sin \theta)
    \dif \theta
= \frac{1}{2}
  \int_{-\pi}^{\pi}
    Q(\cos \theta, \sin \theta)
    \dif \theta.
$$
If we consider
\begin{align*}
   f(z)
&= \frac{
     Q
     \left(
       \frac{1}{2}
       \left(
         z + \frac{1}{z}
       \right),
       \frac{1}{2i}
       \left(
         z - \frac{1}{z}
       \right)
     \right)
   }
   {i z}
 = \frac{1}{iz}
   \frac{1}{3 + z + \frac{1}{z}} \\
&= \frac{1}{i}
   \frac{1}{z^2 + 3z + 1}.
\end{align*}
The polynomial $z^2 + 3z + 1$ has zeros at
$$
z_1 = \frac{1}{2}(\sqrt{5} - 3), \quad
z_2 = \frac{1}{2}(-\sqrt{5} - 3),
$$
where $|z_1| < 1$ and $|z_2| > 1$.
We can write
$$
  \frac{1}{z^2 + 3z + 1}
= \frac{1}{\sqrt{5}}
  \frac{1}{z - z_1}
- \frac{1}{\sqrt{5}}
  \frac{1}{z - z_2},
$$
and since $\frac{1}{z - z_2}$ is holomorphic
at $z_1$, the Laurent expansion of this term around
$z_1$ has no negative terms, so
$$
  \Res_{z_1}
    \frac{1}{z^2 + 3z + 1}
= \frac{1}{\sqrt{5}}.
$$
Therefore
\begin{align*}
   \int_0^\pi
     Q(\cos \theta, \sin \theta)
     \dif \theta
&= \frac{1}{2}
   \int_{|z| = 1}
     \frac{1}{i}
     \frac{1}{z^2 + 3z + 1}
     \dif z \\
&= \frac{1}{2i} \cdot
   2 \pi i \cdot
   \Res_{z_1}
     \frac{1}{z^2 + 3z + 1}
 = \frac{\pi}{\sqrt{5}}
\end{align*}
as desired.

\subsection{Ch. 6\S 24}
First let $n \geq 0$. We have
\begin{align*}
   \int_0^{2\pi}
     (\cos \theta)^n
     \dif \theta
&= \int_{|z| = 1}
     \frac{1}{iz}
     \left[
       \frac{1}{2}
       \left(
         z + \frac{1}{z}
       \right)
     \right]^n
     \dif z \\
&= \frac{1}{i 2^n}
   \int_{|z| = 1}
     \frac{1}{z}
     \sum_{k = 0}^{n}
       {n \choose k}
       z^{n - k}
       (z^{-1})^k
     \dif z \\
&= \frac{1}{i 2^n}
   \sum_{k = 0}^n
     {n \choose k}
     \int_{|z| = 1}
       z^{n - 2k - 1}
       \dif z \\
&= \frac{\pi}{2^{n-1}}
   \sum_{k = 0}^n
     {n \choose k}
     \Res_0 z^{-2k}.
\end{align*}
Since $n$ is even we can write $n = 2m$ for some nonnegative
integer $m$, and observe that
$2m - 2k - 1 = 2(m - k) - 1 > 0$ for $m > k$, and
$2(m - k) - 1 = -1$ for $m = k$. But nonnegative powers of
$z$ are entire, so
$$
  \Res_0 z^{n - 2k - 1}
= \left\{
    \begin{array}{c c}
      0, & \quad k \neq \frac{n}{2} \\
      1, & \quad k = \frac{n}{2}
    \end{array}
  \right..
$$
Then
\begin{align*}
   \int_0^{2\pi}
     (\cos \theta)^n
     \dif \theta
&= \frac{\pi}{2^{n-1}}
     {n \choose \frac{n}{2}}
 = \frac{\pi}{2^{n-1}}
 = \frac{\pi}{2^{n-1}}
     \frac{n!}
          {\left(\frac{n}{2}\right)!
           \left(n - \frac{n}{2}\right)!} \\
&= \frac{\pi}{2^{n-1}}
     \frac{n!}
          {\left[
             \left(\frac{n}{2}\right)!
           \right]^2}
, \quad n \geq 0.
\end{align*}
If $n < 0$ then $(\cos \theta)^n = (\sec \theta)^{-n} > 0$, since
$\sec \theta$ is strictly positive and $-n > 0$. But $\sec \theta$
is unbounded on any open neighborhood of $\frac{\pi}{2}$, so in this
case the integral $\int_0^{2\pi} (\cos \theta)^n \dif \theta$ does not
exist.

\subsection{Ch. 7 \S 2.1}
Without loss of generality, suppose $b \neq 0$.
Since $|f(z)| < 1$, $f(\mathbb{D}) \subset \mathbb{D}$.
If $a = 0$ then $f(0) = 0$, and we may apply the Schwarz lemma
to see that since $b \in \mathbb{D} \setminus \{ 0 \}$ and
$f(b) = b$ that $f = \mathrm{id}$. Therefore let $a, b \neq 0$.

Consider the function
$$
  g(z)
= \frac{a - f(a - z)}
       {2}
+ \frac{f(z + b) - b}
       {2}.
$$
% FIXME: show that g(\mathbb{D}) \subset \mathbb{D}
Then
$$
  g(0)
= \frac{a - f(a)}{2}
+ \frac{f(b) - b}{2}
= 0
$$
and
$$
  g(a - b)
= \frac{a - f(b)}{2}
+ \frac{f(a) - b}{2}
= a - b
$$
so that $g(z) = z$. But then for all $z \in \mathbb{D}$
we have
$$
2z = a - b + f(z + b) - f(a - z).
$$
In particular, for $z = a$,
$$
2a - a + b = a + b = f(a + b) - f(0)
$$

\subsection{Ch. 7 \S 2.2}
We define
$$
g(z) = \frac{a - z}
            {1 - \bar{a} z}
$$
and
$$
h(z) = \frac{f(a) - z}
            {1 - \overline{f(a)} z}
$$
so that
\begin{align*}
   F(z)
&= (h \circ f \circ g)(z)
 = \frac{f(a) - f(g(z))}
        {1 - \overline{f(a)} f(g(z))} \\
&= \frac{ f(a)
        - f\left(
             \frac{a - z}
                  {1 - \bar{a} z}
           \right)}
        { 1
        - \overline{f(a)}
          f\left(
            \frac{a - z}
                 {1 - \bar{a} z}
          \right)}
\end{align*}
and in particular $F(0) = 0$. From the Schwarz lemma,
$|F^\prime(z)| \leq 1$, and we have
\begin{align*}
   F^\prime
&= (h \circ (f \circ g))^\prime
 = (h^\prime \circ (f \circ g)) \cdot (f \circ g)^\prime \\
&= (h^\prime \circ (f \circ g)) \cdot
   (f^\prime \circ g) \cdot
   g^\prime
\end{align*}
and since
\begin{align*}
  g^\prime(z)
&= \frac{(1 - \bar{a} z)(-1) - (a - z)(-\bar{a})}
        {(1 - \bar{a} z)^2}
 = \frac{-1 + \bar{a} z + \bar{a} a - \bar{a} z}
        {(1 - \bar{a} z)^2} \\
&= \frac{|a|^2 - 1}
        {(1 - \bar{a} z)^2},
\end{align*}
\begin{align*}
   (f^\prime \circ g)(z)
&= f^\prime\left(\frac{a - z}{1 - \bar{a} z}\right),
\end{align*}
\begin{align*}
  (h^\prime \circ f \circ g)(z)
= h^\prime\left(f\left(\frac{a - z}{1 - \bar{a}z}\right)\right)
= \frac{ -|f(a)|^2
       + 2 \overline{f(a)}
           f\left(\frac{a - z}{1 - \bar{a}z}\right)
       - 1}
       {\left(
          1
        - \overline{f(a)}
          f\left(\frac{a - z}{1 - \bar{a} z}\right)
        \right)^2},
\end{align*}
we have
\begin{align*}
   F^\prime(0)
&= \left[
     \frac{-|f(a)|^2 + 2|f(a)|^2 - 1}
          {(1 - \overline{f(a)} f(a))^2}
   \right]
   \cdot
   f^\prime(a)
   \cdot
   (|a|^2 - 1) \\
&= \frac{(|f(a)|^2 - 1)(|a|^2 - 1)}
         {(1 - |f(a)|^2)^2}
    f^\prime(a) \\
&= \frac{|a|^2 - 1}
        {1 - |f(a)|^2}
   f^\prime(a).
\end{align*}
But from the Schwarz lemma, $|F^\prime(0)| \leq 1$,
and since the image of $f$ lies in the unit disk we have
$|f(a)| < 1$, whence $|f(a)|^2 < 1$, so $1 - |f(a)|^2 > 0$
and thus $|1 - |f(a)|^2| = 1 - |f(a)|^2$. Similarly since
$a \in \mathbb{D}$, $|1 - |a|^2| = 1 - |a|^2$. It follows
that
$$
     \frac{|f^\prime(a)|}
          {1 - |f(a)|^2}
\leq \frac{1}{1 - |a|^2}.
$$

\section{Additional Problems}

\begin{Problem}
  Find the coefficients of the Laurent series of
  $\cot z$ centered at $0$ up to $z^3$.
\end{Problem}

\begin{Answer}
Since $\cot z = \frac{\cos z}{\sin z}$, we first expand
these functions around 0:
\begin{align*}
   \cos z
&= \sum_{n=0}^\infty
     \frac{(-1)^{n}}
          {(2n)!}
     z^{2n}
 = \sum_{n=0}^\infty
     a_n z^n, \\
   \sin z
&= \sum_{n=0}^\infty
     \frac{(-1)^{n}}
          {(2n + 1)!}
     z^{2n + 1}
 = \sum_{n=0}^\infty
     b_n z^n
\end{align*}
where
\begin{align*}
   a_n
&= \left\{
     \begin{array}{c c}
       \frac{(-1)^{\frac{n}{2}}}{n!}, & \quad n \text{ even} \\
       0,                          & \quad n \text{ odd}
     \end{array}
   \right., \\
   b_n
&= \left\{
     \begin{array}{c c}
       \frac{(-1)^\frac{n-1}{2}}{n!}, & \quad n \text{ odd} \\
       0,                            & \quad n \text{ even}
     \end{array}
   \right.. \\
\end{align*}
Since $\cos(0) = 1$, $\sin(0) = 0$, and
$\sin^\prime(0) = 1$,
the pole of $\cot z$ at 0 has order 1, so
$\ord_0 \cot = -1$, and we can write
$\cot z = \sum_{n=-1}^\infty c_n z^n$ for some
coefficients $c_n$. Since
$\cot z \cdot \sin z = \cos z$, we have
\begin{align*}
   \cos z
&= \left(
     \sum_{n=0}^\infty
       b_n
       z^n
   \right)
   \cdot
   \left(
     \sum_{n=-1}^\infty
       c_n
       z^n
   \right) \\
&= \left(
     \sum_{n=0}^\infty
       b_n
       z^n
   \right)
   \cdot
   \left(
     \sum_{n=0}^\infty
       c_{n-1}
       z^{n-1}
   \right) \\
&= \frac{1}{z}
   \sum_{n=0}^\infty
     \left(
       \sum_{k=0}^n
         b_{n-k}
         c_{k-1}
     \right)
     z^n \\
\end{align*}
so that
\begin{align*}
   \sum_{n=0}^\infty
     \left(
       \sum_{k=0}^n
         b_{n-k}
         c_{k-1}
     \right)
     z^n
&= z \cos z
 = \sum_{n=0}^\infty
     a_n z^{n+1} \\
&= \sum_{n=1}^\infty
     a_{n-1} z^n.
\end{align*}
Then we have
\begin{align*}
  b_0 c_{-1}
&= 0 \cdot c_{-1}
 =  a_{-1}
 =  0, \\
   b_1 c_{-1}
 + b_0 c_0
&= c_{-1}
 =  a_0  = 1, \\
   b_2 c_{-1}
 + b_1 c_0
 + b_0 c_1
&= c_0
 =  a_1 =  0, \\
   b_3 c_{-1}
 + b_2 c_0
 + b_1 c_1
 + b_0 c_2
&= b_3 c_{-1}
 + b_1 c_1 \\
&= -\frac{1}{6} c_{-1}
   + c_1
 =  a_2 = -\frac{1}{2}, \\
   b_4 c_{-1}
 + b_3 c_0
 + b_2 c_1
 + b_1 c_2
 + b_0 c_3
&= b_3 c_0
 + b_1 c_2 \\
&=  -\frac{1}{6} c_0
 + c_2
 =  a_3 = 0, \\
   b_5 c_{-1}
 + b_4 c_0
 + b_3 c_1
 + b_2 c_2
 + b_1 c_3
 + b_0 c_4
&= b_5 c_{-1}
 + b_3 c_1
 + b_1 c_3 \\
&= \frac{1}{5!} c_{-1}
 - \frac{1}{6} c_1
 + c_3
 =  a_4 = \frac{1}{4!}
\end{align*}
which can be recursively solved to find
\begin{align*}
c_{-1} &= 1, \\
c_0   &= 0, \\
c_1   &= -\frac{1}{3}, \\
c_2   &= 0, \\
c_3   &= \frac{1}{4!} - \frac{1}{18} - \frac{1}{5!}
       = -\frac{1}{45}.
\end{align*}
\end{Answer}

\begin{Problem}
  \begin{enumerate}[(i)]
    \item{
       Compute
       $$
       \sum_{n=0}^\infty \frac{1}{n^2}
       $$
       by integrating
       $$
       f(z) = \frac{\cot z}{z^2}
       $$
       along the boundary of a square with vertices
       $\{ \pm C_N \pm i C_N \}$, where $C_N = (N + \frac{1}{2})\pi$
       and $N \in \mathbb{N}$, and letting $N \to \infty$.
    }
    \item{
      Sketch the computation of
      $\sum_{n=1}^\infty \frac{1}{n^4}$. Note: You need to prove
      that $\int_{\partial S} f \to 0$ as $N \to \infty$, and give an answer.
    }
  \end{enumerate}
\end{Problem}

\begin{Problem}
Let $f \in \mathrm{Iso}(\mathbb{D}, U)$. Show that if
$D(f(0), R) \subset U$, then $R \leq |f^\prime(0)|$.
Hint: Consider the restriction of $f^{-1}$ to $D(f(0), R)$.
\end{Problem}

\begin{Answer}
Note that the function
$$
g(z) = R z + f(0)
$$
is holomorphic and satisfies
$$
  |g(z) - f(0)|
= |Rz|
< R
$$
whenever $|z| < 1$, and thus this function maps $\mathbb{D}$ to
$D_f = D(f(0), R)$. Thus we have a holomorphic function
$$
h = f^{-1} \big|_{D_f} \circ g : \mathbb{D} \to \mathbb{D}
$$
with
$$
  h(0)
= f^{-1}(0 + f(0))
= 0
$$
so that $|h^\prime(z)| \leq 1$. But
\begin{align*}
   h^\prime(z)
&= (f^{-1}\big|_{D_f} \circ g)^\prime(z)
 = [ ((f^{-1}\big|_{D_f})^\prime
     \circ g)
     \cdot g^\prime](z) \\
&= (f^{-1}\big|_{D_f})^\prime(Rz + f(0)) \cdot R.
\end{align*}
In particular,
$$
     |h^\prime(0)|
=    R |(f^{-1})^\prime(f(0))|
\leq 1,
$$
or $R \leq \frac{1}{|(f^{-1})^\prime(f(0))|}$.
But $f$ is an isomorphism, so
$$
  f^\prime(0)
= \frac{1}{(f^{-1})^\prime(f(0))}
$$
and therefore $R \leq |f^\prime(0)|$.
\end{Answer}

\end{document}
