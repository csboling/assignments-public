
\documentclass{article}

\usepackage{amsmath}
\usepackage{amsfonts}
\usepackage{amssymb}
\usepackage{enumitem}
\usepackage[lastexercise]{exercise}

\newcommand\dif{\mathop{}\!\mathrm{d}}
\newcommand\Res{\mathrm{Res}}
\newcommand\ord{\mathrm{ord}}
\newcommand\horline{\noindent\makebox[\linewidth]{\rule{\textwidth}{0.4pt}}}

\newcounter{Problem}
\newenvironment{Problem}{\begin{Exercise}[name={Problem},
                                          counter={Problem}]}
                        {\end{Exercise}}
\title{MATH 829 Homework \#10}
\date{March 11, 2015}
\author{Sam Boling}

\begin{document}

\begin{titlepage}
\maketitle
\end{titlepage}

\section{Textbook Problems}
\subsection*{Chapter VI, \S1.12}
$\Res_0 \frac{e^z}{z^4}$

Since
$$
e^z = \sum_{n=0}^\infty \frac{z^n}{n!}
$$
we have
$$
  \frac{e^z}{z^4}
= \sum_{n=0}^\infty \frac{z^{n-4}}{n!}
= \sum_{n=-4}^\infty \frac{z^n}{(n+4)!}
= \sum_{n=-\infty}^\infty a_n z^n
$$
so that 
$$
  \Res_0 \frac{e^z}{z^4} 
= a_{-1} 
= \frac{1}{3!} 
= \frac{1}{6}.
$$

\subsection*{Chapter VI, \S1.14}
Noting that $\sin(0) = 0$ and 
$\sin^\prime(0) = \cos(0) = 1 \neq 0$
so that $\ord_0 \sin = 1$, we have 
$$
  \Res_0 \frac{e^z}{\sin z}
= \frac{e^0}{\sin^\prime(0)}
= \frac{1}{1}
= 1.
$$

\subsection*{Chapter VI, \S1.15}
Since $z^4$ is holomorphic on $D(1, 1)$ and
$e^z$ is entire, $\frac{e^z}{z^4}$ is holomorphic on the
star domain $D(1,1)$ and so has a primitive here,
so $\Res_1 \frac{e^z}{z^4} = 0$.

\subsection*{Chapter VI, \S1.18}
Write 
$$
g(z) = (z - z_1), \quad
h(z) = \frac{1}{\prod_{k=2}^n (z - z_k)}
$$
and observe that $g(z_1) = 0$, 
$g^\prime(z_1) = 1 \neq 0$ so that $\ord_1 g = 1$,
and $\frac{1}{f} = \frac{h}{g}$. Furthermore
$\prod_{k=2}^n (z - z_k)$ is a polynomial with no zero inside 
$C$, so $h$ and $g$ are holomorphic on $\mathrm{Int}(C) \cup C$.
Therefore 
\begin{align*}
   \int_C \frac{1}{f} 
&= 2 \pi i \Res_{z_1} \frac{1}{f}
 = 2 \pi i \Res_{z_1} \frac{h}{g}
 = 2 \pi i \frac{h(z_1)}{g^\prime(z_1)} \\
&= 2 \pi i \frac{1}{\prod_{k=2}^n (z_1 - z_k)}.
\end{align*}

\subsection*{Chapter VI, \S1.20}
\begin{enumerate}[(a)]
  \item{
    The zeros of $z^2 - 3z + 5$ are at
    $$
      \frac{3 \pm \sqrt{9 - 20}}{2}
    = \frac{3}{2} \pm i\frac{\sqrt{11}}{2}
    $$
    so that $z_1 = \frac{3}{2} + i \frac{\sqrt{11}}{2}$ lies inside
    $C$ and $z_2 = \frac{3}{2} - i \frac{\sqrt{11}}{2}$ does not.
    Therefore, from problem 18 above, we have
    \begin{align*}
       \int_C \frac{1}{z^2 - 3z + 5} \dif z
    &= \int_C \frac{1}{(z - z_1)(z - z_2)} \dif z
     = 2 \pi i \frac{1}{(z_1 - z_2)} \\
    &= 2 \pi i \frac{1}{i \sqrt{11}} = \frac{2 \pi}{\sqrt{11}}.
    \end{align*}
  }
  \item{
    $z^2 + z + 1$ has zeros at 
    $$
      \frac{-1 \pm \sqrt{1 - 4}}{2}
    = -1 \pm i \frac{\sqrt{3}}{2}
    $$ 
    which both lie outside $C$.
    Therefore $\int_C f = 0$.
  }
  \item{
    $z^2 - z + 1$ has zeros at $1 \pm i \frac{\sqrt{3}}{2}$, so that
    $z_1 = 1 + i\frac{\sqrt{3}}{2}$ lies inside $C$ and 
    $z_2 = 1 - i\frac{\sqrt{3}}{2}$ does not. Therefore from problem 18,
    \begin{align*}
       \int_C \frac{1}{z^2 - z + 1} \dif z
    &= \int_C \frac{1}{(z - z_1)(z - z_2) \dif z
     = 2 \pi i \frac{1}{z_1 - z_2} \\
    &= 2 \pi i \frac{1}{-i \sqrt{3}}
     = -\frac{2 \pi}{\sqrt{3} = -\frac{2}{3} \sqrt{3} \pi.
    \end{align*}
  }
\end{enumerate}

\subsection*{Chapter VI, \S1.26}
$C = \{ |z| = 8 \}$
\begin{enumerate}
  \item[(a)]{
    $\sin z$ has zeros at $z_k = k \pi$, $k \in \mathbb{Z}$, so at
    each zero we have 
    $$
      \sin^\prime(k\pi) 
    = \cos(k \pi) 
    = \pm 1 \neq 0.
    $$
    Therefore $\ord_{z_k} \sin = 1$ for each $k$, so
    $$
      \Res_{z_k} \frac{1}{\sin z} 
    = \frac{1}{\cos(k\pi)
    = (-1)^k.
    $$
    Since $2 \pi < 7$ and $3 \pi > 9$, the poles of $\frac{1}{\sin z}$
    which lie inside $C$ are $z_{-2}, z_{-1}, z_{0}, z_{1}, z_{2}$, and then
    $$
      \int_C \frac{1}{\sin z} \dif z
    = 2 \pi i \sum_{k=-2}^2 \Res_{z_k} 
    = 2 \pi i \sum_{k=-2}^2 (-1)^k
    = 2 \pi i.
    $$
  }
  \item[(d)]{
    $\cos z$ has zeros at $z_k = k \pi + \frac{\pi}{2}$,
    $k \in \mathbb{Z}$, so at each zero we have
    $$
      \cos^\prime\left(k\pi + \frac{\pi}{2}\right)
    = -\sin\left(k \pi + \frac{\pi}{2}\right)
    = -(-1)^k \neq 0.
    $$
    Then we have
    \begin{align*}
       \Res_{z_k} \tan z
    &= \Res_{z_k} \frac{\sin z}{\cos z}
     = \frac{\sin(z_k)}{-\sin(z_k)}
     = -1.
    \end{align*}
    Since $z_k$ lies inside $C$ for 
    $k \in \{ -2, -1, 0, 1, 2 \}$ we then have
    $$
      \int_C \tan z 
    = 2 \pi i \sum_{k=-2}^{2} \Res_{z_k}
    = -10 \pi i.
    $$
  }
\end{enumerate}

\subsection*{Chapter VI, \S1.31}
\begin{enumerate}[(a)]
  \item{
    Taking $P(z) = z^{87} + 36z^{57} + 71z^4 + z^3 - z + 1$ and
    $f(z) = 71z^4$, we have for $|z| = 1$ that
    $$
         |P(z) - f(z)| 
    \leq |z|^{87} + |36||z|^{57} + |z|^3 + |-z| + |1|
    =    40
    $$
    while 
    $$
    |f(z)| = |71||z|^4 = 71 > |P(z) - f(z)|
    $$
    so that $|P(z) - f(z)| < |f(z)|$, $z \in \{|z| = 1\}$. Therefore
    $P$ has the same number of zeros counting multiplicity 
    as $f$ inside $\{|z| = 1\}$, namely 4.
  }
  \item{
    In this case we have
    $$
         |P(z) - f(z)|
    \leq |z|^{87} + |36||z|^{57} + |z|^3 + |-z| + |1|
    $$
  }
  \item{
  }
\end{enumerate}

\subsection*{Chapter VI, \S1.32}
Since $h$ is analytic on $\bar{D}(0, R)$, it attains its maximum on
the boundary, i.e. there is a $z_{\max h} \in \partial \bar{D} (0, R)$
such that $|h(z)| \leq |h(z_{\max h})|$. Since $f$ is analytic on $\bar{D}(0, R)$,
it attains its minimum on the boundary, i.e. there is a 
$z_{\min f} \in \partial D(0, R)$ such that $|f(z)| \geq |f(z_{\min f})|$. But by 
assumption $f \neq 0$ on $\partial \bar{D}(0, R)$, so 
$|f(z_{\min f})| > 0$, and in particular we can find an $\varepsilon$ 
such that $|f(z_{\min f})| > \varepsilon |h(z_{\max h})|$. Then
we have
\begin{align*}
      |f(z) - (f(z) + \varepsilon h(z))| 
&=    \varepsilon |h(z)| 
\leq  \varepsilon |h(z_{\max h})| \\
&<     |f(z_{\min f})|
 <     |f(z)|
\end{align*}
for all $z \in \partial \bar{D}(0, R)$. From Rouche's theorem it follows that
$f(z)$ and $f(z) + \varepsilon h(z)$ have the same number of zeros, counting
multiplicity, on 
$\mathrm{Int}(\partial\bar{D}(0, R)) = D(0, R)$.

\subsection*{Chapter VI, \S1.35}
Since $e^z = \sum_{k=0}^\infty \frac{z^k}{k!}$, we have
$$
  \left|e^z - P_n(z)\right|
= \left|\sum_{k=n+1}^\infty \frac{z^k}{k!}\right|
$$


\section{Additional Problems}
\begin{Problem}
Let $f$ be an entire function that satisfies
$\lim_{|z| \to \infty} |f(z)| = \infty$. Show that $f$ is a polynomial.
Hint: Consider the type of the singularity of $g(z) = f(\frac{1}{z})$
at 0.
\end{Problem}

\end{document}
