
\documentclass{article}

\usepackage{amsmath}
\usepackage{amsfonts}
\usepackage{amssymb}
\usepackage{enumerate}
\usepackage{mathtools}
\usepackage{xfrac}
\usepackage[lastexercise]{exercise}

\DeclarePairedDelimiter\floor{\lfloor}{\rfloor}

\newcounter{Problem}
\newenvironment{Problem}{\begin{Exercise}[name={Problem},
                                          counter={Problem}]}
                        {\end{Exercise}}
\title{MATH 818 Homework \#5}
\date{January 21, 2015}
\author{Sam Boling}

\begin{document}

\begin{titlepage}
\maketitle
% TODO: I.3.4 1
\end{titlepage}

\section{Textbook Problems}

\subsection*{Ch I \S 1.2}
\begin{itemize}
  \item[(c)]{
    \begin{align*}
       \frac{2 + i}{2 - i}
    &= (2+i)\left(\frac{1}{2 - i} + i\frac{1}{2 - i}\right) \\
    &= (2+i)\left(\frac{2}{4+1} + i\frac{1}{4 + 1}\right) \\
    &= (2+i)\left(\frac{2}{5} + i\frac{1}{5}\right) \\
    &= \frac{4}{5} + i\frac{2}{5} + i\frac{2}{5} - \frac{1}{5} \\
    &= \frac{3}{5} + i\frac{4}{5}.
    \end{align*}
  }
  \item[(f)]{
    \begin{align*}
       \frac{i}{1+i}
    &= i\left(\frac{1}{2} - i\frac{1}{2}\right) \\
    &= \frac{1}{2} + i\frac{1}{2}.
    \end{align*}
  }
\end{itemize}

\subsection*{Ch. I \S 1.7}
We have that $(1 + i) = \sqrt{2}e^{\frac{\pi}{4} i}$.
Therefore
$$
  (1 + i)^{100}
= (\sqrt{2}^2)^{50} e^{25 \pi i}
= 2^{50} e^{\pi i}
= -2^{50},
$$
so $\mathrm{Re}((1 + i)^{100}) = -2^{50}$ and
$\mathrm{Im}((1 + i)^{100}) = 0$.

\subsection*{Ch. I \S 1.8}
\begin{enumerate}[(a)]
  \item{
    Let $x = z - w$ and $y = w$. From the triangle inequality,
    $|x + y| \leq |x| + |y|$. But $x + y = z$, so this gives
    $|z - w + w| \leq |z - w| + |w|$ or $|z| \leq |z - w| + |w|$ as
    desired.
  }
  \item{
    Since $|w|$ is real and nonnegative, we may subtract $|w|$ from
    both sides of the previous inequality to get $|z| - |w| \leq |z - w|$.
  }
  \item{
    Note that
    \begin{align*}
       (|z| - |w|)^2
    &= |z|^2 - 2 |z| |w| + |w|^2
     = |z|^2 - 2 |z| |\overline{w}| + |w|^2 \\
    &= |z|^2 + |w|^2 - 2|z \overline{w}|
    \end{align*}
    and
    \begin{align*}
          |z + w|^2
    &=    (z + w)\overline{(z + w)}
     =    (z + w)(\overline{z} + \overline{w}) \\
    &=    z \overline{z} + z \overline{w} + \overline{z} w + w \overline{w} \\
    &=    |z|^2 + z \overline{w} + \overline{z \overline{w}} + |w|^2 \\
    &=    |z|^2 + 2 \mathrm{Re}(z \overline{w}) + |w|^2.
    \end{align*}
    Subtracting gives
    $$
    |z + w|^2 - (|z| - |w|)^2 = 2 \mathrm{Re}(z \overline{w}) - 2|z \overline{w}|.
    $$

    Since $|u| \geq \mathrm{Re}(u) \geq -|u|$ for any $u \in \mathbb{C}$,
    this means that $|z + w|^2 - (|z| - |w|)^2 \geq 0$, and it follows that
    $|z| - |w| \leq |z + w|$.
  }
\end{enumerate}

\subsection*{Ch. I \S 1.10}
\begin{itemize}
  \item[(b)]
  {
    $|z - i + 3| = |z - (i - 3)| > 5$ corresponds to the complement of
    the closed disk of radius $5$ centered at the point $-3 + i$,
    i.e. $(-3, 1)$.
  }
  \item[(c)]
  {
    $|z - i + 3| = |z - (i - 3)| \leq 5$ corresponds to the closed
    disk of radius $5$ centered at the point $-3 + i$, i.e. $(-3, 1)$.
  }
  \item[(d)]
  {
    $|z + 2i| = |z - (-2i)| \leq 1$ corresponds to the closed disk of
    radius 1 centered at the point $-2i$, i.e. $(0, -2)$.
  }
  \item[(f)]
  {
    $\mathrm{Im}~z \geq 0$ corresponds to the closed upper half-plane.
  }
  \item[(h)]
  {
    $\mathrm{Re}~z \geq 0$ corresponds to the closed right half-plane.
  }
\end{itemize}

\subsection*{Ch. I \S 2.1}
\begin{itemize}
  \item[(a)]{
    \begin{align*}
      |1 + i| &= \sqrt{1^2 + 1^2} = \sqrt{2} \\
      \mathrm{Arg}(1 + i) &= \arctan \frac{1}{1} = \frac{\pi}{4}
    \end{align*}
    so $1 + i = \sqrt{2}e^{\frac{\pi}{4} i}$
  }
  \item[(c)]{
    $|-3| = 3$ and $\mathrm{Arg}(-3) = \pi$, so
    $-3 = 3e^{\pi i}$
  }
  \item[(d)]{
    $|4i| = 4$ and $\mathrm{Arg}(4i) = \frac{\pi}{2}$, so
    $4i = 4e^{\frac{\pi}{2} i}$
  }
  \item[(h)]{
    $$
      -1 - i
    = -(1+i)
    = e^{\pi i} (1 + i)
    = \sqrt{2}e^{\frac{7\pi}{4} i}.
    $$
  }
\end{itemize}

\subsection*{Ch. I \S 2.2}
\begin{itemize}
  \item[(b)]{
    $\cos \frac{2 \pi}{3} = -\frac{1}{2}$,
    $\sin \frac{2 \pi}{3} = \frac{\sqrt{3}}{2}$,
    so $e^{2 i \pi / 3} = -\frac{1}{2} + \frac{\sqrt{3}}{2} i$.
  }
  \item[(c)]{
    $\cos \frac{\pi}{4} = \sqrt{2}{2}$,
    $\sin \frac{\pi}{4} = \sqrt{2}{2}$,
    so
    $$
      3 e^{i \pi / 4}
    = 3(\cos \frac{\pi}{4} + i \sin \frac{\pi}{4})
    = \frac{3 \sqrt{2}}{2} + \frac{3 \sqrt{2}}{2} i.
    $$
  }
  \item[(f)]{
    $\cos \left(-\frac{\pi}{2}\right) = 0$,
    $\sin \left(-\frac{\pi}{2}\right) = -1$
    so $e^{-i \pi / 2} = -i$.
  }
  \item[(g)]{
    $\cos (-\pi) = -1$,
    $\sin (-\pi) = 0$,
    so $e^{-i\pi} = -1$.
  }
\end{itemize}

\subsection*{Ch. I \S 2.8}

\begin{enumerate}[(a)]
  \item{
    Suppose $z = x + iy$ for $x, y \in \mathbb{R}$ and $e^z = 1$. Then
    $e^z = e^x e^{iy} = 1$, so $e^{iy} = e^{-x}$. Taking the modulus of both sides
    gives $1 = |e^{-x}| = e^{-x}$, so $x = -\ln 1 = 0$. Therefore we have
    $e^z = e^{iy} = 1$, so this set is the unit circle.
  }
  \item{
    Take $\alpha = x_0 + i y_0$, with $x_0, y_0 \in \mathbb{R}$
    so that $w = e^\alpha = e^{x_0} e^{iy_0}$.
    Let $z \in \mathbb{C}$, so that $z = x + iy$ for some
    $x, y \in \mathbb{R}$, and assume $w = e^z$.

    Since $e^z = e^x e^{iy}$, this means
    $e^x e^{iy} = e^{x_0} e^{iy_0}$ and therefore
    $e^{x - x_0} = e^{-i(y - y_0)}$. Taking the modulus of
    both sides gives $|e^{x - x_0}| = 1$, so that
    $x - x_0 = \ln 1 = 0$ and therefore $x = x_0$.
    Then we have $e^x e^{iy} = e^x e^{iy_0}$, so
    $e^{iy} = e^{iy_0}$. Therefore $y$ and $y_0$ differ by
    a multiple of $2\pi$. We conclude that
    $z = x_0 + i(y_0 + 2\pi k)$ for some $k \in \mathbb{Z}$,
    i.e. the complex numbers $z$ lie on the same vertical line
    with real part $x_0$ and imaginary part differing from $y_0$
    by a multiple of $2\pi$.
  }
\end{enumerate}


\subsection*{Ch. I \S 2.11}
Let $s = 1 + z + z^2 + \cdots + z^n$ and note that
\begin{align*}
   (1 - z)s
&= (1 - z)\sum_{k=0}^n z^k
 = \sum_{k=0}^n z^k - z\sum_{k=0}^n z^k \\
&= \sum_{k=0}^n z^k - \sum_{k=0}^n z^{k+1}
 = \sum_{k=0}^n z^k - \sum_{k=1}^{n+1} z^{k} \\
&= \left(1 + \sum_{k=1}^n z^k\right) - \left(\sum_{k=1}^n z^k +
  z^{n+1}\right) \\
&= 1 - z^{n+1}
\end{align*}
so that when $z \neq 1$ we can divide by $(1 - z)$ and find
$$
  s
= \sum_{k=0}^n z^k = \frac{1 - z^{n+1}}{1 - z}
= \frac{z^{n+1} - 1}{z - 1}.
$$

\subsection*{Ch. I \S 2.12}
First we observe that
\begin{align*}
    e^{i \theta} - e^{-i \theta}
&= (\cos \theta + i \sin \theta) - (\cos(-\theta) + i \sin(-\theta))
 = 2i \sin \theta
\end{align*}
for any real number $\theta$.

Let $\theta \in (0, 2\pi)$. Then $e^{i\theta} \neq 1$,
so letting $z = e^{i \theta} = \cos \theta + i \sin \theta$ and using the
previous exercise we have
\begin{align*}
   \sum_{k=0}^n z^k
&= \frac{z^{n+1} - 1}{z - 1}
 = \frac{e^{i(n+1)\theta} - 1}{e^{i\theta} - 1} \\
&= \frac{e^{i\frac{1}{2}\theta}(e^{i(n+\frac{1}{2})\theta} - e^{-i\frac{1}{2}\theta})}
        {e^{i\frac{1}{2}\theta}(e^{i\frac{1}{2}\theta} - e^{-i\frac{1}{2}\theta})} \\
&= \frac{e^{i(n + \frac{1}{2})\theta} - e^{-i\frac{1}{2}\theta}}
        {2 i \sin \frac{\theta}{2}} \\
&= -i
    \frac{
      \cos((n + \frac{1}{2})\theta)
  + i \sin((n + \frac{1}{2})\theta)
  -   \cos(-\frac{1}{2}\theta)
  - i \sin(-\frac{1}{2}\theta)
    }
    {2 \sin \frac{\theta}{2}} \\
&=
  \frac{
    \sin(\frac{\theta}{2})
+   \sin ((n + \frac{1}{2})\theta)
+ i (\cos\frac{\theta}{2} - \cos((n + \frac{1}{2})\theta))
  }
  { 2 \sin \frac{\theta}{2} } \\
&= \frac{1}{2}
 + \frac{\sin ((n + \frac{1}{2})\theta)}
        { 2 \sin \frac{\theta}{2} }
 + i\frac{\cos\frac{\theta}{2} - \cos((n + \frac{1}{2})\theta)}
         { 2 \sin \frac{\theta}{2} }.
\end{align*}
But
$$
  \sum_{k=0}^n z^n
= \sum_{k=0}^n e^{i n \theta}
= \sum_{k=0}^n (\cos n \theta + i \sin n \theta),
$$
so taking real parts of both sides we have
$$
  \sum_{k=0}^n \cos n \theta
= \frac{1}{2}
+ \frac{\sin ((n + \frac{1}{2})\theta)}
       { 2 \sin \frac{\theta}{2} }
$$
as desired.

\subsection*{Ch. I \S 3.4}
\begin{itemize}
  \item[(a)]{
    For $x \leq 1$ and $0 \leq y \leq \pi$ we have
    $e^z = e^x e^{iy}$, so these are points with radius
    $e^x$ and argument $y$. Therefore this is the closed
    half-disk in the upper half-plane of radius $e$, excepting the point
    $z = 0$.
  }
  \item[(b)]{
    For $0 \leq y \leq \pi$, this is the closed upper half-plane, excepting
    the point $z = 0$.
  }
  \item[(c)]{
    Considering the set $\{ z \in \mathbb{C} : |z| \geq 100 \}$,
    $e^z = e^{r e^{i \theta}} = e^{r \cos \theta} e^{i r \sin \theta}$.
    With $r \geq 100$, $r \cos \theta$ and $r \sin \theta$
    can take any real value, so the image of this set is the entire
    complex plane except for 0.
  }
\end{itemize}

\subsection*{Ch. II \S 3.4}
Suppose $\sin z = 0$ for some $z = x + iy$. Given that
$$
\sin z = \frac{e^{iz} - e^{-iz}}{2 i},
$$
this gives $e^{iz} = e^{-iz}$ so that
$e^{-y + ix} = e^{y - ix}$. But this means
$|e^{-y + ix}| = |e^{y - ix}|$ or
$e^{-y} = e^y$, and taking the natural logarithm
of these real quantities gives $-y = y$, so $y = 0$
and thus $z$ is purely real. Then we have
$e^{ix} = e^{-ix}$ or
$$
\cos x + i \sin x = \cos x - i \sin x,
$$
leaving $2 i \sin x = 0$ or $\sin x = 0$. This equation
is satisfied only by integer multiples of $\pi$,
$x = \pi k$ and thus $z = x = \pi k$ for some $k \in \mathbb{Z}$.

\section{Additional Problems}

\begin{Problem}
Compute the following principal logarithms using
$\mathrm{Arg} : \mathbb{C} \to [0, 2\pi)$ and
$\mathrm{Arg} : \mathbb{C} \to (-\pi, \pi]$.
\begin{itemize}
  \item{
    $\mathrm{Log}(-i)$
  }
  \item{
    $\mathrm{Log}(\sqrt{3} + i)$
  }
  \item{
    $\mathrm{Log}(-1 - i)$
  }
\end{itemize}
\end{Problem}

\begin{Answer}
  \begin{itemize}
    \item{
      Since $-i = e^{-i \frac{\pi}{2}} = e^{i \frac{3 \pi}{2}}$,
      $\mathrm{Log}(-i) = \frac{3 \pi}{2}i$ when the principal logarithm
      is taken to have argument in $[0, 2 \pi)$ and
      $\mathrm{Log}(-i) = -\frac{\pi}{2}i$ when
      taken to have argument in $(-\pi, \pi]$.
    }
    \item{
      Since $|\sqrt{3} + i| = \sqrt{3 + 1} = 2$ and
      $\mathrm{arg}(\sqrt{3} + i) = \frac{\pi}{6} + 2 \pi k$,
      we have $\sqrt{3} + i  = 2e^{i\frac{\pi}{6}}$, which has the
      same principal logarithm $\ln 2 + \frac{\pi}{6} i$ in either case.
    }
    \item{
      Since
      $$
        -1 - i
      = -(1 + i)
      = -\sqrt{2}e^{i\frac{\pi}{4}}
      = \sqrt{2}e^{i\frac{5 \pi}{4}}
      = \sqrt{2}e^{-i\frac{3 \pi}{4}},
      $$
      we have $\mathrm{Log}(-1 - i) = \ln \sqrt{2} + \frac{5 \pi}{4}i$
      when the principal logarithm is taken to have argument in
      $[0, 2 \pi)$ and $\mathrm{Log}(-1 - i) = \ln \sqrt{2} - \frac{3 \pi}{4}i$
      when the principal logarithm is taken to have argument in $(-\pi, \pi]$.
    }
  \end{itemize}
\end{Answer}

\begin{Problem}
Find all $z \in \mathbb{C}$ in rectangular form which solve
$z^6 + 1 = 0$. Trigonometric functions must be evaluated.
\end{Problem}

\begin{Answer}
The solutions of $z^6 + 1 = 0$ are the 6th roots of -1. Since $|-1| =
1$ and $\mathrm{Arg}(-1) = \pi$, these are the numbers $e^{i\theta}$ with unit
modulus and argument given by
$$
  \theta
= \frac{\pi}{6} + \frac{2 \pi k}{6},
\quad k \in \{ 0, 1, 2, \dots \}.
$$
Explicitly,
\begin{align*}
   e^{i\frac{\pi}{6}}
&= \cos \frac{\pi}{6} + i \sin \frac{\pi}{6}
 = \frac{\sqrt{3}}{2} + i \frac{1}{2} \\
   e^{i\frac{pi}{2}}
&= i \\
   e^{i\frac{5 \pi}{6}}
&= \cos \frac{5\pi}{6} + i \sin \frac{5\pi}{6}
 = -\frac{\sqrt{3}}{2} + i \frac{1}{2} \\
   e^{i\frac{7\pi}{6}}
&= \cos \frac{7\pi}{6} + i \sin \frac{7\pi}{6}
 = -\frac{\sqrt{3}}{2} - i \frac{1}{2} \\
   e^{i\frac{9\pi}{6}}
&= \cos \frac{3\pi}{2} + i \sin \frac{3\pi}{2}
 = -i \\
   e^{i\frac{11\pi}{6}}
&= \cos \frac{11\pi}{6} + i \sin \frac{11\pi}{6}
 = \frac{\sqrt{3}}{2} - i \frac{1}{2}
\end{align*}

\end{Answer}

\begin{Problem}
Let $n \in \mathbb{N}$ with $n \geq 2$ and
$z_0 \in \mathbb{C} - \{ 0 \}$.
Let $w_1, \dots, w_n$ be the distinct $n$-th roots of $z_0$.
Prove that $\sum_{k=1}^n w_k = 0$.
\end{Problem}

\begin{Answer}
Write $z_0$ in polar form as $z_0 = r e^{i \theta}$. The
distinct $n$-th roots of $z_0$ are
$$
  w_k
= r^{-n} e^{i\left(\frac{\theta}{n} + \frac{2 \pi k}{n}\right)}
= r^{-n} e^{i \frac{\theta}{n}} e^{i \frac{2 \pi k}{n}}
\quad k = 0, 1, 2, \dots, n-1
$$

First suppose $n = 2$. Then we have
$$
w_0 = r^{-2} e^{i \frac{\theta}{2}} e^{i \pi} = -r^{-2} e^{i \frac{\theta}{2}}
$$
and
$$
w_1 = r^{-2} e^{i \frac{\theta}{2}} e^{i 2 \pi} = r^{-2} e^{i
  \frac{\theta}{2}} = -w_0
$$
so $w_0 + w_1 = 0$.

Otherwise, $n > 2$. Then
\begin{align*}
   \sum_{k = 0}^{n-1} w_k
&= \sum_{k=0}^{n-1} r^{-n} e^{i\frac{\theta}{n}} e^{i\frac{2 \pi k}{n}} \\
&= r^{-n} e^{i\frac{\theta}{n}} \sum_{k=0}^{n-1} e^{i\frac{2 \pi k}{n}} \\
&= r^{-n} e^{i\frac{\theta}{n}} \sum_{k=0}^{n-1} (e^{i\frac{2 \pi}{n}})^k \\
&= r^{-n} e^{i\frac{\theta}{n}} \frac{(e^{i\frac{2 \pi}{n}})^n -
  1}{e^{i \frac{2\pi}{n}} - 1} \\
&= 0,
\end{align*}
using the result from Ch. I \S 2.11 and the fact that
$e^{\frac{2 \pi}{n}} \neq 1$ for $n > 2$.
\end{Answer}

\begin{Problem}
Let $z_1, z_2, z_3, z_4 \in \mathbb{C}$ be distinct. State
conditions in terms of computation of complex numbers which make
$z_1, z_2, z_3, z_4$ vertices of a square in the counterclockwise
direction.
\end{Problem}

\begin{Answer}
  Let $z_1, z_2, z_3, z_4$ form the vertices of a square which is
  length $l$ on a side. This square inscribes a circle with
  a radius $\frac{\sqrt{2}}{2}l$. First, consider the case where
  this circle (and hence the square) is centered at zero. Then
  this means $z_1 = \frac{\sqrt{2}}{2} e^{i \theta}$ for some
  $\theta \in [0, 2 \pi)$.
  Furthermore, the line segment between $z_2$ and the
  origin makes an angle of $\frac{\pi}{4}$ with the line segment
  between $z_1$ and the origin, and the same is true of the angle
  between $z_3$ and $z_2$, and so on, so that we have
  $z_2 = e^{i\frac{\pi}{2}} z_1 = i z_1$, $z_3 = i z_2$, $z_4 = i z_3$.
  In other words we have
  \begin{align*}
    z_1 &= \frac{\sqrt{2}}{2} e^{i \theta} \\
    z_2 &= i   z_1 \\
    z_3 &= i^2 z_1 \\
    z_4 &= i^3 z_1.
  \end{align*}

  However, it is possible for such a square to be centered anywhere in the
  plane rather than at the origin. Correspondingly, all of these points may
  be simultaneously translated by an arbitrary complex number $w$. Therefore
  we have
  $$
  z_k = \frac{\sqrt{2}}{2} e^{i (\theta + \frac{\pi}{4}(k-1))} + w
  $$
  for some $\theta \in [0, 2 \pi)$ and some $w \in \mathbb{C}$.

\end{Answer}

\begin{Problem}
Express $\sin z$ and $\cos z$ in rectangular form.
\end{Problem}

\begin{Answer}
We have
\begin{align*}
   \sin z
&= \frac{1}{2i}(e^{iz} - e^{-iz})
 = \frac{1}{2i}(e^{-y + ix} - e^{y - ix}) \\
&= \frac{1}{2i}
     [   e^{-y}(\cos x + i \sin x)
       - e^{y}(\cos x - i \sin x)
     ] \\
&= \frac{1}{2}
     [
       -  i e^{-y} \cos x
       +    e^{-y} \sin x
       +  i e^{y}
       +    e^{y}  \sin x
     ] \\
&= \frac{1}{2}[
       (e^y + e^{-y}) \sin x
    + i(e^y - e^{-y}) \cos x
   ] \\
&= \sin x \cosh y + i \cos x \sinh y.
\end{align*}

Furthermore
\begin{align*}
   \cos z
&= \frac{1}{2}[
     e^{-y}(\cos x + i \sin x)
   + e^{y} (\cos x - i \sin x)
   ] \\
&= (e^{y} + e^{-y}) \cos x + i (e^{-y} - e^y) \sin x \\
&= \cos x \cosh y - i \sin x \sinh y.
\end{align*}

\end{Answer}

\end{document}
