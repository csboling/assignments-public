
\documentclass{article}

\usepackage{amsmath}
\usepackage{amsfonts}
\usepackage{amssymb}
\usepackage{enumerate}
\usepackage{mathtools}
\usepackage{xfrac}
\usepackage[lastexercise]{exercise}

\DeclarePairedDelimiter\floor{\lfloor}{\rfloor}

\newcounter{Problem}
\newenvironment{Problem}{\begin{Exercise}[name={Problem},
                                          counter={Problem}]}
                        {\end{Exercise}}
\title{MATH 818 Homework \#5}
\date{January 21, 2015}
\author{Sam Boling}

\begin{document}

\begin{titlepage}
\maketitle
% TODO: I.1.8.c, I.2.12, I.3.4, 1, 4, 5
\end{titlepage}

\section{Textbook Problems}

\subsection*{Ch I \S 1.2}
\begin{itemize}
  \item[(c)]{
    \begin{align*}
       \frac{2 + i}{2 - i}
    &= (2+i)\left(\frac{1}{2 - i} + i\frac{1}{2 - i}\right) \\
    &= (2+i)\left(\frac{2}{4+1} + i\frac{1}{4 + 1}\right) \\
    &= (2+i)\left(\frac{2}{5} + i\frac{1}{5}\right) \\
    &= \frac{4}{5} + i\frac{2}{5} + i\frac{2}{5} - \frac{1}{5} \\
    &= \frac{3}{5} + i\frac{4}{5}.
    \end{align*}
  }
  \item[(f)]{
    \begin{align*}
       \frac{i}{1+i}
    &= i\left(\frac{1}{2} - i\frac{1}{2}\right) \\
    &= \frac{1}{2} + i\frac{1}{2}.
    \end{align*}
  }
\end{itemize}

\subsection*{Ch. I \S 1.7}
We have that $(1 + i) = \sqrt{2}e^{\frac{\pi}{4} i}$.
Therefore
$$
  (1 + i)^{100}
= (\sqrt{2}^2)^{50} e^{25 \pi i}
= 2^{50} e^{\pi i}
= -2^{50},
$$
so $\mathrm{Re}((1 + i)^{100}) = -2^{50}$ and
$\mathrm{Im}((1 + i)^{100}) = 0$.

\subsection*{Ch. I \S 1.8}
\begin{enumerate}[(a)]
  \item{
    Let $x = z - w$ and $y = w$. From the triangle inequality,
    $|x + y| \leq |x| + |y|$. But $x + y = z$, so this gives
    $|z - w + w| \leq |z - w| + |w|$ or $|z| \leq |z - w| + |w|$ as
    desired.
  }
  \item{
    Since $|w|$ is real and nonnegative, we may subtract $|w|$ from
    both sides of the previous inequality to get $|z| - |w| \leq |z - w|$.
  }
  \item{
    \begin{align*}
    \end{align*}
  }
\end{enumerate}

\subsection*{Ch. I \S 1.10}
\begin{itemize}
  \item[(b)]
  {
    $|z - i + 3| = |z - (i - 3)| > 5$ corresponds to the complement of
    the closed disk of radius $5$ centered at the point $-3 + i$,
    i.e. $(-3, 1)$.
  }
  \item[(c)]
  {
    $|z - i + 3| = |z - (i - 3)| \leq 5$ corresponds to the closed
    disk of radius $5$ centered at the point $-3 + i$, i.e. $(-3, 1)$.
  }
  \item[(d)]
  {
    $|z + 2i| = |z - (-2i)| \leq 1$ corresponds to the closed disk of
    radius 1 centered at the point $-2i$, i.e. $(0, -2)$.
  }
  \item[(f)]
  {
    $\mathrm{Im}~z \geq 0$ corresponds to the closed upper half-plane.
  }
  \item[(h)]
  {
    $\mathrm{Re}~z \geq 0$ corresponds to the closed right half-plane.
  }
\end{itemize}

\subsection*{Ch. I \S 2.1}
\begin{itemize}
  \item[(a)]{
    \begin{align*}
      |1 + i| &= \sqrt{1^2 + 1^2} = \sqrt{2} \\
      \mathrm{Arg}(1 + i) &= \arctan \frac{1}{1} = \frac{\pi}{4}
    \end{align*}
    so $1 + i = \sqrt{2}e^{\frac{\pi}{4} i}$
  }
  \item[(c)]{
    $|-3| = 3$ and $\mathrm{Arg}(-3) = \pi$, so
    $-3 = 3e^{\pi i}$
  }
  \item[(d)]{
    $|4i| = 4$ and $\mathrm{Arg}(4i) = \frac{\pi}{2}$, so
    $4i = 4e^{\frac{\pi}{2} i}$
  }
  \item[(h)]{
    $$
      -1 - i
    = -(1+i)
    = e^{\pi i} (1 + i)
    = \sqrt{2}e^{\frac{7\pi}{4} i}.
    $$
  }
\end{itemize}

\subsection*{Ch. I \S 2.2}
\begin{itemize}
  \item[(b)]{
    $\cos \frac{2 \pi}{3} = -\frac{1}{2}$,
    $\sin \frac{2 \pi}{3} = \frac{\sqrt{3}}{2}$,
    so $e^{2 i \pi / 3} = -\frac{1}{2} + \frac{\sqrt{3}}{2} i$.
  }
  \item[(c)]{
    $\cos \frac{\pi}{4} = \sqrt{2}{2}$,
    $\sin \frac{\pi}{4} = \sqrt{2}{2}$,
    so
    $$
      3 e^{i \pi / 4}
    = 3(\cos \frac{\pi}{4} + i \sin \frac{\pi}{4})
    = \frac{3 \sqrt{2}}{2} + \frac{3 \sqrt{2}}{2} i.
    $$
  }
  \item[(f)]{
    $\cos \left(-\frac{\pi}{2}\right) = 0$,
    $\sin \left(-\frac{\pi}{2}\right) = -1$
    so $e^{-i \pi / 2} = -i$.
  }
  \item[(g)]{
    $\cos (-\pi) = -1$,
    $\sin (-\pi) = 0$,
    so $e^{-i\pi} = -1$.
  }
\end{itemize}

\subsection*{Ch. I \S 2.8}

\subsection*{Ch. I \S 2.11}
Let $s = 1 + z + z^2 + \cdots + z^n$ and note that
\begin{align*}
   (1 - z)s
&= (1 - z)\sum_{k=0}^n z^k
 = \sum_{k=0}^n z^k - z\sum_{k=0}^n z^k \\
&= \sum_{k=0}^n z^k - \sum_{k=0}^n z^{k+1}
 = \sum_{k=0}^n z^k - \sum_{k=1}^{n+1} z^{k} \\
&= \left(1 + \sum_{k=1}^n z^k\right) - \left(\sum_{k=1}^n z^k +
  z^{n+1}\right) \\
&= 1 - z^{n+1}
\end{align*}
so that when $z \neq 1$ we can divide by $(1 - z)$ and find
$$
  s
= \sum_{k=0}^n z^k = \frac{1 - z^{n+1}}{1 - z}
= \frac{z^{n+1} - 1}{z - 1}.
$$

\subsection*{Ch. I \S 2.12}
Let $z = e^{i \theta} = \cos \theta + i \sin \theta$. From the
previous exercise we have
\begin{align*}
   \sum_{k=0}^n z^k
&= \frac{z^{n+1} - 1}{z - 1}
 = \frac{e^{i(n+1)\theta} - 1}{e^{i\theta} - 1} \\
&= \frac{\cos ((n+1)\theta) - 1 + i \sin ((n+1)\theta)}
        {\cos \theta -  1 + i \sin \theta}
\end{align*}
where
\begin{align*}
   (\cos \theta - 1 + i \sin \theta)^{-1}
&= \frac{\cos \theta - 1}
        {(\cos \theta - 1)^2 + \sin^2 \theta}
 + i\frac{\sin \theta}
        {(\cos \theta - 1)^2 + \sin^2 \theta} \\
&= \frac{\cos \theta + i \sin \theta - 1}
        {cos^2 \theta - 2 \cos \theta + 1 + \sin^2 \theta} \\
&= \frac{1}{2}\left(
        \frac{\cos \theta - 1}
             {\cos \theta + 1}
   +   i\frac{\sin \theta - 1}
             {\cos \theta + 1}
   \right)
\end{align*}

\subsection*{Ch. I \S 3.4}
\begin{itemize}
  \item[(a)]{
  }
  \item[(b)]{
  }
  \item[(c)]{
    The set $\{ z \in \mathbb{C} : |z| \geq 100 \}$
  }
\end{itemize}

\subsection*{Ch. II \S 3.4}

\section{Additional Problems}

\begin{Problem}
Compute the following principal logarithms using
$\mathrm{Arg} : \mathbb{C} \to [0, 2\pi)$ and
$\mathrm{Arg} : \mathbb{C} \to (-\pi, \pi]$.
\begin{itemize}
  \item{
    $\mathrm{Log}(-i)$
  }
  \item{
    $\mathrm{Log}(\sqrt{3} + i)$
  }
  \item{
    $\mathrm{Log}(-1 - i)$
  }
\end{itemize}
\end{Problem}

\begin{Answer}
  \begin{itemize}
    \item{

    }
  \end{itemize}
\end{Answer}

\begin{Problem}
Find all $z \in \mathbb{C}$ in rectangular form which solve
$z^6 + 1 = 0$. Trigonometric functions must be evaluated.
\end{Problem}

\begin{Answer}
The solutions of $z^6 + 1 = 0$ are the 6th roots of -1. Since $|-1| =
1$ and $\mathrm{Arg}(-1) = \pi$, these are the numbers $e^{i\theta}$ with unit
modulus and argument given by
$$
  \theta
= \frac{\pi}{6} + \frac{2 \pi k}{6},
\quad k \in \{ 0, 1, 2, \dots \}.
$$
Explicitly,
\begin{align*}
   e^{i\frac{\pi}{6}}
&= \cos \frac{\pi}{6} + i \sin \frac{\pi}{6}
 = \frac{\sqrt{3}}{2} + i \frac{1}{2} \\
   e^{i\frac{pi}{2}}
&= i \\
   e^{i\frac{5 \pi}{6}}
&= \cos \frac{5\pi}{6} + i \sin \frac{5\pi}{6}
 = -\frac{\sqrt{3}}{2} + i \frac{1}{2} \\
   e^{i\frac{7\pi}{6}}
&= \cos \frac{7\pi}{6} + i \sin \frac{7\pi}{6}
 = -\frac{\sqrt{3}}{2} - i \frac{1}{2} \\
   e^{i\frac{9\pi}{6}}
&= \cos \frac{3\pi}{2} + i \sin \frac{3\pi}{2}
 = -i \\
   e^{i\frac{11\pi}{6}}
&= \cos \frac{11\pi}{6} + i \sin \frac{11\pi}{6}
 = \frac{\sqrt{3}}{2} - i \frac{1}{2}
\end{align*}

\end{Answer}

\begin{Problem}
Let $n \in \mathbb{N}$ with $n \geq 2$ and
$z_0 \in \mathbb{C} - \{ 0 \}$.
Let $w_1, \dots, w_n$ be the distinct $n$-th roots of $z_0$.
Prove that $\sum_{k=1}^n w_k = 0$.
\end{Problem}

\begin{Answer}
Write $z_0$ in polar form as $z_0 = r e^{i \theta}$. The
distinct $n$-th roots of $z_0$ are
$$
  w_k
= r^{-n} e^{i\left(\frac{\theta}{n} + \frac{2 \pi k}{n}\right)}
= r^{-n} e^{i \frac{\theta}{n}} e^{i \frac{2 \pi k}{n}}
\quad k = 0, 1, 2, \dots, n-1
$$

First suppose $n = 2$. Then we have
$$
w_0 = r^{-2} e^{i \frac{\theta}{2}} e^{i \pi} = -r^{-2} e^{i \frac{\theta}{2}}
$$
and
$$
w_1 = r^{-2} e^{i \frac{\theta}{2}} e^{i 2 \pi} = r^{-2} e^{i
  \frac{\theta}{2}} = -w_0
$$
so $w_0 + w_1 = 0$.

Otherwise, $n > 2$. Then
\begin{align*}
   \sum_{k = 0}^{n-1} w_k
&= \sum_{k=0}^{n-1} r^{-n} e^{i\frac{\theta}{n}} e^{i\frac{2 \pi k}{n}} \\
&= r^{-n} e^{i\frac{\theta}{n}} \sum_{k=0}^{n-1} e^{i\frac{2 \pi k}{n}} \\
&= r^{-n} e^{i\frac{\theta}{n}} \sum_{k=0}^{n-1} (e^{i\frac{2 \pi}{n}})^k \\
&= r^{-n} e^{i\frac{\theta}{n}} \frac{(e^{i\frac{2 \pi}{n}})^n -
  1}{e^{i \frac{2\pi}{n}} - 1} \\
&= 0,
\end{align*}
using the result from Ch. I \S 2.11 and the fact that
$e^{\frac{2 \pi}{n}} \neq 1$ for $n > 2$.
\end{Answer}

\begin{Problem}
Let $z_1, z_2, z_3, z_4 \in \mathbb{C}$ be distinct. State
conditions in terms of computation of complex numbers which make
$z_1, z_2, z_3, z_4$ vertices of a square in the counterclockwise
direction.
\end{Problem}

\begin{Problem}
Express $\sin z$ and $\cos z$ in rectangular form.
\end{Problem}

\end{document}
