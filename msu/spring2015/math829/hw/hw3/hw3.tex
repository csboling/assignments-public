
\documentclass{article}

\usepackage{amsmath}
\usepackage{amsfonts}
\usepackage{amssymb}
\usepackage{enumerate}
\usepackage{mathtools}
\usepackage{xfrac}
\usepackage[lastexercise]{exercise}

\DeclarePairedDelimiter\floor{\lfloor}{\rfloor}

\newcounter{Problem}
\newenvironment{Problem}{\begin{Exercise}[name={Problem},
                                          counter={Problem}]}
                        {\end{Exercise}}
\title{MATH 829 Homework \#3}
\date{February 4, 2015}
\author{Sam Boling}

\begin{document}

\begin{titlepage}
\maketitle
% TODO: III.6.2?
\end{titlepage}

\section{Textbook Problems}

\subsection*{Ch. II, \S 2.4}
\begin{itemize}
  \item[(a)]{
    The series
    $$
    \sum_{n=0}^\infty n^n z^n
    $$
    has $a_n = n^n$ so that
    $$
      \limsup |a_n|^{1 / n}
    = \limsup (n^n)^{1 / n}
    = \limsup n = \infty,
    $$
    and therefore $R = 0$.
  }
  \item[(c)]{
    The series
    $$
    \sum_{n=0}^\infty 2^n z^n
    $$
    has
    $$
      \left|\frac{a_{n+1}}{a_n}\right|
    = \frac{2^{n+1}}{2^n}
    = 2
    $$
    so that $R = \frac{1}{2}$.
  }
  \item[(f)]{
    The series
    $$
    \sum_{n=0}^\infty n^2 z^n
    $$
    has
    $$
      \left|\frac{a_{n+1}}{a_n}\right|
    = \frac{n^2 + 2n + 1}{n^2}
    = 1 + 2\frac{1}{n} + \frac{1}{n^2} \to 1
    $$
    so that $R = 1$.
  }
  \item[(g)]{
    The series
    $$
    \sum_{n=0}^\infty \frac{n!}{n^n} z^n
    $$
    has
    $$
      \frac{|a_{n+1}|}{|a_n|}
    = \frac{n!}{(n+1)!}\frac{n^n}{n^{n+1}}
    = \frac{1}{n+1}\frac{1}{n}
    = \frac{1}{n^2 + n} \to 0
    $$
    so that $R = \infty$.
  }
\end{itemize}

\subsection*{Ch. II \S 2.10}
\begin{enumerate}[(a)]
  \item{
    Noting that
    $\sum (a_n + b_n) z^n = \sum a_n z^n + \sum b_n z^n$
    so that
    $$
      \lim_{n \to \infty} \sum_{n=0}^\infty (a_n + b_n) z^n
    = \lim_{n \to \infty} \sum_{n=0}^\infty a_n z^n
    + \lim_{n \to \infty} \sum_{n=0}^\infty b_n z^n,
    $$
    we see that this series is convergent if and only if
    both $\sum a_n z^n$ and $\sum b_n z^n$ converge. Therefore
    the region of convergence is the intersection of the
    regions of convergence of these two series, i.e.
    $D(0, r) \cap D(0, s)$, so the region of convergence is
    given by $\min(r, s)$.
  }
  \item{
    We see that
    \begin{align*}
       \lim_{n \to \infty} \left|\frac{a_{n+1} b_{n+1}}{a_n b_n}\right|
    &= \lim_{n \to \infty} \left|\frac{a_{n+1}}{a_n}\right|
                       \left|\frac{b_{n+1}}{b_n}\right|
     = \left(\lim_{n \to \infty}
        \left|\frac{a_{n+1}}{a_n}\right|
       \right)
       \left(\lim_{n \to \infty}
         \left|\frac{b_{n+1}}{b_n}\right|
       \right) \\
    &= \frac{1}{r} \frac{1}s}
     = \frac{1}{rs}
    \end{align*}
    so that the radius of convergence is $rs$.
  }
\end{enumerate}

\subsection*{Ch. III \S 6.1}
\begin{itemize}
  \item[(d)]{
    $i^i = e^{i \log i}$,
    and $i = e^{i \frac{\pi}{2}}$ so for this branch of the logarithm
    we have $\log i = i\frac{\pi}{2}$. Then
    $e^{i \log i} = e^{-\frac{\pi}{2}}$.
  }
  \item[(e)]{
    $(-i)^i = e^{i \log (-i)}$, and $-i = e^{-i\frac{\pi}{2}}$ so for
    this branch of the logarithm we have $\log i = -i\frac{\pi}{2}$,
    so $(-i)^i = e^{\frac{\pi}{2}}$.
  }
\end{itemize}

\subsection*{Ch. III \S 6.2}
\begin{itemize}
  \item[(d)]{
    $i^i = e^{i \log i}$,
    and $i = e^{i \frac{\pi}{2}}$ so for this branch of the logarithm
    we have $\log i = i\frac{\pi}{2}$. Then
    $e^{i \log i} = e^{-\frac{\pi}{2}}$.
  }
  \item[(e)]{
    $(-i)^i = e^{i \log (-i)}$, and $-i = e^{-i\frac{\pi}{2}}$ so for
    this branch of the logarithm we have $\log i = -i\frac{\pi}{2}$,
    so $(-i)^i = e^{\frac{\pi}{2}}$.
  }
\end{itemize}

\section{Additional Problems}

\begin{Problem}
  Let $\mathrm{Log} z$ be the principal logarithm function
  defined using $\mathrm{Arg}~z \in (-\pi, \pi]$. Prove that
  $\mathrm{Log}~z$ is holomorphic on
  $\mathbb{C} \backslash \{x \in \mathbb{R} : x \leq 0}$ and
  that $\frac{d}{dz} \mathrm{Log}~z = \frac{1}{z}$.
  (Hint: Divide the domain into the union of the right, lower, and
  upper half-planes. Apply the Cauchy-Riemann equations.)
\end{Problem}

\begin{Answer}
  We have $\mathrm{Log}~z = \log |z| + i \mathrm{Arg}~z$.
  Let $z = x + iy = e^{i \theta}$, and write
  $\mathrm{Log}~(x + iy) = u(x, y) + i v(x, y)$.

  \begin{itemize}
    \item{
      In the half-plane $\{ z : \mathrm{Im}~z < 0 \}$ we have
      $\theta \neq 0$, $\theta \neq \pi$, so $\sin \theta \neq 0$
      and thus
      $\cot \theta = \frac{\cos \theta}{\sin \theta} = \frac{x}{y}$ is defined.
      On this domain we then have $\mathrm{Arg}~z = \cot^{-1} \frac{y}{x}$, so
      $$
      \mathrm{Log}~z = \log \sqrt{x^2 + y^2} + i \cot^{-1} \frac{y}{x}
      $$
      so that
      $$
      u(x, y) = \log \sqrt{x^2 + y^2}, \quad
      v(x, y) = \cot^{-1} \frac{x}{y}
      $$
      and then
      \begin{align*}
      u_x &= \left(\frac{\partial}{\partial x} \sqrt{x^2 + y^2}\right)
             \left(\frac{1}{\sqrt{x^2 + y^2}}\right)
           = (2x)\left(\frac{1}{2}(x^2 + y^2)^{-1/2}\right)
             \left(\frac{1}{\sqrt{x^2 + y^2}}\right) \\
          &= \frac{x}{x^2 + y^2}, \\
      u_y &= \frac{y}{x^2 + y^2} \\
      v_x &= \frac{1}{y}\left(-\frac{1}{1 + \left(\frac{x}{y}\right)^2}\right)
           = -\frac{y}{x^2 + y^2} \\
      v_y &= -\frac{x}{y^2}\left(-\frac{1}{1 + \left(\frac{x}{y}\right)^2}\right)
           = \frac{x}{x^2 + y^2},
      \end{align*}
      so the Cauchy-Riemann equations are satisfied.
    }
    \item{
      In the half-plane $\{z : \mathrm{Im}~z > 0\}$ we have
      $\theta \neq 0$, $\theta \neq \pi$, so $\sin \theta \neq 0$ and
      $\theta = \cot^{-1} \frac{y}{x}$, so that
      $$
      u(x, y) = \log \sqrt{x^2 + y^2}, \quad
      v(x, y) = \cot^{-1} \frac{x}{y}
      $$
      in this case as well, so the Cauchy-Riemann
      equations are satisfied in this region as well.
    }
    \item{
      In the half-plane $\{z : \mathrm{Re}~z > 0\}$ we have
      $\theta \neq \pm \frac{\pi}{2}$, so $\cos \theta \neq 0$ and
      $\tan \theta = \frac{\sin \theta}{\cos \theta}$ is defined.
      Therefore $\theta = \tan^{-1} \theta$ so
      $$
      u(x, y) = \log \sqrt{x^2 + y^2}, \quad
      v(x, y) = \tan^{-1} \frac{y}{x}
      $$
      and then
      \begin{align*}
      u_x &= \frac{x}{x^2 + y^2}, \\
      u_y &= \frac{y}{x^2 + y^2} \\
      v_x &= -\frac{y}{x^2}\frac{1}{\frac{y^2}{x^2} + 1}
           = -\frac{y}{x^2 + y^2}, \\
      v_y &= \frac{1}{x}\frac{1}{\frac{y^2}{x^2} + 1}
           = \frac{x}{x^2 + y^2}.
      \end{align*}
    }
\end{Answer}

\begin{Problem}
  \begin{enumerate}[(i)]
    \item{
      Let $f$, $g$ be two entire functions such that
      $f^\prime(z) = g^\prime(z)$ for all $z$. Show that there is a
      constant $C \in \mathbb{C}$ such that $f = g + C$.
    }
    \item{
      Let $f$ be an entire function and $n \in \mathbb{N}$.
      Suppose $f^{(n)} = 0$. Show that $f$ is a polynomial
      with $\deg f \leq n - 1$.
    }
  \end{enumerate}
\end{Problem}

\begin{Problem}
  Let $f$ be an entire function. Suppose that $|f|$ is constant. Prove
  that $f$ is constant.
  (Hint: $|f| = C$ implies $u^2 + v^2 = C^2$. Take partial derivatives
  and apply the Cauchy-Riemann equations.)
\end{Problem}

\begin{Answer}
  Since $|f| = C$, $|f|^2 = u^2 + v^2 = C^2$. Since $f$ is
  holomorphic, we must have
  \begin{align*}
    0 &= \frac{\partial}{\partial x} (u^2 + v^2)
       = u_x \cdot 2 u + v_x \cdot 2 v \\
      &= 2 u u_x - 2 v u_y
  \end{align*}
  and
  \begin{align*}
    0 &= \frac{\partial}{\partial y} (u^2 + v^2)
       = u_y \cdot 2 u + v_y \cdot 2 v \\
      &= 2 u v_x - 2 v v_y,
  \end{align*}
  so that
  $$
  u u_x = v u_y, \quad
  u v_x = v v_y.
  $$
  Multiplying the first equation by $i$ and adding it to the second we
  have
  \begin{align*}
  u u_x + i u v_x &= v u_y + i v v_y \\
  u (u_x + i v_x) &= v (u_y + i v_y).
  \end{align*}
  But $u_x + i v_x = f^\prime$, and since
  $$
  i f^\prime = i(v_y - i u_y) = u_y + i v_y,
  $$
  we have $u f^\prime = i v f^\prime$ or
  $(u - iv)f^\prime = \bar{f} f^\prime = 0$. Then it must be that
  $f = 0$ or $f^\prime = 0$, and in either case this implies that $f$
  is constant.
\end{Answer}

\begin{Problem}
  Let $f$ be an entire function Suppose that $f^\prime(z) = f(z)$ for
  any $z \in \mathbb{C}$. Prove that $f(z) = C e^z$ for some
  $C \in \mathbb{C}$.
\end{Problem}

\begin{Problem}
  Let $\sum_{n=0}^\infty a_n z^n$ be a power series with radius $R$.
  \begin{enumerate}[(i)]
    \item{
      If the series diverges at $z = 3 - 4i$, what can you say about $R$?
    }
    \item{
      If the series converges at every $z \in D(0, 1)$, what can you say about
      $R$?
    }
    \item{
      What is the radius of $\sum_{n=0}^\infty a_n z^{2n}$?
    }
  \end{enumerate}
\end{Problem}

\begin{Answer}
  \begin{enumerate}
    \item{
      We must have that the series converges for every $z$ with $|z| <
      R$, so since $|3 - 4i| = \sqrt{3^2 + 4^2} = 5$ it must be that
      $R \leq 5$.
    }
    \item{
      Similarly, since $0 \in D(0,1)$ and $|0| = 0$, we have that
      $R = 0$ since $R \geq 0$ must also be satisfied.
    }
    \item{
    }
  \end{enumerate}
\end{Answer}

\end{document}
