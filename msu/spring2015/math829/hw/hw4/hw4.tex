
\documentclass{article}

\usepackage{amsmath}
\usepackage{amsfonts}
\usepackage{amssymb}
\usepackage{enumerate}
\usepackage{mathtools}
\usepackage{xfrac}
\usepackage[lastexercise]{exercise}

\DeclarePairedDelimiter\floor{\lfloor}{\rfloor}

\newcounter{Problem}
\newenvironment{Problem}{\begin{Exercise}[name={Problem},
                                          counter={Problem}]}
                        {\end{Exercise}}
\title{MATH 829 Homework \#4}
\date{February 11, 2015}
\author{Sam Boling}

\begin{document}

\begin{titlepage}
\maketitle
\end{titlepage}

\section{Textbook Problems}

\subsection*{Ch. II \S 5.1}
Define
$$
b_k
=
\left\{\begin{array}{l l}
  \frac{1}{k!}, & \quad k \text{ even} \\
  0,            & \quad k \text{ odd}
\end{array}\right.
$$
so that
$$
f(z) = \sum_{n=0}^{\infty} \frac{z^{2n}}{(2n)!} = \sum_{k=0}^\infty b_k z^k.
$$
Then
\begin{align*}
   f^{\prime\prime}(z)
&= \sum_{k=2}^\infty k (k-1) b_k z^{k-2}
&= \sum_{k \in 2\mathbb{N}} \frac{z^{k-2}}{(k-2)!} \\
&= \sum_{n \in \mathbb{N}} \frac{z^{2n - 2}}{(2n - 2)!}
\end{align*}
under the substitution $k = 2n$, so that
\begin{align*}
   f^{\prime\prime}(z)
&= \sum_{n=1}^\infty \frac{z^{2n - 2}}{(2n - 2)!}
 = \sum_{n=1}^\infty \frac{z^{2(n-1)}}{(2(n-1))!} \\
&= \sum_{n=0}^\infty \frac{z^{2n}}{(2n)!}
 = f(z).
\end{align*}

\subsection*{Ch. II \S 5.4}
Defining
$$
a_k
=
\left\{\begin{array}{l l}
  \frac{(-1)^{\frac{k}{2}}}{\left(\frac{k}{2}!\right)^2 2^k},
    & \quad k \text{ even} \\
  0 & \quad k \text{ odd}
\end{array}\right.,
$$
we can write
$$
  J(z)
= \sum_{n=0}^\infty \frac{(-1)^n}{(n!)^2}\left(\frac{z}{2}\right)^{2n}
= \sum_{k=0}^\infty a_k z^k
$$
so that
\begin{align*}
   J^\prime(z)
&= \frac{d}{dz} \sum_{k=0}^\infty a_k z^k
 = \sum_{k=1}^\infty k a_k z^{k-1} \\
&= \sum_{k=2}^\infty k a_k z^{k-1}
 = \sum_{k \in 2\mathbb{N}} k a_k z^{k-1} \\
&= \sum_{k \in 2\mathbb{N}} k \frac{(-1)^{\frac{k}{2}}}{\left(\frac{k}{2}!\right)^2 2^k} z^{k-1}
\end{align*}
since $a_1 = 0$. But then by substituting $n = 2k$,
\begin{align*}
   J^\prime(z)
&= \sum_{n=1}^\infty 2n \frac{(-1)^{n}}{(n!)^2 2^{2n}} z^{2n-1}
 = \sum_{n=1}^\infty \frac{(-1)^n}{(n-1)! \cdot n! \cdot 2^{2n-1}} z^{2n-1}.
\end{align*}
By the same procedure
\begin{align*}
   J^{\prime\prime}(z)
&= \sum_{k=2}^\infty k (k-1) a_k z^{k-2}
 = \sum_{n=1}^\infty 2n (2n-1) \frac{(-1)^{n}}{(n!)^2 2^{2n}} z^{2n-2} \\
&= \sum_{n=1}^\infty (2n-1) \frac{(-1)^n}{(n-1)! \cdot n! \cdot 2^{2n-1}} z^{2n-2}.
\end{align*}
Then
\begin{align*}
   z^2 J^{\prime\prime}(z) + z J^\prime(z) + z^2 J(z)
&= \sum_{n=1}^\infty (2n-1) \frac{(-1)^n}{(n-1)! \cdot n! \cdot 2^{2n-1}} z^{2n} \\
&+ \sum_{n=1}^\infty \frac{(-1)^n}{(n-1)! \cdot n! \cdot 2^{2n-1}} z^{2n} \\
&+ \sum_{n=0}^\infty \frac{(-1)^n}{(n!)^2 2^{2n}} z^{2n + 2} \\
&= z^2 + \sum_{n=1}^\infty
\end{align*}

\subsection*{Ch. II \S 5.6(a)}
Observe that by a change of index $k = n+1$
\begin{align*}
   \log z
&= \sum_{n=1}^\infty \frac{(-1)^{n-1}}{n}(z-1)^n
 = \sum_{n=0}^\infty \frac{(-1)^n}{n+1} (z-1)^{n+1} \\
&= (z-1)\sum_{n=0}^\infty \frac{(-1)^n}{n+1}
\end{align*}
so that
\begin{align*}
   \frac{d}{dz} \log z
&= \left[\frac{d}{dz} (z-1)\right]
   \sum_{n=0}^\infty \frac{(-1)^n}{n+1} (z-1)^n
 + (z-1)\sum_{n=1}^\infty n\frac{(-1)^n}{n+1} (z - 1)^{n-1} \\
&= \sum_{n=0}^\infty \frac{(-1)^n}{n+1} (z-1)^n
 + \sum_{n=1}^\infty n \frac{(-1)^n}{n+1} (z-1)^n \\
&= 1 + \sum_{n=1}^\infty (1 + n) \frac{(-1)^n}{n+1} (z-1)^n \\
&= 1 + \sum_{n=1}^\infty (-1)^n (z-1)^n \\
&= \sum_{n=0}^\infty (1-z)^n,
\end{align*}
and when $|z - 1| < 1$ this geometric series converges to
$$
\frac{d}{dz} \log z = \frac{1}{1 - (1 - z)} = \frac{1}{z}.
$$

\section{Additional Problems}

\begin{Problem}
  Prove that $\sum_{n=0}^\infty z^n$ does not converge uniformly on
  $D(0, 1)$.
\end{Problem}

\begin{Answer}
  We consider the sequence $(f_n)$ given by
  $$
  f_n(z) = \sum_{k=0}^n z^k,
  $$
  since by definition $\sum_{n=0}^\infty z^n$ is the limit
  of this sequence.

  For any $n$, we can compute the supremum norm
  \begin{align*}
    \left\|
      f_{n+1} - f_n
    \right\|_{D(0,1)}
  &=
    \left\|
      \sum_{k=0}^{n+1} z^k
    - \sum_{k=0}^{n} z^k
    \right\|_{D(0,1)}
   =
  \left\|
    z^{n+1}
  \right\|_{D(0,1)} \\
  &=
  \sup_{D(0,1)} |z^{n+1}|
   =
  \sup_{D(0,1)} |z|^{n+1}
   = 1
  \end{align*}
  and therefore the sequence $(f_n)$ is not uniformly
  Cauchy.
\end{Answer}

\begin{Problem}
  Let $f$ be analytic on an open set $U$. Let
  $V = \{ z \in \mathbb{C} : \bar{z} \in U \}$. Define $g$ on $V$ by
  $g(z) = \overline{f(\bar{z})}$. Show that $g$ is analytic on $V$.
\end{Problem}

\begin{Answer}
For $z_0 \in V$ we have $\bar{z_0} \in U$, so
$f$ is analytic at $\bar{z_0}$, so there is an
$r > 0$ and a sequence $(a_n)$ such that
$$
  f(w)
= \sum_{n=0}^\infty a_n(w - \bar{z_0})^n, \quad
  |w - z_0| < r.
$$

Now let $z \in D(z_0, r)$, and write $z = a + ib$ and
$z_0 = a_0 + ib_0$. Then $|z - z_0| < r$, but we note that
\begin{align*}
  |z - z_0|
&= |(a + ib) - (a_0 + ib_0)|
 = |(a - a_0) + i(b - b_0)| \\
&= \sqrt{(a - a_0)^2 + (b - b_0)^2}
\end{align*}
while
\begin{align*}
  |\bar{z} - \bar{z_0}|
&= |(a - ib) - (a_0 - ib_0)|
 = |(a - a_0) + i(-b + b_0)| \\
&= \sqrt{(a - a_0)^2 + (b_0 - b)^2}
 = \sqrt{(a - a_0)^2 + (b - b_0)^2} \\
&= |z - z_0|,
\end{align*}
so if $z \in D(z_0, r)$ then $\bar{z} \in D(\bar{z_0}, r)$. Therefore
the expression
$$
f(\bar{z})
= \sum_{n=0}^\infty a_n (\bar{z} - \bar{z_0})^n
$$
holds, so
$$
  g(z)
= \overline{f(\bar{z})}
= \sum_{n=0}^\infty \frac{\overline{f^{(n)}(\bar{z_0})}}{n!}
                 \overline{(\bar{z} - \bar{z_0})^n}
= \sum_{n=0}^\infty \frac{\overline{f^{(n)}(\bar{z_0})}}{n!}
                 (z - z_0)^n, \quad
|z - z_0| < r.
$$
Since $z_0$ was chosen arbitrarily
in $V$, this means $g$ is analytic on $V$.
\end{Answer}

\begin{Problem}
  Let $U$ be a domain such that $\forall z \in U$, $\bar{z} \in U$.
  \begin{enumerate}[(i)]
    \item{
      Show that $U \cap \mathbb{R}$ contains an open interval.
    }
    \item{
      Let $f$ be analytic on $U$. Suppose that $f(x) \in \mathbb{R}$
      $\forall x \in \mathbb{R}$. Show that $f(\bar{z}) = \overline{f(z)}$
      for all $z \in U$.
    }
  \end{enumerate}
\end{Problem}

\begin{Problem}
  Show that there does not exist a function which is analytic on
  $\mathbb{C}$ that satisfies
  $f\left(\frac{1}{n}\right) = \left|\frac{1}{n^3}\right|$ for all
  $n \in \mathbb{Z} \backslash \{ 0 \}$.
\end{Problem}

\begin{Answer}
  The function $g(z) = z^3$ is a polynomial and therefore analytic
  on $\mathbb{C}$. But $g(z) = f(z)$ for any $z = \frac{1}{n}$ where
  $n \in \mathbb{N} \subset \mathbb{Z} \backslash \{0\},$, and the set
  $$
  \left\{\frac{1}{n} : n \in \mathbb{N} \right\}
  $$
  has an accumulation point at $0 \in \mathbb{C}$. From the uniqueness theorem
  this means that $f = g$ in $\mathbb{C}$, which is a contradiction.
\end{Answer}

\end{document}
