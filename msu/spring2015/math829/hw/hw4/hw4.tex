
\documentclass{article}

\usepackage{amsmath}
\usepackage{amsfonts}
\usepackage{amssymb}
\usepackage{enumerate}
\usepackage{mathtools}
\usepackage{xfrac}
\usepackage[lastexercise]{exercise}

\DeclarePairedDelimiter\floor{\lfloor}{\rfloor}

\newcounter{Problem}
\newenvironment{Problem}{\begin{Exercise}[name={Problem},
                                          counter={Problem}]}
                        {\end{Exercise}}
\title{MATH 829 Homework \#4}
\date{February 11, 2015}
\author{Sam Boling}

\begin{document}

\begin{titlepage}
\maketitle
\end{titlepage}

\section{Textbook Problems}

\subsection*{Ch. II \S 5.1}
We can observe that in general
$$
  \sum_{n=0}^\infty \frac{(-1)^n + 1}{2} a_n z^n
= \sum_{k=0}^\infty a_{2k} z^{2k}
$$
since $\frac{(-1)^n + 1}{2}$ is 0 when $n$ is odd and 1 when $n$ is
even. Then we have
\begin{align*}
   f^{(2)}(z)
&= \sum_{n=2}^{\infty} n (n - 1) \frac{(-1)^n + 1}{2} \frac{z^{n-2}}{n!}  \\
&= \sum_{n=2}^{\infty} \frac{(-1)^n + 1}{2} \frac{z^{n-2}}{(n-2)!}.
\end{align*}
Taking $k = n - 2$ we see that $n = k + 2$, and $(-1)^{k + 2} =
(-1)^k$ so this means
\begin{align*}
   f^{(2)}(z)
&= \sum_{k=0}^\infty \frac{(-1)^k + 1}{2} \frac{z^k}{k!}
 = \sum_{m=0}^{\infty} \frac{z^{2m}}{(2m)!} \\
&= f(z).
\end{align*}

To find the radius of convergence we observe that
$$
  f(z)
= \frac{1}{2}\left[
    \sum_{k=0}^\infty (-1)^k \frac{z^k}{k!}
  + \sum_{k=0}^\infty \frac{z^k}{k!}
  \right]
$$
and that $|(-1)^k a_k|^{1 / k} = |a_k|^{1 / k}$ for any $a_k$,
so the radius of convergence is that of
$$
e^z = \sum_{n=0}^\infty \frac{z^n}{n!}
$$
which is $\infty$.

\subsection*{Ch. II \S 5.4}
We have
\begin{align*}
   J^\prime(z)
&= \sum_{n=1}^\infty 2n
                   \frac{(-1)^n}{(n!)^2 2^n}
                   z^{2n-1} \\ \\
&= \sum_{n=1}^\infty \frac{n(-1)^n}{(n!)^2 \cdot 2^{2n-1}}
                   z^{2n-1}
\end{align*}
so that
\begin{align*}
   J^{\prime\prime}(z)
&= \sum_{n=2}^\infty (2n-1)
                   \frac{n(-1)^n}{(n!)^2 \cdot 2^{2n-1}}
                   z^{2n-2}
\end{align*}
and then
\begin{align*}
   z^2 J^{\prime\prime}(z) + zJ^\prime(z) + z^2 J(z)
&= \sum_{n=2}^\infty (2n-1)
                   \frac{n(-1)^n}{(n!)^2 \cdot 2^{2n-1}}
                   z^{2n-2} \\
&+ \sum_{n=1}^\infty \frac{n(-1)^n}{(n!)^2 \cdot 2^{2n-1}}
                   z^{2n-1} \\
&+ \sum_{n=0}^\infty \frac{(-1)^n}{(n!)^2 \cdot 2^{2n}}
                   z^{2(n+1)} \\
&= z^2 \\
&+ \left(-\frac{1}{4}z^2 - \frac{1}{2}z\right) \\
&+ \sum_{n=2}^\infty \left[
     (2n-1)
     \frac{n(-1)^n}{(n!)^2 \cdot 2^{2n-1}}
     z^{2n-2}
   + \frac{n(-1)^n}{(n!)^2 \cdot 2^{2n-1}}
     z^{2n-1}
   + \frac{(-1)^n}{(n!)^2 \cdot 2^{2n}}
     z^{2(n+1)}
   \right]\\
&=
\end{align*}


%begin{align*}
%  J(z)
%= \sum_{k=0}^\infty
%    \frac{(-1)^k + 1}{2}
%    \frac{(-1)^{\left\lfloor\frac{k}{2}\right\rfloor}}
%         {2^k \left(\left\lfloor\frac{k}{2}\right\rfloor!\right)^2}
%    z^k
%end{align*}
%o that the derivative is given by
%$
%^\prime(z)
% \sum_{k=1}^\infty
%   k
%   \frac{(-1)^k + 1}{2}
%   \frac{(-1)^{\left\lfloor\frac{k}{2}\right\rfloor}}
%        {2^k \left(\left\lfloor\frac{k}{2}\right\rfloor!\right)^2}
%   z^{k-1}
% \sum_{m=0}^\infty
%   (m+1)
%   \frac{(-1)^{m+1} + 1}{2}
%   \frac{(-1)^{}{}
%$
%
%$
% k
% \frac{(-1)^{\left\lfloor\frac{k}{2}\right\rfloor}}
%      {\left\lfloor\frac{k}{2}\right\rfloor!
%       \cdot \left\lfloor\frac{k}{2}\right\rfloor!
%       \cdot 2^{k}}
% 2n\frac{(-1)^n}{(n!)}
%$
%hen $J^{\prime\prime}(z)$ has zero odd coefficients and even
%oefficients given by
%$
% k\frac{(-1)^k}{(k-1) \cdot k! \cdot 2^k}
% \frac{(-1)^k}{(k-1)^2 2^k}.
%$
%e compute
%begin{align*}
%  z^2 J^{\prime\prime}(z) + zJ^\prime(z) + z^2 J(z)
%=
%end{align*}

\subsection*{Ch. II \S 5.6(a)}
First we find
\begin{align*}
   \frac{d}{dz} \sum_{n=0}^\infty (-1)^{n-1} \frac{(z-1)^n}{n}
&= \sum_{n=0}^\infty n (
\end{align*}

\section{Additional Problems}

\begin{Problem}
  Prove that $\sum_{n=0}^\infty z^n$ does not converge uniformly on
  $D(0, 1)$.
\end{Problem}

\begin{Problem}
  Let $f$ be analytic on an open set $U$. Let
  $V = \{ z \in \mathbb{C} : \bar{z} \in U \}$. Define $g$ on $V$ by
  $g(z) = \overline{f(\bar{z})}$. Show that $g$ is analytic on $V$.
\end{Problem}

\begin{Problem}
  Let $U$ be a domain such that $\forall z \in U$, $\bar{z} \in U$.
  \begin{enumerate}[(i)]
    \item{
      Show that $U \cap \mathbb{R}$ contains an open interval.
    }
    \item{
      Let $f$ be analytic on $U$. Suppose that $f(x) \in \mathbb{R}$
      $\forall x \in \mathbb{R}$. Show that $f(\bar{z}) = \overline{f(z)}$
      for all $z \in U$.
    }
  \end{enumerate}
\end{Problem}

\begin{Problem}
  Show that there does not exist a function which is analytic on
  $\mathbb{C}$ that satisfies
  $f\left(\frac{1}{n}\right) = \left|\frac{1}{n^3}\right|$ for all
  $n \in \mathbb{Z} \backslash \{ 0 \}$.
\end{Problem}

\end{document}
