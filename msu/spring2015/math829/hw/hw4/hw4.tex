
\documentclass{article}

\usepackage{amsmath}
\usepackage{amsfonts}
\usepackage{amssymb}
\usepackage{enumerate}
\usepackage{mathtools}
\usepackage{xfrac}
\usepackage[lastexercise]{exercise}

\DeclarePairedDelimiter\floor{\lfloor}{\rfloor}

\newcounter{Problem}
\newenvironment{Problem}{\begin{Exercise}[name={Problem},
                                          counter={Problem}]}
                        {\end{Exercise}}
\title{MATH 829 Homework \#4}
\date{February 11, 2015}
\author{Sam Boling}

\begin{document}

\begin{titlepage}
\maketitle
\end{titlepage}

\section{Textbook Problems}

\subsection*{Ch. II \S 5.1}
Let $w = z^2$ so that
$$
\frac{dz}{dw} = 2z, \quad
\frac{d^2z}{dw^2} = 2.
$$
Then
\begin{align*}
   f(z)
&= \sum_{n=0}^\infty \frac{(z^2)^n}{(2n)!}
 = \sum_{n=0}^\infty \frac{w^n}{(2n)!}
\end{align*}
so that
\begin{align*}
   \frac{d^2 f}{dz^2}
&= \frac{d^2 f}{dw^2} \left(\frac{dw}{dz}\right)^2
 + \frac{df}{dw} \frac{d^2w}{dz^2} \\
&= 4z^2
   \sum_{n=2}^\infty
     \frac{n(n-1)}{(2n)!}
     w^{n-2}
 + 2
   \sum_{n=1}^\infty
     \frac{n}{(2n)!}
     w^{n-1} \\
&= 4
   \sum_{n=2}^\infty
     \frac{n(n-1)}{(2n)!}
     w^{n-1}
 + 2
   \sum_{n=1}^\infty
     \frac{n}{(2n)!}
     w^{n-1} \\
&= 1
 + \sum_{n=2}^\infty
     \left[
       2\frac{n}{(2n)!}
       w^{n-1}
     + 4\frac{n(n-1)}{(2n)!}
       w^{n-1}
     \right] \\
&= 1
 + 2
   \sum_{n=2}^\infty
     \frac{n}{(2n)!}
     w^{n-1}
     (1 + 2(n-1)) \\
&= 1
 + \sum_{n=2}^\infty
     \frac{2n}{(2n)!}
     w^{n-1}
     (2n-1) \\
&= 1
 + \sum_{n=2}^\infty
     \frac{1}{(2(n-1))!}
     w^{n-1} \\
&= 1
 + \sum_{n=1}^\infty
     \frac{1}{(2n)!}
     w^n \\
&= \sum_{n=0}^\infty
     \frac{w^n}{(2n)!} \\
&= f(z).
\end{align*}
%Define
%$$
%a_k
%=
%\left\{\begin{array}{l l}
%  \frac{1}{k!}, & \quad k \text{ even} \\
%  0,            & \quad k \text{ odd}
%\end{array}\right.
%$$
%so that
%$$
%f(z) = \sum_{n=0}^{\infty} \frac{z^{2n}}{(2n)!} = \sum_{k=0}^\infty a_k z^k.
%$$
%Then
%\begin{align*}
%   f^{\prime\prime}(z)
%&= \sum_{k=2}^\infty k (k-1) a_k z^{k-2}
%&= \sum_{k \in 2\mathbb{N}} \frac{z^{k-2}}{(k-2)!} \\
%&= \sum_{n \in \mathbb{N}} \frac{z^{2n - 2}}{(2n - 2)!}
%\end{align*}
%under the substitution $k = 2n$, so that
%\begin{align*}
%   f^{\prime\prime}(z)
%&= \sum_{n=1}^\infty \frac{z^{2n - 2}}{(2n - 2)!}
% = \sum_{n=1}^\infty \frac{z^{2(n-1)}}{(2(n-1))!} \\
%&= \sum_{n=0}^\infty \frac{z^{2n}}{(2n)!}
% = f(z).
%\end{align*}

We observe also that under the substitution $w = z^2$ this is
the series
$$
\sum_{n=0}^{\infty} \frac{w^n}{(2n)!}
$$
which has radius of convergence $\infty$ since
$$
\lim \frac{|(2n)!|}{|(2n+1)!} = \frac{1}{2n+1} = 0
$$
and so $f(z)$ has radius of convergence $\infty$ as well.

\subsection*{Ch. II \S 5.4}
Let $w = \left(\frac{z}{2}\right)^2$ so that
$\frac{dw}{dz} = \frac{z}{2}$
and $\frac{d^2w}{dz^2} = \frac{1}{2}$.
Then
$$
J(z) = \sum_{n=0}^\infty \frac{(-1)^n}{(n!)^2} w^n
$$
so that
\begin{align*}
   \frac{dJ}{dz}
&= \frac{dJ}{dw}\frac{dw}{dz}
 = \frac{z}{2}\sum_{n=1}^\infty \frac{n (-1)^n}{(n!)^2} w^{n-1} \\
&= \frac{z}{2}
   \sum_{n=1}^\infty
     \frac{n (-1)^n}{(n!)^2}
     \left(\frac{z}{2}\right)^{2(n-1)}
\end{align*}
and
\begin{align*}
   \frac{d^2J}{dz^2}
&= \frac{d^2J}{dw^2}\left(\frac{dw}{dz}\right)^2
 + \frac{dJ}{dw}\frac{d^2w}{dz^2} \\
&= \left(\frac{z}{2}\right)^2
   \sum_{n=2}^\infty
     n (n-1)
     \frac{(-1)^n}{(n!)^2}
     w^{n-2}
  + \frac{1}{2}
    \sum_{n=1}^\infty
      n \frac{(-1)^n}{(n!)^2} w^{n-1} \\
&= \sum_{n=2}^\infty
     n (n-1)
     \frac{(-1)^n}{(n!)^2}
     w^{n-1}
  + \frac{1}{2}
    \sum_{n=1}^\infty
      n \frac{(-1)^n}{(n!)^2} w^{n-1} \\
&= -\frac{1}{2}
 +  \sum_{n=2}^\infty
      n \frac{(-1)^n}{(n!)^2}
      \left(n - 1 + \frac{1}{2}\right)
      w^{n-1} \\
&= -\frac{1}{2}
 +  \sum_{n=2}^\infty
      n \frac{(-1)^n}{(n!)^2}
      \left(n - \frac{1}{2}\right)
      w^{n-1}
\end{align*}
Observing next that
\begin{align*}
   z^2 J(z)
&= 4w
   \sum_{n=0}^\infty
     \frac{(-1)^n}{(n!)^2} w^n \\
&= 4w
 + 4w
   \sum_{n=1}^\infty
     \frac{(-1)^n}{(n!)^2}
     w^n, \\
   z J^\prime(z)
&= \frac{z^2}{2}
   \sum_{n=1}^\infty
     \frac{n(-1)^n}{(n!)^2} w^{n-1} \\
&= 2w
   \sum_{n=1}^\infty
     \frac{n(-1)^n}{(n!)^2} w^{n-1} \\
&= 2
   \sum_{n=1}^\infty
     \frac{n(-1)^n}{(n!)^2} w^{n} \\
&= -2w
 + 2\sum_{n=2}^\infty
     \frac{n(-1)^n}{(n!)^2}
     w^n, \\
   z^2 J^{\prime\prime}(z)
&= -2w
 + 4w
   \sum_{n=2}^\infty
     n \frac{(-1)^n}{(n!)^2}
     \left(n - \frac{1}{2}\right)
     w^{n-1} \\
&= -2w
 + 2
   \sum_{n=2}^\infty
     n \frac{(-1)^n}{(n!)^2}
     (2n-1)
     w^n
\end{align*}
so that
\begin{align*}
   z J^\prime(z) + z^2 J^{\prime\prime}(z)
&= -4w
 + 2
   \sum_{n=2}^\infty
     \frac{n(-1)^n}{(n!)^2}
     w^n(2n - 1 + 1) \\
&= -4w
 + 4
   \sum_{n=2}^\infty
     \frac{n^2 (-1)^n}{(n!)^2}
     w^n \\
&= -4w
 + 4
   \sum_{n=2}^\infty
     \frac{(-1)^n}{((n-1)!)^2}
     w^n \\
&= -4w
 + 4
   \sum_{n=1}^\infty
     \frac{(-1)^{n+1}}{(n!)^2} w^{n+1} \\
&= -4w
 - 4w
   \sum_{n=1}^\infty
     \frac{(-1)^{n+1}}{(n!)^2} w^{n}
\end{align*}
so that
$$
z^2 J(z) + z J^\prime(z) + z^2 J^{\prime\prime}(z)
$$
as desired.

We also have that
\begin{align*}
   \lim \left|\frac{(-1)^n}{(n+1)!)^2}\right|
        \left|\frac{(n!)^2}{(-1)}\right|
&= \lim \frac{n! \cdot n!}{(n+1)! \cdot (n+1)!} \\
&= \lim \frac{1}{(n+1)^2} = 0
\end{align*}
so that $R = \infty$ for the series
$$
\sum_{n=0}^\infty \frac{(-1)^n}{(n!)^2} w^n
$$
and thus for $J(z)$ as well.

\subsection*{Ch. II \S 5.6(a)}
Observe that by a change of index $k = n+1$
\begin{align*}
   \log z
&= \sum_{n=1}^\infty \frac{(-1)^{n-1}}{n}(z-1)^n
 = \sum_{n=0}^\infty \frac{(-1)^n}{n+1} (z-1)^{n+1} \\
&= (z-1)\sum_{n=0}^\infty \frac{(-1)^n}{n+1}
\end{align*}
so that
\begin{align*}
   \frac{d}{dz} \log z
&= \left[\frac{d}{dz} (z-1)\right]
   \sum_{n=0}^\infty \frac{(-1)^n}{n+1} (z-1)^n
 + (z-1)\sum_{n=1}^\infty n\frac{(-1)^n}{n+1} (z - 1)^{n-1} \\
&= \sum_{n=0}^\infty \frac{(-1)^n}{n+1} (z-1)^n
 + \sum_{n=1}^\infty n \frac{(-1)^n}{n+1} (z-1)^n \\
&= 1 + \sum_{n=1}^\infty (1 + n) \frac{(-1)^n}{n+1} (z-1)^n \\
&= 1 + \sum_{n=1}^\infty (-1)^n (z-1)^n \\
&= \sum_{n=0}^\infty (1-z)^n,
\end{align*}
and when $|z - 1| < 1$ this geometric series converges to
$$
\frac{d}{dz} \log z = \frac{1}{1 - (1 - z)} = \frac{1}{z}.
$$

\section{Additional Problems}

\begin{Problem}
  Prove that $\sum_{n=0}^\infty z^n$ does not converge uniformly on
  $D(0, 1)$.
\end{Problem}

\begin{Answer}
  We consider the sequence $(f_n)$ given by
  $$
  f_n(z) = \sum_{k=0}^n z^k,
  $$
  since by definition $\sum_{n=0}^\infty z^n$ is the limit
  of this sequence.

  For any $n$, we can compute the supremum norm
  \begin{align*}
    \left\|
      f_{n+1} - f_n
    \right\|_{D(0,1)}
  &=
    \left\|
      \sum_{k=0}^{n+1} z^k
    - \sum_{k=0}^{n} z^k
    \right\|_{D(0,1)}
   =
  \left\|
    z^{n+1}
  \right\|_{D(0,1)} \\
  &=
  \sup_{D(0,1)} |z^{n+1}|
   =
  \sup_{D(0,1)} |z|^{n+1}
   = 1
  \end{align*}
  and therefore the sequence $(f_n)$ is not uniformly
  Cauchy.
\end{Answer}

\begin{Problem}
  Let $f$ be analytic on an open set $U$. Let
  $V = \{ z \in \mathbb{C} : \bar{z} \in U \}$. Define $g$ on $V$ by
  $g(z) = \overline{f(\bar{z})}$. Show that $g$ is analytic on $V$.
\end{Problem}

\begin{Answer}
For $z_0 \in V$ we have $\bar{z_0} \in U$, so
$f$ is analytic at $\bar{z_0}$, so there is an
$r > 0$ and a sequence $(a_n)$ such that
$$
  f(w)
= \sum_{n=0}^\infty a_n(w - \bar{z_0})^n, \quad
  |w - z_0| < r.
$$

Now let $z \in D(z_0, r)$, and write $z = a + ib$ and
$z_0 = a_0 + ib_0$. Then $|z - z_0| < r$, but we note that
\begin{align*}
  |z - z_0|
&= |(a + ib) - (a_0 + ib_0)|
 = |(a - a_0) + i(b - b_0)| \\
&= \sqrt{(a - a_0)^2 + (b - b_0)^2}
\end{align*}
while
\begin{align*}
  |\bar{z} - \bar{z_0}|
&= |(a - ib) - (a_0 - ib_0)|
 = |(a - a_0) + i(-b + b_0)| \\
&= \sqrt{(a - a_0)^2 + (b_0 - b)^2}
 = \sqrt{(a - a_0)^2 + (b - b_0)^2} \\
&= |z - z_0|,
\end{align*}
so if $z \in D(z_0, r)$ then $\bar{z} \in D(\bar{z_0}, r)$. Therefore
the expression
$$
f(\bar{z})
= \sum_{n=0}^\infty a_n (\bar{z} - \bar{z_0})^n
$$
holds, so
$$
  g(z)
= \overline{f(\bar{z})}
= \sum_{n=0}^\infty \frac{\overline{f^{(n)}(\bar{z_0})}}{n!}
                 \overline{(\bar{z} - \bar{z_0})^n}
= \sum_{n=0}^\infty \frac{\overline{f^{(n)}(\bar{z_0})}}{n!}
                 (z - z_0)^n, \quad
|z - z_0| < r.
$$
Since $z_0$ was chosen arbitrarily
in $V$, this means $g$ is analytic on $V$.
\end{Answer}

\begin{Problem}
  Let $U$ be a domain such that $\forall z \in U$, $\bar{z} \in U$.
  \begin{enumerate}[(i)]
    \item{
      Show that $U \cap \mathbb{R}$ contains an open interval.
    }
    \item{
      Let $f$ be analytic on $U$. Suppose that $f(x) \in \mathbb{R}$
      $\forall x \in U \cap \mathbb{R}$. Show that $f(\bar{z}) = \overline{f(z)}$
      for all $z \in U$.
    }
  \end{enumerate}
\end{Problem}

\begin{Answer}
  \begin{enumerate}
    \item{
      Since $U$ is nonempty, let $z_0 \in U$.

      If $\mathrm{Im}~z_0 = 0$
      then $z_0 \in \mathbb{R}$, and since $U$ is open there is a disk
      $D(z_0, r) \subset U$ for some $r > 0$. Then
      $$
        D(z_0, r) \cap \mathbb{R}
      = \{ z \in \mathbb{R} : |z - z_0| < r \}
      = (z-r, z+r),
      $$
      an open interval in
      $\mathbb{R} \cap D(z_0, r) \subset \mathbb{R} \cap U$.

      Suppose $\mathrm{Im}~z_0 \neq 0$. Without loss of generality,
      let $\mathrm{Im}~z_0 > 0$ so that $z_0$ lies in the upper half-plane
      $\{ z \in \mathbb{C} : \mathrm{Im}~z > 0 \}$. Then
      by assumption $\bar{z_0} \in U$, and
      $\mathrm{Im}~\bar{z_0} = -\mathrm{Im}~\bar{z_0}$ so
      $\bar{z_0} \in \{ z \in \mathbb{C} : \mathrm{Im}~z < 0 \}$.
      Then since $U$ is open and connected it is path-connected, so
      there exists a continuous function $\gamma: [0, 1] \to U$ such
      that $\gamma(0) = z_0$ and $\gamma(1) = \bar{z_0}$. But
      $\mathrm{Im}$ is continuous and thus so is
      $\mathrm{Im} \circ \gamma$, so from the intermediate value
      theorem there must be some $x \in [0, 1]$ such that
      $\Im(\gamma(x)) = 0$. Then since $\gamma(x) \in U$, the case
      where $\mathrm{Im}~z_0 = 0$ can be applied to find an open
      interval $(\gamma(x) - r, \gamma(x) + r)$ for some $r > 0$.
    }
    \item{
      Under our assumption the set
      $$
      V = \{ z \in \mathbb{C} : \bar{z} \in U \}
      $$
      is equal to $U$, and in the previous problem we saw that
      this means $g(z) = \overline{f(\bar{z})}$ is analytic on
      $U$.

      Let $x \in U \cap \mathbb{R}$. Then
      $x = \bar{x}$ and by assumption $f(x) \in \mathbb{R}$,
      so $g(x) = \overline{f(\bar{x})} = f(x)$ on $U \cap \mathbb{R}$.
      But we saw that $U \cap \mathbb{R}$ contains an open
      interval $(x_0 - r, x_0 + r)$ for some $x_0 \in \mathbb{R}$,
      $r > 0$, and certainly either of these endpoints is an
      accumulation point of $U \cap \mathbb{R}$ since
      the sequence $x_0 - r + \frac{1}{n} \to x_0 - r$.
      From the uniqueness theorem for analytic functions this means
      that $f(z) = g(z)$ in $U$, i.e. $f(z) = \overline{f(\bar{z})}$ so
      $\overline{f(z)} = f(\bar{z})$ on $U$ as desired.
    }
  \end{enumerate}
\end{Answer}

\begin{Problem}
  Show that there does not exist a function which is analytic on
  $\mathbb{C}$ that satisfies
  $f\left(\frac{1}{n}\right) = \left|\frac{1}{n^3}\right|$ for all
  $n \in \mathbb{Z} \backslash \{ 0 \}$.
\end{Problem}

\begin{Answer}
  The function $g(z) = z^3$ is a polynomial and therefore analytic
  on $\mathbb{C}$. But $g(z) = f(z)$ for any $z = \frac{1}{n}$ where
  $n \in \mathbb{N} \subset \mathbb{Z} \backslash \{0\},$, and the set
  $$
  \left\{\frac{1}{n} : n \in \mathbb{N} \right\}
  $$
  has an accumulation point at $0 \in \mathbb{C}$. From the uniqueness theorem
  this means that $f = g$ in $\mathbb{C}$, which is a contradiction
  since $f(-1) = -1 \neq 1 = g(-1)$ by definition.
\end{Answer}

\end{document}
