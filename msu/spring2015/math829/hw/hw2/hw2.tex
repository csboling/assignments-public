
\documentclass{article}

\usepackage{amsmath}
\usepackage{amsfonts}
\usepackage{amssymb}
\usepackage{enumerate}
\usepackage[lastexercise]{exercise}


\newcounter{Problem}
\newenvironment{Problem}{\begin{Exercise}[name={Problem},
                                          counter={Problem}]}
                        {\end{Exercise}}
\title{MATH 829 Homework \#2}
\date{January 28, 2015}
\author{Sam Boling}

\begin{document}

\begin{titlepage}
\maketitle
\end{titlepage}

\begin{Problem}
  Prove that for $z \in \mathbb{C}$, the sequence $(z^n)_{n=1}^\infty$
  converges if and only if $|z| < 1$ or $z = 1$.
\end{Problem}

\begin{Answer}
  \begin{itemize}
    \item[($\implies$)]{
      Let $z \in \mathbb{C}$ such that $(z^n)_{n=1}^\infty$ converges.
      Then there exists some $\alpha \in \mathbb{C}$ and some
      $N \in \mathbb{N}$ such that $\forall \varepsilon > 0$,
      $|z^n - \alpha| < \varepsilon$ whenever $n \geq N$.
      But $|z^n - \alpha| \geq |z|^n - |\alpha|$, and if
      $|z| > 1$ then the sequence $(|z|^n)_{n=1}^\infty$ is unbounded,
      so $|z|^n - |\alpha| < \varepsilon$ fails
      for infinitely many $n$.

      If $|z| = 1$ and $z \neq 1$, then we can take $z = e^{i\theta}$
      for some $\theta \in (0, 2 \pi)$. But if
      $z^n \to \alpha$ then we must have
      $\mathrm{Re}(z^n) \to \mathrm{Re}(\alpha)$ and so
      $\cos n\theta \to x$ for some $x \in \mathbb{R}$.
      But the sequence $(\cos n \theta)_{n=0}^\infty$ does
      not converge for any $\theta \in (0, 2 \pi)$:
      Since $\cos x \in [-1, 1]$ for any $x$, if
      the $\cos n\theta \to L$ then $L \in [-1, 1]$,
      so $L = \cos \varphi$ for some
      $\varphi \in [0, 2 \pi)$.
      Suppose
      $|\cos n\theta - \cos \varphi| < \cos(2 \theta -
      \varphi$ for some $n$. Then this means
      $$
      -\cos(2 \theta - \varphi) < \cos n\theta < \cos(2 \theta - \varphi
      $$
      or
      $$
      2 \sin \theta \sin(\varphi - \theta) < \cos n \theta < 2 \cos
      \theta \cos (\theta - \varphi).
      $$
      But then
      \begin{align*}
         |\cos 2 n \theta - \cos \varphi|
      &= |2 \cos^2 n \theta - 1 - \cos \theta| \\
      &> |2 \sin^2 \theta \sin^2(\varphi - \theta) - 1 - \cos \varphi|
      \end{align*}
      which has a positive lower bound, so the sequence diverges.
%     Suppose that $\theta \in (0, \pi]$, so
%     $(n+1)\theta = n\theta + \theta \leq n\theta + \pi$
%     for any $n \in \mathbb{N}$. Then
%     $$
%       f(x)
%     = |\cos x - \cos \varphi|
%     = \left|2 \sin\left(\frac{x - \varphi}{2}\right)
%               \sin\left(\frac{x + \varphi}{2}\right)
%       \right|
%     = 2\left|\sin\left(\frac{x - \varphi}{2}\right)\right|
%        \left|\sin\left(\frac{x + \varphi}{2}\right)\right|
%     $$
%     is increasing on
%     $[\varphi, \varphi + \pi]$. This means
%     $|\cos n\theta - \cos \varphi| <
%      |\cos ((n+1)\theta) - \cos \varphi|$
%     for any $n$, so the sequence must diverge
%     in this case. A symmetric argument applies when
%     $\theta \in [\pi, 2 \pi)$.

%     \begin{itemize}
%       \item{
%         Suppose $\theta / \pi$ is rational. Then
%         $\theta = \frac{k}{l} \pi$ for some
%         $k, l \in \mathbb{Z}$. If $k$ is odd, then
%         $\cos ((l + m) \theta) = -1$ and
%         $\cos ((2l + m) \theta = 1$ for any even $m$,
%         so the sequence does not converge. If $k$ is
%         even, then $\cos ((l + m) \theta) = 1$ and
%         $\cos ((2l + m) \theta) = -1$ for any even
%         $m$, so the sequence does not converge.
%       }
%       \item{
%         Suppose $\theta / \pi$ is irrational.
%       }
%     \end{itemize}

      Therefore $|z| < 1$ or $z = 1$.
    }
    \item[($\impliedby$)]{
      Suppose $z = 1$. Then $|z - 1| = 0$, so
      $\lim_{n \to \infty} z^n = 1$.

      Suppose $|z| < 1$. We have
      $|z^n - 0| = |z^n| = |z|^n$. Since $|z| < 1$
      the real sequence $(|z|^n)_{n=1}^\infty$ has limit
      0, so for any $\varepsilon > 0$ there exists an
      $N \in \mathbb{N}$ such that $||z|^n - 0| = |z|^n = |z^n - 0| < \varepsilon$
      whenever $n \geq N$. Therefore $(z^n)_{n=1}^\infty$ converges.
    }
  \end{itemize}
\end{Answer}

\begin{Problem}
  Let $K$ be a nonempty compact subset of an open set
  $U  \subset \mathbb{C}$. Show that there is $r > 0$
  such that $D(z, r) \subset U$ for any $z \in K$.
  Note that $r$ does not depend on $z \in K$.
\end{Problem}

\begin{Answer}
Observe that since $U$ is open, its complement
$V = \mathbb{C} \backslash U$ is closed. Then
$K$ is a compact set and $V$ is a closed set,
so there exists a $z_0 \in K$ and $w_0 \in V$
such that
$$
  |z_0 - w_0|
= \mathrm{dist}(K, V)
= \inf \{ |z - w| : z \in K, w \in V \}.
$$
Then for any $z \in K$ and any $w \in V$
we have that $|z - w| \geq r$. Since
$K \subset U = \mathbb{C} \backslash V$,
this means $K \cap V = \varnothing$, so it cannot
be the case that $r = \mathrm{dist}(K, V) = 0$.
Therefore $|z - w| \neq 0$ for any $w \in V$,
so $w \notin D(z,r)$, i.e.
$D(z, r) \cap V = \varnothing$, which
means that $D(z, r) \subset U$.
\end{Answer}


\begin{Problem}
  Let $S \subset \mathbb{C}$. We say that $z_0$ is an accumulation
  point of $S$ if for every $r > 0$, the intersection
  $D(z_0, r) \cap S$ is an infinite set.

  Let $U$ be open and $S \subset U$. Suppose that $S$ has no accumulation
  points in $U$. Show that for any compact set $K \subset U$,
  $K \cap S$ is finite.
\end{Problem}

\begin{Answer}
  Suppose $K \cap S$ is infinite, so we may choose a sequence of distinct
  points $(z_n)$ in $K \cap S$. Since $(z_n)$ is a sequence in a compact set
  $K$, it has a convergent subsequence $(z_{n_k})$, where each
  $z_{n_k} \in K \cap S$ and also the limit
  $l = \lim_{k \to \infty} (z_{n_k}) \in K$.
  But by definition of the limit, for any $\varepsilon > 0$
  there is an $N \in \mathbb{N}$ so that
  $|l - z_{n_k}| < \varepsilon$ whenever $k \geq N$.
  Therefore there are infinitely many $k$ such that
  $|l - z_{n_k}| < \varepsilon$, i.e. such that
  $z_{n_k} \in D(l, \varepsilon)$. Since each $z_{n_k} \in K \cap S$,
  $z_{n_k} \in S$, so $z_{n_k} \in S \cap D(l, \varepsilon)$ for
  infinitely many $k$. This means $l$ is an accumulation point of $S$,
  and since $l \in K \subset U$ this is a contradiction.
\end{Answer}

\begin{Problem}
  Let $z_0 \in \mathbb{C}$ and $f(z) = |z - z_0|$. Show that
  $f$ is continuous on $\mathbb{C}$ without using real analysis.
\end{Problem}

\begin{Answer}
  Observe that
  \begin{align*}
          |z - \alpha|
    &=    |z - z_0 + z_0 - \alpha| \\
    &=    |(z - z_0) - (\alpha - z_0)| \\
    &\geq |z - z_0| - |\alpha - z_0|
  \end{align*}
  so $|z - z_0| - |\alpha - z_0| < \delta$ whenever $|z - \alpha| < \delta$.
  Similarly
  \begin{align*}
          |z - \alpha|
    &=    |\alpha - z|
     =    |\alpha - z_0 + z_0 - z| \\
    &=    |(\alpha - z_0) - (z - z_0)| \\
    &\geq |\alpha - z_0| - |z - z_0|
  \end{align*}
  so $|\alpha - z_0| - |z - z_0| < \delta$ whenever $|z - \alpha| < \delta$,
  which means $|z - z_0| - |\alpha - z_0| > -\delta$ in this case. Then
  $$
  -\delta < |z - z_0| - |\alpha - z_0| < \delta
  $$
  or $\left||z - z_0| - |\alpha - z_0|\right| < \delta$, or
  $|f(z) - f(\alpha)| < \delta$ when $|z - \alpha| < \delta$. Choosing
  $\delta = \varepsilon$ yields $\lim_{z \to \alpha} f(z) = f(\alpha)$ for
  every $\alpha \in \mathbb{C}$, so $f$ is continuous on $\mathbb{C}$.
\end{Answer}

\end{document}
