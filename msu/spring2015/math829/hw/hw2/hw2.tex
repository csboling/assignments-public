
\documentclass{article}

\usepackage{amsmath}
\usepackage{amsfonts}
\usepackage{amssymb}
\usepackage{enumerate}
\usepackage{mathtools}
\usepackage{xfrac}
\usepackage[lastexercise]{exercise}

\DeclarePairedDelimiter\floor{\lfloor}{\rfloor}

\newcounter{Problem}
\newenvironment{Problem}{\begin{Exercise}[name={Problem},
                                          counter={Problem}]}
                        {\end{Exercise}}
\title{MATH 829 Homework \#2}
\date{January 28, 2015}
\author{Sam Boling}

\begin{document}

\begin{titlepage}
\maketitle
\end{titlepage}

\section{Textbook Problems}

\section{Additional Problems}

\begin{Problem}
  Prove that for $z \in \mathbb{C}$, the sequence $(z^n)_{n=1}^\infty$
  converges if and only if $|z| < 1$ or $z = 1$.
\end{Problem}

\begin{Answer}
  \begin{itemize}
    \item[($\implies$)]{
      Let $z \in \mathbb{C}$ such that $(z^n)_{n=1}^\infty$ converges.
      Then there exists some $\alpha \in \mathbb{C}$ and some
      $N \in \mathbb{N}$ such that $\forall \varepsilon > 0$,
      $|z^n - \alpha| < \varepsilon$ whenever $n \geq N$.
      But $|z^n - \alpha| \geq |z|^n - |\alpha|$, and if
      $|z| > 1$ then $|z|^n - |\alpha| < \varepsilon$ fails
      for infinitely many $n$. If $|z| = 1$ and $z \neq 1$, then
      $|z^n - \alpha| \geq |z|^n - |\alpha| = 1 - |\alpha|$
      independent of $n$, in which case this sequence is not
      convergent. Therefore $|z| < 1$ or $z = 1$.
    }
    \item[($\impliedby$)]{
      Suppose $z = 1$. Then $|z - 1| = 0$, so
      $\lim_{n \to \infty} z^n = 1$.

      Suppose $|z| < 1$. We have
      $|z^n - 0| = |z^n| = |z|^n$. Since $|z| < 1$
      the real sequence $(|z|^n)_{n=1}^\infty$ has limit
      0, so for any $\varepsilon > 0$ there exists an
      $N \in \mathbb{N}$ such that $||z|^n - 0| = |z|^n = |z^n - 0| < \varepsilon$
      whenever $n \geq N$. Therefore $(z^n)_{n=1}^\infty$ converges.
    }
  \end{itemize}
\end{Answer}

\begin{Problem}
  Let $K$ be a nonempty compact subset of an open set
  $U  \subset \mathbb{C}$. Show that there is $r > 0$
  such that $D(z, r) \subset U$ for any $z \in K$.
  Note that $r$ does not depend on $z \in K$.
\end{Problem}

\begin{Problem}
  Let $S \subset \mathbb{C}$. We say that $z_0$ is an accumulation
  point of $S$ if for every $r > 0$, the intersection
  $D(z_0, r) \cap S$ is an infinite set.

  Let $U$ be open and $S \subset U$. Suppose that $S$ has no accumulation
  points in $U$. Show that for any compact set $K \subset U$,
  $K \cap S$ is finite.
\end{Problem}

\begin{Answer}
  Suppose $K \cap S$ is infinite, so we may choose a sequence of distinct
  points $(z_n)$ in $K \cap S$. Since $(z_n)$ is a sequence in a compact set
  $K$, it has a convergent subsequence $(z_{n_k})$, where each
  $z_{n_k} \in K \cap S$ and also the limit
  $l = \lim_{k \to \infty} (z_{n_k}) \in K$.
  But by definition of the limit, for any $\varepsilon > 0$
  there is an $N \in \mathbb{N}$ so that
  $|l - z_{n_k}| < \varepsilon$ whenever $k \geq N$.
  Therefore there are infinitely many $k$ such that
  $|l - z_{n_k}| < \varepsilon$, i.e. such that
  $z_{n_k} \in D(l, \varepsilon)$. Since each $z_{n_k} \in K \cap S$,
  $z_{n_k} \in S$, so $z_{n_k} \in S \cap D(l, \varepsilon)$ for
  infinitely many $k$. This means $l$ is an accumulation point of $S$,
  and since $l \in K \subset U$ this is a contradiction.
\end{Answer}

\begin{Problem}
  Let $z_0 \in \mathbb{C}$ and $f(z) = |z - z_0|$. Show that
  $f$ is continuous on $\mathbb{C}$ without using real analysis.
\end{Problem}

\begin{Answer}
  Observe that
  \begin{align*}
          |z - \alpha|
    &=    |z - z_0 + z_0 - \alpha| \\
    &=    |(z - z_0) - (\alpha - z_0)| \\
    &\geq |z - z_0| - |\alpha - z_0|
  \end{align*}
  so $|z - z_0| - |\alpha - z_0| < \delta$ whenever $|z - \alpha| < \delta$.
  Similarly
  \begin{align*}
          |z - \alpha|
    &=    |\alpha - z|
     =    |\alpha - z_0 + z_0 - z| \\
    &=    |(\alpha - z_0) - (z - z_0)| \\
    &\geq |\alpha - z_0| - |z - z_0|
  \end{align*}
  so $|\alpha - z_0| - |z - z_0| < \delta$ whenever $|z - \alpha| < \delta$,
  which means $|z - z_0| - |\alpha - z_0| > -\delta$ in this case. Then
  $$
  -\delta < |z - z_0| - |\alpha - z_0| < \delta
  $$
  or $\left||z - z_0| - |\alpha - z_0|\right| < \delta$, or
  $|f(z) - f(\alpha)| < \delta$ when $|z - \alpha| < \delta$. Choosing
  $\delta = \varepsilon$ yields $\lim_{z \to \alpha} f(z) = f(\alpha)$ for
  every $\alpha \in \mathbb{C}$, so $f$ is continuous on $\mathbb{C}$.
\end{Answer}

\end{document}
