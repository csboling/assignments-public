
\documentclass{article}

\usepackage{amsmath}
\usepackage{amsfonts}
\usepackage{amssymb}
\usepackage{enumitem}
\usepackage[lastexercise]{exercise}

\newcommand\dif{\mathop{}\!\mathrm{d}}
\newcommand\horline{\noindent\makebox[\linewidth]{\rule{\textwidth}{0.4pt}}}

\newcounter{Problem}
\newenvironment{Problem}{\begin{Exercise}[name={Problem},
                                          counter={Problem}]}
                        {\end{Exercise}}
\title{MATH 829 Homework \#8}
\date{March 18, 2015}
\author{Sam Boling}

\begin{document}

\begin{titlepage}
\maketitle
\end{titlepage}

\section{Textbook Problems}
\subsection*{Ch. VIII \S 1.7}
\begin{itemize}
  \item[(a)]{
    We have
    $$
    u_x = 6xy, \quad
    u_y = 3x^2 - 3y^2
    $$
    so
    $$
    u_{xx} = 6y, \quad
    u_{yy} = -6y
    $$
    and thus $u_{xx} + u_{yy} = 0$, so $u$ is harmonic. Let $v$
    be a harmonic conjugate of $u$. Then 
    $
    v_x = -u_y = 3y^2 - 3x^2
    $
    so 
    $$
    v = 3xy^2 - x^3 + h(y)
    $$
    for some real differentiable function $h$ and 
    $$
    v_y = u_x = 6xy = 6xy + h^\prime(y),
    $$
    so $h^\prime(y) = 0$ and therefore
    $$
    f = u + i v = (3x^2 y - y^3) + i (3xy^2 - x^3) + C
    $$
    is holomorphic for any $C \in \mathbb{C}$.
  }
  \item[(b)]{
    We have
    $$
    u_{x} = 1 - y, \quad
    u_{y} = -x
    $$
    so
    $$
    u_{xx} = 0, \quad
    u_{yy} = 0
    $$
    and therefore $u$ is harmonic.
    Let $v$ be a harmonic conjugate of $u$.
    Then
    $v_x = -u_y = x$
    so $v = \frac{x^2}{2} + h(y)$ and
    $$
    v_y = u_x = 1 - y = h^\prime(y)
    $$
    so that $h(y) = y - \frac{y^2}{2} + C$ for some $C$. Then
    $$
    f = u + i v = x - xy + i\left(y - \frac{y^2}{2}\right) + C
    $$
    is holomorphic.
  }
  \item[(c)]{
    Observing that $z \bar{z} = |z|^2$,
    $$
      \frac{1}{z} 
    = \frac{\bar{z}}{|z|^2} 
    = \frac{x}{x^2 + y^2} - i\frac{y}{x^2 + y^2}
    $$
    so that
    $$
      \frac{i}{z}
    = \frac{y}{x^2 + y^2} + i\frac{x}{x^2 + y^2}
    $$
    has real part $\frac{y}{x^2 + y^2}$ and is holomorphic on
    $\mathbb{C} \backslash \{ 0 \}$.
  }
  \item[(e)]{
    For $t \in \mathbb{R}$, 
    $$
      \mathrm{Re}~\frac{1}{z - t} 
    = \mathrm{Re}\frac{1}{(x - t) + iy} 
    = \frac{y}{(x - t)^2 + y^2}
    $$
    and $\frac{1}{z - t}$ is analytic except at $z = t$.
  }
\end{itemize}

\section{Additional Problems}
\begin{Problem}
Prove that any positive harmonic function in $\mathbb{R}^2$ is constant.
\end{Problem}

\begin{Answer}
Let $u : \mathbb{R}^2 \to \mathbb{R}$ be harmonic and strictly
positive. Then $u$ has a harmonic conjugate $v$, so that
$f = u + i v$ is holomorphic on $\mathbb{C}$. Then
$e^{-f} = e^{-u} e^{-iv}$ is entire as well, and $|e^{-f}| = e^{-u}$.
But since $-u < 0$, this means $|e^{-f}| \leq 1$ and so $e^{-f}$
is bounded and therefore constant. Thus
$\frac{d}{dz} e^{-f} = -f^\prime e^{-f} \equiv 0$  so $f^\prime \equiv
0$ and thus $f$ is constant. This means $f \equiv x + i y$ for some
fixed $x, y$, so $u = \mathrm{Re} f \equiv x$ and thus $u$ is constant.
\end{Answer}

\begin{Problem}
Let $u$ be a nonconstant harmonic function on $\mathbb{C}$. Show that
for any $c \in \mathbb{R}$, $u^{-1}(\{ c \})$ is unbounded.
Hint: The set $\{ |z| > R \}$ is connected for any $R$.
\end{Problem}

\begin{Answer}
Let $c \in \mathbb{R}$, and suppose by way of contradiction that
$u^{-1}(\{ c \})$ is unbounded. Then there is some $R > 0$ such that
$u^{-1}(\{ c \}) \subset D(0, R)$, so $u(z) \neq c$ for any
$z \in \{ z \in \mathbb{C} : |z| > R \}$. Define
$U = \{ z \in \mathbb{C} : |z| > R \}$.

Let $z_1, z_2 \in U$ and suppose $u(z_1) < c$, $u(z_2) > c$.
Since $U$ is connected, there is a path
$\gamma : [0, 1] \to U$ such that $\gamma(0) = z_1$ and
$\gamma(1) = z_2$. Then
$u \circ \gamma : [0, 1] \to \mathbb{R}$ is a continuous
real function on an interval
with $(u \circ \gamma)(0) < c$ and
$(u \circ \gamma)(1) > 1$, so from the intermediate value
theorem it must be that $u(\gamma(t_0)) = c$ for some
$t_0 \in [0, 1]$ and thus
$u(z_0) = c$ for $z_0 = \gamma(t_0) \in C$, a contradiction.
Therefore it suffices to assume $u(z) < c$
and observe that the argument is symmetric for $u(z) > c$.

Now since $u$ is continuous on $\bar{D}(0, R + \varepsilon)$
for any $\varepsilon > 0$, there is a
point $z_{\max} \in \partial D(0, R + \varepsilon)$ such that
$$
u(z_{\max}) = \max\{ u(z) : z \in D(0, R + \varepsilon) \} \geq c.
$$
But $\partial D(0, R + \varepsilon) \subset U$ so that
$z_{\max} \in U$ and $u(z) < c \leq u(z_{max})$
for all $z \in D(z_{max}, \varepsilon)$, so $u \equiv C$ on 
$U$ for a constant $C \in \mathbb{C}$. Since $u$ is continuous on
$\bar{U} \subset \mathbb{C}$ it must be that $u \equiv C$ on
$\partial U$.
Finally, $\partial U = \partial D(0, R)$, so $u \equiv C$ on
$\partial D(0, R)$ and therefore $u$ is constant on $D(0, R)$.
We conclude that $u \equiv C$ on
$D(0, R) \cup \bar{U} = \mathbb{C}$, a contradiction.
Therefore $u^{-1}(\{ c \})$ is unbounded.
\end{Answer}

\begin{Problem}
Find the winding numbers of the curve $\gamma$ at the given points.
\end{Problem}

\begin{Answer}
\begin{enumerate}
  \item{
    $z_1$ lies in the unbounded component of
    $\mathbb{C} \backslash \gamma$, so $W(\gamma, z_1) = 0$.
  }
  \item{
    A line segment from $z_1$ to $z_2$ crosses the contour from its
    right to its left, so $W(\gamma, z_2) = 1 + W(\gamma, z_1) = 1$.
  }
  \item{
    A line segment from $z_1$ to $z_3$ crosses the contour from its
    left to its right, so $W(\gamma, z_3) = -1 + W(\gamma, z_1) = -1$.
  }
  \item{
    A line segment from $z_3$ to $z_4$ crosses the contour from its
    right to its left, so $W(\gamma, z_4) = 1 + W(\gamma, z_3) = 0$.
  }
  \item{
    A line segment from $z_4$ to $z_5$ crosses the contour from its
    left to its right, so $W(\gamma, z_5) = -1 + W(\gamma, z_4) = -1$.
  }
\end{enumerate}
\end{Answer}

\end{document}
