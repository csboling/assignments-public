
\documentclass{article}

\usepackage{amsmath}
\usepackage{amsfonts}
\usepackage{amssymb}
\usepackage{enumitem}
\usepackage[lastexercise]{exercise}

\newcommand\dif{\mathop{}\!\mathrm{d}}
\newcommand\horline{\noindent\makebox[\linewidth]{\rule{\textwidth}{0.4pt}}}

\newcounter{Problem}
\newenvironment{Problem}{\begin{Exercise}[name={Problem},
                                          counter={Problem}]}
                        {\end{Exercise}}
\title{MATH 829 Homework \#7}
\date{March 4, 2015}
\author{Sam Boling}

\begin{document}

\begin{titlepage}
\maketitle
\end{titlepage}

\section{Textbook Problems}
\subsection*{Ch. III \S 7.3}
Let $f$ be an entire function, and let $\|f\|_R$ be its supremum
norm on the circle of radius $R$. Suppose that there exists a
constant $C$ and a positive integer $k$ such that
$\|f\|_R \leq C R^k$ for arbitrarily large $R$. Show that $f$ is a
polynomial of degree $\leq k$.

\horline

\subsection*{Ch. III \S 6.6}
Let $U$ be a simply connected open set. Let $f$ be analytic on
$U$ and assume that $f(z) \neq 0$ for all $z \in U$. Show that there
exists an analytic function $g$ on $U$ such that $g^2 = f$. Does this
last assertion remain true if 2 is replaced by an arbitrary positive
integer $n$?

\horline



\subsection*{Ch. III \S 1.2}
Let $U$ be a bounded open connected set, $\{f_n\}$ a sequence of
continuous functions on the closure of $U$, analytic on $U$. Assume
that $\{f_n\}$ converges uniformly on the boundary of $U$. Prove that
$\{f_n\}$ converges uniformly on $U$.

\horline

Let $\varepsilon > 0$. 
By assumption, there exists an $N \in \mathbb{N}$ such that
$$
\| f_m - f_n \|_{\partial U} < \varepsilon, \quad
\forall m, n \geq N.
$$
Let $m, n \geq N$. Then since $f_m, f_n$ are analytic, so is their difference, and so from
the maximum modulus principle $|f_m - f_n|$ must attain its maximum on
$\partial U$. Therefore 
\begin{align*}
     \| f_m - f_n \|_U 
&=    \sup_{z \in U} |f_m(z) - f_n(z)|
 \leq \sup_{z \in \partial U} |f_m(z) - f_n(z)| \\
&=    \| f_m - f_n \|_{\partial U}
 <    \varepsilon.
\end{align*}
Therefore $\{ f_n \}$ converges uniformly to $f$ on $U$.

\subsection*{Ch. III \S 1.3}
Let $a_1, \dots, a_n$ be points on the unit circle. Prove that there
exists a point $z$ on the unit circle so that the product of the 
distances from $z$ to $a_j$ is at least 1. 

\horline

Define the function $f : D(0, 1) \to \mathbb{R}$ by
$$
f(z) = \prod_{k=1}^n (z - a_k).
$$
Note that $f$ is a polynomial and thus analytic, and that
$$
|f(0)| = \prod_{k=1}^n |a_k| = 1
$$
since each $a_k$ lies on the unit circle and thus $|a_k| = 1$.

Since $\bar{D(0, 1)}$ is compact, $f$ must have a maximum
on $\bar{D(0, 1)}$, and in particular a point $z_0 \in \bar{D(0,1)}$
such that $|f(z_0)| \geq |f(0)| = 1$. But from the maximum modulus
principle we must have that such a maximum
$z_0$ lies in $\partial D(0, 1) = \{ |z| = 1 \}$. This means that
$$
1 \leq |f(z_0)| = \prod_{k=1}^n |z - a_k|
$$
as desired.

\section{Additional Problems}
\begin{Problem}
Find the radius of
$$
\sum_{n=0}^\infty \frac{\tanh^{(n)}(0)}{n!} z^n.
$$
Justify your answer.
\end{Problem}

\begin{Answer}
Observe that
$$
\tanh = \frac{\sinh z}{\cosh z},
$$
and
\begin{align*}
   2\cosh z
&= e^z + e^{-z} = e^x e^{iy} + e^{-x} e^{-iy} \\
&= e^x (\cos y + i \sin y) + e^{-x} (\cos (-y) - i \sin y) \\
&= \cos y (e^x e^{-x}) + i \sin y (e^x - e^{-x} \\
&= 2 \cos y \cosh x + i 2 \sin y \sinh x.
\end{align*}
Suppose $\cos (x + iy) = 0$. Then either
$\sinh x = 0$ or $\sin y = 0$. 

If $\sinh x = 0$ then $x = 0$, so $\cosh x = 1$, 
and then we must have
$\cos y = 0$ so that 
$y = \pi k + \frac{\pi}{2}$ for some $k \in \mathbb{Z}$.

If $\sin y = 0$ then $y = \pi k$ for some $k \in \mathbb{Z}$,
so $\cos y = \pm 1$ and then we must have $\cosh x = 0$, which is
not possible.

Therefore the zeros of $\cosh z$ are all of the form
$i \left( \pi k + \frac{\pi}{2} \right)$ for some $k \in \mathbb{Z}$.
Since $\sinh z$ and $\cosh z$ are entire, it follows that
$\tanh z$ is holomorphic on the open disk $D\left(0, \frac{\pi}{2}\right)$
and not holomorphic on $D(0, r)$ for any $r > \frac{\pi}{2}$.
Therefore
$$
\tanh z = \sum_{n=0}^\infty \frac{\tanh^{(n)}(0)}{n!} z^n, \quad
z \in D\left(0, \frac{\pi}{2}\right)
$$
and the series converges with radius $\frac{\pi}{2}$.
\end{Answer}

\begin{Problem}
Let $f$ be an entire function. Suppose there exists $r > 0$ such that
$|f(z)| \geq r$ for all $z \in \mathbb{C}$. Prove that $f$ is constant.
\end{Problem}

\begin{Answer}
Since $|f(z)| > r > 0$, $|f(z)|$ is nonzero on $\mathbb{C}$ and thus
$\frac{1}{f}$ is also entire. Then we have
$\left|\frac{1}{f(z)}\right| < \frac{1}{r}$ since $|f(z)|, r > 0$.
This means $\frac{1}{f}$ is a bounded entire function and therefore
$\frac{1}{f} \equiv C$ for some $C \in \mathbb{C} - \{ 0 \}$.
Note that it cannot be that $C = 0$ for in this case we have 
$1 = C f(z) = 0$. Therefore $f \equiv \frac{1}{C}$ as desired.
\end{Answer}

\begin{Problem}
Let $\alpha \in \mathbb{C}$. The function $(1 + z)^\alpha$ in
$D(0, 1)$ is defined as $e^{\alpha L(z)}$, where $L$ is a branch of
$\log (1 + z)$ in $D(0, 1)$ that satisfies $L(0) = 0$.
Prove that $(1 + z)^\alpha$ is analytic on $D(0, 1)$ and has
power series expansion
$$
(1 + z)^\alpha = \sum_{n=0}^\infty {\alpha \choose n} z^n, z \in D(0, 1),
$$
where
$$
{\alpha \choose n} =
\left\{\begin{array}{l l}\
  1,                                                     & \quad n = 0 \\
  \frac{\alpha(\alpha - 1) \cdots (\alpha - n + 1)}{n!}, & \quad n \geq 1
\end{array}\right.
$$
\end{Problem}

\begin{Answer}
We have
$$
   e^{x + iy} \\
 = e^x \cos y + i e^x \sin y
$$
so the Jacobian is given by
\begin{align*}
   \left.\mathbf{J}(\exp)\right|_{x + iy}
&= \left[
     \begin{array}{r r}
        e^x \cos y
     & -e^x \sin y \\
        e^x \sin y
     &  e^x \cos y
     \end{array}
   \right].
\end{align*}
Since 
is holomorphic in 
$D(0, 1)$, it has the Jacobian
\begin{align*}
   \mathbf{J}(\alpha L)|_{x + iy}
&= \alpha
   \left[
     \begin{array}{c c}
       \frac{\partial u}{\partial x}
     & \frac{\partial u}{\partial y} \\
       \frac{\partial v}{\partial x}
     & \frac{\partial v}{\partial y}
     \end{array}
   \right]
 = \alpha
   \left[
     \begin{array}{r r}
        \frac{\partial u}{\partial x}
     & -\frac{\partial v}{\partial y} \\
        \frac{\partial v}{\partial y} 
     &  \frac{\partial u}{\partial x}
     \end{array}
   \right].
\end{align*}
Then we have
\begin{align*}
   \mathbf{J}(\exp \circ \alpha L)|_{x + iy}
&= \left[
     \begin{array}{r r}
       \frac{\partial}{\partial x} 
       \mathrm{Re}(e^{\alpha L(z)})
     & \frac{\partial}{\partial y}
       \mathrm{Re}(e^{\alpha L(z)}) \\
       \frac{\partial}{\partial x}
       \mathrm{Im}(e^{\alpha L(z)})
     & \frac{\partial}{\partial y}
       \mathrm{Im}(e^{\alpha L(z)})
     \end{array}
   \right] \\
&= \mathbf{J}(\exp)|_{\alpha L(x + iy)}
   \mathbf{J}(\alpha L)|_{x + iy} \\
&= 
   \left[
     \begin{array}{r r}
        \alpha e^{\alpha u} \cos (\alpha v)
     & -\alpha e^{\alpha u} \sin (\alpha v) \\
        \alpha e^{\alpha u} \sin (\alpha v)
     &  \alpha e^{\alpha u} \cos (\alpha v)
     \end{array}
   \right]
   \left[
     \begin{array}{r r}
        \frac{\partial u}{\partial x}
     & -\frac{\partial v}{\partial y} \\
        \frac{\partial v}{\partial y} 
     &  \frac{\partial u}{\partial x}
     \end{array}
   \right] \\
&= \alpha e^{\alpha u}
   \left[
     \begin{array}{r r}
         u_x \cos(\alpha v) 
       - v_y \sin(\alpha v)
     & - v_y \cos(\alpha v)
       - u_x \sin(\alpha v) \\
         u_x \sin(\alpha v)
       + v_y \cos(\alpha v)
     & - v_y \sin(\alpha v)
       + u_x \cos(\alpha v)
     \end{array}
   \right]
\end{align*}
which is skew-symmetric, verifying that the Cauchy-Riemann
equations
$$
\left[
  \begin{array}{r r}
    \frac{\partial}{\partial x} 
 & -\frac{\partial}{\partial y} \\
    \frac{\partial}{\partial y}
 &  \frac{\partial}{\partial x}
  \end{array}
\right]
\left[
  \begin{array}{c}
    \mathrm{Re}(e^{\alpha L(z)}) \\
    \mathrm{Im}(e^{\alpha L(z)})
  \end{array}
\right]
=
\left[
  \begin{array}{c}
    0 \\ 0
  \end{array}
\right]
$$
are satisfied. Therefore $(1 + z)^\alpha$ is holomorphic and thus
analytic wherever $L(z)$ is.

Define $f(z) = e^{\alpha L(z)}$. We show that
$$
f^{(n)}(z) = n! {\alpha \choose n} \frac{e^{\alpha L(z)}}{(1 + z)^n}.
$$
for $n \in \mathbb{N} \cup \{0\}$, from which the claim follows by
Cauchy's formula taking $w_0 = 0$.

First we have that
$$
  f^{(0)}(z) 
= e^{\alpha L(z)}
= 0! {\alpha \choose 0} e^{\alpha L(z)}.
$$

Next, suppose 
$$
f^{(k)}(z) = k! {\alpha \choose k} \frac{e^{\alpha L(z)}}{(1 + z)^k}
$$
for some $k \in \mathbb{N} \cup \{ 0 \}$. Then
\begin{align*}
   f^{(k+1)}(z)
&= k! {\alpha \choose k}
   \frac{d}{dz}
   \frac{e^{\alpha L(z)}}{(1 + z)^k} \\
&= k! {\alpha \choose k}
   \frac{ (1 + z)^k \alpha L^\prime(z) e^{\alpha L(z)}
        - e^{\alpha L(z)} \frac{d}{dz} (1 + z)^k     
   }
   {(1 + z)^{2k}} \\
&= k! {\alpha \choose k}
   e^{\alpha L(z)}
   \frac{(1 + z)^k \alpha L^{\prime}(z) - k (1 + z)^{k-1}}
        {(1 + z)^{2k}} \\
&= k! {\alpha \choose k}
   e^{\alpha L(z)}
   \frac{(1 + z) \alpha L^\prime(z) - k}{(1 + z)^{k + 1}}.
\end{align*}
But $L(z)$ is a branch of $\log (1 + z)$, so $L^\prime(z) = \frac{1}{1 + z}$.
Therefore
$$
  f^{(k+1)}(z) 
= e^{\alpha L(z)} k!
  {\alpha \choose k}
  \frac{\alpha - k}{(1 + z)^{k+1}}
= (k + 1)! {\alpha \choose k + 1} \frac{e^{\alpha L(z)}}{(1 + z)^{k+1}},
$$
since
$$
  {\alpha \choose k + 1} 
= \frac{\alpha - k}{k+1} {\alpha \choose k}.
$$

Since $L(0) = 0$ by assumption, we can evaluate
$$
  f^{(n)}(0) 
= n! {\alpha \choose n} e^{\alpha L(0)}
= n! {\alpha \choose n}
$$
so that from Cauchy's formula
$$
  f(z)
= \sum_{n=0}^\infty \frac{f^{(n)}}{n!} z^n
= \sum_{n=0}^\infty {\alpha \choose n} z^n.
$$

\end{Answer}

\begin{Problem}
Let $g$ be holomorphic on a simply connected domain $U$. Show that there is a
holomorphic function $f$ on $U$ such that $g = \frac{f^\prime}{f}$.
\end{Problem}

\begin{Problem}
Find all real-valued $C^2$ functions $h$ on $(0, \infty)$ such that
$h(x^2 + y^2)$ is harmonic on $\mathbb{C} \backslash \{ 0 \}$.
\end{Problem}

\begin{Answer}
We have
$$
\frac{\partial}{\partial x} h(x^2 + y^2) = 2x h^\prime(x^2 + y^2), \quad
\frac{\partial}{\partial y} h(x^2 + y^2) = 2y h^\prime(x^2 + y^2)
$$
so that
$$
  \frac{\partial^2}{\partial x^2} h(x^2 + y^2) 
= 2 h^\prime(x^2 + y^2) + (2x)^2 h^{\prime\prime}(x^2 + y^2), \quad
  \frac{\partial^2}{\partial y^2} h(x^2 + y^2)
= 2 h^\prime(x^2 + y^2) + (2y)^2 h^{\prime\prime}(x^2 + y^2)
$$
and therefore $\Delta f = 0$ requires
$$
  4 h^\prime (x^2 + y^2) 
+ 4x^2 h^{\prime\prime}(x^2 + y^2) 
+ 4y^2 h^{\prime\prime}(x^2 + y^2) = 0
$$
or taking $t = x^2 + y^2$,
$$
h^\prime(t) + t h^{\prime\prime}(t) = 0.
$$

Since 
$x + i y \in \mathbb{C} \backslash \{ 0 \}$ we have
$t = |x + iy|^2 \neq 0$,
so this equation may be rewritten
$$
h^{\prime\prime}(t) = -\frac{h^\prime(t)}{t},
$$
which may be solved by taking $k(t) = h^\prime(t)$ so that
we have
$$
k^\prime(t) = -\frac{k(t)}{t}.
$$
Then
$$
\int \frac{\dif k}{k} = -\int \frac{\dif t}{t} 
$$
gives
$$
k(t) = \frac{c_1}{t}
$$
for some $c_1 \in \mathbb{R}$, and then since $k(t) = h^\prime(t)$
this differential equation has solutions of the form
$$
h(t) = c_1 \log t + c_2
$$ 
for some $c_2$, for $t \in (0, \infty)$.
\end{Answer}

\end{document}
