\documentclass{article}

\usepackage{amsmath}
\usepackage{enumitem}

\newcommand\dif{\mathop{}\!\mathrm{d}}

\title{ECE 813 Homework \#2}
\author{Sam Boling}
\date{February 25, 2015}

\begin{document}

\maketitle

\section*{Problem \#4.7}
A capacitor's contribution to the delay
is determined by the resistances that are shared by the path
to that capacitor and the path to the load, so we have
\begin{align*}
\tau_{DB}
&=
    R_1                     C_1
  + R_1                     C_2
  + (R_1 + R_3)             C_3 
  + R_1                     C_4
  + (R_1 + R_3)             C_5 \\
 &+ (R_1 + R_3 + R_6)        C_6
  + (R_1 + R_3)             C_7
  + (R_1 + R_3 + R_6 + R_8) C_8 \\
&= 2.525 \times 10^{-10} ~\mathrm{s}
 \approx 0.25 ~\mathrm{ns}.
\end{align*}

\section*{Problem \#5.15}
\begin{enumerate}[label=(\alph*)]
  \item{
    The relative sizing factor is given by
    $$
      f 
    = \sqrt[N]{\frac{C_L}{C_{g,1}}} 
    = \sqrt[3]{\frac{20 \times 10^{-12}}{10 \times 10^{-15}}}
    = \sqrt[3]{2000}
    \approx 12.6
    $$
    so that stage 2 should be $12.6$ times larger than stage 1 and
    stage 3 should be $12.6^2 \approx 158.75$ times larger than stage 1.
    This achieves a propagation delay of 2.86 ns.
  }
  \item{
    The delay through an optimally sized inverter chain is given by
    $$
    t_p = N t_{p0} \left(1 + \frac{\sqrt[N]{F}}{\gamma}\right).
    $$
    Numerically, we find that $N = 6$ achieves a minimum delay 
    of 1.91 ns.
  }
  \item{
    It is possible to achieve superior propagation delay with twice the
    number of inverters in this case.
    Since the size of inverters in the chain increase geometrically,
    the area consumed by the inverter chain grows very fast with $N$.
    Area is always at a premium in integrated circuits, and furthermore
    adding increasingly large inverters is likely to cost extra power.
  }
  \item{
    Considering gate capacitances only we have that the load capacitance
    of the $k$-th stage is $C_{g,k} = C_i f^k$. Summing
    these load capacitances gives a total capacitance of
    $$
      C_{chain}
    = C_i \sum_{k=1}^{N} f^k
    = C_i \frac{1 - f^{N}}{1 - f}
    = C_i \frac{1 - F}{1 - \sqrt[N]{f}}
    $$
    so that the dynamic power dissipation is
    $$
      P 
    = C_{chain} V_{DD}^2 f_{0 \to 1}
    = C_{chain} V_{DD}^2 f_{clk} \alpha_{0 \to 1}
    $$
    where $f_{clk}$ is the circuit clock frequency and
    $\alpha_{0 \to 1}$ is the activity factor.
    For 3 stages, this yields
    $$
      P
    = 6.25 C_{chain} f_{clk}
    = 6.25 \cdot (1.72 \times 10^{-12}) f_{clk}
    = f_{clk} \cdot 10.77 \times 10^{-11}.
    $$
  }
\end{enumerate}

\section*{Problem \#5.16}
\begin{enumerate}
  \item{
    We have
    \begin{align*}
             P_{av} 
    &=       I_{av} V
     \propto C_{ox} \frac{W}{L} V^3
     \propto \frac{1}{t_{ox}} \frac{W}{L} V^3 \\
    &\propto \frac{1}{1 / S} \frac{W}{1 / S} 
     =       \frac{W}{S^2},
    \end{align*}
    so $W$ must increase by a factor of $S^2$ to keep
    the power constant as $L$ and $t_{ox}$ are decreased
    by a factor of $1 / S$.
  }
  \item{
    We have
    \begin{align*}
              t_p 
    &\propto  \frac{C_L V}{I_{av}}
     \propto  \frac{C_{ox} W L V}{C_{ox} \frac{W}{L} V^2}
     \propto  \frac{L^2}{V} \\
    &\propto  \frac{1}{S^2}
    \end{align*}
    as the new scaling factor for the intrinsic delay,
    just the same as in normal fixed-voltage scaling.
  }
  \item{
    In this case
    \begin{align*}
               P
      &=       I_{av} V
       \propto C_{ox} W V^2
       \propto \frac{1}{t_{ox}} W
       \propto \frac{1}{1 / S} W
    \end{align*}
    so here scaling the width by $\frac{1}{S}$ maintains constant
    average power.
  }
\end{enumerate}

\end{document}
