\documentclass{article}

\usepackage{amsmath}
\usepackage{amsfonts}

\usepackage{commath}
\usepackage{enumerate}

\usepackage{graphicx}

\title{ECE/ME 851 Homework \#4}
\date{October 3, 2014}
\author{Sam Boling \\ PID A48788119}

\begin{document}

\maketitle

\section*{Problem \#1}

\begin{enumerate}[(a)]
  \item{
  The system
  $$
  A = \left[\begin{array}{c c c c}
        0    & 1   & 0 & 0  \\
        3w^2 & 0   & 0 & 2w \\
        0    & 0   & 0 & 1  \\
        0    & -2w & 0 & 0
      \end{array}\right], \quad
  B = \left[\begin{array}{c c}
        0 & 0 \\
        1 & 0 \\
        0 & 0 \\
        0 & 1
      \end{array}\right]
  $$
  has
  \begin{align*}
  A^2 &= 
  \left[\begin{array}{c c c c}
    3w^2 &     0 & 0 & 2w   \\
    0    & - w^2 & 0 &  0   \\
    0    & -2w   & 0 &  0   \\
   -6w^3 &     0 & 0 & -4w^2
  \end{array}\right], \\
  A^3 &=
  \left[\begin{array}{c c c c}
        0 & -w^2 & 0 &     0 \\
    -3w^4 &    0 & 0 & -2w^3 \\
    -6w^3 &    0 & 0 & -4w^2 \\
        0 & 2w^3 & 0 &     0
  \end{array}\right],
  \end{align*}
  and thus has controllability matrix $\mathcal{C}$ given by
  \begin{align*}
  \mathcal{C}
  &= \left[\begin{array}{c c c c}
       B & AB & \cdots & A^{n-1} B
     \end{array}\right] \\
  &= \left[\begin{array}{c c c c c c c c}
       0 & 0 &   1 &  0 & 0    & 2w    & -w^2 & 0     \\
       1 & 0 &   0 & 2w & -w^2 & 0     &    0 & -2w^3 \\
       0 & 0 &   0 &  1 & -2w  & 0     &    0 & -4w^2 \\
       0 & 1 & -2w &  0 & 0    & -4w^2 & 2w^3 & 0 
     \end{array}\right].
  \end{align*}
  This has reduced row-echelon form
  \begin{align*}
    \mathcal{C} & \sim 
    \left[\begin{array}{c c c c c c c c}
      1 & 0 & 0 & 0 & 3w^2 &  0 &    0 &  6w^3 \\
      0 & 1 & 0 & 0 &    0 &  0 &    0 &     0 \\
      0 & 0 & 1 & 0 &    0 & 2w & -w^2 &     0 \\
      0 & 0 & 0 & 1 &  -2w &  0 &    0 & -4w^2
    \end{array}\right]
  \end{align*}
  and so $\mathrm{rank}(\mathcal{C}) = 4$, and the controllability
  matrix has full row rank and therefore the system is controllable
  from $u$.   When $y_1 = x_1$, $y_2 = x_3$, we have
  $$
  C = \left[\begin{array}{c c c c}
        1 & 0 & 0 & 0 \\
        0 & 0 & 1 & 0
      \end{array}\right]
  $$
  so the observability matrix is
  \begin{align*}
    \mathcal{O} &=
    \left[\begin{array}{c}
      C \\ CA \\ CA^2 \\ CA^3
    \end{array}\right] \\
  &= \left[\begin{array}{c c c c}
       1    & 0    & 0 & 0  \\
       0    & 0    & 1 & 0  \\
       0    & 1    & 0 & 0  \\
       0    & 0    & 0 & 1  \\
       3w^2 & 0    & 0 & 2w \\
       0    & -2w  & 0 & 0  \\
       0    & -w^2 & 0 & 0  \\
     -6w^3  & 0    & 0 & -4w^2
     \end{array}\right] \\
  &\sim
    \left[\begin{array}{c c c c}
      1 & 0 & 0 & 0 \\
      0 & 1 & 0 & 0 \\
      0 & 0 & 1 & 0 \\
      0 & 0 & 0 & 1 \\
      0 & 0 & 0 & 0 \\
      0 & 0 & 0 & 0 \\
      0 & 0 & 0 & 0 \\
      0 & 0 & 0 & 0
    \end{array}\right],
  \end{align*}
  so the observability matrix has column rank 4 and thus the system is
  observable from $y$ in this case.
  }
  \item{If the radial thruster $u_1$ fails, then the matrix $B$ is
        instead
        $$
        B = \left[\begin{array}{c c}
              0 & 0 \\ 0 & 0 \\ 0 & 0 \\ 0 & 1
            \end{array}\right]
        $$
        so that the controllability matrix is
        \begin{align*}
        \mathcal{C} &=
          \left[\begin{array}{c c c c c c c c}
            0 & 0 & 0 & 0  & 0 & 2w    & 0 & 0 \\
            0 & 0 & 0 & 2w & 0 & 0     & 0 & -2w^3 \\
            0 & 0 & 0 & 1  & 0 & 0     & 0 & -4w^2 \\
            0 & 1 & 0 & 0  & 0 & -4w^2 & 0 & 0
          \end{array}\right] \\
        &\sim
          \left[\begin{array}{c c c c c c c c}
            0 & 1 & 0 & 0 & 0 & 0 & 0 & 0 \\
            0 & 0 & 0 & 1 & 0 & 0 & 0 & 0 \\
            0 & 0 & 0 & 0 & 0 & 1 & 0 & 0 \\
            0 & 0 & 0 & 0 & 0 & 0 & 0 & 1
          \end{array}\right]
        \end{align*}
        which still has rank 4, so the system is still controllable.
  }
  \item{Ignoring $y_2$ gives the output mapping
        $$
        C = \left[\begin{array}{c c c c}
            1 & 0 & 0 & 0 \\ 0 & 0 & 0 & 0
            \end{array}\right]
        $$
        and the observability matrix
        \begin{align*}
          \mathcal{O} &=
          \left[\begin{array}{c c c c}
            1    & 0    & 0 & 0 \\
            0    & 0    & 0 & 0 \\
            0    & 1    & 0 & 0 \\
            3w^2 & 0    & 0 & 2w \\
            0    & 0    & 0 & 0  \\
            0    & -w^2 & 0 & 0  \\
            0    & 0    & 0 & 0
          \end{array}\right] \\
        &\sim
          \left[\begin{array}{c c c c}
            1 & 0 & 0 & 0 \\
            0 & 1 & 0 & 0 \\
            0 & 0 & 0 & 1 \\
            0 & 0 & 0 & 0 \\
            0 & 0 & 0 & 0 \\
            0 & 0 & 0 & 0 \\
            0 & 0 & 0 & 0 \\
            0 & 0 & 0 & 0 \\
          \end{array}\right]
        \end{align*}
        which has rank 3. Therefore the system is not controllable
        from $y_1$ only.

        Ignoring $y_1$ gives the output mapping
        $$
        C = \left[\begin{array}{c c c c}
              0 & 0 & 0 & 0 \\ 0 & 0 & 1 & 0
            \end{array}\right]
        $$
        and the observability matrix
        \begin{align*}
          \mathcal{O} &=
          \left[\begin{array}{c c c c}
                0 &   0 & 0 &    0 \\
                0 &   0 & 1 &    0 \\
                0 &   0 & 0 &    0 \\
                0 &   0 & 0 &    1 \\
                0 &   0 & 0 &    0 \\
                0 & -2w & 0 &    0 \\
                0 &   0 & 0 &    0 \\
            -6w^3 &   0 & 0 & -4w^2
          \end{array}\right] \\
        &\sim
          \left[\begin{array}{c c c c}
            1 & 0 & 0 & 0 \\
            0 & 1 & 0 & 0 \\
            0 & 0 & 1 & 0 \\
            0 & 0 & 0 & 1 \\
            0 & 0 & 0 & 0 \\
            0 & 0 & 0 & 0 \\
            0 & 0 & 0 & 0 \\
            0 & 0 & 0 & 0
          \end{array}\right],
        \end{align*}
        which has rank 4. Therefore the system is controllable from
        $y_2$ only.
      }
\end{enumerate}

\section*{Problem \#2}
\begin{enumerate}[(a)]
  \item{
  The given state equation
  $$
  \left[\begin{array}{c}
    \dot{x}_1 \\ \dot{x}_2
  \end{array}\right]
= \left[\begin{array}{r r}
    -\frac{1}{2} & 0 \\ 0 & -1
  \end{array}\right]
  \left[\begin{array}{c}
    x_1 \\ x_2
  \end{array}\right]
+ \left[\begin{array}{c}
    \frac{1}{2} \\ 1
  \end{array}\right]
  u
  $$
  with
  $$
  x(0) = \left[\begin{array}{c}
    a \\ b
  \end{array}\right]
  $$
  has controllability matrix
  \begin{align*}
    \mathcal{C} &=
    \left[\begin{array}{c c}
      B & AB
    \end{array}\right] \\
  &=
    \left[\begin{array}{c c}
      \frac{1}{2} & -\frac{1}{4} \\
      1           & -1
    \end{array}\right] \\
  \end{align*}
  which has rank 2, so the system is controllable
  and thus can be driven to the origin in time $T$
  by the control law
  \begin{align*}
    u(t) &=  B^\top e^{A^\top(T - t)} W_R^{-1}(0, T)
             [x_f - e^{AT} x(0)] \\
         &= -B^\top e^{A^\top(T - t)} W_R^{-1}(0, T) e^{AT} x(0)
  \end{align*}
  where $W_R(0, T)$ is the reachability Gramian.

   We first compute
   \begin{align*}
   e^{At} &= \mathcal{L}^{-1}\{(sI - A)^{-1}\} \\
         &= \mathcal{L}^{-1}\left\{
             \left[\begin{array}{c c}
               s + \frac{1}{2} & 0 \\
               0               & s + 1
             \end{array}\right]^{-1}\right\} \\
         &= \mathcal{L}^{-1}\left\{
             \left[\begin{array}{c c}
               \frac{1}{s + \frac{1}{2}} & 0 \\
               0                         & \frac{1}{s + 1}
             \end{array}\right]\right\} \\
        &= \left[\begin{array}{c c}
              e^{-\frac{t}{2}} & 0 \\
              0             & e^{-t}
            \end{array}\right]
   \end{align*}
   and note that $A = A^\top$ so that $e^{At} = e^{A^\top
     t}$. Furthermore
   $$
   B B^\top = \left[\begin{array}{c c}
               \frac{1}{4} & \frac{1}{2} \\
               \frac{1}{2} & 1
             \end{array}\right]
   $$
   so that
  \begin{align*}
    W_R(0, T) 
  &=
    \int_0^T e^{A(T - \tau)} B B^\top e^{A^\top (T - \tau)} ~d\tau \\
  &= \int_0^T 
     \left[\begin{array}{c c}
       e^{-\frac{(T - \tau)}{2}} & 0 \\ 0 & e^{-(T - \tau)}
     \end{array}\right]
     \left[\begin{array}{c c}
       \frac{1}{4} & \frac{1}{2} \\
       \frac{1}{2} & 1
     \end{array}\right]
     \left[\begin{array}{c c}
       e^{-\frac{(T - \tau)}{2}} & 0 \\ 0 & e^{-(T - \tau)}
     \end{array}\right] ~d\tau\\
  &= \int_0^T 
     \left[\begin{array}{c c}
       \frac{1}{4} e^{-(T - \tau)} 
     & \frac{1}{2} e^{-\frac{3}{2} (T - \tau)} \\
       \frac{1}{2} e^{-\frac{3}{2} (T - \tau)}
     & e^{-2(T - \tau)}
     \end{array}\right] ~d\tau \\
  &= \left[\begin{array}{c c}
       \frac{1}{4} e^{-T} \int_0^T e^\tau ~d\tau
     & \frac{1}{2} e^{-\frac{3}{2} T} \int_0^T e^{\frac{3}{2} T}
     ~d\tau \\
       \frac{1}{2} e^{-\frac{3}{2} T} \int_0^T e^{\frac{3}{2} T}
     ~d\tau
     & e^{-2T} \int_0^T e^{2\tau} ~d\tau
     \end{array}\right] \\
  &= \left[\begin{array}{c c}
       \frac{1}{4}(1 - e^{-T})
     & \frac{1}{3}(1 - e^{-\frac{3}{2} T}) \\
       \frac{1}{3}(1 - e^{-\frac{3}{2} T})
     & \frac{1}{2}(1 - e^{-\frac{3}{2} T})
     \end{array}\right].
 \end{align*}
 Then we find that
 \begin{align*}
 W_R^{-1}(0, T) &=
 \frac{6e^T}{(e^{T/2} - 1)^3(4e^{T/2} + e^T + 1)}
 \left[\begin{array}{c c}
    6(1 + e^{T/2} + e^T + e^{3T / 2})
 & -4(e^{T/2} + e^T + e^{3T / 2}) \\
   -4(e^{T/2} + e^T + e^{3T / 2})
 &  3(e^T + e^{3T / 2})
 \end{array}\right]
 \end{align*}
 and so
 \begin{align*}
 u(t) &=
 \frac{6e^T}{(e^{T/2} - 1)^3(4e^{T/2} + e^T + 1)}
 \left[\begin{array}{c c}
   \frac{1}{2} & 1
 \end{array}\right]
 \left[\begin{array}{c c}
   e^{-\frac{T - t}{2}} & 0 \\
   0                & e^{-(T - t)}
 \end{array}\right] \\
&\cdot
 \left[\begin{array}{c c}
    6(1 + e^{T/2} + e^T + e^{3T / 2})
 & -4(e^{T/2} + e^T + e^{3T / 2}) \\
   -4(e^{T/2} + e^T + e^{3T / 2})
 &  3(e^T + e^{3T / 2})
 \end{array}\right]
 \left[\begin{array}{c c}
   e^{-\frac{T}{2}} & 0 \\
   0                & e^{-T}
 \end{array}\right]
 \left[\begin{array}{c}
   a \\ b
 \end{array}\right] \\
&= 
 \frac{6e^T}{(e^{T/2} - 1)^3(4e^{T/2} + e^T + 1)}
 \left[\begin{array}{c c}
   \frac{1}{2} e^{-\frac{T - t}{2}}
 & e^{-(T - t)}
 \end{array}\right]
 \left[\begin{array}{c c}
    6(e^{-T/2} + 1 + e^{T/2} + e^T)
 & -4(e^{-T/2} + 1 + e^{T/2}) \\
   -4(e^{-T/2} + 1 + e^{T/2})
 &  3(1 + e^{T/2})
 \end{array}\right]
 \left[\begin{array}{c}
   a \\ b
 \end{array}\right] \\
 &=
 \frac{6e^T}{(e^{T/2} - 1)^3(4e^{T/2} + e^T + 1)}
 \left[\begin{array}{c}
  3e^{t/2}(e^{-T} + e^{-T/2} + 1 + e^{T/2})
 -4e^t   (e^{-T} + e^{-T/2}) \\
 -2e^{t/2}(e^{-3T/2} + e^{-T/2} + 1)
 +3e^t   (e^{-T} + e^{-T/2})
 \end{array}\right]
 \left[\begin{array}{c}
   a \\ b
 \end{array}\right] \\
&=
   \frac{6}{e^{T/2} - 1}
   \frac{a[  3e^{t/2} (1 + e^{T/2} + e^T + e^{3T/2}) 
           - 4e^t (1 + e^{T/2} + e^T)
          ]
        }{4e^{T/2} + e^T + 1} \\
&+ \frac{b[- 2e^{T/2} (e^{-T/2} + e^{3T/2} + e^T)
           + 3e^t    (1 + e^{T/2})
          ]
        }{4e^{T/2} + e^T + 1}.
 \end{align*}
 % &= \int_0^T
%    \left[\begin{array}{c c}
%      e^{\frac{\tau}{4}} 
%    & \frac{1}{2} e^{\frac{3}{2}\tau} \\
%      \frac{1}{2}e^{\frac{3}{2}\tau}
%    & e^{2\tau}
%    \end{array}\right] ~d\tau \\
% &= \left[\begin{array}{c c}
%      4(e^{\frac{T}{4}} - 1)
%    & \frac{1}{3}(e^{\frac{3}{2} T} - 1) \\
%      \frac{1}{3}(e^{\frac{3}{2} T} - 1)
%    & \frac{1}{2}(e^{2T} - 1)
%    \end{array}\right]
  % \begin{align*}
%   u(t) &= B^\top e^{A^\top(T - t)} 
%           W_C^{-1}(0, T)
%           [0 - e^{A T} x(0)] \\
%        &= -B^\top
%           e^{A^\top (T - t)} W_C^{-1}(0,T) e^{A T}
%           x(0).
% \end{align*}
%   \begin{align*}
%     W_C(0,T) &= 
%     \int_0^T e^{A(T - \tau)} B B^\top e^{A^\top(T -  \tau)} ~d\tau \\
%  &= \int_0^T
%     \left[\begin{array}{c c}
%       e^{-\frac{(T - \tau)}{2}} & 0 \\
%       0                    & e^{-(T - \tau)}
%     \end{array}\right]
%     \left[\begin{array}{c c}
%       \frac{1}{4} & \frac{1}{2} \\
%       \frac{1}{2} & 1
%     \end{array}\right]
%    \left[\begin{array}{c c}
%       e^{-\frac{(T - \tau)}{2}} & 0 \\
%       0                    & e^{-(T - \tau)}
%     \end{array}\right]
%     ~d\tau \\
% &= \int_0^T
%    \left[\begin{array}{c c}
%      \frac{1}{4}e^{-(T - \tau)}
%    & \frac{1}{2}e^{-\frac{3}{2}(T - \tau)} \\
%      \frac{1}{2}e^{-\frac{3}{2}(T - \tau)}
%    & e^{-2(T - \tau)}
%    \end{array}\right] ~d\tau \\
% &= \int_0^T
%      \left[\begin{array}{c c}
%        e^{-\frac{T - \tau}{4}} 
%      & \frac{1}{2} e^{-\frac{3}{2}(T - \tau)} \\
%        \frac{1}{2} e^{-\frac{3}{2}(T - \tau)} \\
%      & e^{-2(T - \tau)}
%      \end{array}\right] ~d\tau \\
% &= \left[\begin{array}{c c}
%      \frac{1}{4}(1 - e^{-T})
%    & \frac{1}{3}(1 - e^{-\frac{3}{2}T}) \\
%      \frac{1}{3}(1 - e^{-\frac{3}{2}T})
%    & \frac{1}{2}(1 - e^{-2T})
%    \end{array}\right]
%  \end{align*}
%  and
%  \begin{align*}
%  W_C^{-1(0, T) &=
%
%  \end{align*}
%  Then
%  \begin{align*}
%    u(t) 
% &= -\left[\begin{array}{c c}
%       \frac{1}{2} & 1
%     \end{array}\right]
%     \left[\begin{array}{c c}
%       e^{-\frac{t}{2}} & 0 \\
%       0             & e^{-t}
%     \end{array}\right]
%     \left[\begin{array}{c c}
%       4(1 - e^{-\frac{T}{4}})
%     & \frac{1}{3}(1 - e^{-\frac{3}{2}T}) \\
%       \frac{1}{3}(1 - e^{-\frac{3}{2}T})
%     & \frac{1}{2}(1 - e^{-2T})
%     \end{array}\right]
% \end{align*}
  }
  \item{
    Plots are attached. When the controller is required to reach the
    target over shorter timescales, we observe considerable overshoot
    since a large input is being applied to oppose the system dynamics
    as quickly as possible. For the case $T = 5$, the trajectory
    approaches the target in a much more orderly way, avoiding a large
    deviation away from the target before the system stabilizes.
  }
\end{enumerate}

\clearpage
\begin{figure}[h]
  \includegraphics[width=0.75\textwidth]{problem2-1}
  \caption{Controlled system trajectory for problem 2, $T = 1$.}
\end{figure}
\begin{figure}[h]
  \includegraphics[width=0.75\textwidth]{problem2-2}
  \caption{Controlled system trajectory for problem 2, $T = 2$.}
\end{figure}
\begin{figure}[h]
  \includegraphics[width=0.75\textwidth]{problem2-5}
  \caption{Controlled system trajectory for problem 2, $T = 5$.}
\end{figure}

\pagebreak

\section*{Problem \#3}
From the state equations for this discrete-time system
we find the matrices
$$
A = 
\left[\begin{array}{c c c}
  1 & 1 & 0 \\
  0 & 1 & 0 \\
  0 & 0 & 1
\end{array}\right], \quad
B =
\left[\begin{array}{c}
  0 \\ 1 \\ 1
\end{array}\right], \quad
C =
\left[\begin{array}{c c c}
  1 & 1 & 0 \\
  0 & 1 & 0
\end{array}\right]
$$
so that the controllability matrix is
\begin{align*}
\mathcal{C} &=
\left[\begin{array}{c c c}
  B & A B & A^2 B
\end{array}\right] \\
&=
\left[\begin{array}{c c c}
  0 & 1 & 2 \\
  1 & 1 & 1 \\
  1 & 1 & 1
\end{array}\right]
\end{align*}

\begin{enumerate}
  \item{
    We note that
    $$
    \left[\begin{array}{c}
      3 \\ 2 \\ 2
    \end{array}\right]
    =
    \left[\begin{array}{c c c}
      0 & 1 & 2 \\
      1 & 1 & 1 \\
      1 & 1 & 1
    \end{array}\right]
    \left[\begin{array}{c}
      \alpha \\ \beta \\ \gamma
    \end{array}\right]
    $$
    is satisfied when
    $$
    \beta = 1 - 2\alpha, \quad \gamma = 1 + \alpha
    $$
    so the target $x^1$ can be reached by the input sequence
    \begin{align*}
    u(0) &= 1 + \alpha, \\
    u(1) &= 1 - 2\alpha, \\
    u(2) &= \alpha
    \end{align*}
    for any $\alpha$.
  }
  \item{
    We see that
    \begin{align*}
      \mathcal{C} &= 
      \left[\begin{array}{c c c}
        0 & 1 & 2 \\
        1 & 1 & 1 \\
        1 & 1 & 1
      \end{array}\right]
      \sim
      \left[\begin{array}{c c c}
        0 & 1 & 2 \\
        1 & 1 & 1 \\
        0 & 0 & 0
      \end{array}\right]
      \sim
      \left[\begin{array}{c c c}
        0 & 1 &  2 \\
        1 & 0 & -1 \\
        0 & 0 &  0
      \end{array}\right]
    \end{align*}
    and therefore that the third column is a linear combination of the
    first two. Therefore
    $$
    \mathcal{R}(\mathcal{C}) = \mathrm{span}\left\{
      \left[\begin{array}{c}
        0 \\ 1 \\ 1
      \end{array}\right],
      \left[\begin{array}{c}
        1 \\ 1 \\ 1
      \end{array}\right],
    \right\}
    $$
    so every reachable state $x$ has the form
    $$
    x = 
      \alpha 
      \left[\begin{array}{c}
        0 \\ 1 \\ 1
      \end{array}\right]
    + \beta
      \left[\begin{array}{c}
        1 \\ 1 \\ 1
      \end{array}\right]
    $$
    for some $\alpha$ and $\beta$.
  }
  \item{
    The observability matrix is
    \begin{align*}
      \mathcal{O} &=
        \left[\begin{array}{c}
          C    \\
          CA   \\
          CA^2
        \end{array}\right] \\
      &= 
        \left[\begin{array}{c c c}
          1 & 1 & 0 \\
          0 & 1 & 0 \\
          1 & 2 & 0 \\
          0 & 1 & 0 \\
          1 & 3 & 0 \\
          0 & 1 & 0 
        \end{array}\right] \sim
        \left[\begin{array}{c c c}
          1 & 0 & 0 \\
          0 & 1 & 0 \\
          1 & 2 & 0 \\
          0 & 1 & 0 \\
          1 & 3 & 0 \\
          0 & 1 & 0 
        \end{array}\right] \sim
        \left[\begin{array}{c c c}
          1 & 0 & 0 \\
          0 & 1 & 0 \\
          0 & 0 & 0 \\
          0 & 0 & 0 \\
          0 & 0 & 0 \\
          0 & 0 & 0 
        \end{array}\right]
      \end{align*}
      and therefore has null space given by
      $$
      \mathcal{N}(\mathcal{O}) =
      \mathrm{span}\left\{
        \left[\begin{array}{c}
          0 \\ 0 \\ 1
        \end{array}\right]
      \right\},
      $$
      so any state of the form
      $$
      x = 
      \alpha
      \left[\begin{array}{c}
        0 \\ 0 \\ 1
      \end{array}\right]
      $$
      is unobservable.
  }
\end{enumerate}

\pagebreak
\section*{Problem \#4}
The system has state equation
$$
\left[\begin{array}{c}
\dot{x}_1 \\ \dot{x}_2
\end{array}\right] 
=
  \left[\begin{array}{r r}
    0 &  1 \\
   -2 & -3
  \end{array}\right]
  \left[\begin{array}{c}
    x_1 \\ x_2
  \end{array}\right]
+
  \left[\begin{array}{c}
    0 \\ 1
  \end{array}\right]
  u
$$
and output equation
$$
y = 
\left[\begin{array}{c c}
  1 & 0
\end{array}\right].
$$
\begin{enumerate}
  \item{
    The system has controllability matrix
    \begin{align*}
      \mathcal{C} &= 
      \left[\begin{array}{c c}
        B & AB
      \end{array}\right]
      =
      \left[\begin{array}{r r}
        0 &  1 \\
        1 & -3
      \end{array}\right]
   \end{align*}
   and observability matrix
   \begin{align*}
     \mathcal{O} &=
     \left[\begin{array}{c c}
       C & CA
     \end{array}\right]
     =
     \left[\begin{array}{c c}
       1 & 0 \\ 
       0 & 1
     \end{array}\right]
   \end{align*}
   and so both the controllability and observability matrices are of
   full rank. Therefore the system is controllable and observable.
  }
  \item{
    We first compute that
    \begin{align*}
      e^{At} &= 
      \left[\begin{array}{c c}
        2e^{-t} - e^{-2t}  & e^{-t} - e^{-2t} \\
        2e^{-2t} - 2e^{-t} & 2e^{-2t} - e^{-t}
      \end{array}\right]
    \end{align*}
    so that
    \begin{align*}
      W_r(0,T) &=
      \int_0^T e^{A(T - \tau)} B B^\top e^{A^\top(T - \tau)} ~d\tau \\
      &=
      \int_0^T
      \left[\begin{array}{c c}
        2e^{-(T - \tau)} - e^{-2(T - \tau)} 
      & e^{-(T - \tau)} - e^{-2(T - \tau)} \\
        2e^{-2(T - \tau)} - 2e^{-(T - \tau)} 
      & 2e^{-2(T - \tau)} - e^{-(T - \tau)}
      \end{array}\right] \\
      & \cdot
      \left[\begin{array}{c c}
        2e^{-(T - \tau)} - e^{-2(T - \tau)}  
      & 2e^{-2(T - \tau)} - 2e^{-(T - \tau)} \\
        e^{2(T - \tau)} - e^{-2(T - \tau)}   
      & 2e^{-2(T - \tau)} - e^{-(T - \tau)}
      \end{array}\right] ~d\tau \\
    &=
      \int_0^T
      \left[\begin{array}{c c}
        (e^{-(T - \tau)} - e^{-2(T - \tau)})^2
      & 0 \\
        (e^{-(T - \tau)} - e^{-2(T - \tau)})
        (2e^{-2(T - \tau)} - e^{-(T - \tau)})
      & 0
      \end{array}\right] \\
      & +
      \left[\begin{array}{c c}
        0 
      & (e^{-(T - \tau)} - e^{-2(T - \tau)})
        (2e^{-2(T - \tau)} - e^{-(T - \tau)})  \\
        0 
      & (2e^{-2(T - \tau)} - e^{-(T - \tau)})^2
      \end{array}\right]
       ~d\tau \\
    &=
    \left[\begin{array}{c c}
      \frac{1}{12} 
    - \frac{1}{2} e^{-2T}
    + \frac{2}{3} e^{-3T}
    - \frac{1}{4} e^{-4T}
  & - \frac{2}{3}
    - \frac{1}{2} e^{-2T}
    - \frac{1}{3} e^{-3T}
    + \frac{1}{2} e^{-4T} \\
    - \frac{2}{3}
    - \frac{1}{2} e^{-2T}
    - \frac{1}{3} e^{-3T}
    + \frac{1}{2} e^{-4T}
  &   \frac{5}{6}
    - \frac{1}{2} e^{-2T}
    + \frac{2}{3} e^{-3T}
    -             e^{-4T}
    \end{array}\right]
    \end{align*}
    so that
    $$
    W_r(0,1) \approx
    \left[\begin{array}{r r}
      0.0443 & -0.7418 \\
     -0.7418 &  0.7805
    \end{array}\right]
    $$
    and
    $$
    W_r^{-1}(0,1) \approx
    \left[\begin{array}{r r}
     -1.5137 & -1.4385 \\
     -1.4385 & -0.0859
    \end{array}\right].
    $$
    Then a control input can be computed as
    \begin{align*}
      u(t) &= B^\top e^{A^\top (1 -t)} W_r^{-1}(0,1)[x_f - e^A x_0] \\
      &\approx
      \left[\begin{array}{c c}
        e^{-(1 - t)}   - e^{-2(1 - t)}
      & 2e^{-2(1 - t)} - e^{-(1-t)}
      \end{array}\right] \\
      & \cdot
      \left[\begin{array}{r r}
       -1.5137 & -1.4385 \\
       -1.4385 & -0.0859
      \end{array}\right]
      \left(
        \left[\begin{array}{c}
          1 \\ 1
        \end{array}\right]
      - \left[\begin{array}{r r}
          0.6004 x_{10} &  0.2325 x_{20} \\
         -0.4651 x_{10} & -0.0972 x_{20}
        \end{array}\right]
      \right) \\
      & \approx
    \end{align*}
  }
  \item{
    Plots and Matlab code are attached.
  }
\end{enumerate}

\begin{figure}[h!]
  \includegraphics[width=\textwidth]{problem4}
  \caption{Controlled system trajectory for problem 4.}
\end{figure}

\end{document}
