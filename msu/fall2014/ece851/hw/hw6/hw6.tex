\documentclass{article}

\usepackage{amsmath}
\usepackage{amsfonts}

\usepackage{commath}
\usepackage{enumerate}

\usepackage{graphicx}

\title{ECE/ME 851 Homework \#6}
\date{October 29, 2014}
\author{Sam Boling \\ PID A48788119}

\begin{document}

\maketitle

\section*{Problem \#1}
Given the system
$$
A =
\left[\begin{array}{r r r r}
  0 & 0 &  1 &  0 \\
  3 & 0 & -3 &  1 \\
 -1 & 1 &  4 & -1 \\
  1 & 0 & -1 &  0
\end{array}\right], \quad
B =
\left[\begin{array}{c c}
  0 & 0 \\
  1 & 0 \\
  0 & 1 \\
  0 & 0
\end{array}\right]
$$
we have the controllability matrix
$$
\mathcal{C} =
\left[\begin{array}{c c c c}
B & A B & A^2 B & A^3 B
\end{array}\right] =
\left[\begin{array}{r r r r r r r r}
  0 &  0 &  0 &  1 &  1 &   4 &   4 &  13 \\
  1 &  0 &  0 & -3 & -3 &  -9 &  -9 & -27 \\
  0 &  1 &  1 &  4 &  4 &  13 &  13 &  42 \\
  0 &  0 &  0 & -1 & -1 &  -3 &  -3 &  -9
\end{array}\right]
$$
which has linearly independent columns
$$
\left[\begin{array}{r}
0 \\ 1 \\ 0 \\ 0
\end{array}\right] = b_1,
\left[\begin{array}{r}
0 \\ 0 \\ 1 \\ 0
\end{array}\right] = b_2,
\left[\begin{array}{r}
1 \\ -3 \\ 4 \\ -1
\end{array}\right] = Ab_2,
\left[\begin{array}{r}
4 \\ -9 \\ 13 \\ -3
\end{array}\right] = A^2 b_2
$$
and therefore the controllability indices are
$\mu_1 = 1$, $\mu_2 = 3$. Therefore we have
$$
\bar{\mathcal{C}} =
\left[\begin{array}{r r r r}
0 &  0 &  1 &  4 \\
1 &  0 & -3 & -9 \\
0 &  1 &  4 & 13 \\
0 &  0 & -1 & -3
\end{array}\right]
$$
and
$$
\bar{\mathcal{C}}^{-1} =
\left[\begin{array}{r r r r}
  0 &  1 &  0 & -3 \\
 -1 &  0 &  1 &  3 \\
 -3 &  0 &  0 & -4 \\
  1 &  0 &  0 &  1
\end{array}\right]
$$
so that
$$
q_1 =
\left[\begin{array}{r r r r}
  0 & 1 & 0 & -3
\end{array}\right], \quad
q_2 =
\left[\begin{array}{r r r r}
  1 & 0 & 0 & 1
\end{array}\right].
$$

Then we have the similarity transformation
$$
P =
\left[\begin{array}{c}
  q_1   \\
  q_2   \\
  q_2 A \\
  q_2 A^2
\end{array}\right]
=
\left[\begin{array}{r r r r}
   0 & 1 & 0 & -3 \\
   1 & 0 & 0 &  1 \\
   1 & 0 & 0 &  0 \\
   0 & 0 & 1 &  0
\end{array}\right]
$$
so that
$$
P^{-1} =
\left[\begin{array}{r r r r}
  0 & 0 &  1 & 0 \\
  1 & 3 & -3 & 0 \\
  0 & 0 &  0 & 1 \\
  0 & 1 & -1 & 0
\end{array}\right].
$$
Then the controller forms are
$$
A_C
= P A P^{-1}
= \left[\begin{array}{r r r r}
    0 & 1 & -1 & 0 \\
    0 & 0 &  1 & 0 \\
    0 & 0 &  0 & 1 \\
    1 & 2 & -3 & 4
  \end{array}\right], \quad
B_C
= P B
= \left[\begin{array}{r r}
    1 & 0 \\
    0 & 0 \\
    0 & 0 \\
    0 & 1
  \end{array}\right].
$$

\pagebreak

\section*{Problem \#2}
The system in exercise 3.37 has the state space representation
$$
A =
\left[\begin{array}{r r r r}
  0      &       1 &       0 &       0 \\
 -0.1910 & -0.0536 &  0.0910 &  0.0036 \\
       0 &       0 &       0 &       1 \\
  0.0910 &  0.0036 & -0.1910 & -0.0536
\end{array}\right], \quad
B =
\left[\begin{array}{r r}
  0 &  0 \\
  1 &  0 \\
  0 &  0 \\
  0 & -1
\end{array}\right].
$$
\begin{enumerate}[(a)]
  \item{
    The system has the controllability matrix
    $$
    \mathcal{C} =
    \left[\begin{array}{r r r r r r r r}
         0 &  0 &       1 &       0 & -0.0536 & -0.0036 & -0.1881 & -0.0906 \\
         1 &  0 & -0.0536 & -0.0036 & -0.1881 & -0.0906 &  0.0210 &  0.0111 \\
         0 &  0 &       0 &      -1 &  0.0036 &  0.0536 &  0.0906 &  0.1881 \\
         0 & -1 &  0.0036 &  0.0536 &  0.0906 &  0.1881 & -0.0111 & -0.0210
    \end{array}\right]
    $$
    which is full rank, so the system is controllable from
    $\left[\begin{array}{c} f_1 \\ f_2 \end{array}\right]$.

    The first 4 columns of $\mathcal{C}$ are linearly independent, so
    we have $\mu_1 = \mu_2 = 2$ and therefore
    $$
    \bar{\mathcal{C}} =
    \left[\begin{array}{c c c c}
      b_1 & Ab_1 & b_2 & Ab_2
    \end{array}\right] =
    \left[\begin{array}{r r r r}
      0 &       1 &  0 &       0 \\
      1 & -0.0536 &  0 & -0.0036 \\
      0 &       0 &  0 &      -1 \\
      0 &  0.0036 & -1 &  0.0536
    \end{array}\right]
    $$
    and
    $$
    \bar{\mathcal{C}}^{-1} =
    \left[\begin{array}{r r r r}
    0.0536 & 1 & -0.0036 &  0 \\
         1 & 0 &       0 &  0 \\
    0.0036 & 0 & -0.0536 & -1 \\
         0 & 0 &      -1 &  0
    \end{array}\right].
    $$
    Therefore
    $$
    P =
    \left[\begin{array}{r r r r}
      1 & 0 &  0 &  0 \\
      0 & 1 &  0 &  0 \\
      0 & 0 & -1 &  0 \\
      0 & 0 &  0 & -1
    \end{array}\right] = P^{-1}
    $$
    so
    $$
    A_C = P A P^{-1} =
    \left[\begin{array}{r r r r}
          0 &       1 &       0 &       0 \\
    -0.1910 & -0.0536 & -0.0910 & -0.0036 \\
          0 &       0 &       0 &       1 \\
    -0.0910 & -0.0036 & -0.1910 & -0.0536
    \end{array}\right]
    $$
    and
    $$
    B_C = P B =
    \left[\begin{array}{r r}
      0 & 0 \\
      1 & 0 \\
      0 & 0 \\
      0 & 1
    \end{array}\right].
    $$
  }
  \item{
    When $f_2 = 0$ we have
    $$
    B =
    \left[\begin{array}{r r}
      0 & 0 \\
      1 & 0 \\
      0 & 0 \\
      0 & 0
    \end{array}\right]
    $$
    which results in $\mathrm{rank}(\mathcal{C}) = 4$,
    so the system is controllable from $f_1$ only.

    When $f_1 = 0$ we have
    $$
    B =
    \left[\begin{array}{r r}
      0 &  0 \\
      0 &  0 \\
      0 &  0 \\
      0 & -1
    \end{array}\right]
    $$
    which results in $\mathrm{rank}(\mathcal{C}) = 4$,
    so the system is controllable from $f_2$ only.

    Comparing the matrices $A$ and $B$
    with the system they are modeling, we see that these choices of
    values correspond to the case where $K_1 = K_2$ and $M_1 = M_2$,
    so it makes sense that the system behaves symmetrically in
    response to its inputs.
  }
  \item{
      With the output matrix
      $$
      C = \left[\begin{array}{c c c c}
            1 & 0 & 0 & 0 \\
            0 & 1 & 0 & 0
          \end{array}\right]
      $$
      we have the observability matrix
      $$
      \mathcal{O} =
      \left[\begin{array}{r r r r}
              1 &       0 &       0 &       0 \\
              0 &       1 &       0 &       0 \\
              0 &       1 &       0 &       0 \\
        -0.1910 & -0.0536 &  0.0910 &  0.0036 \\
        -0.1910 & -0.0536 &  0.0910 &  0.0036 \\
         0.0106 & -0.1881 & -0.0056 &  0.0906 \\
         0.0106 & -0.1881 & -0.0056 &  0.0906 \\
         0.0442 &  0.0210 & -0.0344 & -0.0111
      \end{array}\right]
      $$
      This matrix has rank 4, so the system is observable from $y$.
      The first 4 linearly independent rows of this system are
      $$
      \left[\begin{array}{r}
        1 \\ 0 \\ 0 \\ 0
      \end{array}\right] = c_1^\top, \quad
      \left[\begin{array}{r}
        0 \\ 1 \\ 0 \\ 0
      \end{array}\right] = c_2^\top, \quad
      \left[\begin{array}{r}
        -0.1910 \\ -0.0536 \\ 0.0910 \\ 0.0036
      \end{array}\right] = (c_2 A)^\top, \quad
      \left[\begin{array}{r}
        0.0106 \\ -0.1881 \\ -0.0056 \\ 0.0906
      \end{array}\right] = (c_2 A^2)^\top, \quad
      $$
      so we have observability indices $\nu_1 = 1$,
      $\nu_2 = 3$ and
      $$
      \bar{\mathcal{O}} =
      \left[\begin{array}{c}
        c_1     \\
        c_2     \\
        c_2 A   \\
        c_2 A^2
      \end{array}\right] =
      \left[\begin{array}{r r r r}
        1      &       0 &       0 &      0 \\
        0      &       1 &       0 &      0 \\
       -0.1910 & -0.0536 &  0.0910 & 0.0036 \\
        0.0106 & -0.1881 & -0.0056 & 0.0906
      \end{array}\right]
      $$
      so that
      $$
      \bar{\mathcal{O}}^{-1}
      \left[\begin{array}{r r r r}
             1 &      0 &       0 &       0 \\
             0 &      1 &       0 &       0 \\
        2.0984 & 0.5056 & 10.9622 & -0.4356 \\
        0.0127 & 2.1074 &  0.6776 & 11.0106
      \end{array}\right].
      $$
      Then
      $$
      P^{-1} =
      \left[\begin{array}{r r r r}
             1 &       0 &       0 &       0 \\
             0 &       0 &       0 &  1.0001 \\
        2.0984 & -0.4356 & 11.0106 & -0.5070 \\
        0.0127 & 11.0106 & -0.5070 & -2.0759
      \end{array}\right]
      $$
      and
      $$
      P =
      \left[\begin{array}{r r r r}
             1 &      0 &      0 &      0 \\
       -0.0099 & 0.1910 & 0.0042 & 0.0910 \\
       -0.1910 & 0.0536 & 0.0910 & 0.0036 \\
             0 & 0.9999 &      0 &      0
      \end{array}\right],
      $$
      so the observer form is given by
      $$
      A_O = P A P^{-1} =
      \left[\begin{array}{r r r r}
             0 & 0 & 0 &  1.0001 \\
       -0.0282 & 0 & 0 & -0.0199 \\
             0 & 1 & 0 & -0.3849 \\
             0 & 0 & 1 & -0.1072
      \end{array}\right]
      $$
      and
      $$
      C_O = CP^{-1} =
      \left[\begin{array}{r r r r}
        1 & 0 & 0 &      0 \\
        0 & 0 & 0 & 1.0001
      \end{array}\right].
      $$
}
\end{enumerate}

\pagebreak

\section*{Problem \#3}
We have
$$
A =
\left[\begin{array}{r r r r}
  0 & 1 &  0 & 0 \\
  0 & 0 &  1 & 0 \\
  0 & 0 &  0 & 1 \\
  1 & 1 & -3 & 4
\end{array}\right], \quad
B =
\left[\begin{array}{r r}
  1 & 0 \\
  0 & 0 \\
  0 & 0 \\
  0 & 1
\end{array}\right]
$$
and desire eigenvalues at $-1 \pm i$ and $-2 \pm i$.
A suitable state feedback matrix for this eigenvalue placement can be
derived by any of the following means.

\begin{itemize}
\item{
First we notice
that
$$
BF =
\left[\begin{array}{c c c c}
  f_{11} & f_{12} & f_{13} & f_{14} \\
  0     & 0     &       & 0      \\
  0     & 0     &       & 0      \\
  f_{21} & f_{22} & f_{23} & f_{24}
\end{array}\right].
$$
Since $A$ is already in controller form, we observe that we can then
adjust the characteristic polynomial of $A + BF$ arbitrarily by letting
$f_{1i} = 0$ and setting $f_{2i}$ appropriately for
$i \in \{ 1, \dots, 4 \}$. To achieve the desired eigenvalues we see
$$
\alpha_d(s) =
  (s - (-1 + i))
  (s - (-1 - i))
  (s - (-2 + i))
  (s - (-2 - i)) =
  s^4 + 6s^3 + 15s^2 + 18s + 10
$$
so we desire
$$
A + BF =
\left[\begin{array}{r r r r}
  0 & 1 &  0 & 0 \\
  0 & 0 &  1 & 0 \\
  0 & 0 &  0 & 1 \\
  1 & 1 & -3 & 4
\end{array}\right]
+
\left[\begin{array}{r r r r}
0     &     0 &     0 &     0 \\
0     &     0 &     0 &     0 \\
0     &     0 &     0 &     0 \\
f_{21} & f_{22} & f_{23} & f_{24}
\end{array}\right]
\left[\begin{array}{r r r r}
0   &   1 &   0 &  0 \\
0   &   0 &   1 &  0 \\
0   &   0 &   0 &  1 \\
-10 & -18 & -15 & -6
\end{array}\right]
$$
so that $f_{21} = -10 - 1 = -11$, $f_{22} = -18 - 1 = -19$,
$f_{23} = -15 - (-3) = -12$, $f_{24} = -6 - 4 = -10$. Therefore a feedback
matrix can be given by
$$
F =
\left[\begin{array}{r r r r}
   0   &   0 &   0 &   0 \\
   -11 & -19 & -12 & -10
\end{array}\right].
$$
We confirm that $A + BF$ has the desired eigenvalues.
}
\item
{
}
\end{itemize}

\pagebreak

\section*{Problem \#4}
We have
$$
A =
\left[\begin{array}{r r r}
  0 & 1 &  0 \\
  0 & 0 &  1 \\
  1 & 0 & -1
\end{array}\right], \quad
B =
\left[\begin{array}{r}
  0 \\
  0 \\
  1
\end{array}\right], \quad
C =
\left[\begin{array}{r r r}
  1 & 2 &  0
\end{array}\right].
$$
and desire
$$
H(s) = \frac{1}{s^2 + 3s + 2}.
$$

The desired transfer function has poles when
$$
0 = s^2 + 3s + 2 = (s + 2)(s + 1),
$$
or at $-2$ and $-1$. The existing $A$ matrix is in controller form and
has characteristic polynomial $s^3 + s^2 - 1$, and furthermore this
system has the transfer function
$$
H(s) = C (sI - A)^{-1} B =
\frac{2s + 1}{s^3 + s^2 - 1}
$$
and therefore has a zero at $s = -\frac{1}{2}$. By changing the
characteristic polynomial to
$$
(2s + 1)(s + 2)(s + 1) = 2s^3 + 7s^2 + 7s + 2
$$
we can adjust the transfer function to have the desired frequency
response. We therefore choose $F$ to satisfy
$$
A + BF =
\left[\begin{array}{r r r}
   0 &            1 &            0 \\
   0 &            0 &            1 \\
  -1 & -\frac{7}{2} & -\frac{7}{2}
\end{array}\right]
=
\left[\begin{array}{r r r}
  0 & 1 &  0 \\
  0 & 0 &  1 \\
  1 & 0 & -1
\end{array}\right]
+
\left[\begin{array}{r r r}
  0   &   0 &  0 \\
  0   &   0 &  0 \\
  f_1 & f_2 & f_3
\end{array}\right]
$$
so we choose
$$
F =
\left[\begin{array}{r r r}
  -2 & -\frac{7}{2} & -\frac{5}{2}
\end{array}\right].
$$
The resulting transfer function is
$$
H(s) = \frac{4s + 2}{2s^3 + 7s^2 + 7s + 2}
     = \frac{2}{s^2 + 3s + 2}.
$$

This feedback strategy produces the desired frequency behavior, but
cannot eliminate the DC gain in the system,
which exists because the numerator $(2s + 1) = 2(s + \frac{1}{2})$ in the open-loop
transfer function has a DC gain of 2. Feedback of this kind is
unaffected by the output transformation, which can introduce gain of
this kind.

\end{document}
