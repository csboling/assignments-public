\documentclass{article}

\usepackage{amsmath}
\usepackage{amsfonts}

\usepackage{commath}
\usepackage{enumerate}

\usepackage{graphicx}

\title{ECE/ME 851 Homework \#10}
\date{November 12, 2014}
\author{Sam Boling \\ PID A48788119}

\begin{document}

\maketitle

\section*{Problem \#1}
The desired transfer function is given by
$$
H(s) =
\left[\begin{array}{c c}
  \frac{1}{s}   & \frac{s+3}{s+1} \\
  \frac{1}{s+3} & \frac{s}{s+1}
\end{array}\right].
$$
\begin{enumerate}
  \item{
    The denominator polynomial is given by
    $$
    d(s) = s(s+3)(s+1)
    $$
    which has distinct real roots, so taking
    $$
    \bar{H}(s) = H(s) - H(\infty) =
    \left[\begin{array}{c c}
      \frac{1}{s}   & \frac{s+3}{s+1} - 1 \\
      \frac{1}{s+3} & \frac{s}{s+1} - 1
    \end{array}\right] =
    \left[\begin{array}{c c}
      \frac{1}{s}   &  \frac{2}{s+1} \\
      \frac{1}{s+3} & -\frac{1}{s+1}
    \end{array}\right]
    $$
    we have
    \begin{align*}
      K_1 = \lim_{s \to 0} s \bar{H}(s) &=
      \left[\begin{array}{c c}
        1 & 0 \\
        0 & 0
      \end{array}\right] =
      \left[\begin{array}{c}
        1 \\
        0
      \end{array}\right]
      \left[\begin{array}{c c}
        1 & 0
      \end{array}\right], \\
      K_2 = \lim_{s \to -1} (s+1)\bar{H}(s) &=
      \left[\begin{array}{r r}
        0 &  2 \\
        0 & -1
      \end{array}\right] =
      \left[\begin{array}{r}
         2 \\
        -1
      \end{array}\right]
      \left[\begin{array}{c c}
        0 & 1
      \end{array}\right], \\
      K_3 = \lim_{s \to -3} (s+3)\bar{H}(s) &=
      \left[\begin{array}{c c}
        0 & 0 \\
        1 & 0
      \end{array}\right] =
      \left[\begin{array}{c c}
        0 \\
        1
      \end{array}\right]
      \left[\begin{array}{c c}
        1 & 0
      \end{array}\right].
    \end{align*}
    Since $\rho_1 = \rho_2 = \rho_3 = 1$, we then have
    \begin{align*}
      A &=
      \left[\begin{array}{r r r}
        0 &  0 &  0 \\
        0 & -1 &  0 \\
        0 &  0 & -3
      \end{array}\right], \\
      B &=
      \left[\begin{array}{c c}
        1 & 0 \\
        0 & 1 \\
        1 & 0
      \end{array}\right], \\
      C &=
      \left[\begin{array}{r r r}
        1 &  2 & 0 \\
        0 & -1 & 1
      \end{array}\right], \\
      D &=
      \left[\begin{array}{c c}
        0 & 1 \\
        0 & 1
      \end{array}\right].
    \end{align*}
  }
  \item{
    The denominators are already factored, so we have
    $$
    H(s) - H(\infty) =
    \left[\begin{array}{c c}
      \frac{1}{s}   & \frac{s+3}{s+1} - 1 \\
      \frac{1}{s+3} & \frac{s}{s+1} - 1
    \end{array}\right] =
    \left[\begin{array}{c c}
      \frac{1}{s}   &  \frac{2}{s+1} \\
      \frac{1}{s+3} & -\frac{1}{s+1}
    \end{array}\right].
    $$
    Next we find the denominator polynomials
    $$
    d_1(s) = s(s+3) = s^2 + 3s, \quad
    d_2(s) = s + 1
    $$
    so that
    $$
    D(s) =
    \left[\begin{array}{c c}
      s^2 & 0 \\
      0   & s
    \end{array}\right]
    +
    \left[\begin{array}{c c c}
      0 & 3 & 0 \\
      0 & 0 & 1
    \end{array}\right]
    \left[\begin{array}{c c}
      1 & 0 \\
      s & 0 \\
      0 & 1
    \end{array}\right] =
    \left[\begin{array}{c c}
      s^2 + 3s & 0 \\
      0        & s + 1
    \end{array}\right]
    $$
    Then
    \begin{align*}
    N(s) = H(s)D(s) &=
    \left[\begin{array}{c c}
      s+3 & s+3 \\
      s   & s
    \end{array}\right], \\
    H(\infty)D(s) &=
    \left[\begin{array}{c c}
      0 & s+1 \\
      0 & s+1
    \end{array}\right], \\
    N(s) - H(\infty)D(s) &=
    \left[\begin{array}{c r}
      s+3 &  2 \\
      s   & -1 \\
    \end{array}\right] =
    \left[\begin{array}{r r r}
      3 & 1 &  2 \\
      0 & 1 & -1
    \end{array}\right]
    \left[\begin{array}{c c}
      1 & 0 \\
      s & 0 \\
      0 & 1
    \end{array}\right].
    \end{align*}
    Furthermore
    \begin{align*}
      A_m &=
      \left[\begin{array}{c c c}
        0 & 1 & 0 \\
        0 & 0 & 0 \\
        0 & 0 & 0
      \end{array}\right], \\
      B_m &=
      \left[\begin{array}{c c}
        0 & 0 \\
        1 & 0 \\
        0 & 1
      \end{array}\right]
    \end{align*}
    so that the controller form realization is given by
    \begin{align*}
      A_C = A_m - B_m L &=
      \left[\begin{array}{r r r}
        0 &  1 &  0 \\
        0 & -3 &  0 \\
        0 &  0 & -1
      \end{array}\right], \\
      B_C = B_m &=
      \left[\begin{array}{c c}
        0 & 0 \\
        1 & 0 \\
        0 & 1
      \end{array}\right], \\
      C_C = M &=
      \left[\begin{array}{r r r}
        3 & 1 &  2 \\
        0 & 1 & -1
      \end{array}\right], \\
      D_C = H(\infty) &=
      \left[\begin{array}{c c}
        0 & 1 \\
        0 & 1 \\
    \end{array}\right].
    \end{align*}
    The observability matrix has full rank, so this is an observable
    realization. Since it is in controller form, it is a controllable
    realization as well, so it is a minimal realization. A
    diagonalizing transformation can then be found by finding a matrix
    of eigenvectors
    $$
    P =
    \left[\begin{array}{c c c}
      1 & -1 & 0 \\
      0 &  3 & 0 \\
      0 &  0 & 1
    \end{array}\right]
    $$
    so that
    $$
    P^{-1} =
    \left[\begin{array}{c c c}
      1 & \frac{1}{3} & 0 \\
      0 & \frac{1}{3} & 0 \\
      0 &           0 & 1
    \end{array}\right]
    $$
    and the diagonal minimal realization is
    \begin{align*}
      A = P^{-1} A P &=
      \left[\begin{array}{c c c}
        0 &  0 &  0 \\
        0 & -3 &  0 \\
        0 &  0 & -1
      \end{array}\right], \\
      B = P^{-1} B &=
      \left[\begin{array}{c c}
        \frac{1}{3} & 0 \\
        \frac{1}{3} & 0 \\
        0           & 1
      \end{array}\right], \\
      C = CP &=
      \left[\begin{array}{r r r}
        3 & 0 &  2 \\
        0 & 3 & -1
      \end{array}\right],
      D = D_C &=
      \left[\begin{array}{c c}
        0 & 1 \\
        0 & 1 \\
      \end{array}\right].
    \end{align*}
  }
\end{enumerate}

\section*{Problem \#2}
\begin{enumerate}[(a)]
  \item{
    The scalar
    $$
    A(t) = -\frac{t}{1 + t}
    $$
    yields the transition matrix (via integration by parts)
    \begin{align*}
      \phi(t, t_0)
      &= \exp\left\{
           -\int_{t_0}^{t} \frac{t}{1 + t}
         \right\} \\
      &= \exp\left\{
           -\left[
              t \ln (1 + t) - \int \ln (1 + t) \dif t
            \right]_{t_0}^{t}
         \right\} \\
      &= \exp\left\{
           \left[
             -t \ln (1 + t) + \left((t + 1) \ln (1 + t) - t\right)
           \right]_{t_0}^{t}
         \right\} \\
      &= \exp\left\{
           \left[
             \log (t + 1) - t
           \right]_{t_0}^{t}
         \right\} \\
      &= \exp\left\{
           \ln\frac{1 + t}{1 + t_0} - (t - t_0)
         \right\} \\
      &= \frac{1 + t}{1 + t_0}e^{-(t - t_0)}.
     \end{align*}
  }
  \item{
    We can write
    $$
    A =
    \left[\begin{array}{c c}
      t & 0 \\
      1 & t
    \end{array}\right]
    =
    t
    \left[\begin{array}{c c}
      1 & 0 \\
      0 & 1
    \end{array}\right]
    +
    \left[\begin{array}{c c}
      0 & 0 \\
      1 & 0
    \end{array}\right]
    =
    \alpha_1(t) M_1 + \alpha_2(t) M_2,
    $$
    which is a sum of scalar functions multiplied by constant
    matrices. Then
    \begin{align*}
    \phi(t, t_0)
    &=
    \exp \left\{
      \int_{t_0}^{t} \alpha_1(t) M_1 \dif t
    \right\}
    \exp \left\{
      \int_{t_0}^{t} \alpha_2(t) M_2 \dif t
    \right\} \\
    &=
    \exp\left\{
      \frac{1}{2}(t^2 - t_0^2) I
    \right\}
    e^{M_2(t - t_0)} \\
    &=
    \left[\begin{array}{c c}
      e^{\frac{1}{2}(t^2 - t_0^2)} & 0 \\
      0                      & e^{\frac{1}{2}(t^2 - t_0^2)}
    \end{array}\right]
    \left[\begin{array}{c c}
      1         & 0 \\
      e^{t - t_0} & 1
    \end{array}\right] \\
    &=
    e^{\frac{1}{2}(t^2 - t_0^2)}
    \left[\begin{array}{c c}
      1         & 0 \\
      e^{t - t_0} & 1
    \end{array}\right].
    \end{align*}
  }
  \item{
    We can write
    $$
    A =
    \left[\begin{array}{r r}
     -1 & 0 \\
      t & 0
    \end{array}\right]
    =
    t
    \left[\begin{array}{c c}
      0 & 0 \\
      1 & 0
    \end{array}\right]
    +
    \left[\begin{array}{r r}
     -1 & 0 \\
      0 & 0
    \end{array}\right]
    =
    \alpha_1(t) M_1 + \alpha_2(t) M_2,
    $$
    which is a sum of scalar functions multiplied by constant
    matrices. Then
    \begin{align*}
      \phi(t, t_0)
      &=
      \exp \left\{
        \int_{t_0}^{t} \alpha_1(t) M_1 \dif t
      \right\}
      \exp \left\{
        \int_{t_0}^{t} \alpha_2(t) M_2 \dif t
      \right\} \\
      &=
      \left[\begin{array}{c c}
        1       & 0 \\
        \lambda & 1
      \end{array}\right]
      \left[\begin{array}{c c}
        e^{-(t - t_0)} & 0 \\
        0           & 1
      \end{array}\right]
    \end{align*}
    where
    $$
    \lambda = \int_{t_0}^t \tau \dif \tau = \frac{1}{2}(t^2 - t_0^2)
    $$
    so that
    $$
    \phi(t, t_0) =
    \left[\begin{array}{c c}
      e^{-(t - t_0)}                         & 0 \\
      \frac{1}{2}(t^2 - t_0^2) e^{-(t - t_0)} & 1
    \end{array}\right].
    $$
  }
\end{enumerate}

\section*{Problem \#3}
\begin{enumerate}[(a)]
  \item{
    The matrix $A$ has the form
    $$
      A(t)
    = t^2
      \left[\begin{array}{r r}
        1 &  0 \\
        2 & -1
      \end{array}\right]
    = \alpha(t) M
    $$
    The transition matrix is then given by
    \begin{align*}
         \phi(t, t_0)
      &= \exp\left\{
           \int_{t_0}^{t} A(\tau) \dif\tau
         \right\} \\
      &= \exp\left\{
           \left[
             \int_{t_0}^t \alpha(\tau) \dif\tau
           \right]
           \left[\begin{array}{r r}
             1 &  0 \\
             2 & -1
           \end{array}\right]
         \right\} \\
      &= \left[\begin{array}{c c}
           e^\lambda         & 0 \\
           2 \sinh \lambda & e^{-\lambda}
         \end{array}\right]
    \end{align*}
    where
    $$
      \lambda
    = \int_{t_0}^{t} \tau^2 \dif\tau
    = \frac{1}{3}(t^3 - t_0^3).
    $$
  }
  \item{
    The matrix $A$ can be rewritten
    $$
    A
    =
    \left[\begin{array}{r r}
      -\frac{2t}{1 + t} & 1                 \\
       1                & -\frac{2t}{1 + t}
    \end{array}\right]
    =
    -\frac{2t}{1 + t}
    \left[\begin{array}{r r}
       1 & 0 \\
       0 & 1
    \end{array}\right]
    +
    \left[\begin{array}{r r}
       0 & 1 \\
       1 & 0
    \end{array}\right]
    =
    \alpha_1(t) M_1 + \alpha_2(t) M_2,
    $$
    which is a sum of scalar functions multiplied by constant
    matrices. Then
    \begin{align*}
    \phi(t, t_0)
    &=
    \exp \left\{
      \int_{t_0}^{t} \alpha_1(t) M_1 \dif t
    \right\}
    \exp \left\{
      \int_{t_0}^{t} \alpha_2(t) M_2 \dif t
    \right\} \\
    &=
    2\frac{1 + t}{1 + t_0}e^{-(t - t_0)}
    \left[\begin{array}{c c}
       \cosh (t - t_0) & \sinh (t - t_0) \\
       \sinh (t - t_0) & \cosh (t - t_0)
    \end{array}\right].
    \end{align*}
  }
\end{enumerate}

\section*{Problem \#4}

\begin{enumerate}
  \item{
    The uniform asymptotic stability of the transition matrices is as
    follows:
    \begin{itemize}
      \item{
        The transition matrix in problem 2(a) is uniformly
        asymptotically stable, since the decaying exponential term
        dominates the response.
      }
      \item{
        The matrix in problem 2(b) is not uniformly asymptotically
        stable, since the transition matrix contains growing
        exponentials.
      }
      \item{
        Although the response in problem 2(c) consists of decaying
        exponentials, there is a constant entry in the transition
        matrix which keeps the system response from decaying to
        zero. Therefore this system is not uniformly asymptotically
        stable.
      }
      \item{
        The system in problem 3(b) is not uniformly asymptotically
        stable since it contains a term which grows as $e^{t^3}$.
      }
      \item{
        Note that
        $$
        2e^{-z}
        \left[\begin{array}{c c}
          \cosh z & \sinh z \\
          \sinh z & \cosh z
        \end{array}\right]
        =
        e^{-z}
        \left[\begin{array}{c c}
          e^{z} + e^{-z} & e^{z} - e^{-z} \\
          e^{z} - e^{-z} & e^{z} + e^{-z}
        \end{array}\right]
        =
        \left[\begin{array}{c c}
          1 + e^{-2z} & 1 - e^{-2z} \\
          1 - e^{-2z} & 1 + e^{-2z}
        \end{array}\right]
        $$
        so that
        $$
        \phi(t, t_0) =
        \frac{1 + t}{1 + t_0}
        \left[\begin{array}{c c}
          1 + e^{-2z} & 1 - e^{-2z} \\
          1 - e^{-2z} & 1 + e^{-2z}
        \end{array}\right].
        $$
        Therefore this system is not uniformly asymptotically stable.
      }
    \end{itemize}
  }
  \item{
    None of the $2 \times 2$ systems are uniformly asymptotically stable.
  }
\end{enumerate}

\end{document}
