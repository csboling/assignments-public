\documentclass{article}

\usepackage{amsmath}
\usepackage{amsfonts}
\usepackage{commath}
\usepackage{enumerate}

\title{ECE/ME 851 Homework \#3}
\date{September 26, 2014}
\author{Sam Boling \\ PID A48788119}

\begin{document}

\maketitle

\section*{Problem \#1}
\begin{enumerate}[(a)]
\item{
Since $\tilde{x} = P x$, we have
$x = P^{-1} \tilde{x}$ and so
$\dot{x} = P^{-1} \dot{\tilde{x}}$.
Then we make the transformation
\begin{align*}
\dot{x} &= A x + B u \\
P^{-1}\dot{\tilde{x}} &= A P^{-1} \tilde{x} + B u \\
\dot{\tilde{x}} &= P A P^{-1} \tilde{x} + P B u
\end{align*}
so that $\tilde{A} = P A P^{-1}$ and $\tilde{B} = P B$ and furthermore
\begin{align*}
  y &= C x = C P^{-1} \tilde{x}
\end{align*}
so $\tilde{C} = C P^{-1}$. Then since $A = P^{-1} \tilde{A} P$, we can
express $\tilde{A}$ as a Jordan matrix for $A$.

Using the matrices
$$
  A =
  \left[\begin{array}{r r r r}
     0 &  1 &  0 &  0 \\
     3 &  0 &  0 &  2 \\
     0 &  0 &  0 &  1 \\
     0 & -2 &  0 &  0
  \end{array}\right], \quad
  B =
  \left[\begin{array}{c c}
     0 &  0 \\
     1 &  0 \\
     0 &  0 \\
     0 &  1 
  \end{array}\right], \quad
  C =
  \left[\begin{array}{c c c c}
     1 &  0 &  1 &  0
  \end{array}\right]
$$

we find the eigenvalues of $A$ as
\begin{align*}
\det (sI - A)
&= \left|\begin{array}{r r r r}
      s & -1 & 0 &  0 \\
     -3 &  s & 0 & -2 \\
      0 &  0 & s & -1 \\
      0 &  2 & 0 &  s
  \end{array}\right|
 = s\left|\begin{array}{r r r}
      s & -1 &  0 \\
     -3 &  s & -2 \\
      0 &  2 &  s
    \end{array}\right| \\
&= s\left(
     s\left|\begin{array}{r r}
        s & -2 \\ 2 & s
      \end{array}\right|
   +  \left|\begin{array}{r r}
        -3 & -2 \\ 0 & s
      \end{array}\right|\right) \\
&= s(s(s^2 +4) + (-3s))
 = s(s^3 + s) = s^2(s^2 + 1),
\end{align*}
so the matrix $A$ has eigenvalues 0, 0, and $\pm i$. Next we find the
true eigenvectors by setting
\begin{align*}
   0 
&= Av
 = \left[\begin{array}{c}
     v_2 \\ 3v_1 + 2v_4 \\ v_4 \\ -2v_2
   \end{array}\right]
\implies
  v = \left[\begin{array}{c}
        0 \\ 0 \\ v_3 \\ 0
      \end{array}\right], \\
   0
&= (A - iI)v
 = \left[\begin{array}{c}
     -i v_1 + v_2 \\
      3 v_1 - i v_2 + 2 v_4  \\
     -i v_3 + v_4 \\
     -2 v_2 - i v_4
   \end{array}\right]
\implies
  v = \left[\begin{array}{c}
        -\frac{i}{2} v_3 \\
        \frac{1}{2} v_3 \\
        v_3 \\
        i v_3
      \end{array}\right],       
\end{align*}
so eigenvectors are
$$
\left[\begin{array}{c}
0 \\ 0 \\ 1 \\ 0
\end{array}\right], \quad
\left[\begin{array}{c}
-\frac{i}{2} \\ \frac{1}{2} \\ 1 \\ i
\end{array}\right], \quad
\left[\begin{array}{c}
\frac{i}{2} \\ \frac{1}{2} \\ 1 \\ -i
\end{array}\right]
$$
since eigenvectors with complex entries must appear as
conjugates. The eigenvalue 0 has multiplicity 2 and the corresponding
eigenspace has dimension 1, so we find a generalized eigenvector for
this eigenvalue by solving
$$
\left[\begin{array}{c}
  0 \\ 0 \\ 1 \\ 0  
\end{array}\right]
= Av
$$
to find
$$
\left[\begin{array}{r}
  -\frac{2}{3} \\ 0 \\ x_3 \\ 1
\end{array}\right]
$$
and therefore a matrix of generalized eigenvectors is given by
$$
P^{-1} =
\left[\begin{array}{r r r r}
  0 & -\frac{2}{3} & -\frac{i}{2} & \frac{i}{2} \\
  0 & 0            &  \frac{1}{2} &  \frac{1}{2} \\
  1 & 1            &  1           &  1 \\
  0 & 1            &  i           &  -i
\end{array}\right]
$$
so
$$
P = \left[\begin{array}{r r r r}
  6 & -2 & 1 &   3 \\
 -6 &  0 & 0 &  -3 \\
  3 &  1 & 0 &  2i \\
-3i &  1 & 0 & -2i
\end{array}\right]
$$
and taking $\tilde{A} = P A P^{-1}$ to be the corresponding Jordan
matrix we have
\begin{align*}
\tilde{A} &=
\left[\begin{array}{r r r r}
0 & 1 &  0 &  0 \\
0 & 0 &  0 &  0 \\
0 & 0 &  i &  0 \\
0 & 0 &  0 & -i
\end{array}\right], \\
\tilde{B} &= PB 
= \left[\begin{array}{r r}
    -2 & 3 \\ 0 & -3 \\ 1 & 2i \\ 1 & -2i
  \end{array}\right], \\
\tilde{C} &= CP^{-1}
= \left[\begin{array}{r r r r}
    1 & -\frac{1}{3} & 1 - \frac{i}{2} & 1 + \frac{i}{2}
  \end{array}\right].
\end{align*}
}
\item{
Since the transfer function is independent of state representation,
we have
$$
H(s) = \tilde{C} (sI - \tilde{A})^{-1} \tilde{B}
$$
Then we note that
\begin{align*}
(sI - A)^{-1} &=
\left[\begin{array}{c c c c}
s & -1 &    0   &  0 \\
0 &  s &    0   &  0 \\
0 &  0 &  s - i &  0 \\
0 &  0 &    0   & s + i
\end{array}\right]^{-1} 
= \left[\begin{array}{r r r r}
    \frac{1}{s} & \frac{1}{s^2} & 0               & 0 \\
    0           & \frac{1}{s}   & 0               & 0 \\
    0           & 0             & \frac{1}{s - i} & 0 \\
    0           & 0             & 0               & \frac{1}{s+i}
  \end{array}\right], \\
\tilde{C}(sI - \tilde{A})^{-1} &=
  \left[\begin{array}{c}
      \frac{1}{s} \\
      \frac{1}{s^2} - \frac{1}{3s} \\
      \left(1 - \frac{i}{2}\right)\frac{1}{s - i} \\
      \left(1 + \frac{i}{2}\right)\frac{1}{s + i}
    \end{array}\right]^\top, \\
\tilde{C}(sI - \tilde{A})^{-1}\tilde{B} &=
\left[\begin{array}{c}
  -\frac{2}{s} 
 + \frac{1 - \frac{i}{2}}{s-i}
 + \frac{1 + \frac{i}{2}}{s+i} \\
   \frac{3}{s}
 - \frac{3}{s^2}
 - \frac{1}{s}
 + \frac{2i + 1}{s-i}
 + \frac{-2i + 1}{s+i}
\end{array}\right]^\top
= \left[\begin{array}{c}
 \frac{2s + 1}{s^2 + 1} - \frac{2}{s} \\
 \frac{2s - 4}{s^2 + 1} + \frac{2}{s} - \frac{3}{s^2}
\end{array}\right].
\end{align*}
}
\end{enumerate}

\pagebreak

\section*{Problem \#2}
The general solution for the discrete-time equation is given by
\begin{align*}
x(k) &= A^k x(0) + \sum_{j=0}^{k-1} A^{k - (j+1)}Bu(j),
\end{align*}
so taking $x(k) = x(0) \theta(k)$, where $\theta(k)$ is the unit step
function, we have
\begin{align*}
x(0) \theta(k) &= A^k x(0)
      + \sum_{j=0}^{k-1} A^{(k-1) - j} B u(j) \\
     &= A^k x(0)
      + (f \ast u)(k-1),
\end{align*}
where $f(k) = A^k B$ and $\ast$ denotes convolution. Then taking a
$z$-transform we have
\begin{align*}
\frac{z}{z-1}x(0) &=
  \mathcal{Z}\{A^k\}x(0)
+ \mathcal{Z}\{(f \ast u)(k - 1)\} \\
&= z(zI - A)^{-1} x(0)
 + z^{-1}\mathcal{Z}\{(f \ast u)(k)\} \\
&= z(zI - A)^{-1} x(0)
 + z^{-1}\mathcal{Z}\{f(k)\} U(z) \\
&= z(zI - A)^{-1} x(0)
 + z^{-1}z(zI - A)^{-1} B U(z)
\end{align*}
so that
\begin{align*}
\frac{z}{z-1}(zI - A)x(0) &=
  zx(0) + BU(z) \\
B U(z) 
&= \left[\frac{z}{z-1}(zI - A) - zI\right]x(0) \\
&= \left[\left(\frac{z}{z-1} - 1\right)zI 
       - \frac{z}{z-1}A\right] x(0)\\
&= \left[\frac{z - (z-1)}{z-1}zI 
       - \frac{z}{z-1} A\right] x(0) \\
&= \left[\frac{z}{z-1}(I - A)\right]x(0) \\
&= \mathcal{Z}\{\theta(k)\} (I - A) x(0),
\end{align*}
so we have
$$
B u(k) = (I - A) x(0) \theta(k).
$$
When $B$ is a vector, we then have
$$
u(k) = \frac{1}{\| B \|^2} B^\top (I - A) x(0) \theta(k).
$$
In the case
$$
A = \left[\begin{array}{r r}
      2 & 0 \\ 0 & -1
    \end{array}\right], \quad
B = \left[\begin{array}{c}
      1 \\ 1
    \end{array}\right], \quad
x(0) = \left[\begin{array}{r}
        -2 \\ 1
       \end{array}\right]
$$
this gives
$$
\frac{1}{2} 
\left[\begin{array}{c c} 
  1 & 1
\end{array}\right]
\left[\begin{array}{c c}
  -1 & 0 \\ 0 & 2
\end{array}\right]
\left[\begin{array}{r}
  -2 \\ 1
\end{array}\right] \theta(k)
= 2 \theta(k).
$$

\pagebreak

\section*{Problem \#3}

\begin{enumerate}[(a)]
\item{
The matrix
$$
A 
= \left[\begin{array}{r r r r}
    0       &       1 & 0       & 0      \\
    -0.1910 & -0.0536 & 0.0910  & 0.0036 \\
    0       & 0       & 0       & 1      \\
    0.0910  & 0.0036  & -0.1910 & -0.0536
  \end{array}\right]
$$
has eigenvalues
\begin{align*}
& \lambda_1 = -0.0286 + 0.53027i,
& \lambda_2 = -0.0286 - 0.53027i, \\
& \lambda_3 = -0.025  + 0.31524i,
& \lambda_4 = -0.025  - 0.31524i
\end{align*}
and eigenvectors
\begin{align*}
& v_1 
= \left[\begin{array}{c}
    -0.62451 \\
     0.01786 - 0.33116i \\
     0.62451 \\
    -0.01786 + 0.33116i
  \end{array}\right],
& v_2
= \left[\begin{array}{c}
    -0.62451 \\
     0.01786 + 0.33116i \\
     0.62451 \\
    -0.01786 - 0.33116i
  \end{array}\right], \\
& v_3 
= \left[\begin{array}{c}
    0.67420 \\
    -0.01685 + 0.21253i \\
    0.67420 \\
    -0.01685 + 0.21253i
  \end{array}\right],
& v_4 
= \left[\begin{array}{c}
    0.67420 \\
    -0.01685 - 0.21253i \\
    0.67420 \\
    -0.01685 - 0.21253i
  \end{array}\right].
\end{align*}
Since the system has distinct eigenvalues, its modes may be written as
\begin{align*}
A_1 = v_1 w_1^\top 
&= \left[\begin{array}{c c c c}
     0.25 - 0.013484i 
   &        0.471461i 
   &-0.25 + 0.013484i
   &        0.471461i \\
            0.132952i
   & 0.25 + 0.013484i
   &        0.132952i
   &-0.25 - 0.013484i \\
    -0.25 + 0.013484i 
   &        0.471461i
   & 0.25 - 0.013484i 
   &        0.471461i \\
            0.132952i 
   &-0.25 - 0.013484i 
   &        0.132952i 
   & 0.25 + 0.013484i
   \end{array}\right], \\
A_2 = v_2 w_2^\top 
&= \left[\begin{array}{c c c c}
     0.25 + 0.013484i 
  &         0.471461i 
  & -0.25 - 0.013484i 
  &         0.471461i \\
            0.132952i 
  &  0.25 - 0.013484i 
  &         0.132952i 
  & -0.25 + 0.013484i \\
            0.013484i 
  &         0.471461i 
  &  0.25 + 0.013484i 
  &         0.471461i \\
            0.132952i 
  & -0.25 + 0.013484i 
  &         0.132952i 
  &  0.25 - 0.013484i
   \end{array}\right], \\
A_3 = v_3 w_3^\top 
&= \left[\begin{array}{c c c c}
    0.25 - 0.019826i 
 &         0.793052i 
 &  0.25 - 0.019826i 
 &         0.793052i
           0.079305i \\
 &  0.25 + 0.019826i 
 &         0.079305i 
 &  0.25 + 0.019826i \\
    0.25 - 0.019826i 
 &         0.793052i 
 &  0.25 - 0.019826i 
 &         0.793052i \\
           0.079305i 
 &  0.25 + 0.019826i 
 &         0.079305i 
 &  0.25 + 0.019826i
   \end{array}\right], \\
A_4 = v_4 w_4^\top 
&= \left[\begin{array}{c c c c}
    0.25 + 0.019826i 
 &         0.793052i 
 &  0.25 + 0.019826i 
 &         0.793052i \\
           0.079305i 
 &  0.25 - 0.019826i 
 &         0.079305i 
 &  0.25 - 0.019826i \\
    0.25 + 0.019826i 
 &         0.793052i 
 &  0.25 + 0.019826i 
 &         0.793052i \\
           0.079305i 
 &  0.25 - 0.019826i 
 &         0.079305i 
 &  0.25 - 0.019826i
   \end{array}\right], \\
\end{align*}
where the $w_i$ are the left eigenvectors of $A$ found from the
inverse of the matrix possessing the eigenvectors $v_i$ as columns.

A plot of the system's evolution with the given initial condition is
attached.
}
\end{enumerate}

\pagebreak

\section*{Problem \#4}
\begin{enumerate}[(a)]
  \item{
  The matrix
  $$
  A  = \left[\begin{array}{r r r r}
          2 &  3 &  2 &  1 \\
         -2 & -3 &  0 &  0 \\
         -2 & -2 & -4 &  0 \\
         -2 & -2 & -2 & -5
       \end{array}\right]
  $$
  has distinct eigenvalues
  $$
  \lambda_1 = -1, \quad
  \lambda_2 = -2, \quad
  \lambda_3 = -4, \quad
  \lambda_4 = -3
  $$
  and corresponding eigenvectors
  $$
  v_1 = \left[\begin{array}{r}
          -1 \\ 1 \\ 0 \\ 0
        \end{array}\right], \quad
  v_2 = \left[\begin{array}{r}
          -1 \\ 2 \\ -1 \\ 0
        \end{array}\right], \quad
  v_3 = \left[\begin{array}{r}
           0 \\ 0 \\  1 \\ -2
        \end{array}\right], \quad
  v_4 = \left[\begin{array}{r}
           0 \\ 1 \\ -2 \\ 1
        \end{array}\right], 
  $$
  so the transformation matrix is written
  $$
  P = \left[\begin{array}{r r r r}
        -1 & -1 &  0 &  0 \\
         1 &  2 &  0 &  1 \\
         0 & -1 &  1 & -2 \\
         0 &  0 & -2 &  1
      \end{array}\right]
  $$
  so that
  $$
  P^{-1} = \left[\begin{array}{r r r r}
            6 &  3 &  2 &  1 \\
           -2 &  1 &  0 &  0 \\
           -2 & -2 &  0 &  0 \\
           -2 & -2 & -2 & -1
         \end{array}\right]
  $$
  and we have
  \begin{align*}
  A &= P \tilde{A} P^{-1} \\
    &= \left[\begin{array}{r r r r}
        -1 & -1 &  0 &  0 \\
         1 &  2 &  0 &  1 \\
         0 & -1 &  1 & -2 \\
         0 &  0 & -2 &  1
      \end{array}\right]
      \left[\begin{array}{r r r r}
        -1 &  0 &  0 &  0 \\
         0 & -2 &  0 &  0 \\
         0 &  0 & -4 &  0 \\
         0 &  0 &  0 & -3
      \end{array}\right]
      \left[\begin{array}{r r r r}
        -4 &  -3 &  -2 &  -1 \\
         3 &   3 &   2 &   1 \\
        -1 &  -1 &  -1 &  -1 \\
        -2 &  -2 &  -2 &  -1
      \end{array}\right]
  \end{align*}
  so that
  \begin{align*}
  e^{At} &= P \Lambda P^{-1} \\
  &= 
      \left[\begin{array}{r r r r}
        -1 & -1 &  0 &  0 \\
         1 &  2 &  0 &  1 \\
         0 & -1 &  1 & -2 \\
         0 &  0 & -2 &  1
      \end{array}\right]
      \left[\begin{array}{r r r r}
        e^{-t} &  0     & 0      & 0 \\
         0    & e^{-2t} & 0      & 0 \\
         0    &  0     & e^{-4t} & 0 \\
         0    &  0     & 0      & e^{-3t}
      \end{array}\right]
      \left[\begin{array}{r r r r}
        -4 &  -3 &  -2 &  -1 \\
         3 &   3 &   2 &   1 \\
        -1 &  -1 &  -1 &  -1 \\
        -2 &  -2 &  -2 &  -1
      \end{array}\right] \\
  &= \left[\begin{array}{c c c c}
       4 e^{-t} - 3e^{-2t}
     & 3 e^{-t} - 3e^{-2t}
     & 2 e^{-t} - 2e^{-2t}
     & e^{-t}   - e^{-2t} \\
       -4e^{-t} + 6e^{-2t} - 2e^{-3t}
     & -3e^{-t} + 6e^{-2t} - 2e^{-3t}
     & -2e^{-2t} + 4e^{-3t} 
     & 2e^{-4t} - e^{-3t} \\
       -3e^{-2t} + 4e^{-3t}
     & -3e^{-2t} + 4e^{-3t}
     & -2e^{-2t} + 4e^{-3t}
     & -e^{-2t}  + 2e^{-3t} \\
       2e^{-4t}  - 2e^{-3t}
     & 2e^{-4t}  - 2e^{-3t}
     & 2e^{-4t}  - 2e^{-3t}
     & 2e^{-4t}  -  e^{-3t}
     \end{array}\right].
  \end{align*}
}
\item
{
  We have that $e^{At} = \mathcal{L}\{(sI - A)^{-1}\}$, so by the
  inverse Laplace transform
  $$
  (sI - A)^{-1} = \left[\begin{array}{c c c c}
    \frac{4}{s + 1} - \frac{3}{s + 2}
  & \frac{3}{s + 1} - \frac{3}{s + 2}
  & \frac{2}{s + 1} - \frac{2}{s + 2}
  & \frac{1}{s + 1} - \frac{1}{s + 2} \\
   -\frac{4}{s + 1} + \frac{6}{s + 2} - \frac{2}{s + 3}
  &-\frac{3}{s + 1} + \frac{6}{s + 2} - \frac{2}{s + 3}
  &-\frac{2}{s + 2} + \frac{4}{s + 3} 
  &-\frac{2}{s + 4} - \frac{1}{s + 3} \\
   -\frac{3}{s + 2} + \frac{4}{s + 3}
  &-\frac{3}{s + 2} + \frac{4}{s + 3}
  &-\frac{2}{s + 2} + \frac{4}{s + 3}
  &-\frac{1}{s + 2} + \frac{2}{s + 3} \\
    \frac{2}{s + 4} - \frac{2}{s + 3}
  & \frac{2}{s + 4} - \frac{2}{s + 3}
  & \frac{2}{s + 4} - \frac{2}{s + 3}
  & \frac{2}{s + 4} - \frac{1}{s + 3}
  \end{array}\right].
  $$
  Then
  \begin{align*}
  H(s) &= C(sI - A)^{-1}B \\
       &= \left[\begin{array}{c}
            \frac{4}{s+1} 
          + \frac{3}{s+2} 
          - \frac{4}{s+3} 
          + \frac{4}{s+4} \\
            \frac{3}{s+1}
          + \frac{3}{s+3}
          + \frac{4}{s+4} \\
            \frac{14}{s+1}
          - \frac{34}{s+2}
          + \frac{36}{s+3}
          + \frac{4}{s+4} \\
            \frac{7}{s+1}
          - \frac{11}{s+2}
          - \frac{8}{s+4}
        \end{array}\right]^\top B \\
     &= \frac{19}{s+1} - \frac{54}{s+2} + \frac{62}{s+3} + \frac{12}{s+4}.
   \end{align*}
}
\item
{
  Since the eigenvalues of the state dynamics matrix $A$ are all real
  and strictly negative, the system is internally asymptotically
  stable and therefore BIBO stable.
}
\end{enumerate}

\end{document}
