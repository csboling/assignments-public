\documentclass{article}

\usepackage{amsmath}
\usepackage{amsfonts}
\usepackage{commath}
\usepackage{enumerate}

\title{ECE/ME 851 Homework \#2}
\date{September 19, 2014}
\author{Sam Boling \\ PID A48788119}

\begin{document}

\maketitle

\section*{Problem \#1}
\begin{enumerate}[(a)]
  \item{We have that
        \begin{align*}
          sI - A &=
            \left[\begin{array}{c c c}
              s - \frac{1}{2} & 1    & 0 \\
              0               & s + 1 & 0 \\
              0               & 0     & s + 2
            \end{array}\right], \\
          (sI - A)^{-1} &=
            \left[\begin{array}{c c c}
              \frac{1}{s - \frac{1}{2}}
            & -\frac{1}{s+1}\frac{1}{s - \frac{1}{2}}
            & 0 \\
              0
            & \frac{1}{s+1}
            & 0 \\
              0
            & 0
            & \frac{1}{s+2}
            \end{array}\right] \\
          &= \left[\begin{array}{c c c}
               \frac{1}{s - \frac{1}{2}}
             &  \frac{2}{3}\left(
                  \frac{1}{s + 1}
                 -\frac{1}{s - \frac{1}{2}}
                \right)
             & 0 \\
               0
             & \frac{1}{s+1}
             & 0 \\
               0
             & 0
             & \frac{1}{s+2}
             \end{array}\right],
        \end{align*}
        so that
        \begin{align*}
          e^{At} &= \mathcal{L}^{-1}\{(sI - A)^{-1}\}
                 = \left[\begin{array}{c c c}
                     e^{\frac{t}{2}}
                   & \frac{2}{3}\left(
                       e^{-t} - e^{\frac{t}{2}}
                     \right)
                   & 0 \\
                     0 
                   & e^{-t}
                   & 0 \\
                     0
                   & 0
                   & e^{-2t}
                   \end{array}\right].
        \end{align*}
       }
       \item{The characteristic polynomial of $A$ is
             \begin{align*}
               \det (zI - A) &=
               \left|\begin{array}{c c c}
                 z - \frac{1}{2} & 1     & 0 \\
                 0               & z + 1 & 0 \\
                 0               & 0     & z + 2
               \end{array}\right| =
               \left(z - \frac{1}{2}\right)(z + 1)(z+2)
             \end{align*}
             which has roots 
             $$
             \lambda_1 = \frac{1}{2}, \quad 
             \lambda_2 = -1, \quad
             \lambda_3 = -2
             $$ 
             and thus has distinct real eigenvalues. Next we compute
             \begin{align*}
               (A - \lambda_1 I) &= 
               \left[\begin{array}{r r r}
                 0 & -1           & 0            \\
                 0 & -\frac{3}{2} & 0            \\
                 0 & 0            & -\frac{5}{2}
               \end{array}\right]
             \sim
               \left[\begin{array}{r r r}
                 0 & 0 & 0 \\ 0 & 1 & 0 \\ 0 & 0 & 1
               \end{array}\right], \\
               (A - \lambda_2 I) & =
               \left[\begin{array}{r r r}
                 \frac{3}{2}  & -1 &  0 \\
                 0            & 0  &  0 \\
                 0            & 0  & -1
               \end{array}\right]
             \sim
               \left[\begin{array}{r r r}
                 1 & -\frac{2}{3} & 0 \\
                 0 & 0            & 0 \\
                 0 & 0            & 1
               \end{array}\right], \\
             (A - \lambda_3 I) & =
             \left[\begin{array}{r r r}
               \frac{5}{2} & -1 & 0  \\
               0           &  1 & 0  \\
               0           &  0 & 0 
             \end{array}\right]
             \sim
             \left[\begin{array}{r r r}
               1 & 0 & 0  \\
               0 & 1 & 0  \\
               0 & 0 & 0 
             \end{array}\right]. \\             
             \end{align*}
             We then find
            \begin{align*}
              & 0
              = (A - \lambda_1 I)v_1
              = (A - \lambda_1 I)
                \left[\begin{array}{c}
                  \alpha_1 \\ \beta_1 \\ \gamma_1
                \end{array}\right] \\
              \implies &
              0 =
                \left[\begin{array}{c} 
                  0 \\ \beta_1 \\ \gamma_1
                \end{array}\right]
              \implies 
                \beta_1 = \gamma_1 = 0 \\
              \implies &
                v_1 = 
                \left[\begin{array}{c}
                  \alpha_1 \\ 0 \\ 0
                \end{array}\right], \\
              & 0
              = (A - \lambda_2 I)v_2 
              = (A - \lambda_2 I)
                \left[\begin{array}{c}
                  \alpha_2 \\ \beta_2 \\ \gamma_2
                \end{array}\right] \\
              \implies &
              0 =
                \left[\begin{array}{c}
                  \alpha_2 - \frac{2}{3} \beta_2 \\
                  0 \\
                  \gamma_2
                \end{array}\right]
              \implies \gamma_2 = 0, \beta_2 = \frac{3}{2} \alpha_2 \\
              \implies &
              v_2 = 
              \left[\begin{array}{r}
                \alpha_2 \\ \frac{3}{2}\alpha_2 \\ 0
              \end{array}\right], \\
              & 0 
              = (A - \lambda_3 I)v_1
              = (A - \lambda_3 I)
                \left[\begin{array}{c}
                  \alpha_3 \\ \beta_3 \\ \gamma_3
                \end{array}\right] \\
              \implies &
              0 =
              \left[\begin{array}{c}
                \alpha_3 \\ \beta_3 \\ 0
              \end{array}\right]
              \implies
              \alpha_3 = \beta_3 = 0 \\
              \implies &
              v_3 =
              \left[\begin{array}{c}
                0 \\ 0 \\ \gamma_3
              \end{array}\right],
            \end{align*}
             so a basis of eigenvectors is given by
             $$
               v_1 = \left[\begin{array}{c}
                       1 \\ 0 \\ 0
                     \end{array}\right], \quad
               v_2 = \left[\begin{array}{c}
                       1 \\ \frac{3}{2} \\ 0
                     \end{array}\right], \quad
               v_3 = \left[\begin{array}{c}
                       0 \\ 0 \\ 1
                     \end{array}\right].
             $$
             We then form the matrix $P$ as
             \begin{align*}
             P &= \left[\begin{array}{c c c}
                  v_1 & v_2 & v_3 
                  \end{array}\right]
                = \left[\begin{array}{c c c}
                  1 & 1 & 0 \\ 0 & \frac{3}{2} & 0 \\ 0 & 0 & 1
                  \end{array}\right]
             \end{align*}
             so that
             \begin{align*}
             P^{-1} &= \left[\begin{array}{r r r}
                        1 & -\frac{2}{3} & 0 \\ 
                        0 & \frac{2}{3}  & 0 \\
                        0 & 0            & 1
                      \end{array}\right]
             \end{align*}
             and then
             \begin{align*}
             e^{At} 
             &= P \Lambda P^{-1} \\
             &= \left[\begin{array}{c c c}
                  1 & 1           & 0 \\
                  0 & \frac{3}{2} & 0 \\ 
                  0 & 0           & 1
                \end{array}\right]
                \left[\begin{array}{c c c}
                  e^{\lambda_1 t} & 0            & 0 \\
                  0            & e^{\lambda_2 t} & 0 \\
                  0            & 0            & e^{\lambda_3 t}
                \end{array}\right]
                \left[\begin{array}{r r r}
                   1 & -\frac{2}{3} & 0 \\
                   0 & \frac{2}{3}  & 0 \\
                   0 & 0            & 1
                \end{array}\right] \\
            &= \left[\begin{array}{c c c}
                 1 & 1           & 0 \\
                 0 & \frac{3}{2} & 0 \\ 
                 0 & 0           & 1
               \end{array}\right]
               \left[\begin{array}{r r r}
                 e^{\lambda_1 t}
               & - \frac{2}{3} e^{\lambda_1 t} 
               & 0 \\
                 0
               & \frac{2}{3}e^{\lambda_2} t
               & 0 \\
                 0
               & 0
               & e^{\lambda_3 t}
              \end{array}\right] \\
           &= \left[\begin{array}{c c c}
                e^{\lambda_1 t}
              & \frac{2}{3}
                  \left(-e^{\lambda_1 t}
                      + e^{\lambda_2 t}
                  \right)
              & 0 \\
                0
              & e^{\lambda_2 t}
              & 0 \\
                0
              & 0
              & e^{\lambda_3 t}
              \end{array}\right] \\
            &=
            \left[\begin{array}{c c c}
              e^{\frac{t}{2}}
              & \frac{2}{3}\left(
                  e^{-t} - e^{\frac{t}{2}}
                \right)
              & 0 \\
                0 
              & e^{-t}
              & 0 \\
                0
              & 0
              & e^{-2t}
            \end{array}\right].
            \end{align*}
        }
\end{enumerate}

\pagebreak

\section*{Problem \#2}

The characteristic polynomial of $A$ is given by
\begin{align*}
\det (zI - A) &=
\left|\begin{array}{c c c c}
z - 2 & 0     & 0     & 0 \\
0     & z - 2 & 1     & 0 \\
0     & 0     & z - 2 & 0 \\
1     & 0     & 0     & z - 2
\end{array}\right|
=
(z - 2)
\left|\begin{array}{c c c}
z - 2 & 1     & 0 \\
0     & z - 2 & 0 \\
0     & 0     & z - 2
\end{array}\right| \\
&=
(z - 2)^4,
\end{align*}
so the eigenvalue $\lambda = 2$ has arithmetic multiplicity
4. Furthermore the matrix
\begin{align*}
(A - \lambda I) =
\left[\begin{array}{c c c c}
0 & 0 & 0 & 0 \\
0 & 0 & 1 & 0 \\
0 & 0 & 0 & 0 \\
1 & 0 & 0 & 0
\end{array}\right]
\end{align*}
has rank 2, so this eigenvalue has geometric multiplicity 2.
Eigenvectors can then by found by solving
\begin{align*}
& 0 = (A - \lambda I)x =
  \left[\begin{array}{c c c c}
    0 & 0 & 0 & 0 \\
    0 & 0 & 1 & 0 \\
    0 & 0 & 0 & 0 \\
    1 & 0 & 0 & 0
  \end{array}\right]
  \left[\begin{array}{c}
    \alpha \\ \beta \\ \gamma \\ \delta
  \end{array}\right] \\
\implies &
0 =
  \left[\begin{array}{c}
    0 \\ \gamma \\ 0 \\ \alpha
  \end{array}\right]
\implies
  \alpha = \gamma = 0 \\
\implies &
  x = \left[\begin{array}{c}
    0 \\ \beta \\ 0 \\ \delta
  \end{array}\right],
\end{align*}
so $A$ admits eigenvectors
$$
v_1 = \left[\begin{array}{c}
  0 \\ 1 \\ 0 \\ 0
\end{array}\right], \quad
v_2 = \left[\begin{array}{c}
  0 \\ 0 \\ 0 \\ 1
\end{array}\right].
$$
Next we solve
\begin{align*}
& v_1 = (A - \lambda I) v_3 =
\left[\begin{array}{c c c c}
0 & 0 & 0 & 0 \\
0 & 0 & 1 & 0 \\
0 & 0 & 0 & 0 \\
1 & 0 & 0 & 0 
\end{array}\right]
\left[\begin{array}{c}
\alpha \\ \beta \\ \gamma \\ \delta
\end{array}\right] \\
\implies &
\left[\begin{array}{c}
0 \\ 1 \\ 0 \\ 0
\end{array}\right]
= \left[\begin{array}{c}
  0 \\ \gamma \\ 0 \\ \alpha
\end{array}\right]
\implies
\alpha = 0, \gamma = 1 \\
\implies &
v_3 = \left[\begin{array}{c}
  0 \\ \beta \\ 1 \\ \delta
\end{array}\right], \\
& v_2 = (A - \lambda I) v_4 =
\left[\begin{array}{c c c c}
0 & 0 & 0 & 0 \\
0 & 0 & 1 & 0 \\
0 & 0 & 0 & 0 \\
1 & 0 & 0 & 0 
\end{array}\right]
\left[\begin{array}{c}
\alpha \\ \beta \\ \gamma \\ \delta
\end{array}\right] \\
\implies &
\left[\begin{array}{c}
0 \\ 0 \\ 0 \\ 1
\end{array}\right]
= \left[\begin{array}{c}
  0 \\ \gamma \\ 0 \\ \alpha
\end{array}\right]
\implies \alpha = 1, \gamma = 0 \\
\implies &
v_4 = \left[\begin{array}{c}
1 \\ \beta \\ 0 \\ \delta
\end{array}\right],
\end{align*}
so $A$ admits generalized eigenvectors
$$
v_3 = \left[\begin{array}{r r}
0 \\ -1 \\ 1 \\ 0
\end{array}\right], \quad
v_4 = \left[\begin{array}{c c}
1 \\ 0 \\ 0 \\ 0
\end{array}\right].
$$

Ordering these vectors so that the eigenvectors are paired with their
associated generalized eigenvectors, we write the state transformation matrix
$$
P 
= \left[\begin{array}{r r r r}
v_1 & v_3 & v_1 & v_4
\end{array}\right]
= \left[\begin{array}{r r r r}
0 &  0 & 0 & 1 \\
1 & -1 & 0 & 0 \\
0 &  1 & 0 & 0 \\
0 &  0 & 1 & 0
\end{array}\right]
$$
so we have
$$
P^{-1} = \left[\begin{array}{c c c c}
0 & 1 & 1 & 0 \\
0 & 0 & 1 & 0 \\
0 & 0 & 0 & 1 \\
1 & 0 & 0 & 0
\end{array}\right]
$$
and thus the Jordan form
$$
J = P^{-1} A P
  = \left[\begin{array}{c c c c}
      2 & 1 & 0 & 0 \\
      0 & 2 & 0 & 0 \\
      0 & 0 & 2 & 1 \\
      0 & 0 & 0 & 2
    \end{array}\right].
$$

\pagebreak

\section*{Problem \# 3}

Letting
$$
A = \left[\begin{array}{c c}
      -1 & 0 \\ 0 & 1
    \end{array}\right], \quad
B = \left[\begin{array}{c}
       1 \\ 1
    \end{array}\right]
$$
the general solution is given by
\begin{align*}
x(t) &= x(0) + \int_0^t e^{A(t - \tau)} Bu(t) \dif \tau \\
     &= x(0) + \int_0^t e^{A(t - \tau)} B \dif \tau,
\end{align*}
since $u(t) = 1$ for $t \geq 0$. The matrix $e^{At}$ can then be
computed by first finding
\begin{align*}
(s I - A) 
&= \left[\begin{array}{c c}
     s + 1 & 0     \\
     0     & s - 1
\end{array}\right], \\
(s I - A)^{-1} 
&= \frac{1}{\det (s I - A)}
   \left[\begin{array}{c c}
     s - 1 & 0 \\
     0     & s + 1
   \end{array}\right]
 = \frac{1}{(s+1)(s-1)}
   \left[\begin{array}{c c}
     s - 1 & 0 \\
     0     & s + 1
   \end{array}\right] \\
&=
   \left[\begin{array}{c c}
     \frac{1}{s + 1} & 0 \\
     0               & \frac{1}{s - 1}
   \end{array}\right],
\end{align*}
so
\begin{align*}
e^{A t} 
&= \mathcal{L}^{-1}\{(s I - A)^{-1}\}
 = \left[\begin{array}{c c}
     e^{-t} & 0 \\ 0 & e^{t}
   \end{array}\right].
\end{align*}
Then
$$
e^{A(t - \tau)}B 
= \left[\begin{array}{c c}
    e^{-(t - \tau)} & 0 \\ 0 & e^{t - \tau}
  \end{array}\right]
  \left[\begin{array}{c}
    1 \\ 1
  \end{array}\right]
= \left[\begin{array}{c}
    e^{-t}e^{\tau} \\ e^{t}e^{-\tau}
  \end{array}\right]
$$
so
\begin{align*}
\int_0^t e^{A(t - \tau)} B ~\dif \tau
&= \left[\begin{array}{c c}
     e^{-t}\int_0^t e^\tau \dif \tau & 0 \\
     0                              & e^t \int_0^t e^{-\tau} \dif \tau
   \end{array}\right]
 = \left[\begin{array}{c c}
     e^{-t}(e^t - 1) \\
     e^{t}(1 - e^{-t})
   \end{array}\right] \\
&= \left[\begin{array}{c}
     1 - e^{-t} \\
     e^{t} - 1
   \end{array}\right],
\end{align*}
so for $x(0) = \left[\begin{array}{c} 1 \\ 0 \end{array}\right]$,
$$
x(t) 
= \left[\begin{array}{c}
    2 - e^{-t} \\ e^t - 1
  \end{array}\right]
$$
and for other initial conditions
$$
x(t) 
= \left[\begin{array}{c}
    a + 1 - e^{-t} \\
    b - 1 + e^t
  \end{array}\right].
$$

Thus $x_1(t)$ will asymptotically approach $a + 1$ and
$x_2(t)$ will diverge.

\pagebreak

\section*{Problem \#4}
With
$$
A = \left[\begin{array}{r r r}
      -1 & 1 & 0 \\ 0 & -1 & 0 \\ 0 & 0 & 2
    \end{array}\right], \quad
B = 0, \quad
C = \left[\begin{array}{c c c}
      1 & 1 & 1
    \end{array}\right],
$$
we have
\begin{align*}
s I - A 
&= \left[\begin{array}{c c c}
     s + 1 & -1    & 0 \\
     0     & s + 1 & 0 \\
     0     & 0     & s - 2
   \end{array}\right], \\
(s I - A)^{-1}
&= \left[\begin{array}{c c c}
     \frac{1}{s + 1} & \frac{1}{(s+1)^2} & 0 \\
     0               & \frac{1}{s + 1}   & 0 \\
     0               & 0                 & \frac{1}{s-2}
   \end{array}\right], \\
e^{At} = \mathcal{L}^{-1} \{ (s I - A)^{-1} \}
&= \left[\begin{array}{c c c}
     e^{-t} & t e^{-t} & 0 \\
     0     & e^{-t}   & 0 \\
     0     & 0       & e^{2t}
   \end{array}\right].
\end{align*}
Then the system response is given by
\begin{align*}
y(t) &= Ce^{A(t - t_0)}x_0 
      + C\int_0^t e^{A(t - s)} Bu(s) \dif s
      + Du(t) \\
     &= \left[\begin{array}{c c c}
          1 & 1 & 1
        \end{array}\right]
        \left[\begin{array}{c c c}
          e^{-t} & t e^{-t} & 0 \\
          0     & e^{-t}   & 0 \\
          0     & 0       & e^{2t}
        \end{array}\right]
        \left[\begin{array}{c}
          x_1(0) \\ x_2(0) \\ x_3(0)
        \end{array}\right] \\
      &= e^{-t}x_1(0) + (t + 1)e^{-t}x_2(0) + e^{2t}x_3(0),
\end{align*}
and since it is not possible to choose any constant $c$ such that
$c(t+1)e^{-t} = te^{-t}$ for all $t \geq 0$, no initial condition can
be chosen as desired.

\end{document}
