\documentclass{article}

\usepackage{amsmath}
\usepackage{amsfonts}
\usepackage{amssymb}
\usepackage{enumerate}
\usepackage{mathtools}
\usepackage{xfrac}
\usepackage[lastexercise]{exercise}

\DeclarePairedDelimiter\floor{\lfloor}{\rfloor}

\newcounter{Problem}
\newenvironment{Problem}{\begin{Exercise}[name={Problem},
                                          counter={Problem}]}
                        {\end{Exercise}}
\title{MATH 818 Homework \#5}
\date{December 3, 2014}
\author{Sam Boling}

\begin{document}

\begin{titlepage}
\maketitle
\end{titlepage}

\begin{Problem}
  Let $A$ be a commutative ring with 1 which is not the zero
  ring. Denote by $A^\ast$ the units of the ring $A$.
  \begin{enumerate}[(a)]
    \item{
      Suppose that $A$ contains a proper ideal $I$ with the property
      that $x$ is a unit of $A$ if any only if $x \notin I$, i.e.
      $A^\ast = A - I$. Prove that $I$ is a maximal ideal of $A$ and
      that, in fact, $I$ is the only maximal ideal of $A$.
    }
    \item{
      Let $P$ be a prime ideal of $A$ and denote $S = A - P$. Prove
      that $S$ is a multiplicative submonoid of $A$ and that the
      localization $S^{-1}A$ of $A$ at $S$ is a local ring.
    }
    \item{
      Show that for every $(a, b) \in \mathbb{R}^2$, the ideal
      $P = (x - a, y - b)$ in the polynomial ring
      $\mathbb{R}[x, y]$ is a maximal ideal. Write down explicitly the
      localization $S^{-1} A$ where $A = \mathbb{R}[x, y]$,
      $S = A - P$ with $P = (x - a, y - b)$ as above.
    }
  \end{enumerate}
\end{Problem}

\begin{Answer}
  \begin{enumerate}[(a)]
    \item{
      Let $J$ be an ideal such that $I \subset J$ and
      $I \neq J$. Then $J - I$ is nonempty, so there exists an
      $x \in J$ such that $x \notin I$. But then
      $x \in A - I = A^\ast$, so $x$ is a unit, and therefore there
      exists some $y \in A$ such that $xy = 1$. But $J$ is an ideal,
      so if $x \in J$ then $xy \in J$ as well. Therefore $1 \in J$,
      whence $J = A$. Therefore $I$ is a maximal ideal.

      Let $M$ be a maximal ideal of $A$. Then $M \neq A$, so
      $1 \notin M$. Let $x \in M$. Then $x$ cannot be a unit of $A$,
      since if it were then we would have $y \in A$ such that $xy = 1$
      and thus $1 \in M$ since $xy \in M$ for all $y$. Therefore
      $x \in A - A^\ast = I$, so $M \subset I$. But since $M$ is
      maximal this means $M = I$.
    }
    \item{
      Let $P$ be a prime ideal of $A$ and let $S = A - P$. Let
      $x, y \in S$. Then $x, y \notin P$, so since $P$ is a prime
      ideal it must be that $xy \notin P$. Therefore $xy \in A - P =
      S$,
      so $S$ is a multiplicative submonoid of $A$.


    }
  \end{enumerate}
\end{Answer}

\pagebreak

\begin{Problem}
  Let $k$ be a field and let $A = k[x, y]$ be the ring of polynomials in
  two variables $x$ and $y$. We know that $A$ is a Noetherian ring which
  is a unique factorization domain.
  \begin{enumerate}[(a)]
    \item{
      Consider the subring $B$ of $A$ generated by the constants $k$
      and the monomials $xy^n$, for all $n \geq 0$. Show that $B$ is
      not a Noetherian ring.
    }
    \item{
      For $f(x, y) \in A$ show that $f(x, y)$ is irreducible if and
      only if the ideal $(f(x,y))$ is a prime ideal of $A$.
    }
    \item{
      Consider the subring $C$ of $A$ generated by the constants $k$
      and the monomials $x^2$, $x^3$ and $y$. Show that $C$ is not a
      unique factorization domain.
    }
  \end{enumerate}
\end{Problem}

\begin{Answer}
  \begin{enumerate}[(a)]
    \item{
    }
  \end{enumerate}
\end{Answer}

\pagebreak

\begin{Problem}
  Consider the ring $\mathbb{C}[[x]]$ of formal power series with
  coefficients in $\mathbb{C}$ and variable $x$, i.e.
  $$
  A = \mathbb{C}[[x]] =
  \left\{
    \left.
      \sum_{k \in \mathbb{N}} a_k X^k
    = a_0 + a_1 X + a_2 X^2 + \cdots + a_n X^n + \cdots
    \right|
      a_i \in \mathbb{C}
  \right\}.
  $$
  \begin{enumerate}
    \item{
      Show that $f(X) = a_0 + a_1 X + a_2 X^2 + \cdots \in A$ is a
      unit in $A$ if any only if $a_0 \neq 0$.
    }
    \item{
      Show that the ideal $(X)$ generated by $X$ is the unique maximal
      ideal of $A$.
    }
    \item{
      Show that $A$ is a principal ideal domain. Find all the ideals
      of $A$.
    }
  \end{enumerate}
\end{Problem}

\begin{Answer}
  \begin{enumerate}[(a)]
    \item{
      The product of the power series
      $$
      a = \sum_{n=0}^\infty a_n X^n, \quad
      b = \sum_{n=0}^\infty b_n X^n \in \mathbb{C}[[x]]
      $$
      is given by
      $$
      a \cdot b = \sum_{n=0}^\infty(a \cdot b)_n X^n, \quad
      (a \cdot b)_n = \sum_{k=0}^n a_k b_{n-k}.
      $$
      \begin{itemize}
        \item[$\implies$]{
          Suppose $a \cdot b = 1$ for some $b \in
          \mathbb{C}[[x]]$. Then
          \begin{align*}
             1
          &= a \cdot b
          &= \sum_{n=0}^\infty \sum_{k=0}^n a_k b_{n-k} X^n \\
          &= a_0 b_0
           + X \sum_{n=1}^\infty \sum_{k=0}^n a_k b_{n-k} X^{n-1},
          \end{align*}
          and since all terms besides $a_0 b_0$ contain $X$ this is
          only possible if $b_0 = a_0^{-1}$, $(a \cdot b)_n = 0$ for
          all $n \geq 1$. Therefore $a_0$ must have a multiplicative
          inverse for this equation to be satisfied, so $a_0 \neq 0$.
        }
        \item[$\impliedby$]{
          Suppose $a_0 \neq 0$. Then $a_0$ has a multiplicative
          inverse in $\mathbb{C}$. Let
          \begin{align*}
            b_0 &= a_0^{-1}, \\
            b_n &= -a_0^{-1} \sum_{k=0}^{n-1} a_{k+1} b_{(n-1)-k}.
          \end{align*}
          and consider the power series $b = \sum_{n=0}^\infty b_n
          X^n$. Then the terms of the product are given by
          \begin{align*}
             (a \cdot b)_0
          &= a_0 a_0^{-1} = 1, \\
             (a \cdot b)_n
          &= a_0 b_n + a_1 b_{n-1} + \cdots + a_{n-1} b_1 + a_n b_0 \\
          &= a_0 b_n + \sum_{k=0}^{n-1} a_{k+1} b_k \\
          &= -a_0 a_0^{-1} \sum_{k=0}^{n-1} a_{k+1} b_k
           + \sum_{k=0}^{n-1} a_{k+1} b_k
           = 0,
          \end{align*}
          and therefore $a \cdot b = 1$.
        }
      \end{itemize}
    }
    \item{
      Note that
      \begin{align*}
         (X)
      &= \left\{
           r_1 X + r_2 X + \cdots + r_n X
         \mid
           r_i \in A
         \right\} \\
      &= \left\{
           X (r_1 + \cdots r_n)
         \mid
           r_i \in A
         \right\} \\
      &= \left\{
           \left.
             X
             \left(
               \sum_{k \in \mathbb{N}} a_{1k} X^k
             + \cdots
             + \sum_{k \in \mathbb{N}} a_{nk} X^k
             \right)
           \right|
             a_{ij} \in \mathbb{C}
         \right\} \\
      &= \left\{
           \left.
             X
             \sum_{k \in \mathbb{N}}
               (a_{1k} + \cdots + a_{nk}) X^k
           \right|
             a_{ij} \in \mathbb{C}
         \right\},
      \end{align*}
      or after relabeling coefficients
      \begin{align*}
         (X)
      &= \left\{
           \left.
             \sum_{k \in \mathbb{N}} a_k X^k
           \right|
             a_0 = 0
         \right\} \\
      &= A -
         \left\{
           \left.
             \sum_{k \in \mathbb{N}} a_k X^k
           \right|
             a_0 \neq 0
         \right\}.
      \end{align*}
      Therefore, from part (a), $(X) = A - A^\ast$
      and from problem 1(a) this means $(X)$ is the unique maximal
      ideal of $A$.
    }
    \item{
    }
  \end{enumerate}
\end{Answer}

\pagebreak

\begin{Problem}
  Consider the subset $\mathbb{Z}[i]$ of complex numbers of the form
  $a + bi$, where $a$ and $b$ are both integers.
  \begin{enumerate}[(a)]
    \item{
      Show that $\mathbb{Z}[i]$ is a subring of $\mathbb{C}$.
    }
    \item{
      Show that $\mathbb{Z}[i]$ ``has division,'' i.e. given
      $z = a + bi$, $w = c + di \in \mathbb{Z}[i]$ with $w \neq 0$, we
      can find $q, r \in \mathbb{Z}[i]$ such that $z = wq + r$ and
      with $0 \leq |r| < |w|$.
    }
    \item{
      Use (b) to show that $\mathbb{Z}[i]$ is a principal ideal domain.
    }
  \end{enumerate}
\end{Problem}

\begin{Answer}
  \begin{enumerate}[(a)]
    \item{
      We have that $1 = 1 + 0i, 0 = 0 + 0i \in \mathbb{Z}[i]$.
      Let $x = a + bi$, $y = c + di$. Then
      $$
      x + y = (a + c) + (b + d)i \in \mathbb{Z}[i]
      $$
      and
      $$
      xy = ac + adi + bci - bd = (ac - bd) + (ad + bc)i
        \in \mathbb{Z}[i],
      $$
      so $\mathbb{Z}[i]$ is a subring of $\mathbb{C}$.
    }
    \item{
      Let $z = a + bi \in \mathbb{Z}[i]$,
      $w = c + di \in \mathbb{Z}[i] - \{0\}$. Then
      $a, b, c, d \in \mathbb{Z}$, and since $w \neq 0$ we know
      $c^2 + d^2 \neq 0$. Therefore Euclidean
      division of $ac + bd$ by $c^2 + d^2$ yields
      $$
      ac + bd = (c^2 + d^2) q_R + r_R
      $$
      for some $q_R, r_R \in \mathbb{Z}$, and Euclidean division of
      $cb - ad$ by $c^2 + d^2$ yields
      $$
      cb - ad = (c^2 + d^2) q_I + r_I
      $$
      for some $q_I, r_I \in \mathbb{Z}$. We rewrite this system of
      equations as an equation of integer-valued matrices and observe
      \begin{align*}
        \left[\begin{array}{c}
          ac + bd \\
          cb - ad
        \end{array}\right]
        &=
        \left[\begin{array}{c c}
          c^2 + d^2 & 0 \\
          0         & c^2 + d^2
        \end{array}\right]
        \left[\begin{array}{c}
          q_R \\
          q_I
        \end{array}\right]
        +
        \left[\begin{array}{c}
          r_R \\
          r_I
        \end{array}\right] \\
        \left[\begin{array}{r r}
          c & d \\
         -d & c
        \end{array}\right]
        \left[\begin{array}{c}
          a \\
          b
        \end{array}\right]
        &=
        \left[\begin{array}{r r}
          c & d \\
         -d & c
        \end{array}\right]
        \left[\begin{array}{r r}
          c & -d \\
          d &  c
        \end{array}\right]
        \left[\begin{array}{c}
          q_R \\
          q_I
        \end{array}\right]
        +
        \left[\begin{array}{c}
          r_R \\
          r_I
        \end{array}\right]
      \end{align*}
      so that
      \begin{align*}
        \left[\begin{array}{c}
          a \\
          b
        \end{array}\right]
        &=
        \left[\begin{array}{r r}
          c & -d \\
          d &  c
        \end{array}\right]
        \left[\begin{array}{c}
          q_R \\
          q_I
        \end{array}\right]
        +
        \left[\begin{array}{c}
          r_R \\
          r_I
        \end{array}\right].
      \end{align*}
      But this means
      \begin{align*}
         z
       = a + bi
      &= (cq_R - dq_I + r_R) + (cq_I + dq_R + r_I)i \\
      &= (cq_R + cq_I i + dq_Ri - dq_I) + (r_R + r_I i) \\
      &= (c + di)(q_R + q_I i) + (r_R + r_I i) \\
      &= wq + r
      \end{align*}
      for some $q, r \in \mathbb{Z}[i]$ as desired. Therefore
      $\mathbb{Z}[i]$ has division.
    }
    \item{
    }
  \end{enumerate}
\end{Answer}

\end{document}