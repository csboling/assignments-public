\documentclass{article}

\usepackage{amsmath}
\usepackage{amsfonts}
\usepackage{amssymb}
\usepackage{enumerate}
\usepackage{mathtools}
\usepackage[lastexercise]{exercise}

\DeclarePairedDelimiter\floor{\lfloor}{\rfloor}

\newcounter{Problem}
\newenvironment{Problem}{\begin{Exercise}[name={Problem},
                                          counter={Problem}]}
                        {\end{Exercise}}
\title{MATH 818 Homework \#4}
\date{November 19, 2014}
\author{Sam Boling}

\begin{document}

\begin{titlepage}
\maketitle
\end{titlepage}

\begin{Problem}
Show that if $P$ is a $p$-Sylow subgroup of a finite group $G$, then
$N_G(N_G(P)) = N_G(P)$, where $N_G(H)$ denotes the normalizer of $H$
in $G$.
\end{Problem}

\begin{Answer}
Note that $[G : N_G(P)] = |G . P|$, where
$$
G . P = \{ H \mid H < G, H = gPg^{-1}, g \in G \},
$$
the orbit of the action of $G$ on its subgroup $P$ by conjugation.
All conjugates have the same order, so each $gPg^{-1}$ has the order
$|P|$ and is thus a $p$-Sylow since $P$ is. But all $p$-Sylows are
conjugate to $P$, and so $P$ has exactly one orbit under this
conjugation. Then $[G : N_G(P)] = |G . P| = 1$.

Since $N_G(P) < N_G(N_G(P))$,
$[G : N_G(P)] \geq [G : N_G(N_G(P))]$. But if
$[G : N_G(P)] > [G : N_G(N_G(P))]$ then $[G : N_G(N_G(P))] = 0$, which
is not possible. Therefore $[G : N_G(P)] = [G : N_G(N_G(P))]$ and so
$|N_G(N_G(P))| = |N_G(P)|$, and since $N_G(P)$ is a subgroup of
$N_G(N_G(P))$ we must have $N_G(N_G(P)) = N_G(P)$.
\end{Answer}

\pagebreak

\begin{Problem}
Let $G$ be a finite group. Consider the sequence of subgroups of $G$
defined inductively by
$$
Z_0(G) = \{ 1 \}, \quad
Z_{i+1}(G) = \pi^{-1}(Z(G/Z_i(G)))
$$
where $\pi : G \to G / Z_i(G)$ is the natural quotient map. Then $G$
is called \emph{nilpotent} if $Z_n(G) = G$ for some $n \geq 0$.

\begin{enumerate}[(a)]
  \item{
    Show that for $i \geq 0$,
    $$
    Z_{i+1}(G) = \{ x \in G
                   \mid
                   xyx^{-1}y^{-1} \in Z_i(G), \forall y \in G
                \}.
    $$
  }
  \item{
    Show that if $G$ is nilpotent then $G$ is solvable.
  }
  \item{
    Show that a $p$-group, where $p$ is prime, is always nilpotent.
  }
  \item{
    Give an example of a group which is not nilpotent.
  }
  \item{
    Assume $G$ is nilpotent. Show that if $H$ is a proper subgroup of $G$, then
    the normalizer of $H$ in $G$ is always strictly larger than $H$.
  }
  \item{
    Suppose that $G$ is nilpotent and $P$ a $p$-Sylow subgroup of
    $G$. Show that $P$ is normal in $G$. (Hint: Use problem 1.)
  }
  \item{
    Show that $G$ is nilpotent if and only if it is isomorphic to the
    direct product of its $p$-Sylow subgroups. (Hint: recall that if
    $P$ and $Q$ are both normal in $G$ and $P \cap Q = \{ 1 \}$, then
    $ab = ba$ for all $a \in P$, $b \in Q$.
  }
\end{enumerate}
\end{Problem}

\begin{Answer}
\begin{enumerate}[(a)]
  \item{
  }
  \item{


    Then $Z_{i}(G) \triangleleft Z_{i+1}(G)$, and so we have
    $$
      Z\left(\sfrac{G}{Z_{i}(G)}\right)
    = \pi(Z_{i+1}(G))
    = \sfrac{Z_{i+1}(G)}{Z_{i}(G)}.
    $$
    Since $Z_{i+1}(G) / Z_{i}(G)$ is the center of a group, it is
    abelian. Since $G$ is nilpotent by assumption, $G = Z_n(G)$ for
    some $n$, and so we have an abelian tower
    $$
    G = Z_n(G)
    \triangleright Z_{n-1}(G)
    \triangleright \cdots
    \triangleright Z_1(G)
    \triangleright Z_0(G) = \{ 1 \}.
    $$
    Therefore $G$ is solvable.
  }
\end{enumerate}
\end{Answer}

\pagebreak

\begin{Problem}
Consider the group given by generators and relations as follows:
$$
G = \langle
      x, y, z
    \mid
      x^3 = y^7 = z^{13} = 1,
      x^{-1} y x = y^4,
      x^{-1} z x = z^{-1},
      yz = zy
    \rangle.
$$
What can you say about the order of $G$?
\end{Problem}

\begin{Answer}
First note that the condition $x^3 = y^7 = z^{13} = 1$ implies that
$G$ is no larger than the free product of cyclic groups
$\mathbb{Z}_3 \ast \mathbb{Z}_7 \ast \mathbb{Z}_{13}$. Furthermore
note that
\end{Answer}

\pagebreak

\begin{Problem}
The \emph{braid group} $B_3$ in 3 strings is given by generators and
relations as
$$
B_3 = \langle x, y \mid xyx = yxy \rangle.
$$

\begin{enumerate}[(a)]
  \item{
    Show that $xyxyxy$ is in the center of $B_3$.
  }
  \item{
    Prove that there exists a surjective homomorphism
    $\varphi : B_3 \to S_3$ such that $\varphi(x) = (12)$,
    $\varphi(y) = (23)$.
  }
  \item{
    Show that the kernel of $\varphi$ contains a group isomorphic to
    the free group with two generators $F(u,v)$.
  }
  \item{
    Show that there is a group homomorphism
    $$
    \psi : B_3
       \to \mathrm{PSL}_2(\mathbb{Z})
         = \mathrm{SL}_2(\mathbb{Z}) / \{ \pm I \}
    $$
    with
    $$
    \psi(x) =
    \left(\begin{array}{r r}
      1 & 1 \\
      0 & 1
    \end{array}\right), \quad
    \psi(y) =
    \left(\begin{array}{r r}
      1 & 0 \\
     -1 & 1
    \end{array}\right)
    $$
  }
  \item{
    Show that $\psi$ is surjective.
  }
\end{enumerate}

\end{Problem}

\begin{Answer}
\begin{enumerate}
  \item{
    Assume that $xyx = yxy$. The following inductive arguments show
    that $xyx \cdot y^n = x^n \cdot xyx$ for any $n \in \mathbb{Z}$.
    If $n = 0$, then $xyx \cdot y^0 = x^0 \cdot xyx$. Let $k \in \mathbb{Z}_+$.
    \begin{itemize}
      \item{
        Suppose $xyx \cdot y^{k-1} = x^{k-1} xyx$. Then
        \begin{align*}
             xyx \cdot y^k
          &= xyx \cdot y \cdot y^{k-1}
           = x \cdot yxy \cdot y^{k-1} \\
          &= x \cdot xyx \cdot y^{k-1}
           = x \cdot x^{k-1} xyx       \\
          &= x^k \cdot xyx
        \end{align*}
        as desired.
      }
      \item{
        Suppose $xyx \cdot y^{-(k-1)} = x^{-(k-1)} xyx$.
        Then
        \begin{align*}
             xyx \cdot y^{-(m-1)}
          &= x^{-(m-1)} \cdot xyx
           = x^{-m} \cdot x \cdot xyx \\
           = x^{-m} \cdot x \cdot yxy \\
          &= x^{-m} \cdot xyx \cdot y.
        \end{align*}
        But $xyx \cdot y^{-(m-1)} = xyx \cdot y^{-m} \cdot y$, so we
        have
        $$
        xyx \cdot y^{-m} \cdot y = x^{-m} \cdot xyx \cdot y
        $$
        so it must be the case that $xyx \cdot y^{-m} = x^{-m} \cdot
        xyx$ as desired.
      }
    \end{itemize}
    Therefore $xyx \cdot y^n = x^n \cdot xyx$. Replacing every $x$ by
    $y$ and vice-versa in the above argument shows we also have
    $xyx \cdot x^n = y^n \cdot xyx$ for any $n$.

    This means that
    \begin{align*}
        xyxyxy \cdot x^n
     &= xyx \cdot y^n \cdot yxy
      = x^n \cdot xyxyxy, \\
        xyxyxy \cdot y^n
     &= xyx \cdot x^n \cdot yxy
      = y^n \cdot xyxyxy,
    \end{align*}
    and it follows that $xyxyxy$ commutes with every element of $G$.
  }
  \item{
  }
  \item{
    Note that
    $$
    \varphi(x^2) = \varphi(x)\varphi(x) = (12)(12) = \mathrm{id}
    $$
    and
    $$
    \varphi(y^2) = \varphi(y)\varphi(y) = (23)(23) = \mathrm{id}
    $$
    so that $x^2, y^2 \in \ker(\varphi)$ and similarly
    $x^{-2}, y^{-2} \in \ker(\varphi)$, so
    $\langle x^2, y^2 \rangle < \ker(\varphi)$.

    Consider the unique group homomorphism
    $f : F(a,b) \to \langle x^2, y^2 \rangle$ such that
    $f(a) = x^2$, $f(b) = y^2$. Since $\langle x^2, y^2 \rangle$ is
    generated by $x^2, y^2$ by construction, $f$ is surjective.
  }
\end{enumerate}
\end{Answer}

\pagebreak

\end{document}
