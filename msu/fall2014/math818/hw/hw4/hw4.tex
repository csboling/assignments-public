\documentclass{article}

\usepackage{amsmath}
\usepackage{amsfonts}
\usepackage{amssymb}
\usepackage{enumerate}
\usepackage{mathtools}
\usepackage{xfrac}
\usepackage[lastexercise]{exercise}

\DeclarePairedDelimiter\floor{\lfloor}{\rfloor}

\newcounter{Problem}
\newenvironment{Problem}{\begin{Exercise}[name={Problem},
                                          counter={Problem}]}
                        {\end{Exercise}}
\title{MATH 818 Homework \#4}
\date{November 19, 2014}
\author{Sam Boling}

\begin{document}

\begin{titlepage}
\maketitle
\end{titlepage}

\begin{Problem}
Show that if $P$ is a $p$-Sylow subgroup of a finite group $G$, then
$N_G(N_G(P)) = N_G(P)$, where $N_G(H)$ denotes the normalizer of $H$
in $G$.
\end{Problem}

\begin{Answer}
Note that
$$
  N_G(S)
= \{ g \in G \mid g S g^{-1} = S \}
= \{ g \in G \mid g . S = S \}
$$
gives the stabilizer of the group action $.$ of $G$ on its subgroups by
conjugation, and so $[G : N_G(P)] = |G . P|$, where
$$
G . P = \{ H < G \mid \exists g \in G, H = gPg^{-1} \},
$$
is the orbit of $P$ under this action.
All conjugates have the same order, so each $gPg^{-1}$ has the order
$|P|$ and is thus a $p$-Sylow since $P$ is. But all $p$-Sylows are
conjugate to $P$, and so $P$ has exactly one orbit under
conjugation of subgroups. Then $[G : N_G(P)] = |G . P| = 1$.

Since $N_G(P) < N_G(N_G(P))$,
$[G : N_G(P)] \geq [G : N_G(N_G(P))]$. But if
$[G : N_G(P)] > [G : N_G(N_G(P))]$ then this means $[G : N_G(N_G(P))] = 0$, which
is not possible. Therefore $[G : N_G(P)] = [G : N_G(N_G(P))]$ and so
$|N_G(N_G(P))| = |N_G(P)|$, and since $N_G(P)$ is a subgroup of
$N_G(N_G(P))$ we must have $N_G(N_G(P)) = N_G(P)$.
\end{Answer}

\pagebreak

\begin{Problem}
Let $G$ be a finite group. Consider the sequence of subgroups of $G$
defined inductively by
$$
Z_0(G) = \{ 1 \}, \quad
Z_{i+1}(G) = \pi^{-1}(Z(G/Z_i(G)))
$$
where $\pi : G \to G / Z_i(G)$ is the natural quotient map. Then $G$
is called \emph{nilpotent} if $Z_n(G) = G$ for some $n \geq 0$.

\begin{enumerate}[(a)]
  \item{
    Show that for $i \geq 0$,
    $$
    Z_{i+1}(G) = \{ x \in G
                   \mid
                   xyx^{-1}y^{-1} \in Z_i(G), \forall y \in G
                \}.
    $$
  }
  \item{
    Show that if $G$ is nilpotent then $G$ is solvable.
  }
  \item{
    Show that a $p$-group, where $p$ is prime, is always nilpotent.
  }
  \item{
    Give an example of a group which is not nilpotent.
  }
  \item{
    Assume $G$ is nilpotent. Show that if $H$ is a proper subgroup of $G$, then
    the normalizer of $H$ in $G$ is always strictly larger than $H$.
  }
  \item{
    Suppose that $G$ is nilpotent and $P$ a $p$-Sylow subgroup of
    $G$. Show that $P$ is normal in $G$. (Hint: Use problem 1.)
  }
  \item{
    Show that $G$ is nilpotent if and only if it is isomorphic to the
    direct product of its $p$-Sylow subgroups. (Hint: recall that if
    $P$ and $Q$ are both normal in $G$ and $P \cap Q = \{ 1 \}$, then
    $ab = ba$ for all $a \in P$, $b \in Q$.)
  }
\end{enumerate}
\end{Problem}

\begin{Answer}
\begin{enumerate}[(a)]
  \item{
    Let $x \in Z_{i+1}(G)$ and let $y \in G$. Then
    $\pi(x) = x Z_{i}(G) \in Z(G / Z_i(G))$,
    so $\pi(x) \pi(y) = \pi(y) \pi(x)$. But then
    $\pi(x)\pi(y)\pi(x)^{-1}\pi(y)^{-1} = \pi(1)$, and since $\pi$ is
    a homomorphism this means
    $$
      \pi(xyx^{-1}y^{-1})
    = xyx^{-1}y^{-1} Z_i(G)
    = \pi(1)
    = Z_i(G),
    $$
    and therefore $xyx^{-1}y^{-1} \in Z_i(G)$ for any $y \in
    G$. Therefore
    $$
    Z_{i+1}(G) \subset \{ x \in G
                         \mid
                         xyx^{-1}y^{-1} \in Z_i(G), \forall y \in G
                      \}.
    $$
    This argument reverses completely and so
    $$
    Z_{i+1}(G) =\{ x \in G
                  \mid
                  xyx^{-1}y^{-1} \in Z_i(G), \forall y \in G
               \}
    $$
    as desired.
  }
  \item{
    First we show that $Z_{i}(G) \triangleleft Z_{i+1}(G)$ for any
    $i$. Let $x \in Z_{i+1}(G)$. Then we know $xyx^{-1}y^{-1} \in
    Z_i(G)$ for any $y \in G$. Let $g \in x Z_{i}(G) x^{-1}$. Then
    $g = xzx^{-1}$ for some $z \in Z_{i}(G)$, so we have
    $gz^{-1} = xzx^{-1}z^{-1} \in Z_{i}(G)$. But then $g \in
    Z_{i}(G)$, so $x Z_{i}(G) x^{-1} \subset Z_{i}(G)$ and therefore
    $Z_{i}(G) \triangleleft Z_{i+1}(G)$.

    Next we have
    $$
      Z(G / Z_{i}(G))
    = \pi(Z_{i+1}(G))
    = Z_{i+1}(G) / Z_{i}(G).
    $$
    Since $Z_{i+1}(G) / Z_{i}(G)$ is the center of a group, it is
    abelian. Since $G$ is nilpotent by assumption, $G = Z_n(G)$ for
    some $n$, and so we have an abelian tower
    $$
    G = Z_n(G)
    \triangleright Z_{n-1}(G)
    \triangleright \cdots
    \triangleright Z_1(G)
    \triangleright Z_0(G) = \{ 1 \}.
    $$
    Therefore $G$ is solvable.
  }
  \item{
    Let $P$ be a $p$-group, where $p$ is a prime.
    Then $|P| = p^n$ for some $n > 0$.

    We claim that $Z_k(P)$ is a $p$-group for any $k > 0$.
    For $k = 1$ the claim holds since $Z(P)$ has a nontrivial order
    that divides $|P|$ and so $Z(P)$ is a $p$-group.
    Suppose $Z_{k-1}(P)$ is a $p$-group. Then (unless $P =
    Z_{k-1}(P)$, and then $Z_k(P) = P$) $P / Z_{k-1}(P)$ is as
    well and so
    $$
      |Z_k(P)|
    = \left|\pi^{-1}\left(Z\left(\frac{P}{Z_{k-1}(P)}\right)\right)\right|
    = \left|Z\left(\frac{P}{Z_{k-1}(P)}\right)\right|,
    $$
    and the center of $P / Z_{k-1}(P)$ is nontrivial and divides a
    power of $p$. Therefore $Z_k(P)$ is a $p$-group.

    Now consider
    $$
    \frac{Z_k(P)}{Z_{k-1}(P)} = Z\left(\frac{P}{Z_{k-1}(P)}\right).
    $$
    If $|P / Z_{k-1}(P)| = 1$, then $Z_{m}(P) = P$ for all $m \geq
    k-1$, as desired. Otherwise $P / Z_{k-1}(P)$ is a $p$-group,
    so it has nontrivial center and then
    $\left|\frac{Z_k(P)}{Z_{k-1}(P)}\right| > 1$, which means
    $|Z_k(P)| > |Z_{k-1}(P)|$. But then
    $|P / Z_k(P)|$ is strictly less than $|P / Z_{k-1}(P)|$, so the
    order of each such quotient is a strictly smaller power of
    $p$ than the last. Therefore this process must terminate in at
    most $n$ steps so that $|P / Z_n(P)| = 1$,
    and then $P = Z_n(P)$ as desired.
  }
  \item{
    Any nontrivial group $G$ with trivial center fails to be nilpotent.
    In this case
    $$
    Z_1(G) = \pi^{-1}(Z(G / Z(G))) = Z(G) = \{ 1 \},
    $$
    and similarly if
    $Z_i(G) = \{ 1 \}$ then $Z_{i+1}(G) = Z(G) = \{ 1 \}$. The free group in
    two generators is an example of such a group. A finite example is
    the symmetric group $S_3$.
  }
  \item{
    Let $G$ be nilpotent and let $H < G$ such that $H \neq G$.
    Then $G = Z_n(G)$ and $\{ 1 \} = Z_0(G)$, so there is some
    $k \in \{0, 1, \dots, n-1\}$ such that
    $Z_k(G) \subset H \subset Z_{k+1}(G)$ and $H \neq Z_{k+1}(G)$.

    Then $Z_{k+1}(G) - H$ is nonempty, so let $x \in Z_{k+1}(G) - H$.
    Since $x \in Z_{k+1}(G)$, the commutator
    $[x, y] \in Z_{k}(G) \subset H$ for all $y$. In particular this means
    $[x, h] \in H$ for any $h \in H$. But then
    $x h x^{-1} h^{-1} \in H$, so multiplying by $h$ on the right we
    have $x h x^{-1} \in H$, or $x h x^{-1} = h^\prime$ for some
    $h^\prime \in H$. But then $x h = h^\prime x$, and since $h$ is an
    arbitrary element of $H$ this means $x H = H x$. Therefore
    $x \in N_G(H)$, and since $x \notin H$ this means
    $N_G(H)$ properly contains $H$.
  }
  \item{
    Let $P$ be a $p$-Sylow of $G$. Then from problem 1, $N_G(N_G(P)) =
    N_G(P)$. Therefore from part (e), it cannot be the case that
    $N_G(P)$ is a proper subgroup of $G$, since in this case
    $|N_G(N_G(P))| > |N_G(P)|$. But then $N_G(P) = G$, so $gP = Pg$
    for every $g \in G$. Thus $P$ is normal in $G$.
  }
  \item{
    Let $G$ be a group, and let its order have prime factorization
    $|G| = p_1^{k_1} p_2^{k_2} \cdots p_n^{k_n}$.
    \begin{itemize}
      \item[$\implies$]{
        Suppose $G$ is nilpotent. Then from part (f) we have that each
        of its Sylow subgroups is normal, and therefore there is a
        unique $p_i$-Sylow for each prime factor $p_i$.

        Let $P_i$ and $P_j$ be distinct Sylow subgroups of $G$, so that $P_i$
        is a $p_i$-Sylow and $P_j$ is a $p_j$-Sylow, with $i \neq j$.
        Then $P_i \cap P_j$ is a subgroup of both $P_i$ and $P_j$, so
        $|P_i \cap P_j|$ divides $|P_i| = p_i^{k_i}$ and divides
        $|P_j| = p_j^{k_j}$, but since $p_i, p_j$ are distinct primes
        this is only possible of $|P_i \cap P_j| = 1$.

        Therefore $P_i$ and
        $P_j$ are normal in $G$ and have trivial intersection, so
        for any $p \in P_i$, $q \in P_j$ the commutator
        $p q p^{-1} q^{-1} \in P_i \cap P_j$ is 1, which means $pq =
        qp$.

        Note also that each $p_i$-Sylow has order
        $p_i^{k_i}$. Letting $P_i$ denote the unique $p_i$-Sylow, we have
        $$
          |P_1 P_2 \cdots P_n|
        = \frac{|P_1||P_2|\cdots|P_n|}
               {\left|\bigcap_{i=1}^n P_i\right|}
        = p_1^{k_1} p_2^{k_2} \cdots p_n^{k_n}
        = |G|,
        $$
        so we must have $P_1 P_2 \cdots P_n = G$. Therefore the
        following conditions are met for the Sylow subgroups of $G$:
        \begin{itemize}
          \item{$G = P_1 P_2 \cdots P_n$, i.e. each element in $G$ can
                be written as a product of elements
                $g_1, g_2, \dots, g_n$ drawn from $P_1, P_2, \dots,
                P_n$ respectively.
               }
          \item{$P_i \cap P_j = \{ 1 \}$ pairwise for every
                $i \neq j$.}
          \item{$P_i \triangleleft G$ for each Sylow subgroup $P_i$.}
        \end{itemize}
        Therefore we can write $G$ as the semidirect product
        $$
                            P_1
        \rtimes_{\psi_{1,2}}   P_2
        \rtimes_{\psi_{2,3}}   \cdots
        \rtimes_{\psi_{n-1, n}} P_n,
        $$
        where $\psi_{i,j} : P_i \to \mathrm{Inn}(P_j)$ is the
        conjugation action of $P_i$ on $P_j$ given by
        $\psi_{i,j}(g_i)(g_j) = g_i g_j g_i^{-1}$.

        Note however that for each such pair $i, j$ we have
        $g_i g_j = g_j g_i$ for all $g_i \in P_i$, $g_j \in P_j$, so
        this means $\psi_{i,j}(g_i) = \mathrm{id}_{P_j}$
        uniformly. But in this case the semidirect product is simply
        the direct product, so
        $$
        G = \prod_{i=1}^n P_i.
        $$
      }
      \item[$\impliedby$]{
        Suppose $G$ is isomorphic to the direct product of its Sylow
        subgroups. Each such subgroup is then a $p_i$-group for some
        $i$, so from part (c) we have that each Sylow subgroup is
        nilpotent. Therefore we show that the direct product of
        nilpotent groups is nilpotent, concluding that $G$ is nilpotent.

        Let $G$ and $H$ be nilpotent groups so that $G = Z_m(G)$,
        $H = Z_n(H)$ for some $m, n \in \mathbb{N}$. Note however
        that, for instance,
        \begin{align*}
          Z_{m+1}(G)
       &= \pi^{-1}\left(Z\left(\frac{G}{Z_{m}(G)}\right)\right)
        = \pi^{-1}\left(Z\left(\frac{Z_m(G)}{Z_m(G)}\right)\right) \\
       &= \pi^{-1}\left(Z(\{ 1 \})\right)
        = \pi^{-1}(1) \\
       &= \ker(\pi)
        = Z_{m}(G)
        \end{align*}
        and therefore $Z_k(G) = Z_m(G) = G$ for all $k \geq m$.
        Without loss of generality then let $m < n$ and write
        $G = Z_n(G)$, $H = Z_n(H)$.

        Next we show the following: for any groups $G$ and $H$ and
        any natural number $n$, if $G$ and $H$ are nilpotent of order
        at least $n$, then
        $G \times H = Z_n(G) \times Z_n(H) = Z_n(G \times H)$.

        Suppose $n = 0$. We have
        $$
          Z_0(G \times H)
        = \{ (1_G, 1_H) \}
        = \{ 1_G \} \times \{ 1_H \}
        = Z_0(G) \times Z_0(H)
        = G \times H.
        $$

        Suppose the claim holds for some $k$, and suppose
        $G$ and $H$ are nilpotent of order $k+1$.
        Let $(g, h) \in G \times H$.
        Then $g \in G = Z_{k+1}(G)$
        and $h \in H = Z_{k+1}(H)$ so
        $$
        g x g^{-1} x^{-1} \in Z_{k}(G), \quad
        h y h^{-1} y^{-1} \in Z_{k}(G), \quad
        \forall x, y \in H.
        $$
        Let $(x, y) \in G \times H$. Then
        \begin{align*}
             (g, h) (x, y) (g, h)^{-1} (x, y)^{-1}
         &=  (g x g^{-1} x^{-1}, h y h^{-1} y^{-1}) \\
        &\in Z_k(G) \times Z_k(H) = Z_k(G \times H)
        \end{align*}
        which means
        $(g, h)(x, y)(g, h)^{-1}(x, y)^{-1} \in Z_k(G \times H)$ and
        so $(g, h) \in Z_{k+1}(G \times H)$. Certainly
        $Z_{k+1}(G \times H) \subset G \times H$. Therefore
        $G \times H = Z_{k+1}(G \times H)$ and so $G \times H$ is
        nilpotent, as desired.
      }
    \end{itemize}
  }
\end{enumerate}
\end{Answer}

\pagebreak

\begin{Problem}
Consider the group given by generators and relations as follows:
$$
G = \langle
      x, y, z
    \mid
      x^3 = y^7 = z^{13} = 1,
      x^{-1} y x = y^4,
      x^{-1} z x = z^{-1},
      yz = zy
    \rangle.
$$
What can you say about the order of $G$?
\end{Problem}

\begin{Answer}
From the relations $y^7 = 1$ and $x^{-1} y x = y^4$ we have
\begin{align*}
   y
&= y \cdot y^7
 = (y^4)^2
 = x^{-1} y x x^{-1} y x
 = x^{-1} y^2 x
\end{align*}
so that $x = y^2 x y^{-1}$ or $xy = y^2 x$.
We also have that $x^{-1}zx = z^{-1}$ which gives
$xz = z^{-1}x$. This yields
\begin{align*}
(xy)^2 &= y^6 x^2 = y^{-1} x^{-1} = (xy)^{-1}, \\
(xy)^3 &= y^7 x^3 = 1, \\
(xz)^2 &= x^2 = x^{-1}, \\
(xz)^3 &= x^3z = z, \\
(xz)^4 &= zxz = x, \\
(xz)^5 &= xzx = z^{-1} x^2 = z^{-1} x^{-1} = (xz)^{-1}, \\
(xz)^6 &= 1.
\end{align*}
From $(xz)^3 = z$ and $(xz)^6 = 1$ we conclude that $z^2 = 1$, so
it follows from $z^{13} = 1$ that $z = z^{-1} = 1$
(since
$z^{12} = z^{-1} = z \implies z^{11} = 1 \implies z^{10} = z^{-1}
\implies \cdots$).
Therefore the representation reduces to
$$
\langle x, y \mid x^3 = y^7 = 1, yx = xy^4 \rangle.
$$
Note also that the rewriting rule $yx = xy^4$ means an arbitrary term
in $G$ can be rearranged into the form $x^k x^l$ for some exponents
$k, l$. After cancelling powers of
$x^3$ and $y^7$, such a term can contain at most $x^2$ and at most
$y^6$. This means the order of the group is at most $2 \cdot 6 + 1 = 13$,
accounting for the 2 possible powers of $x$, the 6 possible powers of
$y$, and the identity. But then $G \simeq \mathbb{Z}_{13}$.
\end{Answer}

\pagebreak

\begin{Problem}
The \emph{braid group} $B_3$ in 3 strings is given by generators and
relations as
$$
B_3 = \langle x, y \mid xyx = yxy \rangle.
$$

\begin{enumerate}[(a)]
  \item{
    Show that $xyxyxy$ is in the center of $B_3$.
  }
  \item{
    Prove that there exists a surjective homomorphism
    $\varphi : B_3 \to S_3$ such that $\varphi(x) = (12)$,
    $\varphi(y) = (23)$.
  }
  \item{
    Show that the kernel of $\varphi$ contains a group isomorphic to
    the free group with two generators $F(u,v)$.
  }
  \item{
    Show that there is a group homomorphism
    $$
    \psi : B_3
       \to \mathrm{PSL}_2(\mathbb{Z})
         = \mathrm{SL}_2(\mathbb{Z}) / \{ \pm I \}
    $$
    with
    $$
    \psi(x) =
    \left(\begin{array}{r r}
      1 & 1 \\
      0 & 1
    \end{array}\right), \quad
    \psi(y) =
    \left(\begin{array}{r r}
      1 & 0 \\
     -1 & 1
    \end{array}\right)
    $$
  }
  \item{
    Show that $\psi$ is surjective.
  }
\end{enumerate}
\end{Problem}

\begin{Answer}
\begin{enumerate}[(a)]
  \item{
    Assume that $xyx = yxy$. The following inductive arguments show
    that $xyx \cdot y^n = x^n \cdot xyx$ for any $n \in \mathbb{Z}$.
    If $n = 0$, then $xyx \cdot y^0 = x^0 \cdot xyx$. Let $k \in \mathbb{Z}_+$.
    \begin{itemize}
      \item{
        Suppose $xyx \cdot y^{k-1} = x^{k-1} \cdot xyx$. Then
        \begin{align*}
             xyx \cdot y^k
          &= xyx \cdot y \cdot y^{k-1}
           = x \cdot yxy \cdot y^{k-1} \\
          &= x \cdot xyx \cdot y^{k-1}
           = x \cdot x^{k-1} xyx       \\
          &= x^k \cdot xyx
        \end{align*}
        as desired.
      }
      \item{
        Suppose $xyx \cdot y^{-(k-1)} = x^{-(k-1)} \cdot xyx$.
        Then
        \begin{align*}
             xyx \cdot y^{-(k-1)}
          &= x^{-(k-1)} \cdot xyx
           = x^{-k} \cdot x \cdot xyx \\
          &= x^{-k} \cdot x \cdot yxy
           = x^{-k} \cdot xyx \cdot y.
        \end{align*}
        But $xyx \cdot y^{-(k-1)} = xyx \cdot y^{-k} \cdot y$, so we
        have
        $$
        xyx \cdot y^{-k} \cdot y = x^{-k} \cdot xyx \cdot y
        $$
        so it must be the case that
        $xyx \cdot y^{-k} = x^{-k} \cdot xyx$ as desired.
      }
    \end{itemize}
    Therefore $xyx \cdot y^n = x^n \cdot xyx$. Replacing every $x$ by
    $y$ and vice-versa in the above argument shows we also have
    $xyx \cdot x^n = y^n \cdot xyx$ for any $n$.

    This means that
    \begin{align*}
        xyxyxy \cdot x^n
     &= xyx \cdot y^n \cdot yxy
      = x^n \cdot xyxyxy, \\
        xyxyxy \cdot y^n
     &= xyx \cdot x^n \cdot yxy
      = y^n \cdot xyxyxy,
    \end{align*}
    and it follows that $xyxyxy$ commutes with every element of $G$.
  }
  \item{
    Let $\bar{\varphi} : \{ x, y \in B_3 \} \to S_3$ be the map given by
    $$
    \bar{\varphi}(x) = (12), \quad
    \bar{\varphi}(y) = (23).
    $$
    We have that
    $$
    (12)(23)(12) = (13) = (23)(12)(23),
    $$
    so
    $\bar{\varphi}(x)\bar{\varphi}(y)\bar{\varphi}(x) = \bar{\varphi}(y)\bar{\varphi}(x)\bar{\varphi}(y)$,
    and thus the
    homomorphism $\varphi : B_3 \to S_3$ that extends $\bar{\varphi}$
    is well-defined.  Since $(12), (23), (13)$ are all the
    transpositions in $S_3$, these generate $S_3$,
    and so $\varphi$ is surjective.
  }
  \item{
    Note that
    $$
    \varphi(x^2) = \varphi(x)\varphi(x) = (12)(12) = \mathrm{id}
    $$
    and
    $$
    \varphi(y^2) = \varphi(y)\varphi(y) = (23)(23) = \mathrm{id}
    $$
    so that $x^2, y^2 \in \ker(\varphi)$ and similarly
    $x^{-2}, y^{-2} \in \ker(\varphi)$, so
    $\langle x^2, y^2 \rangle < \ker(\varphi)$.

    Consider the unique group homomorphism
    $f : F(a,b) \to \langle x^2, y^2 \rangle$ such that
    $f(a) = x^2$, $f(b) = y^2$. Since $\langle x^2, y^2 \rangle$ is
    generated by $x^2, y^2$ by construction, $f$ is surjective.
    Let $z \in \ker(f)$. Then $f(z) = 1$, so either
    $z = 1$ or $f(z) = xyxy^{-1}x^{-1}y^{-1}$. But the latter case is
    not possible, since the image of $f$ contains only even powers of
    $x$ and $y$ and their products. Therefore $f$ is injective, so
    $F(a, b) \simeq \langle x^2, y^2 \rangle$.
  }
  \item{
    \begin{align*}
    \psi(x)\psi(y)\psi(x)
    &=
    \left(\begin{array}{r r}
      1 & 1 \\
      0 & 1
    \end{array}\right)
    \left(\begin{array}{r r}
      1 & 0 \\
     -1 & 1
    \end{array}\right)
    \left(\begin{array}{r r}
      1 & 1 \\
      0 & 1
    \end{array}\right)
    =
    \left(\begin{array}{r r}
      1 & 1 \\
      0 & 1
    \end{array}\right)
     \left(\begin{array}{r r}
      1 & 1 \\
     -1 & 0
    \end{array}\right)
    \\ &=
    \left(\begin{array}{r r}
      0 & 1 \\
     -1 & 0
    \end{array}\right)
    \end{align*}
    while
    \begin{align*}
    \psi(y) \psi(x) \psi(y)
    &=
    \left(\begin{array}{r r}
      1 & 0 \\
     -1 & 1
    \end{array}\right)
    \left(\begin{array}{r r}
      1 & 1 \\
      0 & 1
    \end{array}\right)
    \left(\begin{array}{r r}
      1 & 0 \\
     -1 & 1
    \end{array}\right)
    =
    \left(\begin{array}{r r}
      1 & 0 \\
     -1 & 1
    \end{array}\right)
    \left(\begin{array}{r r}
      0 & 1 \\
     -1 & 1
    \end{array}\right)
    \\ &=
    \left(\begin{array}{r r}
      0 & 1 \\
     -1 & 0
    \end{array}\right)
    \end{align*}
    Thus the homomorphism $\psi : B_3 \to \mathrm{PSL}_2(\mathbb{Z})$
    that extends
    $$
    \psi(x) =
    \left(\begin{array}{r r}
      1 & 1 \\
      0 & 1
    \end{array}\right), \quad
    \psi(y) =
    \left(\begin{array}{r r}
      1 & 0 \\
     -1 & 1
    \end{array}\right)
    $$
    is well-defined.
  }
%\item{
% Observing that the map
% $\bar{\gamma} : \mathrm{PSL}_2(\mathbb{Z}) \to (\mathbb{C} \to \mathbb{C})$ given by
% $$
% \bar{\gamma}\left(
% \left[\begin{array}{c c}
%   a & b \\
%   c & d
% \end{array}\right]
% \right)(z)
% =
% \frac{a z + b}{c z + d}
% $$
% is injective when $ad - bc = 1$. Let $\Gamma =
% \mathrm{Im}(\bar{\gamma})$. Note that
%
% Therefore we have an isomorphism
% $\gamma : \mathrm{PSL}_2(\mathbb{Z}) \to \Gamma$, with group
% operation in $\Gamma$ given by composition of maps. Therefore showing that
% $\psi$ is surjective is equivalent to showing that the maps
% $$
% \gamma\left(
% \left[\begin{array}{r r}
%   1 & 1 \\
%   0 & 1
% \end{array}\right]\right)(z)
% = 1 + z, \quad
% \gamma\left(
% \left[\begin{array}{r r}
%   1 & 0 \\
%  -1 & 1
% \end{array}\right]\right)(z)
% = \frac{z}{1 - z}
% $$
% generate the group $\Gamma$.
%}
\end{enumerate}
\end{Answer}

\end{document}
