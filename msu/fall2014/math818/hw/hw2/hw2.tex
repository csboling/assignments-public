\documentclass{article}

\usepackage{amsmath}
\usepackage{amsfonts}
\usepackage{amssymb}
\usepackage{enumerate}
\usepackage[lastexercise]{exercise}

\newcounter{Problem}
\newenvironment{Problem}{\begin{Exercise}[name={Problem},
                                          counter={Problem}]}
                        {\end{Exercise}}
\title{MATH 818 Homework \#2}
\date{October 6, 2014}
\author{Sam Boling}

\begin{document}

\begin{titlepage}
\maketitle
\end{titlepage}

\begin{Problem}
Start with the data $(K, L, \psi)$ of two groups $K$ and $L$ and a
group homomorphism $\psi : K \to \mathrm{Aut}(L)$. Show that the
Cartesian product $L \times K$ becomes a group with operation
$$
(l, k) \cdot (l^\prime, k^\prime) = (l\psi(k)(l^\prime), k k^\prime).
$$
\end{Problem}

\begin{Answer}
The product so defined yields a group:
\begin{itemize}
  \item[(Associativity)]
    {Let $l_1, l_2, l_3 \in L$ and $k_1, k_2, k_3 \in K$. Then
     \begin{align*}
       (l_1, k_1) \cdot ((l_2, k_2) \cdot (l_3, k_3))
    &= (l_1, k_1) \cdot
       (l_2 \psi(k_2) (l_3), k_2 k_3) \\
    &= (l_1 \psi(k_1)(l_2 \psi(k_2) (l_3)), k_1 k_2 k_3)
    \end{align*}
    while
    \begin{align*}
      ((l_1, k_1) \cdot (l_2, k_2)) \cdot (l_3, k_3)
   &= (l_1 \psi(k_1)(l_2), k_1 k_2) \cdot (l_3, k_3) \\
   &= (l_1 \psi(k_1)(l_2)
       \psi(k_1 k_2) (l_3), k_1 k_2 k_3).
    \end{align*}
    so the operation is associative in the second component.
    Furthermore since $\psi(k)$ is a homomorphism for all $k \in K$,
    \begin{align*}
     l_1 \psi(k_1)(l_2 \psi(k_2)(l_3))
  &= l_1 \psi(k_1)(l_2) \psi(k_1)(\psi(k_2)(l_3)) \\
  &= (l_1 \psi(k_1)(l_2))
      (\psi(k_1) \circ \psi(k_2))(l_3)\\
  &= (l_1 \psi(k_1)(l_2))
      \psi(k_1 k_2)(l_3)
     \end{align*}
     and so the first component agrees as well. Therefore the
     operation is associative.
    }
  \item[(Closure)]
    {
      Since $k \in K$, $\psi(k) \in \mathrm{Aut}(L)$, so
      since $l^\prime \in L$ this means $\psi(k)(l^\prime) \in
      L$. Therefore $l \psi(k) l^\prime \in L$ since $L$ is a group
      and furthermore $k k^\prime \in K$, so
      $(l\psi(k)(l^\prime), k k^\prime) \in L \times K$ and thus the
      set is closed under this operation.
    }
  \item[(Unit)]
    {
      Let $1_L$ and $1_K$ be the identity elements of groups $L$ and
      $K$, respectively, and let $l \in L$, $k \in K$. Then
      $$
      (l, k) \cdot (1_L, 1_K)
    = (l \psi(k)(1_L), k 1_K)
    = (l 1_L, k 1_K)
    = (l, k),
      $$
      since $\psi(k)$ is a homomorphism $L \to L$ which requires
      $\psi(1_L) = 1_L$. Furthermore
      $$
      (1_L, 1_K) \cdot (l, k)
    = (1_L \psi(1_K)(l), 1_K k)
    = (1_L \mathrm{id}(l), 1_K k)
    = (1_L l, 1_K k)
    = (l, k)
      $$
      since $\psi$ is a homomorphism $K \to \mathrm{Aut}(L)$ which
      requires $\psi(1_K) = 1_{\mathrm{Aut}(L)} = \mathrm{id}$.

      Therefore $(1_L, 1_K)$ is an identity for the product so defined.
    }
    \item[(Inverses)]
      {Consider that
       \begin{align*}
         (l, k) \cdot (\psi(k^{-1})(l^{-1}), k^{-1})
      &= (l \psi(k)(\psi(k^{-1})(l^{-1}), k k^{-1}) \\
      &= (l \psi(k)(\psi(k)^{-1}(l^{-1})), k k^{-1}) \\
      &= (l (\psi(k) \circ \psi(k)^{-1})(l^{-1}), k k^{-1}) \\
      &= (l l^{-1}, k k^{-1}) \\
      &= (1_L, 1_K),
       \end{align*}
       and similarly
       \begin{align*}
          (\psi(k^{-1})(l^{-1}), k^{-1}) \cdot (l, k)
       &= (\psi(k^{-1})(l^{-1}) \psi(k^{-1})(l), k^{-1} k) \\
       &= (\psi(k^{-1})(l^{-1} l), k^{-1} k) \\
       &= (\psi(k^{-1})(1_L), 1_K) \\
       &= (1_L, 1_K)
       \end{align*}
       so the operation has inverses of this form.
      }
\end{itemize}
Therefore $L \times K$ is a group with this operation.
\end{Answer}

\pagebreak

\begin{Problem}
Let $G$ be a group, $H < G$, $N \triangleleft G$. Suppose that
$H \cap N = \{ 1 \}$ and that $G = NH$.
\begin{enumerate}[(a)]
  \item{Show that the map $N \times H \to G$ given by
        $(n, h) \mapsto nh$ is a bijection.}
  \item{Show that the map $H \to \mathrm{Aut}(N)$ given by
      $h \mapsto c_h$, where $c_h(n) = h n h^{-1}$ is conjugation by
      $h$, is a group homomorphism.}
  \item{Observe that
        $$
        n_1 h_1 n_2 h_2
      = n_1 h_1 n_2 h_1^{-1} h_1 h_2
      = n_1 c_{h_1}(n_2) h_1 h_2
        $$
        and show that there is a group isomorphism between $G$ and the
        semidirect product $N \rtimes_c H$ (as in problem 1).}
\end{enumerate}
\end{Problem}

\begin{Answer}
\begin{enumerate}[(a)]
  \item{Since $G = NH = \{ nh \mid n \in N, h \in H \}$,
        the map $\varphi((n, h)) = nh$ is surjective.

        Next let $(n, h) \in \ker \varphi \subset N \times H$. Then
        $nh = 1$, so $n = h^{-1}$ and $h = n^{-1}$. But if
        $n = h^{-1}$, then since $H$ is a subgroup $n \in H$
        and thus $n \in N \cap H = \{1\}$.
        Similarly if $h = n^{-1}$, then
        since $N$ is a subgroup $h \in N$ so $h \in N \cap H = \{1\}$.
        Therefore $(n, h) = (1, 1)$, so $\ker \varphi$ is trivial.
        Therefore $\varphi$ is injective.

        Then $\varphi$ is bijective.
       }
  \item{
    Letting $\chi : H \to \mathrm{Aut}(N)$ be given by
    $\chi(h) = c_h$, we have
    \begin{align*}
     \chi(h h^\prime) =  c_{h h^\prime}(n)
  &= (h h^\prime) n (h h^\prime)^{-1} \\
  &= h h^\prime n (h^\prime)^{-1} h^{-1} \\
  &= h c_{h^\prime}(n) h^{-1}
  &= c_h (c_{h^\prime} (n)) \\
  &= (c_h \circ c_{h^\prime})(n) = \chi(h) \chi(h^\prime),
    \end{align*}
    so this is a group homomorphism.
  }
  \item{
    Take $\varphi$ as defined in part (a). We first show that
    $\varphi$ is a homomorphism:
    $$
    \varphi((n_1, h_1) \cdot (n_2, h_2))
  = \varphi((n_1 n_2, h_1 h_2))
  = n_1 n_2 h_1 h_2
    $$

    Since $\varphi$ is
    bijective it has an inverse $\varphi^{-1} : NH \to N \times H$, so
    \begin{align*}
    \varphi^{-1}(n_1 h_1 n_2 h_2)
 &= \varphi^{-1}(n_1 c_{h_1}(n_2) h_1 h_2) \\
 &= (n_1 c_{h_1}(n_2), h_1 h_2) \\
 &= (n_1 c(h_1)(n_2), h_1 h_2) \\
 &= (n_1, h_1) \cdot (n_2, h_2) \\
 &= \varphi^{-1}(n_1 h_1) \varphi^{-1}(n_2 h_2)
    \end{align*}
    where the product
    $\cdot$ is taken from the group $N \rtimes_c H$. Since $G = NH$,
    every $g \in G$ has the form $g = nh$ for some $n \in N$ and $h
    \in H$, so this identity means
    $\varphi^{-1} : G \to N \rtimes_{\psi} H$ is a group
    homomorphism, and since it is invertible this means
    $G \simeq N \rtimes_{\psi} H$.
  }
\end{enumerate}
\end{Answer}

\pagebreak

\begin{Problem}
Suppose that $H$ and $N$ are normal subgroups of a finite group $G$.
\begin{enumerate}[(a)]
  \item{Show that $HN = \{ hn \mid h \in H, n \in H \}$
        is a subgroup of $G$.}
  \item{Now suppose in addition that the orders of $H$ and $N$ are
      relatively prime, i.e. that $\mathrm{gcd}(|H|,|N|) = 1$. Show
      that $hn = nh$ for all $h \in H$, $n \in N$, and that $HN$ is
      isomorphic to the direct product $H \times N$.}
\end{enumerate}
\end{Problem}

\begin{Answer}
\begin{enumerate}[(a)]
  \item{We show that $HN$ contains the unit and is closed under the
      group operations.
  \begin{itemize}
    \item[(Unit)]
    {
      Since $H$ and $N$ are subgroups, $1 \in H$ and $1 \in N$, so
      $1 = 1 \cdot 1 \in HN$.
    }
    \item[(Product)]
    {
      Let $x_1, x_2 \in HN$. Then $x_1 = h_1 n_1$ and $x_2 = h_2 n_2$
      for some $h_1, h_2 \in H$, $n_1, n_2 \in N$. Therefore
      \begin{align*}
        x_1 \cdot x_2 &= h_1 n_1 h_2 n_2 = h_1 h_2 n_1^\prime n_2
      \end{align*}
      since $N \triangleleft G$ and thus $n_1 h_2 = h_2 n_1^\prime$
      for some $n_1^\prime \in N$. But this means $x_1 \cdot x_2 \in HN$ so
      $HN$ is closed under the group product inherited from $G$.
    }
    \item[(Inverses)]
    {
      Let $x \in HN$. Then $x = hn$ for some $h \in H$, $n \in N$.
      Then $x^{-1} = (nh)^{-1} = h^{-1} n^{-1}$. But since $n$ is
      normal in $G$ and since $h^{-1} \in G$, $h^{-1} n^{-1} =
      n^\prime h^{-1}$ for some $n^\prime \in N$, and thus
      $x = n^\prime h^{-1} \in NH$. Therefore the inverse of each
      element in $NH$ is also in $NH$.
    }
  \end{itemize}
  Therefore $NH < G$.
  }
  \item
  {
    Consider the group action of $N$ on $H$ given by
    $$
    . : N \times H \to H, \quad
    n . h = n h n^{-1}.
    $$
    Then since $H$ is the disjoint union of the orbits $N . h$ of this
    action, the order of $H$ is
    \begin{align*}
    |H| &= \sum_h [N : N . h] = \sum_h \frac{|N|}{|N . h|} \\
        &= |N| \sum_h \frac{1}{|N . h|}
    \end{align*}
    where the sum is taken over representative elements $h \in H$.
    But $|H|$ and $|N|$ are coprime by assumption, so since $|H|$ divides
    $|N| \sum_h \frac{1}{|N . h|}$ this means $|H|$ divides
    $\sum_h \frac{1}{|N . h|}$ Therefore it must be the case that
    $\sum_h \frac{1}{|N . h|} \geq |H|$, which is only possible if
    $|N . h| = 1$ for all $h \in H$. But since $N / N_h \simeq N . h$
    this means $\frac{|N|}{|N_h|} = 1$ for every $h$ and therefore
    that $N_h = N$, or
    $$
    \{ n \in N \mid n h n^{-1} = h \} = N
    $$
    for each $h \in H$, so $n h n^{-1} = h$ for all $n \in N, h \in H$
    as desired.

    Furthermore consider the function
    $\psi : H \times N \to HN$ given by $\psi((h, n)) = h n$. Then we
    have
    \begin{align*}
       \psi((h_1, n_1) \cdot (h_2, n_2))
    &= \psi((h_1 h_2, n_1 n_2)) \\
    &= h_1 h_2 n_1 n_2.
    \end{align*}
    But we have just shown that $h_2 n_1 = n_1 h_2$, so this means
    \begin{align*}
       \psi((h_1, n_1) \cdot (h_2, n_2))
    &= h_1 n_1 h_2 n_2 \\
    &= \psi((h_1, n_1)) \psi((h_2, n_2))
    \end{align*}
    and thus that $\psi$ is a homomorphism. Furthermore this function
    is bijective as shown in Problem 2(a), so we have an isomorphism
    $H \times N \to HN$.
  }
\end{enumerate}
\end{Answer}

\pagebreak

\begin{Problem}
\begin{enumerate}[(a)]
  \item{Suppose that $Z(G)$ is the center of the group $G$. Prove that if
      the quotient $G / Z(G)$ is cyclic, then $G$ is abelian.}
  \item{Prove that every group of order $p^2$, where $p$ is prime, is
      abelian.}
  \item{For each prime $p$, give an example of a group of order $p^3$
      which is not abelian.}
\end{enumerate}
\end{Problem}

\begin{Answer}
\begin{enumerate}[(a)]
  \item{
    Let $Z = Z(G)$, and suppose $G / Z$ is cyclic. Then for any
    $g \in G$, there exists an $aZ \in G / Z$ such that
    $gZ = (aZ)^n = a^n Z$ (since $Z$ is normal) for some $n$, so that
    $g = a^n z$ for some $z \in Z$. Then we have
    \begin{align*}
    gh &= a^m z a^n z^\prime
        = a^{m + n} z z^\prime \\
       &= z^\prime a^{m + n} z
        = z^\prime a^{n + m} z
        = z^\prime a^n a^m z \\
       &= a^n z^\prime a^m z = h g,
    \end{align*}
    so $G$ is abelian.
  }
  \item{
    Let $G$ be a group of order $p^2$. Then since $[G : C_x]$ divides
    $|G|$ for each $x \in G$, $[G : C_x]$ divides $p^2$ and so
    $[G : C_x] \in \{ 1, p, p^2 \}$ for each $x$. Since
    $1 \in Z(G)$, $|Z(G)| \neq 0$ and so $[G : C_x] \neq p^2$
    for any $x$. Then the class equation can be written in this case
    to give
    $$
    p^2 = n p + |Z(G)|
    $$
    or
    $$
    p(p - n) = |Z(G)|
    $$
    and therefore $p$ divides $|Z(G)|$. Therefore either $|Z(G)| =
    p^2$,
    in which case $G = Z(G)$ and $G$ is abelian, or $|Z(G)| =
    p$.
    But in the latter case
    $$
    |\frac{G}{Z(G)}| = \frac{|G|}{|Z(G)|} = p
    $$
    for $p$ prime, and since $\frac{G}{Z(G)}$ is of prime order it
    must be cyclic. The result from part (a) then implies that $G$
    is abelian.
  }
  \item{
    Let $p$ be prime and let $G$ be the subgroup of
    $GL_3(\mathbb{Z}_p)$ given by matrices of the form
    $$
    \left[\begin{array}{c c c}
      1     & a_{12} & a_{13} \\
      0     & 1     & a_{23} \\
      0     & 0     & 1
    \end{array}\right].
    $$
    We verify that this is a group since
    $$
    \left[\begin{array}{c c c}
      1 & a_{12} & a_{13} \\
      0 & 1     & a_{23} \\
      0 & 0     & 1
    \end{array}\right]
    \left[\begin{array}{c c c}
      1 & b_{12} & b_{13} \\
      0 & 1     & b_{23} \\
      0 & 0     & 1
    \end{array}\right]
    =
    \left[\begin{array}{c c c}
      1 & a_{12} + b_{12} & a_{13} + b_{13} + a_{12} b_{23} \\
      0 & 1             & a_{23} + b_{23} \\
      0 & 0             & 1
    \end{array}\right] \in G
    $$
    and
    $$
    \left[\begin{array}{c c c}
      1 & a_{12} & a_{13} \\
      0 & 1     & a_{23} \\
      0 & 0     & 1
    \end{array}\right]^{-1}
    =
    \left[\begin{array}{c c c}
      1 & -a_{12} & a_{12} a_{23} - a_{13} \\
      0 & 1      & -a_{23}              \\
      0 & 0      & 1
    \end{array}\right] \in G.
    $$
    However, this group is not abelian since
    \begin{align*}
    &\left[\begin{array}{c c c}
       1 & b_{12} & b_{13} \\
       0 & 1     & b_{23} \\
       0 & 0     & 1
     \end{array}\right]
     \left[\begin{array}{c c c}
       1 & a_{12} & a_{13} \\
       0 & 1     & a_{23} \\
       0 & 0     & 1
     \end{array}\right] \\
    =&
    \left[\begin{array}{c c c}
      1 & a_{12} + b_{12} & a_{13} + b_{13} + a_{23} b_{12} \\
      0 & 1             & a_{23} + b_{23} \\
      0 & 0             & 1
    \end{array}\right] \\
    \neq&
    \left[\begin{array}{c c c}
      1 & a_{12} & a_{13} \\
      0 & 1     & a_{23} \\
      0 & 0     & 1
    \end{array}\right]
    \left[\begin{array}{c c c}
      1 & b_{12} & b_{13} \\
      0 & 1     & b_{23} \\
      0 & 0     & 1
    \end{array}\right].
    \end{align*}
    There is however a bijection
    $f : G \to \mathbb{Z}_p \times \mathbb{Z}_p \times \mathbb{Z}_p$
    given by
    $$
    \left[\begin{array}{c c c}
      1 & a_{12} & a_{13} \\
      0 & 1     & a_{23} \\
      0 & 0     & 1
    \end{array}\right]
    \mapsto
    (a_{11}, a_{13}, a_{23})
    $$
    and so this group has order
    $|\mathbb{Z}_p \times \mathbb{Z}_p \times \mathbb{Z}_p| = p^3$.
  }
\end{enumerate}
\end{Answer}

\pagebreak
\begin{Problem}
\begin{enumerate}[(a)]
  \item{Suppose that $G$ is a finite group and $H$ a proper subgroup
      of $G$. Show that there is at least one element of $G$ which is
      not conjugate to any element of $H$.}
  \item{Let the group $G$ act on the set $X$ and denote by $G_x$ the
      stabilizer of $x$. Show that if $y = g . x$, then $G_y = g
      G_x g^{-1}$.}
  \item{Suppose that a finite group $G$ acts on a set $X$ transitively
      (this means that all elements of $X$ lie in the same
      $G$-orbit). Assuming that $X$ has at least two elements, show
      that there is $g \in G$ such that $g . x \neq x$, for all $x
      \in X$,
      i.e. such that $g$ ``leaves no point fixed''.}
\end{enumerate}
\end{Problem}

\begin{Answer}
  \begin{enumerate}[(a)]
    \item{
      We wish to show that
      $$
      G \neq U = \bigcup_{g \in G} g H g^{-1},
      $$
      the union of all conjugates of $H$ in $G$. Since this conjugation is
      a group action with its stabilizer given by the normalizer
      $N(H)$, there are $[G : N(H)]$ such conjugates. If $|N(H)| = |G|$
      then this means $H$ is the only conjugate so $H = U$, in which
      case $G \neq U$ since $H$ is a proper subgroup by assumption.

      Otherwise, we have more than one conjugate, and since each
      conjugate must have identity as a common element, and since all
      conjugates are of the same order $|g H g^{-1}| = |H|$, this means
      $|U| < [G : N(H)]|H|$. Therefore
      $$
      |U| < [G : N(H)]|H| \leq [G : H] |H| = |G|
      $$
      since $[G : N(H)] \leq [G : H]$ because $H \subset N(H)$.
    }
    \item{
      We write
      $$
      G_y = \{ h \in G \mid h . y = y \}.
      $$
      Let $h \in G_y$. Then
      \begin{align*}
           & h . y = y \\
      \iff & h . (g . x) = g . x \\
      \iff & h g . x = g . x \\
      \iff & g^{-1} h g . x = x,
      \end{align*}
      so $h \in g^{-1} G_x g$. Therefore $G_y = g^{-1} G_x g$.
    }
    \item{
      If $G$ acts transitively on $X$, then there is only one orbit,
      so
      $$
      X = \coprod_x G . x = G . x_0
      $$
      for some $x_0 \in X$.

      Let $x \in X$. Then there is only one orbit, so
      $x = g_x . x_0$ for some $g_x \in G$. Then from part (b) we have
      that $G_x = g_x G_{x_0} g_x^{-1}$, and therefore for all $g \in
      G_x$ we have that $g$ is conjugate to $G_{x_0}$. But
      $$
      |X| = |G . x_0|[G : G_{x_0}] = \frac{|G|}{|G_{x_0}|}
      $$
      so that $|G_{x_0}| |X| = |G|$. But if $X$ contains more than one
      element then this means $|G_{x_0}| < |G|$, in which case
      $G_{x_0}$ is a proper normal subgroup. Therefore from part (a)
      there is an element of $G$ which is not conjugate to $G_{x_0}$
      and therefore cannot lie in $G_x$. Since $x$ was chosen
      arbitrarily this is true for every element of $X$, as desired.
    }
  \end{enumerate}
\end{Answer}

\pagebreak
\begin{Problem}
Let $G$ be a $p$-group where $p$ is a prime. Suppose that $x$ is an
element of order $p$ in $G$ such that the subgroup $\langle x \rangle$
generated by $x$ is normal in $G$. Show that $x$ belongs to the center
$Z(G)$ of $G$.
\end{Problem}

\begin{Answer}
Consider the group action $\psi : G \to \mathrm{Perm}(S)$ given by
$$
\psi(g)(x^i) = g x^i g^{-1},
$$
where $S$ is the set of powers $x^i$ of $x$. Note that the normality
of $\langle x \rangle$ ensures that the image of this map is inside of
$S$, since $\langle x \rangle \triangleleft G$ means that
$g x^i g^{-1} = x^j$ for any $g \in G$.

Decomposing the set $S$ into its orbits under this action we have
\begin{align*}
|S| &= \sum_{i=1}^p |G . x^i| = \sum_{i=1}^p [G : G_{x^i}] \\
    &= \sum_{i=1}^{p-1} [G : G_{x^i}] + [G : G_{x^p}] \\
    &= \sum_{i=1}^{p-1} [G : G_{x^i}] + [G : G_1] \\
    &= \sum_{i=1}^{p-1} [G : G_{x^i}] + 1,
\end{align*}
and since $|S| = p$ this is only possible if $[G : G_{x^i}] = 1$ for
each $i \in \{ 1, \dots, p \}$, which means $[G : G_x] = 1$ and
therefore $gx = xg$ for all $g \in G$. Therefore $x \in Z(G)$.
\end{Answer}

\pagebreak
\begin{Problem}
\begin{enumerate}[(a)]
\item{Let $G$ be a finite group and $H$ a subgroup of $G$. Let $Q$ be
    a $p$-Sylow subgroup of $H$. Show that there is a $p$-Sylow
    subgroup $P$ of $G$ such that $P \cap H = Q$.}
\item{Let $G$ be a finite group and $P$ a $p$-Sylow subgroup of
    $G$. Further let $N$ be a normal subgroup of $G$. Prove that $P
    \cap N$ is a $p$-Sylow subgroup of $N$.}
\end{enumerate}
\end{Problem}

\begin{Answer}
\begin{enumerate}[(a)]
  \item{
    Since $Q$ is a $p$-subgroup of $G$, $Q$ is contained in $p$-Sylow
    $P$ of $G$. Then notice that $|P \cap H|$ divides $|H|$, so
    $P \cap H$ is a $p$-group of $H$. But $Q < P < P \cap H$,
    so $Q < P \cap H < H$. But since $Q$ is a $p$-Sylow of $H$ it is
    of maximal order, so since $P \cap H$ is a $p$-group this means
    $P \cap H = Q$.
  }
  \item{
    Let $Q < N$ be a $p$-Sylow. Then from part (a) there is a
    $p$-Sylow $P^\prime$ of $G$ such that $Q = P^\prime \cap N$. But
    since all $p$-Sylows are conjugate there is some $g \in G$ such
    that $P^\prime = g P g^{-1}$, so
    \begin{align*}
      Q &= g P g^{-1} \cap N \\
        &= \{ x \in G \mid x = g \pi g^{-1} = n,
                           \pi \in P, n \in N \} \\
        &= \{ x \in G \mid g^{-1} x g = \pi = g^{-1} n g \} \\
        &= \{ x \in G \mid g^{-1} x g = \pi = n \} \\
        &= g (P \cap N) g^{-1}.
    \end{align*}
    Since all conjugates have the same order, this means
    $|Q| = |P \cap N|$, but $Q$ is a $p$-Sylow of $N$ and this means
    $P \cap N$ is a $p$-Sylow of $N$.
  }
\end{enumerate}
\end{Answer}

\end{document}
