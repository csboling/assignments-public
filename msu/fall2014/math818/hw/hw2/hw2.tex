\documentclass{article}

\usepackage{amsmath}
\usepackage{amsfonts}
\usepackage{amssymb}
\usepackage{enumerate}
\usepackage[lastexercise]{exercise}

\newcounter{Problem}
\newenvironment{Problem}{\begin{Exercise}[name={Problem},
                                          counter={Problem}]}
                        {\end{Exercise}}
\title{MATH 818 Homework \#2}
\date{October 6, 2014}
\author{Sam Boling}

\begin{document}

\begin{titlepage}
\maketitle
\end{titlepage}

\begin{Problem}
Start with the data $(K, L, \psi)$ of two groups $K$ and $L$ and a
group homomorphism $\psi : K \to \mathrm{Aut}(L)$. Show that the
Cartesian product $L \times K$ becomes a group with operation
$$
(l, k) \cdot (l^\prime, k^\prime) = (l\psi(k)(l^\prime), k k^\prime).
$$
\end{Problem}

\begin{Answer}
The product so defined defines a group:
\begin{itemize}
  \item[(Associativity)]
    {Let $l_1, l_2, l_3 \in L$ and $k_1, k_2, k_3 \in K$. Then
     \begin{align*}
       (l_1, k_1) \cdot ((l_2, k_2) \cdot (l_3, k_3))
    &= (l_1, k_1) \cdot
       (l_2 \psi(k_2) (l_3), k_2 k_3) \\
    &= (l_1 \psi(k_1)(l_2 \psi(k_2) (l_3)), k_1 k_2 k_3)
    \end{align*}
    while
    \begin{align*}
      ((l_1, k_1) \cdot (l_2, k_2)) \cdot (l_3, k_3) 
   &= (l_1 \psi(k_1)(l_2), k_1 k_2) \cdot (l_3, k_3) \\
   &= (l_1 \psi(k_1)(l_2)
       \psi(k_1 k_2) (l_3), k_1 k_2 k_3).
    \end{align*}
    so the operation is associative in the second component.
    Furthermore since $\psi(k)$ is a homomorphism for all $k \in K$,
    \begin{align*}
     l_1 \psi(k_1)(l_2 \psi(k_2)(l_3))
  &= l_1 \psi(k_1)(l_2) \psi(k_1)(\psi(k_2)(l_3)) \\
  &= (l_1 \psi(k_1)(l_2))
      (\psi(k_1) \circ \psi(k_2))(l_3)\\
  &= (l_1 \psi(k_1)(l_2))
      \psi(k_1 k_2)(l_3)
     \end{align*}
     and so the first component agrees as well. Therefore the
     operation is associative.
    }
  \item[(Closure)]
    {
      Since $k \in K$, $\psi(k) \in \mathrm{Aut}(L)$, so
      since $l^\prime \in L$ this means $\psi(k)(l^\prime) \in
      L$. Therefore $l \psi(k) l^\prime \in L$ since $L$ is a group
      and furthermore $k k^\prime \in K$, so 
      $(l\psi(k)(l^\prime), k k^\prime) \in L \times K$ and thus the
      set is closed under this operation.
    }
  \item[(Unit)]
    {
      Let $1_L$ and $1_K$ be the identity elements of groups $L$ and
      $K$, respectively, and let $l \in L$, $k \in K$. Then
      $$
      (l, k) \cdot (1_L, 1_K)
    = (l \psi(k)(1_L), k 1_K)
    = (l 1_L, k 1_K)
    = (l, k),
      $$
      since $\psi(k)$ is a homomorphism $L \to L$ which requires
      $\psi(1_L) = 1_L$. Furthermore
      $$
      (1_L, 1_K) \cdot (l, k)
    = (1_L \psi(1_K)(l), 1_K k)
    = (1_L \mathrm{id}(l), 1_K k)
    = (1_L l, 1_K k)
    = (l, k)
      $$
      since $\psi$ is a homomorphism $K \to \mathrm{Aut}(L)$ which 
      requires $\psi(1_K) = 1_{\mathrm{Aut}(L)} = \mathrm{id}$.

      Therefore $(1_L, 1_K)$ is an identity for the product so defined.
    }
    \item[(Inverses)]
      {Consider that
       \begin{align*}
         (l, k) \cdot (\psi(k^{-1})(l^{-1}), k^{-1})
      &= (l \psi(k)(\psi(k^{-1})(l^{-1}), k k^{-1}) \\
      &= (l \psi(k)(\psi(k)^{-1}(l^{-1})), k k^{-1}) \\
      &= (l (\psi(k) \circ \psi(k)^{-1})(l^{-1}), k k^{-1}) \\
      &= (l l^{-1}, k k^{-1}) \\
      &= (1_L, 1_K),
       \end{align*}
       and similarly
       \begin{align*}
          (\psi(k^{-1})(l^{-1}), k^{-1}) \cdot (l, k)
       &= (\psi(k^{-1})(l^{-1}) \psi(k^{-1})(l), k^{-1} k) \\
       &= (\psi(k^{-1})(l^{-1} l), k^{-1} k) \\
       &= (\psi(k^{-1})(1_L), 1_K) \\
       &= (1_L, 1_K)
       \end{align*}
       so the operation has inverses of this form.
      }
\end{itemize}
Therefore $L \times K$ is a group with this operation.
\end{Answer}

\pagebreak

\begin{Problem}
Let $G$ be a group, $H < G$, $N \triangleleft G$. Suppose that 
$H \cap N = \{ 1 \}$ and that $G = NH$.
\begin{enumerate}[(a)]
  \item{Show that the map $N \times H \to G$ given by
        $(n, h) \mapsto nh$ is a bijection.}
  \item{Show that the map $H \to \mathrm{Aut}(N)$ given by
      $h \mapsto c_h$, where $c_h(n) = h n h^{-1}$ is conjugation by
      $h$, is a group homomorphism.}
  \item{Observe that 
        $$
        n_1 h_1 n_2 h_2 
      = n_1 h_1 n_2 h_1^{-1} h_1 h_2
      = n_1 c_{h_1}(n_2) h_1 h_2
        $$
        and show that there is a group isomorphism between $G$ and the
        semidirect product $N \rtimes_c H$ (as in problem 1).}
\end{enumerate}
\end{Problem}

\begin{Answer}
\begin{enumerate}
  \item{Since $G = NH = \{ nh \mid n \in N, h \in H \}$, 
        the map $\varphi((n, h)) = nh$ is surjective.

        Next let $(n, h) \in \ker \varphi \subset N \times H$. Then 
        $nh = 1$, so $n = h^{-1}$ and $h = n^{-1}$. But if
        $n = h^{-1}$, then since $H$ is a subgroup $n \in H$
        and thus $n \in N \cap H = \{1\}$. 
        Similarly if $h = n^{-1}$, then
        since $N$ is a subgroup $h \in N$ so $h \in N \cap H = \{1\}$.
        Therefore $(n, h) = (1, 1)$, so $\ker \varphi$ is trivial.
        Therefore $\varphi is injective$.

        Then $\varphi$ is bijective.
       }
  \item{
    Letting $\chi : H \to \mathrm{Aut}(N)$ be given by
    $\chi(h) = c_h$, we have
    \begin{align*}
     \chi(h h^\prime) =  c_{h h^\prime}(n)
  &= (h h^\prime) n (h h^\prime)^{-1} \\
  &= h h^\prime n (h^\prime)^{-1} h^{-1} \\
  &= h c_{h^\prime}(n) h^{-1}
  &= c_h (c_{h^\prime} (n)) \\
  &= (c_h \circ c_{h^\prime})(n) = \chi(h) \chi(h^\prime),
    \end{align*}
    so this is a group homomorphism.
  }
  \item{
    The map $\varphi((n, h)) = nh$ was shown in the first problem to
    give an isomorphism $G \simeq N \times H$.

  }
\end{enumerate}
\end{Answer}

\pagebreak

\begin{Problem}
Suppose that $H$ and $N$ are normal subgroups of a finite group $G$.
\begin{enumerate}[(a)]
  \item{Show that $HN = \{ hn \mid h \in H, n \in H \}$ 
        is a subgroup of $G$.}
  \item{Now suppose in addition that the orders of $H$ and $N$ are
      relatively prime, i.e. that $\mathrm{gcd}(|H|,|N|) = 1$. Show
      that $hn = nh$ for all $h \in H$, $n \in N$, and that $HN$ is
      isomorphic to the direct product $H \times N$.}
\end{enumerate}
\end{Problem}

\begin{Answer}
\begin{enumerate}[(a)]
  \item{We show that $HN$ contains the unit and is closed under the
      group operations.
  \begin{itemize}
    \item[(Unit)]
    {
      Since $H$ and $N$ are subgroups, $1 \in H$ and $1 \in N$, so
      $1 = 1 \cdot 1 \in HN$.
    }
    \item[(Product)]
    {
    }
  \end{itemize}
  }
\end{enumerate}
\end{Answer}

\pagebreak

\begin{Problem}
\begin{enumerate}[(a)]
  \item{Suppose that $Z(G)$ is the center of the group $G$. Prove that if
      the quotient $G / Z(G)$ is cyclic, then $G$ is abelian.}
  \item{Prove that every group of order $p^2$, where $p$ is prime, is
      abelian.}
  \item{For each prime $p$, give an example of a group of order $p^3$
      which is not abelian.}
\end{enumerate}
\end{Problem}

\begin{Answer}
\end{Answer}

\pagebreak
\begin{Problem}
\begin{enumerate}
  \item{Suppose that $G$ is a finite group and $H$ a proper subgroup
      of $G$. Show that there is at least one element of $G$ which is
      not conjugate to any element of $H$.}
  \item{Let the group $G$ act on the set $X$ and denote by $G_x$ the
      stabilizer of $x$. Show that if $y = g \cdot x$, then $G_y = g
      G_x g^{-1}$.}
  \item{Suppoes that a finite group $G$ acts on a set $X$ transitively
      (this means that all elements of $X$ lie in the same
      $G$-orbit). Assuming that $X$ has at least two elements, show
      that there is $g \in G$ such that $g \cdot x \neq x$, for all $x
      \in X$, i.e. such that $g$ ``leaves no point fixed''.}
\end{enumerate}
\end{Problem}

\pagebreak
\begin{Problem}
Let $G$ be a $p$-group where $p$ is a prime. Suppose that $x$ is an
element of order $p$ in $G$ such that the subgroup $\langle x \rangle$
generated by $x$ is normal in $G$. Show that $x$ belongs to the center
$Z(G)$ of $G$.
\end{Problem}

\begin{Answer}
\end{Answer}

\pagebreak
\begin{Problem}
\begin{enumerate}[(a)]
\item{Let $G$ be a finite group and $H$ a subgroup of $G$. Let $Q$ be
    a $p$-Sylow subgroup of $H$. Show that there is a $p$-Sylow
    subgroup $P$ of $G$ such that $P \cap H = Q$.}
\item{Let $G$ be a finite group and $P$ a $p$-Sylow subgroup of
    $G$. Further let $N$ be a normal subgroup of $G$. Prove that $P
    \cap N$ is a $p$-Sylow subgroup of $N$.}
\end{enumerate}
\end{Problem}

\begin{Answer}
\end{Answer}

\end{document}
