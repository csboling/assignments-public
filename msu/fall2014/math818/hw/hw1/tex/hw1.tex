\documentclass{article}

\usepackage{amsmath}
\usepackage{amsfonts}
\usepackage{amssymb}
\usepackage{enumerate}
\usepackage[lastexercise]{exercise}

\newcounter{Problem}
\newenvironment{Problem}{\begin{Exercise}[name={Problem},
                                          counter={Problem}]}
                        {\end{Exercise}}
\title{MATH 818 Homework \#1}
\date{September 19, 2014}
\author{Sam Boling}

\begin{document}

\begin{titlepage}
\maketitle
\end{titlepage}

\begin{Problem}
Show that every group of order $\leq 5$ is abelian.
\end{Problem}

\begin{Answer}
Let $G$ be a group of order $\leq 5$. There are the following possibilities:
\begin{itemize}
  \item{$|G| = 1$. Then $G$ must be trivial, and therefore abelian.}
  \item{$|G| \in \{2, 3, 5\}$. Then $|G|$ is prime, so $G$ is cyclic.

        Let $H$ be any cyclic group, and let $h, h^\prime \in
        H$. Since $H$ is cyclic $H = \langle x \rangle$ for some $x
        \in G$, so $h = x^n$ and $h^\prime =
        x^{n^\prime}$ for some $n, n^\prime \in \mathbb{Z}$. Therefore
        $$
        h \cdot h^\prime 
      = x^n \cdot x^{n^\prime} 
      = x^{n + n^\prime}
      = x^{n^\prime} \cdot x^{n}
      = h^\prime \cdot h,
        $$
        so $H$ is abelian.

        Therefore $G$ is abelian if $|G| \in \{2, 3, 5\}$.
      }
  \item{$|G| = 4$. Then since $G$ contains an odd number of non-unit
      elements, there must exist an $x \in G$ such that $x = x^{-1}$.
%      Then consider the function $f(g) = xggx$. This is a homomorphism
%      since $$f(ab) = xabx = xaxxbx = f(a)f(b)$$
      There are two cases:
      \begin{itemize}
        \item{$g = g^{-1}$ for each $g \in G$. Then since $xy \in G$
              and thus $(xy)^{-1} = xy$,
              $$
              xy = (xy)^{-1} = y^{-1} x^{-1} = yx
              $$
              for any $x, y \in G$, so $G$ is abelian.
              }
        \item{$x = x^{-1}$ for one $x \neq e$, and the remaining two
              elements can be labeled $y$ and its distinct inverse.
              Then $G = \{e, x, y, y^{-1}\}$. Since $y \neq x^{-1}$,
              $xy \neq e$, and since $y \neq e \neq x$ this means 
              $x \neq xy \neq y$. Therefore the only remaining option
              is $xy = y^{-1}$, and the same argument means
              $yx = y^{-1}$. This further means
              \begin{align*}
              x y^{-1} &= xxy = y = (y^{-1})^{-1} \\
                      &= (xy)^{-1} = y^{-1} x^{-1} = y^{-1} x,
              \end{align*}
              so $x$ commutes with $y$ and $y^{-1}$. Each other pair
              of elements either contains $e$ or consists of an
              element and its inverse, so all pairs of elements commute.              
             }
      \end{itemize}
       Therefore $G$ is abelian when $|G| = 4$.
       }
\end{itemize}
Hence $G$ is abelian when $|G| \leq 5$.
\end{Answer}

\pagebreak

\begin{Problem}
Show that if $N$ is a normal subgroup of order 2 in a group $G$ then
$N$ is contained in the center $Z(G)$.
\end{Problem}

\begin{Answer}
A group of order 2 has the form $\{ e, x \}$, where $x =
x^{-1}$. Then let $N = \{ e, x \}$ of this form be a normal subgroup
of $G$.

Let $a \in G$ and consider the cosets
$$aN = \{ an | n \in N \} = \{ ae, ax \} = \{ a, ax \}$$
and
$$Na = \{ na | n \in N \} = \{ ea, xa \} = \{ a, xa \}.$$
Since $N$ is normal, $aN = Na$ and thus $ax = xa$, so $x$
commutes with $a$. But $a$ was chosen arbitrarily in $G$, so $x$ commutes
with every $a \in G$ and thus $x \in Z(G)$. Since $e \in Z(G)$ this
means $N \subset Z(G)$, as desired.
\end{Answer}

\pagebreak

\begin{Problem}
Let $G^c$ be the subgroup of $G$ generated by all its commutators.

\begin{enumerate}[(a)]
  \item{Show that $G^c$ is normal and that the quotient $G / G^c$ is
        abelian.}
  \item{Show that every homomorphism $f : G \to G^\prime$ where
      $G^\prime$ is an abelian group factors through the quotient 
      $G / G^c$, i.e. it can be written as a composition
      $G \to G / G^c \to G^\prime$.}
  \item{Suppose that $N$ is a normal subgroup of $G$ such that
      $N \cap G^c = \{ 1 \}$. Show that $N$ is contained in the center
      $Z(G)$ of $G$.}
\end{enumerate}
\end{Problem}

\begin{Answer}
\begin{enumerate}[(a)]
  \item{Consider the function given by
        $f(wxyz) = wyxz$
       }
\end{enumerate}
\end{Answer}

\pagebreak

\begin{Problem}
Let $G$ be a group. For an element $a \in G$, consider the map 
$f_a : G \to G$ given by $f_a(x) = axa^{-1}$ (``conjugation by a'').

\begin{enumerate}[(a)]
  \item{Show that $f_a$ is an automorphism of $G$ and so it gives an
      element in the group of automorphisms $\mathrm{Aut}(G)$.}
  \item{By definition, an automorphism of $G$ which has the form $f_a$
      for some $a \in G$ is called ``inner''. Show that the set 
      $\mathrm{Inn}(G)$ of inner automorphisms is a subgroup of
      $\mathrm{Aut}(G)$.}
  \item{Show that the group $\mathrm{Inn}(G)$ of inner automorphisms
      is isomorphic to the quotient $G / Z(G)$ where $Z(G)$ is the
      center of $G$.}
  \item{Show that $\mathrm{Inn}(G)$ is normal in $\mathrm{Aut}(G)$.}
\end{enumerate}
\end{Problem}

\begin{Answer}
\begin{enumerate}[(a)]
  \item{$f_a$ is a homomorphism because
        $$
        f_a(x) f_a(y) = a x a^{-1} a y a^{-1} 
                      = a x y a^{-1}
                      = f_a(xy).
        $$
        Furthermore
        $$
        f_{a^{-1}}(f_a(x)) = a^{-1} f_a(x) a
                         = a^{-1} a x a^{-1} a
                         = x,
        $$
        so $f_{a^{-1}} \circ f_a = \mathrm{id}$ and since $a$ is
        arbitrary $f_a \circ f_{a^{-1}} = \mathrm{id}$, so $f_a$ is
        invertible. Therefore $f_a : G \to G$ is an isomorphism and
        so $f_a \in \mathrm{Aut}(G)$.
        }
      \item{
        \begin{itemize}
          \item[(Unit)]{
            $f_e(x) = e x e^{-1} = x$, so
            $f_e = \mathrm{id}$ and so $\mathrm{id} \in
            \mathrm{Inn}(G)$. Thus $\mathrm{Inn}(G)$ contains the
            unit.
            }
          \item[(Closure)]{
              Let $f_a, f_b \in \mathrm{Inn}(G)$. Then
              $$
              (f_a \circ f_b)(x) = f_a( b x b^{-1})
                                 = a b x b^{-1} a^{-1}
                                 = (ab) x (ab)^{-1},
              $$
              so $f_a \circ f_b \in \mathrm{Inn}(G)$.
            }
          \item[(Inverses)]{
              Let $f_a \in \mathrm{Inn}(G)$.
              As in (a), $f_{a^{-1}} \circ f_a = \mathrm{id}$, so
              $f_a^{-1} = f_{a^{-1}}$, so $f_a^{-1} \in \mathrm{Inn}(G)$.
            }
        \end{itemize}
        Therefore $\mathrm{Inn}(G) < \mathrm{Aut}(G)$.
      }
      \item{}
      \item{
        Let $g \in \mathrm{Aut}(G)$. Then the coset
        $g \mathrm{Inn}(G) = \{ g \circ f_a | a \in G \}$.
        Then consider an arbitrary $a \in G$. We write
        \begin{align*}
          (g \circ f_a)(x) 
            &= g(f_a(x)) = g(a x a^{-1}) \\
            &= g(a) g(x) g(a^{-1})
             = g(a) g(x) g(a)^{-1} \\
            &= f_{g(a)}(g(x))
             = (f_{g(a)} \circ g)(x),
        \end{align*}
        and $f_{g(a)} \circ g \in \mathrm{Inn}(G) g$. Therefore
        $g \mathrm{Inn}(G) \subset \mathrm{Inn}(G) g$ for any
        $g \in \mathrm{Aut}(G)$, so $\mathrm{Inn}(G)$ is normal
        in $\mathrm{Aut}(G)$.
      }
\end{enumerate}
\end{Answer}

\pagebreak

\begin{Problem}
Let $G$ be a group and $A$ and abelian normal subgroup of $G$. Give a
group homomorphism $f : G/A \to \mathrm{Aut}(A)$.
\end{Problem}

\begin{Problem}
Find the automorphism groups $\mathrm{Aut}(\mathbb{Z}_{45})$,
$\mathrm{Aut}(\mathbb{Z}_3 \times \mathbb{Z}_5)$, and
$\mathrm{Aut}(\mathbb{Z}_5 \times \mathbb{Z}_5)$.
\end{Problem}

\pagebreak

\begin{Problem}
Show that $SL_2(\mathbb{R})$ is perfect.
\end{Problem}

\begin{Answer}

\begin{enumerate}[(a)]
  \item{}
  \item{}
  \item{}
\end{enumerate}

\end{Answer}

\end{document}
