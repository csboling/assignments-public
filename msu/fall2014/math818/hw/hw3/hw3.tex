\documentclass{article}

\usepackage{amsmath}
\usepackage{amsfonts}
\usepackage{amssymb}
\usepackage{enumerate}
\usepackage[lastexercise]{exercise}

\newcounter{Problem}
\newenvironment{Problem}{\begin{Exercise}[name={Problem},
                                          counter={Problem}]}
                        {\end{Exercise}}
\title{MATH 818 Homework \#3}
\date{November 5, 2014}
\author{Sam Boling}

\begin{document}

\begin{titlepage}
\maketitle
\end{titlepage}

\begin{Problem}
Prove that a group of order 255 cannot be simple.
\end{Problem}

\pagebreak

\begin{Problem}
\begin{enumerate}[(a)]
  \item{
    Consider $\sigma = (13452)$ in $S_6$. Find the order of the
    centralizer $C_\sigma$ of $\sigma$ in $S_6$.
  }
  \item{
    How many elements of $S_7$ are conjugate to $a = (123)(45)$?
  }
\end{enumerate}
\end{Problem}

\pagebreak

\begin{Problem}
Let $p$ and $q$ be distinct primes.
\begin{enumerate}[(a)]
  \item{
    Show that a group of order $p^2 q$ is solvable and that one of its
    Sylow subgroups is normal.
  }
  \item{
    Assume in addition that $p < q$ and $p$ does not divide $q -
    1$. Show that a group of order $p^2 q$ has to be abelian.
  }
\end{enumerate}
\end{Problem}

\pagebreak

\begin{Problem}
  \begin{enumerate}[(a)]
    \item{
      Show that $S_n$ is generated by the transpositions
      $(12),(23),\dots,(n-1n)$.
    }
    \item{
      Show that $S_n$ is generated by $(12)$ and the $n$-cycle
      $(12 \cdots n)$.
    }
    \item{
      Show that if $n$ is a prime number, then $S_n$ is generated by
      $(1 2 \cdots n)$ and any transposition $(rs)$, $r \neq s$.
    }
  \end{enumerate}
\end{Problem}

\pagebreak

\begin{Problem}
\begin{enumerate}[(a)]
  \item{
    How many abelian groups of order 3600 are there (up to isomorphism)?
  }
  \item{
    Give an example of an abelian group which is torsion-free but not free.
  }
  \item{
    Given an example of an abelian group which is both torsion and uncountable.
  }
\end{enumerate}
\end{Problem}

\begin{Answer}
\begin{enumerate}[(a)]
  \item{}
  \item{}
  \item{
    The direct product
    $$
    A = \prod_{i=1}^\infty \mathbb{Z}_2
    $$
    is abelian, since $\mathbb{Z}_2$ is abelian and therefore so is
    the direct product.

    Take $x \in \mathbb{Z}_2^n$ so that $x = (x_i)_{i=1}^\infty$. Then
    $x + x = (x_i + x_i)_{i=1}^\infty = (0)_{i=1}^\infty$, so $A$ is torsion.
    But $A$ consists of all infinite sequences of binary digits, and
    therefore is in bijection with the interval $[0, 1)$.
    Therefore $A$ is abelian, torsion, and uncountable.
  }
\end{enumerate}
\end{Answer}

\pagebreak

\begin{Problem}
Consider a $2 \times 2$ matrix
$A = \left(\begin{array}{c c}
       a & b \\ c & d
     \end{array}\right)$
with integer coefficients, i.e.
$a, b, c, d \in \mathbb{Z}$. Let
$\mathbb{Z}^2 = \mathbb{Z} \oplus \mathbb{Z}$ as usual.
\begin{enumerate}[(a)]
  \item{
    Show that $(u, v) \mapsto (au + bv, cu + dv)$ gives a group
    homomorphism $\varphi_A : \mathbb{Z}^2 \to \mathbb{Z}^2$.
  }
  \item{
    Show that $\varphi_A$ is injective if and only if
    $\det(A) \neq 0$.
  }
  \item{
    Show that $\varphi_A$ is surjective if and only if
    $\det(A) = \pm 1$ and then $\varphi_A$ is an isomorphism.
  }
  \item{
    Show that if $\det(A) \neq 0$, then the quotient
    $\mathbb{Z}^2 / \mathrm{Im}(\varphi_A)$ is a finite group.
  }
\end{enumerate}
\end{Problem}

\pagebreak

\begin{Problem}
By definition a \emph{lattice} in $\mathbb{R}^n$ is a subgroup $L$ of
$\mathbb{R}^n$ which contains a basis of the vector space
$\mathbb{R}^n$ in the topology of $\mathbb{R}^n$.
\begin{enumerate}[(a)]
  \item{
    Assume $n = 1$. Show that a lattice $L$ in $\mathbb{R}$ is
    generated by a single real number $r$ and so
    $L = \{ n \cdot r \mid n \in \mathbb{Z} \}$. Conclude that, as an
    abstract group, $L$ is isomorphic to $\mathbb{Z}$.
  }
  \item{
    Assume $n = 2$. Show that every lattice $L$ in $\mathbb{R}^2$ is
    generated by a basis $\{f_1, f_2\}$ of $\mathbb{R}^2$. Conclude
    that every lattice $L$ in $\mathbb{R}^2$ is, as an abstract group,
    isomorphic to $\mathbb{Z}^2$.
  }
  \item{
    Extra credit: use induction to show that every lattice in
    $\mathbb{R}^n$ is isomorphic to $\mathbb{Z}^n$ as an abstract group.
  }
\end{enumerate}
\end{Problem}

\end{document}
