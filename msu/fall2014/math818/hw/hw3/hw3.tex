\documentclass{article}

\usepackage{amsmath}
\usepackage{amsfonts}
\usepackage{amssymb}
\usepackage{enumerate}
\usepackage[lastexercise]{exercise}

\newcounter{Problem}
\newenvironment{Problem}{\begin{Exercise}[name={Problem},
                                          counter={Problem}]}
                        {\end{Exercise}}
\title{MATH 818 Homework \#3}
\date{November 5, 2014}
\author{Sam Boling}

\begin{document}

\begin{titlepage}
\maketitle
\end{titlepage}

\begin{Problem}
Prove that a group of order 255 cannot be simple.
\end{Problem}

\begin{Answer}
Let $G$ be a group of order 255. The prime factorization of 255 is
$3 \cdot 5 \cdot 17$, so each Sylow subgroup of $G$ has prime
order. Therefore all Sylow subgroups of $G$ are disjoint except for
the identity.

The number of 17-Sylows must be congruent to 1 mod 17 and must divide
255. But the only factor of 255 congruent to 1 mod 17 is 1. Therefore
G has a unique 17-Sylow, and a unique $p$-Sylow must be normal. Since
this normal subgroup has order 17, it is a proper nontrivial normal
subgroup. Therefore $G$ cannot be simple.
\end{Answer}

\pagebreak

\begin{Problem}
\begin{enumerate}[(a)]
  \item{
    Consider $\sigma = (13452)$ in $S_6$. Find the order of the
    centralizer $C_\sigma$ of $\sigma$ in $S_6$.
  }
  \item{
    How many elements of $S_7$ are conjugate to $a = (123)(45)$?
  }
\end{enumerate}
\end{Problem}

\begin{Answer}
To answer these questions we first find the size of the
conjugacy classes of a symmetric group.
We know that any permutation $\sigma$ can be written as a product of
disjoint cycles
$$
\sigma =
  (a_{11} a_{12} \cdots a_{1k_1})
  (a_{21} a_{22} \cdots a_{2k_2})
  \cdots
  (a_{m1} a_{m2} \cdots a_{mk_m})
$$
and furthermore from the conjugation formula that conjugation by a
permutation $\tau$ gives
\begin{align*}
\tau \sigma \tau^{-1} &=
  \tau
  (a_{11} a_{12} \cdots a_{1k_1})
  \cdots
  (a_{m1} a_{m2} \cdots a_{mk_m})
  \tau^{-1} \\
&=
  \tau
  (a_{11} a_{12} \cdots a_{1k_1})
  \tau^{-1}
  \cdots
  \tau
  (a_{m1} a_{m2} \cdots a_{mk_m})
  \tau^{-1} \\
&=
  (\tau(a_{11}) \tau(a_{12}) \cdots \tau(a_{1k_1}))
  \cdots
  (\tau(a_{m1}) \tau(a_{m2}) \cdots \tau(a_{mk_m})).
\end{align*}
Furthermore since $\tau$ is injective and since the $a_{ij}$ are
disjoint, so must $\tau(a_{ij})$ be.
Therefore any conjugation of a permutation $\sigma$ has a disjoint
cycle decomposition into cycles of the same lengths as those in the
decomposition of $\sigma$.

Next, take two permutations $\sigma$ and $\tau$ with the same cycle
structure
\begin{align*}
\sigma &=
  (a_{11} a_{12} \cdots a_{1k_1})
  (a_{21} a_{22} \cdots a_{2k_2})
  \cdots
  (a_{m1} a_{m2} \cdots a_{mk_m}), \\
\tau &=
  (b_{11} b_{12} \cdots b_{1k_1})
  (b_{21} b_{22} \cdots b_{2k_2})
  \cdots
  (b_{m1} b_{m2} \cdots b_{mk_m})
\end{align*}
and observe that with
$$
\omega =
  (a_{11} b_{11})
  (a_{12} b_{12})
  \cdots
  (a_{1k_1} b_{1k_1})
  \cdots
  (a_{m1} b_{m1})
  \cdots
  (a_{mk_m} b_{mk_m})
$$
we have $\omega \sigma \omega^{-1} = \tau$. Therefore any permutation
with the same cycle structure is conjugate to $\sigma$, so the
conjugacy classes in a symmetric group consist of exactly those
elements with the same cycle structure.

To enumerate the elements in a given conjugacy class in a symmetric
group we then list all elements with the same structure to their
disjoint cycle decompositions and eliminate repetitions. Ignoring the
separation into cycles, a disjoint cycle decomposition of an element
of $S_n$ can be thought of as an ordered list of $n$ elements, and
there are $n!$ such lists.

Take the disjoint decomposition of some element $a$.
Label the number of $i$-cycles in the decomposition by $r_i$.
 Since disjoint cycles commute, any two cycles in a decomposition
with the same length can be exchanged without affecting the
permutation. Therefore we can eliminate each rearrangement of
cycles of the same length, or divide out $\prod_{i} r_i!$ elements.
Furthermore a given $k$-cycle can be written in $k$ ways, as is seen
by rotating the notation to the left, e.g.
$(12) = (21)$, $(123) = (231) = (312)$. Therefore each of the
$r_k$ $k$-cycles in $\sigma$ admits $k$ representations, so we must
also divide by $\prod_i k^{r_k}$. This results in the formula
$$
\frac{n!}{\left(\prod_{i} r_i!\right) \left(\prod_i k^{r_k}\right)}
$$
for the number of elements in the conjugacy class of an arbitrary
permutation $\sigma$.

\begin{enumerate}[(a)]
  \item{
    The centralizer
    $$
    C_{S_6}(\sigma) =
    \{ \tau \in S_6
       \mid \sigma = \tau \sigma \tau^{-1}
    \}
    $$
    is the stabilizer of $\sigma$ under the group action
    $. : S_6 \times S_6 \to S_6$ given by conjugation. Therefore we have
    that $[S_6 : C_{S_6}(\sigma)] = |S_6 . \sigma|$, the order of the orbit of this
    action on $\sigma$. But this orbit is given by
    $$
    S_6 . \sigma =
    \{ \tau \sigma \tau^{-1} \mid \tau \in S_6 \}
    $$
    and by the conjugation formula
    $$
    \tau \sigma \tau^-1
    = \tau (13452) \tau^{-1}
    = (\tau(1) ~ \tau(3) ~ \tau(4) ~ \tau(5) ~ \tau(2))
    $$
    so elements of the form $\tau \sigma \tau^{-1}$ are exactly the
    5-cycles in $S_6$. of which there are $\frac{6!}{5^1 \cdot 1!}$. Then
    $$
    \frac{6!}{5} = [S_6 : C_{S_6}(\sigma)] = \frac{|S_6|}{|C_{S_6}(\sigma)|}
    $$
    so that
    $$
    |C_{S_6}(\sigma)| = \frac{|S_6|}{\frac{6!}{5}}
                     = \frac{6!}{\frac{6!}{5}}
                     = 5.
    $$
  }
  \item{
    The number of elements conjugate to $a$ is the order of the
    conjugacy class of $a$. Conjugacy classes in $S_n$ are in
    correspondence with the partitions of $n$, and in particular those
    elements conjugate to $(123)(45)$ are those of the same cycle type.
    There are
    $$
    \frac{7!}{2^1 \cdot 3^1 \cdot 1! \cdot 1!} =
    7 \cdot 6 \cdot 5 \cdot 4 = 840.
    $$
  }
\end{enumerate}
\end{Answer}

\pagebreak

\begin{Problem}
Let $p$ and $q$ be distinct primes.
\begin{enumerate}[(a)]
  \item{
    Show that a group of order $p^2 q$ is solvable and that one of its
    Sylow subgroups is normal.
  }
  \item{
    Assume in addition that $p < q$ and $p$ does not divide $q -
    1$. Show that a group of order $p^2 q$ has to be abelian.
  }
\end{enumerate}
\end{Problem}

\pagebreak

\begin{Problem}
  \begin{enumerate}[(a)]
    \item{
      Show that $S_n$ is generated by the transpositions
      $(12),(23),\dots,(n-1 \to n)$.
    }
    \item{
      Show that $S_n$ is generated by $(12)$ and the $n$-cycle
      $(12 \cdots n)$.
    }
    \item{
      Show that if $n$ is a prime number, then $S_n$ is generated by
      $(1 2 \cdots n)$ and any transposition $(rs)$, $r \neq s$.
    }
  \end{enumerate}
\end{Problem}

\begin{Answer}
\begin{enumerate}
  \item{
    Notice that
    $$
    (12)(23)(12) = (13).
    $$
    Furthermore suppose $(1 \to k - 1) \in H$. Then
    $$
    (1 \to k - 1)(k - 1 \to k)(1 \to k - 1) =
    (1k),
    $$
    so $(1k) \in H$ as well.
    Therefore $H$ contains every transposition $(1k)$ for
    $k \in \{ 2, \dots, n \}$, and we know that the transpositions
    $(12)$, $(13)$, \dots, $(1n)$ generate $H$.
  }
  \item{
    Denote $H = \langle (1 2), (1 2 \cdots n) \rangle$.

    First note that from the conjugation formula,
    $$
    (1 2 \cdots n)(1 2)(1 2 \cdots n)^{-1} = (2 3)
    $$
    so $(2, 3) \in H$.
    Similarly,
    $$
    (1 2 \cdots n)(i - 2 \to i - 1)(1 2 \cdots n)^{-1}
    = (i - 1 \to i)
    $$
    for $i \in \{ 3, \dots, n \}$.
    Therefore $H$ contains the elements
    $(12)$, $(23)$, \dots, $(n-1 \to n)$, which from part (a)
    generates $S_n$.
  }
  \item{
  }
\end{enumerate}
\end{Answer}

\pagebreak

\begin{Problem}
\begin{enumerate}[(a)]
  \item{
    How many abelian groups of order 3600 are there (up to isomorphism)?
  }
  \item{
    Give an example of an abelian group which is torsion-free but not free.
  }
  \item{
    Given an example of an abelian group which is both torsion and uncountable.
  }
\end{enumerate}
\end{Problem}

\begin{Answer}
\begin{enumerate}[(a)]
  \item{}
  \item{
    The rationals are torsion-free. Let $\frac{r}{s} \in
    \mathbb{Q}$ and suppose $n \frac{r}{s} = 0$ for $n \geq 1$.
    Then $n r = 0$ and thus $r = 0$, so $\frac{r}{s} = 0$.

    The rationals are not free. Let
    $\varphi : \mathbb{Q} \to \mathbb{Z}[I]$ be a group homomorphism
    for some set $I$, and let $\frac{r_1}{s_1}, \frac{r_2}{s_2} \in
    \mathbb{Q}$. Then without loss of generality
    $$
    \varphi\left(\frac{r_1}{s_1}\right) =
      (k_1, k_2, \dots, k_m, 0, 0, \dots)
    $$
    and
    $$
    \varphi\left(\frac{r_2}{s_2}\right) =
      (l_1, l_2, \dots, l_n, 0, 0, \dots)
    $$
    for some integers $k_i$, $l_i$.
  }
  \item{
    The direct product
    $$
    A = \prod_{i=1}^\infty \mathbb{Z}_2
    $$
    is abelian, since $\mathbb{Z}_2$ is abelian and therefore so is
    the direct product.

    Take $x \in \mathbb{Z}_2^n$ so that $x = (x_i)_{i=1}^\infty$. Then
    $x + x = (x_i + x_i)_{i=1}^\infty = (0)_{i=1}^\infty$, so $A$ is torsion.
    But $A$ consists of all infinite sequences of binary digits, and
    therefore is in bijection with the interval $[0, 1)$ since these
    can be regarded as binary representations of the nonnegative real
    numbers with zero integer part.
    Therefore $A$ is abelian, torsion, and uncountable.
  }
\end{enumerate}
\end{Answer}

\pagebreak

\begin{Problem}
Consider a $2 \times 2$ matrix
$A = \left(\begin{array}{c c}
       a & b \\ c & d
     \end{array}\right)$
with integer coefficients, i.e.
$a, b, c, d \in \mathbb{Z}$. Let
$\mathbb{Z}^2 = \mathbb{Z} \oplus \mathbb{Z}$ as usual.
\begin{enumerate}[(a)]
  \item{
    Show that $(u, v) \mapsto (au + bv, cu + dv)$ gives a group
    homomorphism $\varphi_A : \mathbb{Z}^2 \to \mathbb{Z}^2$.
  }
  \item{
    Show that $\varphi_A$ is injective if and only if
    $\det(A) \neq 0$.
  }
  \item{
    Show that $\varphi_A$ is surjective if and only if
    $\det(A) = \pm 1$ and then $\varphi_A$ is an isomorphism.
  }
  \item{
    Show that if $\det(A) \neq 0$, then the quotient
    $\mathbb{Z}^2 / \mathrm{Im}(\varphi_A)$ is a finite group.
  }
\end{enumerate}
\end{Problem}

\begin{Answer}
\begin{enumerate}
  \item{
    We have that
    \begin{align*}
    \varphi_A((u, v) + (w, x)) &=
      \varphi_A((u + w, v + x)) =
      (a(u + w) + b(v + x), c(u + w) + d(v + x)) \\
      &= ((au + bv) + (aw + bx), (cu + dv) + (cw + vx)) \\
      &= \varphi_A((u, v)) + \varphi_A((w, x))
    \end{align*}
    so $\varphi_A$ is a group homomorphism.
  }
  \item{
    We show $\det(A) \neq 0$ iff. $\varphi_A$ is injective.
    \begin{itemize}
      \item[($\implies$)]
      {
        Let $\varphi_A$ be injective. Then
        $\ker(\varphi_A) = \{(0, 0)\}$. Notice that
        $$
        \varphi_A((d, -c)) = (ad - bc, 0)
        $$
        and
        $$
        \varphi_A((-b, a)) = (0, ad - bc).
        $$
        Suppose $ad - bc = 0$. Then this means
        $(d, -c) \in \ker(\varphi_A)$ and $(-b, a) \in \ker(\varphi_A)$,
        which means $a = b = c = d = 0$. But this is impossible, since in
        this case $\varphi_A(x) = (0, 0)$ for all $x \in \mathbb{Z}^2$,
        and $\varphi_A$ is injective by assumption. Therefore
        $\det(A) = ad - bc \neq 0$.
      }
      \item[($\impliedby$)]
      {
        Suppose $\det(A) = ad - bc \neq 0$. Let $(u, v) \in
        \ker(\varphi_A)$. Then $(au + bv, cu + dv) = (0, 0)$
        so $au + bv = 0$ and $cu + dv = 0$. But this means
        $adu + bdv = 0$ and $bcu + bdv = 0$, so subtracting these
        equations gives $(ad - bc)u = 0$, so since $ad - bc \neq 0$ it
        must be the case that $u = 0$. Similarly, we have
        $acu + bcv = 0$ and $acu + adv = 0$ so that $(ad - bc)v = 0$, so
        $v = 0$ as well. Therefore $(u, v) \in \ker(\varphi_A)$ implies
        $(u, v) = (0, 0)$ in this case, so the kernel is trivial and
        therefore $\varphi_A$ is injective.
      }
    \end{itemize}
  }
  \item
  {
    We show $\varphi_A$ is surjective iff. $\det(A) = \pm 1$.
    \begin{itemize}
      \item[($\implies$)]
      {
        Suppose $\det(A) = ad - bc = 1$, and let $(x, y) \in
        \mathbb{Z}^2$.
        Then
        $$
        \varphi_A((d, -c)) = (ad - bc, 0) = (1, 0)
        $$
        and
        $$
        \varphi_A((-b, a)) = (0, ad - bc) = (0, 1),
        $$
        so
        $$
        (x, y) =
          x (1, 0) + y (0, 1) =
          x \varphi_A((d, -c)) + y \varphi_A((-b, a))
        $$
        and therefore
        $$
        (x, y) = \varphi_A(x(d, -c) + y(-b, a)).
        $$
        Similarly if $\det(A) = -1$ we have
        $$
        (x, y) = \varphi_A(-x(d, -c) - y(-b, a)),
        $$
        and therefore $(x, y) \in \mathrm{Im}(\varphi_A)$ in either of
        these cases. Therefore $\varphi_A$ is surjective.
      }
      \item[($\impliedby$)]
      {
        Suppose $\varphi_A$ is surjective. Then for any $(x, y) \in
        \mathbb{Z}^2$ we have $(x, y) = \varphi_A((u, v))$ for some
        $(u, v) \in \mathbb{Z}^2$, and so
        \begin{align*}
          \left(\begin{array}{c}
            x \\ y
          \end{array}\right) &=
          x
          \left(\begin{array}{c}
            1 \\ 0
          \end{array}\right)
          +
          y
          \left(\begin{array}{c}
            0 \\ 1
          \end{array}\right) \\
          &=
          \left(\begin{array}{c c}
            a & b \\ c & d
          \end{array}\right)
          \left(\begin{array}{c}
            u \\ v
          \end{array}\right) =
          \left(\begin{array}{c}
            au + bv \\ cu + dv
          \end{array}\right) \\
          &=
          u
          \left(\begin{array}{c}
            a \\ c
          \end{array}\right)
          +
          v
          \left(\begin{array}{c}
            b \\ d
          \end{array}\right).
        \end{align*}
      }
    \end{itemize}
    Therefore $\varphi_A$ is surjective iff. $\det(A) = \pm 1$. But in
    this case
    $\det(A) \neq 0$, so $\varphi_A$ is injective as well, and
    therefore an isomorphism.
  }
\end{enumerate}
\end{Answer}

\pagebreak

\begin{Problem}
By definition a \emph{lattice} in $\mathbb{R}^n$ is a subgroup $L$ of
$\mathbb{R}^n$ which contains a basis of the vector space
$\mathbb{R}^n$ in the topology of $\mathbb{R}^n$.
\begin{enumerate}[(a)]
  \item{
    Assume $n = 1$. Show that a lattice $L$ in $\mathbb{R}$ is
    generated by a single real number $r$ and so
    $L = \{ n \cdot r \mid n \in \mathbb{Z} \}$. Conclude that, as an
    abstract group, $L$ is isomorphic to $\mathbb{Z}$.
  }
  \item{
    Assume $n = 2$. Show that every lattice $L$ in $\mathbb{R}^2$ is
    generated by a basis $\{f_1, f_2\}$ of $\mathbb{R}^2$. Conclude
    that every lattice $L$ in $\mathbb{R}^2$ is, as an abstract group,
    isomorphic to $\mathbb{Z}^2$.
  }
  \item{
    Extra credit: use induction to show that every lattice in
    $\mathbb{R}^n$ is isomorphic to $\mathbb{Z}^n$ as an abstract group.
  }
\end{enumerate}
\end{Problem}

\end{document}
