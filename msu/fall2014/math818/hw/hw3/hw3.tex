\documentclass{article}

\usepackage{amsmath}
\usepackage{amsfonts}
\usepackage{amssymb}
\usepackage{enumerate}
\usepackage{mathtools}
\usepackage[lastexercise]{exercise}

\DeclarePairedDelimiter\floor{\lfloor}{\rfloor}

\newcounter{Problem}
\newenvironment{Problem}{\begin{Exercise}[name={Problem},
                                          counter={Problem}]}
                        {\end{Exercise}}
\title{MATH 818 Homework \#3}
\date{November 5, 2014}
\author{Sam Boling}

\begin{document}

\begin{titlepage}
\maketitle
% 4. Show that every lattice in $\mathbb{R}^n$ is isomorphic to $\mathbb{Z}^n$
\end{titlepage}

\begin{Problem}
Prove that a group of order 255 cannot be simple.
\end{Problem}

\begin{Answer}
Let $G$ be a group of order 255. The prime factorization of 255 is
$3 \cdot 5 \cdot 17$, so each Sylow subgroup of $G$ has prime
order. Therefore all Sylow subgroups of $G$ are disjoint except for
the identity.

The number of 17-Sylows must be congruent to 1 modulo 17 and must divide
255. But the only factor of 255 congruent to 1 modulo 17 is 1. Therefore
G has a unique 17-Sylow, and a unique $p$-Sylow must be normal. Since
this normal subgroup has order 17, it is a proper nontrivial normal
subgroup. Therefore $G$ cannot be simple.
\end{Answer}

\pagebreak

\begin{Problem}
\begin{enumerate}[(a)]
  \item{
    Consider $\sigma = (13452)$ in $S_6$. Find the order of the
    centralizer $C_\sigma$ of $\sigma$ in $S_6$.
  }
  \item{
    How many elements of $S_7$ are conjugate to $a = (123)(45)$?
  }
\end{enumerate}
\end{Problem}

\begin{Answer}
To answer these questions we first find the size of the
conjugacy classes of a symmetric group.
We know that any permutation $\sigma$ can be written as a product of
disjoint cycles
$$
\sigma =
  (a_{11} a_{12} \cdots a_{1k_1})
  (a_{21} a_{22} \cdots a_{2k_2})
  \cdots
  (a_{m1} a_{m2} \cdots a_{mk_m})
$$
and furthermore from the conjugation formula that conjugation by a
permutation $\tau$ gives
\begin{align*}
\tau \sigma \tau^{-1} &=
  \tau
  (a_{11} a_{12} \cdots a_{1k_1})
  \cdots
  (a_{m1} a_{m2} \cdots a_{mk_m})
  \tau^{-1} \\
&=
  \tau
  (a_{11} a_{12} \cdots a_{1k_1})
  \tau^{-1}
  \cdots
  \tau
  (a_{m1} a_{m2} \cdots a_{mk_m})
  \tau^{-1} \\
&=
  (\tau(a_{11}) \tau(a_{12}) \cdots \tau(a_{1k_1}))
  \cdots
  (\tau(a_{m1}) \tau(a_{m2}) \cdots \tau(a_{mk_m})).
\end{align*}
Furthermore since $\tau$ is injective and since the $a_{ij}$ are
disjoint, so must $\tau(a_{ij})$ be.
Therefore any conjugate of a permutation $\sigma$ has a disjoint
cycle decomposition into cycles of the same lengths as those in the
decomposition of $\sigma$.

Next, take two permutations $\sigma$ and $\tau$ with the same cycle
structure
\begin{align*}
\sigma &=
  (a_{11} a_{12} \cdots a_{1k_1})
  (a_{21} a_{22} \cdots a_{2k_2})
  \cdots
  (a_{m1} a_{m2} \cdots a_{mk_m}), \\
\tau &=
  (b_{11} b_{12} \cdots b_{1k_1})
  (b_{21} b_{22} \cdots b_{2k_2})
  \cdots
  (b_{m1} b_{m2} \cdots b_{mk_m})
\end{align*}
and observe that with
$$
\omega =
  (a_{11} b_{11})
  (a_{12} b_{12})
  \cdots
  (a_{1k_1} b_{1k_1})
  \cdots
  (a_{m1} b_{m1})
  \cdots
  (a_{mk_m} b_{mk_m})
$$
we have $\omega \sigma \omega^{-1} = \tau$. Therefore any permutation
with the same cycle structure is conjugate to $\sigma$, so the
conjugacy classes in a symmetric group consist of exactly those
elements with the same cycle structure.

To enumerate the elements in a given conjugacy class in a symmetric
group we then list all elements with the same structure to their
disjoint cycle decompositions and eliminate repetitions. Ignoring the
separation into cycles, a disjoint cycle decomposition of an element
of $S_n$ can be thought of as an ordered list of $n$ elements, and
there are $n!$ such lists.

Take the disjoint decomposition of some element $a$.
Label the number of $i$-cycles in the decomposition by $r_i$.
 Since disjoint cycles commute, any two cycles in a decomposition
with the same length can be exchanged without affecting the
permutation. Therefore we can eliminate each rearrangement of
cycles of the same length, or divide out $\prod_{i} r_i!$ elements.
Furthermore a given $k$-cycle can be written in $k$ ways, as is seen
by rotating the notation to the left, e.g.
$(12) = (21)$, $(123) = (231) = (312)$. Therefore each of the
$r_k$ $k$-cycles in $\sigma$ admits $k$ representations, so we must
also divide by $\prod_i k^{r_k}$. This results in the formula
$$
\frac{n!}{\left(\prod_{i} r_i!\right) \left(\prod_i k^{r_k}\right)}
$$
for the number of elements in the conjugacy class of an arbitrary
permutation $\sigma$.

\begin{enumerate}[(a)]
  \item{
    The centralizer
    $$
    C_{S_6}(\sigma) =
    \{ \tau \in S_6
       \mid \sigma = \tau \sigma \tau^{-1}
    \}
    $$
    is the stabilizer of $\sigma$ under the group action
    $. : S_6 \times S_6 \to S_6$ given by conjugation. Therefore we have
    that $[S_6 : C_{S_6}(\sigma)] = |S_6 . \sigma|$, the order of the orbit of this
    action on $\sigma$. But this orbit is given by
    $$
    S_6 . \sigma =
    \{ \tau \sigma \tau^{-1} \mid \tau \in S_6 \}
    $$
    and by the conjugation formula
    $$
    \tau \sigma \tau^-1
    = \tau (13452) \tau^{-1}
    = (\tau(1) ~ \tau(3) ~ \tau(4) ~ \tau(5) ~ \tau(2))
    $$
    so elements of the form $\tau \sigma \tau^{-1}$ are exactly the
    5-cycles in $S_6$. of which there are $\frac{6!}{5^1 \cdot 1!}$. Then
    $$
    \frac{6!}{5} = [S_6 : C_{S_6}(\sigma)] = \frac{|S_6|}{|C_{S_6}(\sigma)|}
    $$
    so that
    $$
    |C_{S_6}(\sigma)| = \frac{|S_6|}{\frac{6!}{5}}
                     = \frac{6!}{\frac{6!}{5}}
                     = 5.
    $$
  }
  \item{
    The number of elements conjugate to $a$ is the order of the
    conjugacy class of $a$. From the formula given above,
    there are
    $$
    \frac{7!}{2^1 \cdot 3^1 \cdot 1! \cdot 1!} =
    7 \cdot 6 \cdot 5 \cdot 4 = 840.
    $$
  }
\end{enumerate}
\end{Answer}

\pagebreak

\begin{Problem}
Let $p$ and $q$ be distinct primes.
\begin{enumerate}[(a)]
  \item{
    Show that a group of order $p^2 q$ is solvable and that one of its
    Sylow subgroups is normal.
  }
  \item{
    Assume in addition that $p < q$ and $p$ does not divide $q -
    1$. Show that a group of order $p^2 q$ has to be abelian.
  }
\end{enumerate}
\end{Problem}

\begin{Answer}
\begin{enumerate}
  \item{
    Let $G$ be a group of order $|G| = p^2 q$.

    Let $P$ be an arbitrary $p$-Sylow of $G$ and let $Q$ be an
    arbitrary $q$-Sylow of $G$. Then $P \cap Q$ is a subgroup of $P$
    and a subgroup of $Q$, so $|P \cap Q|$ divides $|P| = p^2$ and
    $|P \cap Q|$ divides $|Q| = q$, and since $p$ and $q$ are distinct
    primes this is only possible when $|P \cap Q| = 1$. Therefore $P
    \cap Q$ is trivial.

    The number of $p$-Sylows $N_{p^2}$ must divide the order $p^2q$ of the
    group, so either $N_{p^2} \vert p^2$ or $N_{p^2} \vert q$. But
    we also must have $N_{p^2} \equiv 1 \pmod{p}$, so it cannot be true
    that $N_{p^2} \vert p^2$ for in that case $N_{p^2} \equiv 0
    \pmod{p}$.
    Therefore either $N_{p^2} = 1$ or $N_{p^2} = q$.

    Suppose $N_{p^2} = q$. Then each of these $p$-Sylows has order
    $p^2$, so there are $q (p^2 - 1)$ non-identity elements belonging
    to the $p$-Sylows of $G$. Then there are
    $$
    |G| - (qp^2 - q) = qp^2 - (qp^2 - q) = q
    $$
    elements in the $q$-Sylows of $G$. But each $q$-Sylow has order
    $q$, so in this case there is a unique $q$-Sylow. Then there
    is either a unique $p$-Sylow or a unique $q$-Sylow which is
    therefore normal. Thus $G$ has a normal Sylow subgroup.
    Label this group $H$ and note that either $|H| = p^2$ or
    $|H| = q$.

    Consider $G / H$. If $|H| = p^2$ then $|G / H| = q$ is prime and
    so $G / H$ is cyclic and thus abelian. But $H$ is a $p$-group and
    therefore solvable, so it follows that $G$ is solvable.

    If $|H| = q$ then $|G / H| = p^2$. Then this is a $p$-group of
    order greater than 1, so its
    center is nontrivial. Then since $|Z(G / H)|$ divides $p^2$ and is
    not 1, it must be $p^2$ (in which case $Z(G / H) = G / H$ and so
    $G / H$ is abelian) or else $p$. But if $|Z(G / H)| = p$ then
    $\left|\frac{G / H}{Z(G / H)}\right| = p$ and thus
    $\frac{G / H}{Z(G / H)}$ is cyclic, whence $G / H$ is abelian. But
    in this case $H$ is of prime order and thus cyclic, so we have an
    abelian tower $G \triangleright H \triangleright \{ 1 \}$.
  }
  \item
  {
    Let $G$ be a group of order $p^2 q$, where $p$, $q$ are distinct
    primes and $p$ does not divide $q-1$. For any $p$-Sylow $P$
    and any $q$-Sylow $Q$ we have $P \cap Q = \{ 1 \}$, so
    $$
    |PQ| = \frac{|P||Q|}{|P \cap Q|} = |P||Q| = p^2 q = |G|
    $$
    and therefore it must be the case that $G = PQ$.

    We know that there is either a normal $q$-Sylow or a normal
    $p$-Sylow. In fact, under the restriction that $p$ does not divide
    $q - 1$, it must be that there is a unique $q$-Sylow. This is
    because $N_q \equiv 1 \pmod{q}$, so $N_q$ does not divide $q$ and
    therefore must divide $p^2$, but this is only possible if $N_q =
    1$ because $p \not\equiv 1 \pmod{q}$ by assumption.

    Call the unique $q$-Sylow $Q$ and let $P$ be any $p$-Sylow.
    Since $G = PQ$, $Q \triangleleft G$, and $P \cap Q = \{ 1 \}$,
    we have $G = Q \rtimes_\psi P$, where $\psi : P \to
    \mathrm{Aut}(Q)$ is the conjugation action of $P$ on $Q$.

    We know that $\mathrm{Aut}(Q) = \mathbb{Z}_q^\ast$ has order $q -
    1$, and from the first isomorphism theorem
    $$
    P / \ker \psi \simeq \mathrm{Im}(\psi) < \mathrm{Aut}(Q).
    $$
    Therefore $|\mathrm{Im}(\psi)|$ divides
    $|\mathrm{Aut}(Q)| = q - 1$ and furthermore
    $|\mathrm{Im}(\psi)| = |P / \ker \psi|$, so $|\mathrm{Im}(\psi)|$
    divides $|P| = p^2$ as well. This is only possible if
    $|\mathrm{Im}(\psi)| = 1$ and so
    $\mathrm{Im}(\psi) = \{ \mathrm{id} \}$. Therefore
    $\psi(g_P)(g_Q) = g_P g_Q g_P^{-1} = g_Q$ for any $g_P \in P$ and
    $g_Q \in Q$, so $g_P g_Q = g_Q g_P$. Therefore $G = PQ$ is abelian.
  }
\end{enumerate}
\end{Answer}

\pagebreak

\begin{Problem}
  \begin{enumerate}[(a)]
    \item{
      Show that $S_n$ is generated by the transpositions
      $(12),(23),\dots,(n-1 \to n)$.
    }
    \item{
      Show that $S_n$ is generated by $(12)$ and the $n$-cycle
      $(12 \cdots n)$.
    }
    \item{
      Show that if $n$ is a prime number, then $S_n$ is generated by
      $(1 2 \cdots n)$ and any transposition $(rs)$, $r \neq s$.
    }
  \end{enumerate}
\end{Problem}

\begin{Answer}
\begin{enumerate}
  \item{
    Notice that
    $$
    (12)(23)(12) = (13).
    $$
    Furthermore suppose $(1 \to k - 1) \in H$. Then
    $$
    (1 \to k - 1)(k - 1 \to k)(1 \to k - 1) =
    (1k),
    $$
    so $(1k) \in H$ as well.
    Therefore $H$ contains every transposition $(1k)$ for
    $k \in \{ 2, \dots, n \}$, and we know that the transpositions
    $(12)$, $(13)$, \dots, $(1n)$ generate $H$.
  }
  \item{
    It is easier to see why the claim holds if we
    denote the letters of the symmetric group $S_n$ by $\{0, 1, \dots,
    n-1\}$. Then the conjugation
    $$
    (0 \to 1 \to \cdots \to n-1)
    (rs)
    (0 \to 1 \to \cdots \to n-1)^{-1} =
    (r + 1 \bmod n \to s + 1 \bmod n).
    $$
    If $r \equiv s \pm 1 \pmod{n}$ then the orbit of this conjugation consists of
    all transpositions of the form $(i-1 \to i)$ for $i \in \{2, \dots,
    n\}$, i.e. the elements
    $(12)$, $(23)$, \dots, $(n-1 \to n)$, which from part (a)
    generate $S_n$.
  }
  \item{
    Let $n$ be a prime and denote the letters of the symmetric group
    by $\{0, 1, \dots, n - 1\}$ as in part (b). Again we note that
    conjugation of $(rs)$ by $(0 \to 1 \to \cdots \to n-1)$
    corresponds to adding 1 modulo $n$ to both $r$ and $s$.
    Denote $\sigma = (0 \to 1 \to \cdots \to n-1)$. Then we observe
    that we can arrive at $(r + k \bmod n \to s + k \bmod n)$ by
    the conjugation $\sigma^k (rs) \sigma^{-k}$.

    Since $n$ is prime, this means every
    nonzero element $m \in \mathbb{Z}_n$ has a multiplicative inverse so that
    \begin{align*}
    1 &= m^{-1} m
       = \sum_{i=1}^{m^{-1}} m \\
      &= \sum_{i=1}^{m^{-1} - 1} m + m
       = m(m^{-1} - 1) + m \\
      &= m + \sum_{i=1}^{m(m^{-1} - 1)} 1.
    \end{align*}

    If $r \neq 0$, then
    $\sigma^{-r}(rs)\sigma^{r} = (0 \to s - r \bmod n)$ where
    $s - r \bmod n \neq 0$. Therefore it is enough to show that
    $S_n$ is generated by $\sigma$ and a transposition
    of the form $(0q)$.

    Considering $q$ as an integer modulo $n$, there is an inverse $q^{-1}$ in
    $\mathbb{Z}_n$, so
    $$
    \sigma^{q(q^{-1} - 1)} (0q) \sigma^{q(q^{-1} - 1)}
      = (q(q^{-1} - 1) \bmod n \to 1)
      = (1 - q \bmod n \to 1).
    $$

    Since $n$ is prime, $q$ is coprime to $n$, so there exists a $k$
    such that $kq \equiv 1 \pmod{n}$, so  $(k - 1)q + q \equiv 1 \pmod{n}$ or
    $(k - 1)q \equiv 1 - q \pmod{n}$. Therefore there is a multiple of
    $q$ congruent to $1 - q$ modulo $n$. But there is a sequence of
    transpositions
    \begin{align*}
    ((k-2)q \to (k-1)q)
      &= \sigma^{q}((k-3)q \to (k-2)q)\sigma^{-q}, \\
    ((k-3)q \to (k-2)q)
      &= \sigma^{q}((k-4)q \to (k-3)q)\sigma^{-q}, \\
      &\cdots \\
    (q \to 2q)
      &= \sigma^{q}(0 q)\sigma^{-q},
    \end{align*}
    and by conjugating $(0q)$ by each of these in turn we get
    $$
    (0 \to (k - 1)q) = (0 \to 1 - q \bmod n).
    $$
    Then
    $$
    (0 \to 1 - q) (1 - q \bmod n \to 1) (0 \to 1 - q)
      = (01),
    $$
    and from part (a) we know $\{ (01), \sigma \}$ generates $S_n$.
  }
\end{enumerate}
\end{Answer}

\pagebreak

\begin{Problem}
\begin{enumerate}[(a)]
  \item{
    How many abelian groups of order 3600 are there (up to isomorphism)?
  }
  \item{
    Give an example of an abelian group which is torsion-free but not free.
  }
  \item{
    Given an example of an abelian group which is both torsion and uncountable.
  }
\end{enumerate}
\end{Problem}

\begin{Answer}
\begin{enumerate}[(a)]
  \item{
    Let $A$ be an abelian group of order 3600.
    3600 has the prime factorization $2^4 \cdot 3^2 \cdot
    5^2$. Therefore
    $$
    A \simeq A[2^\infty] \times A[3^\infty] \times A[5^\infty]
    $$
    where
    $$
    |A[2^\infty]| = 2^4, \quad
    |A[3^\infty]| = 3^2, \quad
    |A[5^\infty]| = 5^2.
    $$
    $A[2^\infty] = A[2^4]$ has one of the following forms:
    $$
    \mathbb{Z}_{2^4}, \quad
    \mathbb{Z}_{2^3} \times \mathbb{Z}_{2}, \quad
    \mathbb{Z}_{2^2} \times \mathbb{Z}_{2}^2, \quad
    \mathbb{Z}_{2}^4
    $$
    $A[3^\infty] = A[3^2]$ has one of the following forms:
    $$
    \mathbb{Z}_{3^2}, \quad \mathbb{Z}_{3}^2
    $$
    $A[5^\infty] = A[5^2]$ has one of the following forms:
    $$
    \mathbb{Z}_{5^2}, \quad \mathbb{Z}_{5}^2.
    $$

    An abelian group consists of a combination of one option for
    $A[2^4]$, one option for $A[3^2]$, and one option for $A[5^2]$, e.g.
    $$
    \mathbb{Z}_{25} \times \mathbb{Z}_{16} \times \mathbb{Z}_{9},
    \quad
    \mathbb{Z}_{9} \times \mathbb{Z}_{8} \times \mathbb{Z}_5^2 \times \mathbb{Z}_2,
    $$
    et cetera. There are therefore $4 \cdot 2 \cdot 2 = 16$ such
    groups. Written in invariant factors, they are:
    \begin{align*}
      & \mathbb{Z}_{2}^{2} \times \mathbb{Z}_{30} \times \mathbb{Z}_{30} \\
      & \mathbb{Z}_{2}^{2} \times \mathbb{Z}_{6} \times \mathbb{Z}_{150} \\
      & \mathbb{Z}_{2}^{2} \times \mathbb{Z}_{10} \times \mathbb{Z}_{90} \\
      & \mathbb{Z}_{2}^{3} \times \mathbb{Z}_{450} \\
      & \mathbb{Z}_{2} \times \mathbb{Z}_{30} \times \mathbb{Z}_{60}    \\
      & \mathbb{Z}_{2} \times \mathbb{Z}_{6} \times \mathbb{Z}_{300}    \\
      & \mathbb{Z}_{2} \times \mathbb{Z}_{10} \times \mathbb{Z}_{180}   \\
      & \mathbb{Z}_{2}^{2} \times \mathbb{Z}_{900}                      \\
      & \mathbb{Z}_{30} \times \mathbb{Z}_{120}                         \\
      & \mathbb{Z}_{6} \times \mathbb{Z}_{600}                          \\
      & \mathbb{Z}_{10} \times \mathbb{Z}_{360}                         \\
      & \mathbb{Z}_{2} \times \mathbb{Z}_{1800}                         \\
      & \mathbb{Z}_{15} \times \mathbb{Z}_{240}                         \\
      & \mathbb{Z}_{3} \times \mathbb{Z}_{1200}                         \\
      & \mathbb{Z}_{5} \times \mathbb{Z}_{720}                          \\
      & \mathbb{Z}_{3600}
    \end{align*}

  }
  \item{
    The rationals are torsion-free. Let $\frac{r}{s} \in
    \mathbb{Q}$ and suppose $n \frac{r}{s} = 0$ for $n \geq 1$.
    Then $n r = 0$ and thus $r = 0$, so $\frac{r}{s} = 0$.

    The rationals are not free. Suppose $\mathbb{Q}$ is free, and let
    $q \in \mathbb{Q}$. Then $q$ may be written as
    $$
    q = \sum_{i \in I} n_i \frac{r_i}{s_i}
    $$
    for some set $I$ and some $n_i, r_i \in \mathbb{Z}$, $s_i \in \mathbb{Z}_+$. But then
    for instance we have
    $$
    q
      = \sum_{i \in I} n_i
          \frac{r_i\prod_{\substack{j \in I}{j \neq i}} s_j}
               {s_i \prod_{\substack{j \in I}{j \neq i}} s_j}
      = \frac{\sum_{i \in I} n_i r_i
                \prod_{\substack{j \in I}{j \neq i}} s_j}
             {\prod_{i \in I} s_i},
    $$
    so the representation is not unique.
%
%   $\varphi : \mathbb{Q} \to \mathbb{Z}[I]$ be a group homomorphism
%
%   for some set $I$, and let $\frac{r_1}{s_1}, \frac{r_2}{s_2} \in
%   \mathbb{Q}$. Then without loss of generality
%   $$
%   \varphi\left(\frac{r_1}{s_1}\right) =
%     (k_1, k_2, \dots, k_m, 0, 0, \dots)
%   $$
%   and
%   $$
%   \varphi\left(\frac{r_2}{s_2}\right) =
%     (l_1, l_2, \dots, l_n, 0, 0, \dots)
%   $$
%   for some integers $k_i$, $l_i$.
  }
  \item{
    The direct product
    $$
    A = \prod_{i=1}^\infty \mathbb{Z}_2
    $$
    is abelian, since $\mathbb{Z}_2$ is abelian and therefore so is
    the direct product.

    Take $x \in \mathbb{Z}_2^n$ so that $x = (x_i)_{i=1}^\infty$. Then
    $x + x = (x_i + x_i)_{i=1}^\infty = (0)_{i=1}^\infty$, so $A$ is torsion.
    But $A$ consists of all infinite sequences of binary digits, and
    therefore is in bijection with the interval $[0, 1)$ since these
    can be regarded as binary representations of the nonnegative real
    numbers with zero integer part.
    Therefore $A$ is abelian, torsion, and uncountable.
  }
\end{enumerate}
\end{Answer}

\pagebreak

\begin{Problem}
Consider a $2 \times 2$ matrix
$A = \left(\begin{array}{c c}
       a & b \\ c & d
     \end{array}\right)$
with integer coefficients, i.e.
$a, b, c, d \in \mathbb{Z}$. Let
$\mathbb{Z}^2 = \mathbb{Z} \oplus \mathbb{Z}$ as usual.
\begin{enumerate}[(a)]
  \item{
    Show that $(u, v) \mapsto (au + bv, cu + dv)$ gives a group
    homomorphism $\varphi_A : \mathbb{Z}^2 \to \mathbb{Z}^2$.
  }
  \item{
    Show that $\varphi_A$ is injective if and only if
    $\det(A) \neq 0$.
  }
  \item{
    Show that $\varphi_A$ is surjective if and only if
    $\det(A) = \pm 1$ and then $\varphi_A$ is an isomorphism.
  }
  \item{
    Show that if $\det(A) \neq 0$, then the quotient
    $\mathbb{Z}^2 / \mathrm{Im}(\varphi_A)$ is a finite group.
  }
\end{enumerate}
\end{Problem}

\begin{Answer}
\begin{enumerate}[(a)]
  \item{
    We have that
    \begin{align*}
    \varphi_A((u, v) + (w, x)) &=
      \varphi_A((u + w, v + x)) =
      (a(u + w) + b(v + x), c(u + w) + d(v + x)) \\
      &= ((au + bv) + (aw + bx), (cu + dv) + (cw + vx)) \\
      &= \varphi_A((u, v)) + \varphi_A((w, x))
    \end{align*}
    so $\varphi_A$ is a group homomorphism.
  }
  \item{
    We show $\det(A) \neq 0$ iff. $\varphi_A$ is injective.
    \begin{itemize}
      \item[($\implies$)]
      {
        Let $\varphi_A$ be injective. Then
        $\ker(\varphi_A) = \{(0, 0)\}$. Notice that
        $$
        \varphi_A((d, -c)) = (ad - bc, 0)
        $$
        and
        $$
        \varphi_A((-b, a)) = (0, ad - bc).
        $$
        Suppose $ad - bc = 0$. Then this means
        $(d, -c) \in \ker(\varphi_A)$ and $(-b, a) \in \ker(\varphi_A)$,
        which means $a = b = c = d = 0$. But this is impossible, since in
        this case $\varphi_A(x) = (0, 0)$ for all $x \in \mathbb{Z}^2$,
        and $\varphi_A$ is injective by assumption. Therefore
        $\det(A) = ad - bc \neq 0$.
      }
      \item[($\impliedby$)]
      {
        Suppose $\det(A) = ad - bc \neq 0$. Let $(u, v) \in
        \ker(\varphi_A)$. Then $(au + bv, cu + dv) = (0, 0)$
        so $au + bv = 0$ and $cu + dv = 0$. But this means
        $adu + bdv = 0$ and $bcu + bdv = 0$, so subtracting these
        equations gives $(ad - bc)u = 0$, so since $ad - bc \neq 0$ it
        must be the case that $u = 0$. Similarly, we have
        $acu + bcv = 0$ and $acu + adv = 0$ so that $(ad - bc)v = 0$, so
        $v = 0$ as well. Therefore $(u, v) \in \ker(\varphi_A)$ implies
        $(u, v) = (0, 0)$ in this case, so the kernel is trivial and
        therefore $\varphi_A$ is injective.
      }
    \end{itemize}
  }
  \item
  {
    We show $\varphi_A$ is surjective iff. $\det(A) = \pm 1$.
    \begin{itemize}
      \item[($\implies$)]
      {
        Suppose $\det(A) = ad - bc = 1$, and let $(x, y) \in
        \mathbb{Z}^2$.
        Then
        $$
        \varphi_A
          \left(\begin{array}{r}
            d \\ -c
          \end{array}\right) =
          \left(\begin{array}{c}
            ad - bc \\ 0
          \end{array}\right) =
          \left(\begin{array}{c}
            1 \\ 0
          \end{array}\right)
        $$
        and
        $$
        \varphi_A
          \left(\begin{array}{r}
            -b \\ a
          \end{array}\right) =
          \left(\begin{array}{c}
            0 \\ ad - bc
          \end{array}\right) =
          \left(\begin{array}{c}
            0 \\ 1
          \end{array}\right)
        $$
        so
        $$
        \left(\begin{array}{c}
          x \\ y
        \end{array}\right) =
          x
          \left(\begin{array}{c}
            1 \\ 0
          \end{array}\right) +
          y
          \left(\begin{array}{c}
            0 \\ 1
          \end{array}\right) =
          x \varphi_A
          \left(\begin{array}{r}
            d \\ -c
          \end{array}\right) +
          y \varphi_A
          \left(\begin{array}{r}
            -b \\ a
          \end{array}\right)
        $$
        and therefore
        $$
        \left(\begin{array}{c}
          x \\ y
        \end{array}\right) =
        \varphi_A \left(
          x
          \left(\begin{array}{r}
            d \\ -c
          \end{array}\right) +
          y
          \left(\begin{array}{r}
            -b \\ a
          \end{array}\right)
        \right)
        $$
        Similarly if $\det(A) = -1$ we have
        $$
        \left(\begin{array}{c}
          x \\ y
        \end{array}\right) =
        \varphi_A \left(
          -x
          \left(\begin{array}{r}
            d \\ -c
          \end{array}\right) +
          -y
          \left(\begin{array}{r}
            -b \\ a
          \end{array}\right)
        \right)
        $$
        and therefore $(x, y) \in \mathrm{Im}(\varphi_A)$ in either of
        these cases. Therefore $\varphi_A$ is surjective.
      }
      \item[($\impliedby$)]
      {
        Suppose $\varphi_A$ is surjective.
        Note that $\mathbb{Z}^2$ is free.
        Then since $\varphi_A$ is a surjective homomorphism between
        abelian groups with free image, it has a right inverse
        $\varphi_B : \mathbb{Z}^2 \to \mathbb{Z}^2$ given by
        $$
        \varphi_B\left(
          x\left(\begin{array}{c}
            1 \\ 0
          \end{array}\right)
          +
          y\left(\begin{array}{c}
            0 \\ 1
          \end{array}\right)
        \right) =
        x\left(\begin{array}{c}
          u_1 \\ v_1
        \end{array}\right)
        +
        y\left(\begin{array}{c}
          u_2 \\ v_2
        \end{array}\right) =
        \left(\begin{array}{c c}
          u_1 & u_2 \\
          v_1 & v_2
        \end{array}\right)
        \left(\begin{array}{c}
          x \\ y
        \end{array}\right).
        $$
        Then $\varphi_B$ also has a matrix representation
        $$
        B =
        \left(\begin{array}{c c}
          u_1 & u_2 \\
          v_1 & v_2
        \end{array}\right).
        $$
        But since $\varphi_A \circ \varphi_B = \mathrm{id}$ this means
        \begin{align*}
        (\varphi_A \circ \varphi_B)
        \left(\begin{array}{c}
          x \\ y
        \end{array}\right) &=
        A B
        \left(\begin{array}{c}
          x \\ y
        \end{array}\right) \\ &=
        \left(\begin{array}{c}
          x \\ y
        \end{array}\right) =
        I
        \left(\begin{array}{c}
          x \\ y
        \end{array}\right),
        \end{align*}
        where $I$ is the $2 \times 2$ identity matrix. This means the
        matrix product $A B = I$. But then
        $\det(A) \det(B) = \det(I) = 1$, which means $\det(A)$ has a
        multiplicative inverse. Since $\det(A)$ is an integer this can
        only be true if $\det(A) = \pm 1$.
      }
    \end{itemize}
    Therefore $\varphi_A$ is surjective iff. $\det(A) = \pm 1$. But in
    this case
    $\det(A) \neq 0$, so $\varphi_A$ is injective as well, and
    therefore an isomorphism.
  }
  \item
  {
    Consider the matrix
    $$
    \hat{A} =
    \left(\begin{array}{r r}
       d & -b \\
      -c &  a
    \end{array}\right)
    $$
    and the homomorphism $\varphi_{\hat{A}} : \mathbb{Z}^2 \to \mathbb{Z}^2$
    given by $\varphi_{\hat{A}}
                \left(\begin{array}{c}
                  u \\ v
                \end{array}\right) =
                \hat{A}\left(\begin{array}{c}
                  u \\ v
                \end{array}\right)$.
    We find that
    $$
    (\varphi_{\hat{A}} \circ \varphi_A)
      \left(\begin{array}{c}
        u \\ v
      \end{array}\right) =
      \left(\begin{array}{c c}
        ad - bc &       0 \\
              0 & ad - bc
      \end{array}\right)
      \left(\begin{array}{c}
        u \\ v
      \end{array}\right) =
      \left(\begin{array}{c}
        (ad - bc)u \\ (ad - bc)v
      \end{array}\right).
    $$

    For $n \in \mathbb{Z}_+$, define the map
    $\mu_n : \mathbb{Z}^2 \to \mathbb{Z}_n^2$ by
    $$
    \mu_n
    \left(\begin{array}{c}
      u \\ v
    \end{array}\right) =
    \left(\begin{array}{c}
      u \bmod n \\
      v \bmod n
    \end{array}\right).
    $$
    Also define $\psi_A = \mu_{|\det A|} \circ \varphi_{\hat{A}}$.

    We claim the following:
    \begin{itemize}
      \item{
        $\mu_n$ is a homomorphism for any $n \in \mathbb{Z}_+$
        and therefore so is $\psi_A$ for any $A \in \mathrm{GL}_2(\mathbb{Z})$,
      }
      \item{
        if $\det A > 0$ then $\mathrm{Im}(\varphi_A) = \ker(\psi_A)$, and
      }
      \item{
        if $\det A < 0$ then $\mathrm{Im}(\varphi_A) = \ker(\psi_{(-A)})$.
      }
    \end{itemize}

    This is demonstrated below.
    \begin{itemize}
      \item{
        Let $n$ be a positive integer. We have
        \begin{align*}
         \mu_n \left[
           \left(\begin{array}{c}
             x_1 \\ y_1
           \end{array}\right)
           +
           \left(\begin{array}{c}
             x_2 \\ y_2
           \end{array}\right)
         \right]
         &=
          \mu_n
            \left(\begin{array}{c}
              x_1 + x_2 \\
              y_1 + y_2
            \end{array}\right) =
          \left(\begin{array}{c}
            (x_1 + x_2) \bmod n \\
            (y_1 + y_2) \bmod n
          \end{array}\right) \\ &=
          \left(\begin{array}{c}
            x_1 \bmod n + x_1 \bmod n \\
            y_1 \bmod n + y_2 \bmod n
          \end{array}\right) \\ &=
          \left(\begin{array}{c}
            x_1 \bmod n \\
            y_1 \bmod n
          \end{array}\right)
          +
          \left(\begin{array}{c}
            x_2 \bmod n \\
            y_2 \bmod n
          \end{array}\right) \\ &=
          \mu_n\left(\begin{array}{c}
            x_1 \\
            y_1
          \end{array}\right)
          +
          \mu_n\left(\begin{array}{c}
            x_2 \\
            y_2
          \end{array}\right)
        \end{align*}
        so $\mu_n$ is a homomorphism. Let $A \in
        \mathrm{GL}_2(\mathbb{Z})$. Then $\det A \neq 0$ and therefore
        $|\det A|$ is a positive integer, so
        $\psi_A = \mu_{|\det A|} \circ \varphi_{\hat{A}}$ is a
        composition of homomorphisms and thus a homomorphism itself.
      }
      \item{
        Suppose $\det A > 0$, and let $\alpha \in
        \mathrm{Im}(\varphi_A)$.
        Then $|\det A| = \det A = ad - bc$, and
        $\alpha$ has the form
        $$
        \alpha =
        \left(\begin{array}{c}
          a u + b v \\
          c u + d v
        \end{array}\right)
        $$
        for some $u, v \in \mathbb{Z}$. Then
        \begin{align*}
        \varphi_{\hat{A}}(\alpha) &=
        \left(\begin{array}{r r}
           d & -b \\
          -c &  a
        \end{array}\right)
        \left(\begin{array}{c}
          a u + b v \\
          c u + d v
        \end{array}\right) =
        \left(\begin{array}{r}
          d(a u + b v) - b(c u + d v) \\
         -c(a u + b v) + a(c u + d v)
        \end{array}\right) \\ &=
        \left(\begin{array}{r}
          (a d - b c) u \\
          (a d - b c) v
        \end{array}\right)
        \end{align*}
        so
        \begin{align*}
        (\mu_{|\det A|} \circ \varphi_{\hat{A}})(\alpha) &=
        \left(\begin{array}{r}
          (a d - b c) u \bmod (ad - bc) \\
          (a d - b c) v \bmod (ad - bc)
        \end{array}\right) =
        \left(\begin{array}{r}
          0 \\
          0
        \end{array}\right)
        \end{align*}
        and thus $\alpha \in \ker(\mu_{|\det A|} \circ
        \varphi_{\hat{A}}) = \ker(\psi_A)$.

        Suppose next
        $\alpha =
         \left(\begin{array}{c}
           x \\
           y
         \end{array}\right)
         \in \ker(\psi_A)$. Then $\varphi_{\hat{A}}(\alpha) \in \ker(\mu_{|\det A|})$, so
         $$
         \varphi_{\hat{A}}(\alpha) =
         \left(\begin{array}{c}
           (ad - bc)u \\
           (ad - bc)v
         \end{array}\right)
         $$
         for some $u, v \in \mathbb{Z}$. But
         $$
         \varphi_{\hat{A}}(\alpha) =
         \left(\begin{array}{r r}
            d & -b \\
           -c &  a
         \end{array}\right)
         \left(\begin{array}{c}
           x \\
           y
         \end{array}\right)
         $$
         so we have
         $$
         \left(\begin{array}{r}
           xd - yb \\
          -xc + ya
         \end{array}\right) =
         \left(\begin{array}{c}
           (ad - bc)u \\
           (ad - bc)v
         \end{array}\right)
         $$
         for some $u, v \in \mathbb{Z}$. But this means
         $$
         xd = yb + (ad - bc)u
         $$
         so
         $$
         xcd = ybc + (ad - bc)cu
         $$
         and
         $$
         (ad - bc)dv =
           -xcd + yad =
           -ybc - (ad - bc)cu + yad
         $$
         or
         $$
         (y - cu)(ad - bc) = dv(ad - bc)
         $$
         so that
         $$
         y = cu + dv.
         $$
         Similarly
         $$
         yab = xbc + (ad - bc)bv
         $$
         and
         $$
         (ad - bc)au =
           xad - yab =
           (x - bv)(ad - bc)
         $$
         so
         $$
         x = au + bv
         $$
         and therefore
         $$
         \left(\begin{array}{c}
           x \\
           y
         \end{array}\right) =
         \left(\begin{array}{c}
           au + bv \\
           cu + dv
         \end{array}\right)
         \in \mathrm{Im}(\varphi_A).
         $$
         Therefore $\alpha \in \mathrm{Im}(\varphi_A)$. We conclude
         that
         $\ker(\psi_{A}) = \mathrm{Im}(\varphi_A)$.
      }
      \item{
        Suppose $\det A < 0$. Then $|det A| = -\det A = \det (-A) > 0$
        and so $\ker \psi_{-A} = \mathrm{Im}(\varphi_A)$ just as above.
      }
    \end{itemize}

    Therefore when $\det A \neq 0$ there is a homomorphism
    $\psi : \mathbb{Z}^2 \to \mathbb{Z}_{|\det A|}^2$ such that
    $\mathrm{Im}(\varphi_A) = \ker(\psi)$. But then
    $$
    \mathbb{Z}^2 / \mathrm{Im}(\varphi_A) =
    \mathbb{Z}^2 / \ker(\psi) \simeq
    \mathrm{Im}(\psi) \subset \mathbb{Z}_{|\det A|}^2,
    $$
    and $\mathbb{Z}_{|\det A|}^2$ is a finite group so
    $\mathbb{Z}^2 / \mathrm{Im}(\varphi_A)$ is as well.
  }
\end{enumerate}
\end{Answer}

\pagebreak

\begin{Problem}
By definition a \emph{lattice} in $\mathbb{R}^n$ is a subgroup $L$ of
$\mathbb{R}^n$ which contains a basis of the vector space
$\mathbb{R}^n$ and is discrete
$\mathbb{R}^n$ in the topology of $\mathbb{R}^n$.
\begin{enumerate}[(a)]
  \item{
    Assume $n = 1$. Show that a lattice $L$ in $\mathbb{R}$ is
    generated by a single real number $r$ and so
    $L = \{ n \cdot r \mid n \in \mathbb{Z} \}$. Conclude that, as an
    abstract group, $L$ is isomorphic to $\mathbb{Z}$.
  }
  \item{
    Assume $n = 2$. Show that every lattice $L$ in $\mathbb{R}^2$ is
    generated by a basis $\{f_1, f_2\}$ of $\mathbb{R}^2$. Conclude
    that every lattice $L$ in $\mathbb{R}^2$ is, as an abstract group,
    isomorphic to $\mathbb{Z}^2$.
  }
  \item{
    Extra credit: use induction to show that every lattice in
    $\mathbb{R}^n$ is isomorphic to $\mathbb{Z}^n$ as an abstract group.
  }
\end{enumerate}
\end{Problem}

\begin{Answer}
  \begin{enumerate}
    \item{
      Let $L$ be a lattice in $\mathbb{R}$. Then $L$ contains a basis
      $\{ \beta \}$ of $\mathbb{R}$, where $\beta \in
      \mathbb{R}$. Since $L$ is a subgroup of $\mathbb{R}$ it must be
      closed under addition, and so $n \beta \in L$ for any $n \in \mathbb{Z}$.
      Therefore $\{ n \beta \mid n \in \mathbb{Z} \} \subset L$.

      Suppose $l \in L$. Since $\beta$ is a basis for $\mathbb{R}$, we
      have $l = a \beta$ for some $a \in \mathbb{R}$. If $l$ is a
      rational multiple $\frac{r}{s}\beta$ we may
      instead choose $\beta^\prime = \frac{\beta}{s}$ and then $l$ is
      an integer multiple of $\beta^\prime$, and
      $L \supset \{ n \beta^\prime \mid n \in \mathbb{Z} \}$ as
      before, so $L$ consists of integer multiples of $\beta^\prime$.
      Therefore we suppose  $a$ is irrational and derive a contradiction.

      Define the fractional-part function $\{ \cdot \} : \mathbb{R}
      \to [0,1)$ by $\{ x \} = x - \floor{x}$.
      Let $x \in \mathbb{R}$,
      $k \in \mathbb{Z}_+$, and divide $[0,1)$ into subintervals
      $$
      \left[ 0, \frac{1}{k} \right),
      \left[ \frac{1}{k}, \frac{2}{k} \right),
      \dots,
      \left[ \frac{k-1}{k}, 1 \right).
      $$
      By the pigeonhole principle, there exist $s, t \in \mathbb{Z}_+$
      such that $\{ s x \}$, $\{ t x \}$ belong to the same
      subinterval and therefore such that
      $|\{ s x \} - \{t x\}| < \frac{1}{k}$. But
      $$
      |\{ s x \} - \{t x\}|
        = |s x - \floor{s x} - (tx - \floor{t x})|
        = |(s - t)x - \floor{s x} + \floor{t x}|
      $$
      whereas
      $$
      \{ (s - t)x \}
        = (s - t)x - \floor{(s - t)x}
        < (s - t)x - \floor{s x},
      $$
      assuming $s > t$ without loss of generality. Therefore
      $\{ (s - t)x \} < \frac{1}{k}$ where $k$ is arbitrary.

      Therefore there exists an integer multiple of $a \beta$ with arbitrarily
      small fractional part.
      In particular, since $L$ is discrete, let $\delta > 0$ denote the radius of the
      smallest ball centered on an integer multiple $n \beta \in L$.
      There exists an integer $m$ such that
      $m a \beta - \floor{m a} \beta < \delta$, or
      $m a \beta \in B_{\delta}(\floor{m a} \beta)$.
      But if $a \beta \in L$ then so must $m a \beta$ be,
      and since $\floor{m a} \beta$ is an integer multiple of $\beta$
      we know $\floor{m a} \beta \in L$. This implies
      $$
      B_{\delta}(\floor{m a} \beta) \cap L \supset \{ m a \beta,
      \floor{m a} \beta \} \neq \{ \floor{m a} \beta \}
      $$
      which means $L$ is not a lattice. Therefore this case cannot occur.

      We conclude that $\{ n r \mid n \in \mathbb{Z} \} = L$ for some
      $r \in \mathbb{R}$. Then $f : L \to \mathbb{Z}$ given by
      $n r \mapsto n$ is an isomorphism since it has inverse
      $n \mapsto n r$, so $L \simeq \mathbb{Z}$.
    }
    \item{
      Define the group action
      $\psi : \mathbb{R} \to \mathrm{Aut}(\mathbb{R}^2)$
      of rotations
      $$
      \psi(\theta)(x) =
        \left(\begin{array}{r r}
          \cos \theta & -\sin \theta \\
          \sin \theta &  \cos \theta
        \end{array}\right) x,
      $$
      which indeed produces automorphisms since
      $\psi(\theta)$ is an invertible linear map.
      Furthermore this transformation preserves discreteness of
      subgroups, and even the radii of individual balls around points
      in $L$, since it rotates the plane but does not affect the
      distances between points.

      Let
      $\left\{
           v = \left(\begin{array}{c}
                 v_1 \\ v_2
               \end{array}\right)
         , w = \left(\begin{array}{c}
                 w_1 \\ w_2
               \end{array}\right)
       \right\} \subset L$ be a basis for
      $\mathbb{R}^2$, and let
      \begin{align*}
      V &= \{ \alpha \cdot v + 0     \cdot w
             \mid \alpha \in \mathbb{R}
           \} \cap L, \\
      W &= \{ 0      \cdot v + \beta \cdot w
             \mid \beta \in \mathbb{R}
          \} \cap L.
      \end{align*}
      Notice that $V$ and $W$ are subgroups of $L$,
      that $V \cap W = \{ 0 \}$, and $L = V + W$, so that
      $L = V \oplus W$.

      We can obtain the isomorphic image
      $\psi(-\theta)(v) =
       \left(\begin{array}{c}
         v_1^\prime \\ 0
       \end{array}\right)$,
       where $\theta$ is the argument of $v$. But $v_1$ is a basis of
       $\mathbb{R}$, and so the subgroup
       \begin{align*}
         \psi(-\theta)(V)
         &= \psi(-\theta)(L) \cap
            \{   \alpha \cdot \psi(-\theta)(v)
               + 0      \cdot \psi(-\theta)(w)
               \mid \alpha \in \mathbb{R}
            \} \\
         &= \psi(-\theta)(L) \cap
            \{ \alpha \cdot v_1^\prime,
               \mid \alpha \in \mathbb{R}
            \} \\
         &= \psi(-\theta)(L) \cap
            \mathbb{R}v_1^\prime
       \end{align*}
       is a discrete subgroup of $\mathbb{R}$ that contains a basis
       $\{ v_1^\prime \}$ of $\mathbb{R}$ and so it is
       isomorphic to $\mathbb{Z}$ from part (a). Namely
       $\psi(-\theta)(V) = \mathbb{Z} v_1^\prime$, so we have
       \begin{align*}
       \psi(-\theta)(L)
        &= \psi(-\theta)(V) \oplus \psi(-\theta)(W) \\
        &= \mathbb{Z} v_1^\prime \oplus \psi(-\theta)(W).
       \end{align*}
       Then
       \begin{align*}
       L
        &= \mathbb{Z} \psi(\theta)(v_1^\prime) \oplus W \\
        &= \mathbb{Z} v_1 \oplus W.
       \end{align*}
       Repeating this process for
       $\psi(-\varphi)(L)$, where $-\varphi$ is the argument of $w$,
       we arrive at
       $$
       L = \mathbb{Z} v_1 \oplus \mathbb{Z} w_1 \simeq \mathbb{Z}^2.
       $$
    }
    \item{
      Suppose any lattice in $\mathbb{R}^{n-1}$ is isomorphic to
      $\mathbb{Z}^{n-1}$ for some $n \in \mathbb{Z}_+$. The same
      argument in (b) can be extended using the rotation matrices
      $\psi : \mathbb{R}^{n-1} \to \mathrm{Aut}(\mathbb{R}^n)$. Given
      a basis $\{ \beta_1, \dots, \beta_n \}$ of
    }
  \end{enumerate}
\end{Answer}

\end{document}
