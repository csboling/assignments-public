\documentclass{article}

\usepackage{amsmath}
\usepackage{amsfonts}
\usepackage{amssymb}
\usepackage{enumerate}
\usepackage[lastexercise]{exercise}

\newcounter{Problem}
\newenvironment{Problem}{\begin{Exercise}[name={Problem},
                                          counter={Problem}]}
                        {\end{Exercise}}
\title{MATH 818 Midterm}
\author{Sam Boling}

\begin{document}

\begin{Problem}
By definition a subgroup $H$ of a group $G$ is called
``characteristic'' if for every automorphism $\varphi : G \to G$ we
have $\varphi(H) = H$.
\begin{enumerate}[(a)]
  \item{Show that the center $Z(G)$ is a characteristic subgroup of
        $G$.
       }
  \item{Show that if $H$ is a characteristic subgroup of $G$, then $H$
        is normal in $G$.
       }
  \item{Suppose that $K$ is a normal subgroup of $G$ and that $H$ is a
        characteristic subgroup of $K$. Show that then $H$ is normal
        in $G$.
       }
\end{enumerate}
\end{Problem}

\begin{Answer}
\begin{enumerate}
  \item{
    Let $z \in Z(G)$ and $y \in G$, and let $\varphi \in
    \mathrm{Aut}(G)$.
    Then $y = \varphi(x)$ for some $x \in G$ since $\varphi$ is
    surjective, so since $zx = xz$ we have
    $\varphi(zx) = \varphi(xz)$ or
    $\varphi(z)\varphi(x) = \varphi(x)\varphi(z)$ and thus
    $\varphi(z)y = y \varphi(z)$ for all $y \in G$, so
    $\varphi(z) \in Z(G)$ and therefore
    $\varphi(Z(G)) \subset Z(G)$.

    Next let $z \in Z$ so that $z = \varphi(y)$ for some $y \in
    G$. Then $y = \varphi^{-1}(z)$. Let $h \in G$. Then
    $h = \varphi^{-1}(g)$ for some $g \in G$, so we have
    $$
    yh = \varphi^{-1}(z)\varphi^{-1}(g) = \varphi^{-1}(zg)
       = \varphi^{-1}(gz) = \varphi^{-1}(g)\varphi^{-1}(z) = hy,
    $$
    so $y \in Z(G)$. Therefore
    $z = \varphi(y) \in \varphi(Z(G))$ so
    $Z(G) \subset \varphi(Z(G))$. Thus $Z(G) = \varphi(Z(G))$ and so
    $Z(G)$ is characteristic.
  }
  \item{
    Let $h \in H$ where $H$ is a characteristic subgroup. Then for any
    $g \in G$ we have an automorphism $\varphi_g$ of conjugation by
    $g$. Since $H$ is characteristic,
    $\varphi_g(h) = ghg^{-1} \in H$ for every $g$, so $H$ is normal.
  }
  \item{
    Let $h \in H$. Then $\varphi(h) \in H$ for all
    $\varphi \in \mathrm{Aut}(K)$. But since $K$ is normal,
    $g k g^{-1} \in K$ for all $g \in G$, so conjugation by $g$
    restricted to $K$, i.e. $\varphi_g : K \to K$ is an automorphism
    for any $g \in G$. Then we have
    $H \ni \varphi_g(h) = g h g^{-1}$ so $H$ is normal in $G$.
  }
\end{enumerate}
\end{Answer}

\begin{Problem}
\begin{enumerate}[(a)]
  \item{State the Sylow theorems.}
  \item{Show that a group of order 132 cannot be simple.}
\end{enumerate}
\end{Problem}

\begin{Answer}
\begin{enumerate}[(a)]
  \item{
    \begin{enumerate}
      \item{A group $G$ of order $p \cdot n$ for $p$ prime has at
            least one $p$-Sylow subgroup.
           }
      \item{All $p$-Sylows are conjugate, and any normal subgroup is
            contained in some $p$-Sylow.
           }
      \item{The number of $p$-Sylows divides $|G|$ and is congruent to
            $1 mod p$.
           }
    \end{enumerate}
  }
  \item{
    $132 = 2 \cdot 66 = 2 \cdot 2 \cdot 3 \cdot 11$.

    $N_{11} \equiv 1 \mod 11$ so $N_{11} = 1$ or $N_{11} =
    12$. Suppose $N_{11} = 12$. Then these subgroups cover
    $12 \cdot (11 - 1) + 1 = 121$ elements, so 11 elements remain.

    $N_3 = 1$ or $N_3 = 4$. Suppose $N_3 = 4$. Then these cover
    $4 \cdot (3 - 1) = 8$ more elements so 3 remain.

    Therefore there must be exactly one 2-Sylow since it is of order
    $2 \cdot 2 = 4$, which is therefore normal and not trivial, so
    the group has a normal subgroup in any case.
  }
\end{enumerate}
\end{Answer}

\begin{Problem}
\begin{enumerate}[(a)]
  \item{Show that $S_n$ is generated by the transpositions
        $(12), (13), \dots, (1n)$.
       }
  \item{Show that for every $n \geq 3$, the center of $S_n$ is trivial.}
\end{enumerate}
\end{Problem}

\begin{Answer}
\begin{enumerate}
  \item{
    Observe that $(1i)(1j)(1i) = (ij)$ and thus that any transposition
    can be generated by transpositions of this form. Since $S_n$ is
    generated by transpositions, this concludes the proof.
  }
  \item{
    Let $\sigma \in Z(S_n)$. Then $\sigma \tau \sigma^{-1} = \tau$,
    $\forall \tau \in S_n$. Let
    $\tau = (a_1 \cdots a_k)$ be an $k$-cycle. Then
    $$
      \sigma \tau \sigma^{-1}
    = (\sigma(a_1) \cdots \sigma(a_k))
    = (a_1 \cdots a_k).
    $$
    In particular this means
    $(ij) = (\sigma(i) \sigma(j))$, so $\sigma$ permutes $i$ and $j$
    (if $\sigma$ is not the identity)
    for any $i$ and $j$. But for $n \geq 3$, we also have
    $(ijk) = (\sigma(i) \sigma(j) \sigma(k))$, and $(ijk)(j) = k \neq
    i$. Therefore it cannot be true that $\sigma \tau = \tau \sigma$
    for every $\tau \in S_n$ unless $\sigma = \mathrm{id}$.
  }
\end{enumerate}
\end{Answer}

\end{document}