\documentclass{article}

\title{ECE 832 - Homework}
\author{Sam Boling}
\date{\today}

\usepackage{enumitem}
\usepackage{amsmath}
\usepackage{amsfonts}
\usepackage{amssymb}
\usepackage{graphicx}

\usepackage{tikz}
\usepackage{circuitikz}
\usepackage{siunitx}

\renewcommand*{\Re}{\operatorname{\mathfrak{Re}}}
\renewcommand*{\Im}{\operatorname{\mathfrak{Im}}}

\newcommand{\horline}
           {\begin{center}
              \noindent\rule{8cm}{0.4pt}
            \end{center}}

\newcommand\scalemath[2]{\scalebox{#1}{\mbox{\ensuremath{\displaystyle #2}}}}

\begin{document}

\maketitle

\section*{Problem \#1}

Label the gate voltage of $Q_2$ as $V_l$ and of $Q_3$ as $V_r$, and the
voltage at the node fed by the current source as $V_b$. Observe that since
the gates of $Q_2$ and $Q_3$ are tied to their drains, $V_{gs} = V_{ds}$ for
these transistors, so the dependent current source in their small signal models
has a value of $g_m V_{gs} = g_m V_{ds}$. But a current source dependent on the
voltage applied across it is simply a resistor, so $Q_2$ and $Q_3$ can be 
represented as a resistor with value $\frac{1}{g_m}$ in parallel with a 
resistor with value $\frac{1}{g_d}$. $g_m \gg g_d$, so 
$\frac{1}{g_m} \ll \frac{1}{g_d}$, so the equivalent parallel resistance is
about $\frac{1}{g_m}$. Therefore the drain voltages for the left and right
sides are given by $\Delta V_l \approx \frac{\Delta I_l}{g_m}$ and
$\Delta V_r \approx \frac{\Delta I_r}{g_m}$.


\begin{enumerate}
  \item{First consider the situation where $V_{in}^{+}$ increases by a small amount
        $\Delta v$ and $V_{in}^{-}$ is fixed. 
    \begin{enumerate}
      \item{
        $Q_5$ acts as a 
        source-degenerated current sink, so its output impedance is about 
        $R_{out_l} = \frac{g_m}{g_d} R$, where $R = \frac{1}{g_m}$ so 
        $R_{out_l} = \frac{1}{g_d}$, and thus 
        $\Delta V_b = \frac{\Delta I_r}{g_d} \approx 0$ for small changes in 
        $\Delta I_r$. Writing the small signal model for $Q_5$ with 
        $\Delta V_b = \frac{\Delta I_r}{g_d}$, and observing that 
        $\Delta v_s$ for this transistor is the voltage drop across the
        equivalent source resistor $\frac{1}{g_m}$ gives
        \begin{align*}
          \Delta I_r &= g_m(0 - \frac{\Delta I_r}{g_m})
                     + g_d(\Delta V_b - \frac{\Delta I_r}{g_m}) \\
            &\approx -\Delta I_r + \Delta I_r - \frac{g_d}{g_m} \Delta I_r
        \end{align*}
        so $\Delta I_r \approx 0$.
      }
      \item{
        Next the small signal model for $Q_4$ gives
        $$
        \Delta I_l = -g_m \Delta v 
                     -\frac{g_d}{g_m} \Delta I_l 
                     +(g_m + g_d) \Delta V_b
             \approx -g_m \Delta v
        $$
        since $(g_m + g_d)\Delta V_b \approx 
               \frac{g_m}{g_d} \Delta I_r 
               \approx 0$
        and $\frac{g_d}{g_m} \Delta I_l$ is very small for small currents.
      }
      \item{
        Therefore the output current in this case is about $-g_m \Delta v$
        since $\Delta I_l$ and $\Delta I_r$ are reflected by the current 
        mirrors formed by the other transistors.
      }
    \end{enumerate}
   }
  \item{
    Consider the situation where $V_{in}^{+}$ is fixed and
    $V_{in}^{-}$ changes by a small amount $-\Delta v$. 
    \begin{enumerate}
      \item{
        Note that the output impedance of the right branch is 
        $r_{ds}(1 + \frac{g_m}{g_m}) = 2 r_{ds}$, so 
        $\Delta V_b = 2 r_{ds} \Delta I_r$. The small signal
        model for $Q_5$ then gives
        \begin{align*}
        \Delta I_r &= g_m (\Delta v - \frac{\Delta I_r}{g_m})
                    + g_d (\Delta V_b - \frac{\Delta I_r}{g_m}) \\
                   &= g_m \Delta v - \Delta I_r 
                    + 2 \Delta I_r - \frac{g_d}{g_m} \Delta I_r 
        \end{align*}
        so $\frac{g_d}{g_m} \Delta I_r = g_m \Delta v$, or
        $\Delta I_r = \frac{g_m^2}{g_d} \Delta v$, so
        $\Delta V_b = 2\frac{g_m^2}{g_d^2} \Delta v$.
      }
      \item{
        Applying the small signal model for $Q_4$ gives
        \begin{align*}
          \Delta I_l &= -g_m(0 - \Delta V_b)
                        -g_d(\frac{\Delta I_l}{g_m} - \Delta V_b) \\
                     &= 2\frac{g_m^3}{g_d^2} \Delta v
                      - \frac{g_d}{g_m} \Delta I_l - 2\frac{g_m^2}{g_d} \Delta v
        \end{align*}
      }
    \end{enumerate}
  }
\end{enumerate}

%The small signal model for $Q_5$ gives
%\begin{align*}
%\Delta I_r &\approx g_m (0 - \frac{\Delta I_r}{g_m}) 
%                  + g_d(\Delta V_b - \Delta V_r) \\
%           &=      - g_m \frac{\Delta I_r}{g_m} 
%                   + g_d \Delta V_b 
%                   - \frac{g_d}{g_m} \Delta I_r
%\end{align*}
%so since $\frac{g_d}{g_m}$ is very small, 
%$\Delta I_r \approx \frac{g_d}{2} \Delta V_b$ and thus
%$\Delta V_b \approx \frac{2}{g_d} \Delta I_r$. Next we see from the small
%signal model for $Q_4$ that
%\begin{align*}
%\Delta I_l &= -g_m (\Delta V - \Delta V_b)
%              -g_d (\Delta V_l - \Delta V_b) \\
%           &= -g_m \Delta V + \frac{2 g_m}{g_d} \Delta V_b 
%              -\frac{g_d}{g_m} \Delta I_l + g_d \Delta V_b
%\end{align*}
%and thus
%\begin{align*}
%\Delta I_l &\approx -g_m \Delta V + \left(\frac{2 g_m}{g_d} + g_d\right) \Delta V_b \\
%           &\approx -g_m \Delta V + \left(\frac{4 g_m}{g_d^2} + 1\right) \Delta I_r \\
%           &\approx -g_m \Delta V,
%\end{align*}
%
%
%
%First, consider the situation where $V_{in}^{+}$ is fixed and
%$V_{in}^{-}$ increases by $\Delta v$.
%
%\begin{align*}
%g_m (\Delta v - \Delta V_r) + g_d (\Delta V_b - \Delta V_r) &= g_m \Delta V_r \\
%g_m \Delta v + g_d \Delta V_b &= (2 g_m + g_d) \Delta V_r \\
%g_m \Delta v &\approx 2 g_m \Delta V_r \\
%\Delta V_r &\approx \frac{\Delta v}{2} \\
%\Delta I_r &\approx g_m \frac{\Delta v}{2}
%\end{align*}
%and
%\begin{align*}
%-g_m (0 - \Delta V_b) - g_d (\Delta V_l - \Delta V_b) &= g_m \Delta V_l \\
%g_m \Delta V_b + g_d \Delta V_b &= (g_m + g_d) \Delta V_l \\
%\Delta V_l &= \Delta V_b \\
%\Delta I_l &= g_m \Delta V_b
%\end{align*}
%but from above
%\begin{align*}
%\Delta V_b &= \frac{1}{g_d} [(2 g_m + g_d) \Delta V_r - g_m \Delta v] \\
%           &\approx \frac{1}{g_d} [2 g_m \frac{\Delta v}{2} - g_m \Delta v] \\
%           &= 0,
%\end{align*}
%so $\Delta I_l = 0$ and $\Delta I_r = g_m \frac{\Delta v_{in}^{-}}{2}$. For a
%differential input, let $\Delta v_{in}^{-} = -\Delta v$ so that
%$\Delta I_{out} = \frac{1}{2} g_m \Delta v$ in this case.
%
%Next, consider the situation where $V_{in}^{+}$ increases by $\Delta v$ 
%and $V_{in}^{-}$ remains fixed. 
%
%We see that
%\begin{align*}
%  -g_m (\Delta v - \Delta V_b) - g_d (\Delta V_l - \Delta V_b) &= g_m \Delta V_l \\
%  (g_m + g_d) \Delta V_b - g_m \Delta v &= (g_m + g_d) \Delta V_l \\
%  \Delta V_l &= \Delta V_b + \frac{g_m}{g_m + g_d} \Delta v \approx \Delta V_b + \Delta v \\
%  \Delta I_l &= g_m (\Delta V_b + \Delta v)
%\end{align*}
%and that
%\begin{align*}
%g_m (0 - \Delta V_r) + g_d (\Delta V_b - \Delta V_r) &= g_m \Delta V_r \\
%(s g_m + g_d) \Delta V_r &= g_d \Delta V_b \\
%\Delta V_r &= \frac{g_d}{2 g_m + g_d} \Delta V_b
%\end{align*}
%so
%$$
%\Delta I_r = g_m \Delta V_r 
%           \approx g_m \frac{g_d}{2 g_m} \Delta V_b 
%           = \frac{1}{2} g_d \Delta V_b.
%$$
%
%Next observe that
%\begin{enumerate}
%  \item{Since the drain and gate of $Q_3$ are tied together, the change in 
%        current through $Q_3$ is 
%        $(g_m + g_d) \Delta V_r \approx g_m \Delta V_r$.
%       }
%  \item{The change in current through $Q_5$ is
%        $$
%        g_m (0 - \Delta V_r) + g_d (\Delta V_b - \Delta V_r)
%      = -\Delta V_r (g_m + g_d) + g_d \Delta V_b \approx -g_m \Delta V_r
%        $$
%        as long as $\Delta V_b$ is small enough to keep $Q_5$ and $Q_6$ in
%        saturation.
%       }
%   \item{Then $\Delta I_r \approx g_m \Delta V_r \approx -g_m \Delta V_r$, so 
%         $\Delta I_r = \Delta V_r = 0$.
%        }
%\end{enumerate}
%Thus $\frac{g_d}{2} \Delta V_b \approx 0$ so $\Delta V_b \approx 0$, so
%the output current is simply $g_m \Delta v$ for this case.
%
%Superposition of these two results gives
%$\Delta I_{out} = \frac{3}{2} g_m \Delta v$,
%and the output impedance is the parallel impedance of the two cascoded
%current sources
%$$
%R_{out} = g_m r_{ds}^2 \| g_m r_{ds}^2 = \frac{1}{2} g_m r_{ds}^2,
%$$
%so the output voltage is
%$$
%\Delta V_{out} = \Delta I_{out} R_{out} = \frac{3}{4} g_m^2 r_{ds}^2 \Delta v
%$$
%and thus the gain is
%$$
%A_v = \frac{\Delta V_{out}}{\Delta V_{in}^{+} - \Delta V_{in}^{-}} 
%    = \frac{\Delta V_{out}}{2 \Delta v} 
%    = \frac{3}{8} g_{m}^2 r_{ds}^2.
%$$
%
%
%$$
%\begin{tabular}{l | l l l | l}
%    & \Delta v_{g} & \Delta v_{d} & \Delta v_{s} & \Delta{I} \\
%Q_2 & \Delta v_{l} & \Delta v_{l} & 0            & (g_m + g_d) \Delta V_1 \\
%\end{tabular}
%$$

\section*{Problem \#2}
\begin{enumerate}
\item{
Note that $Q_6$ is mirroring the current $I_b$ through $Q_7$ and $Q_1$ is
mirroring the same current through $Q_1$, so since current through $Q_5$ is
flowing towards the top of the circuit diagram, this means that the pin 
connected to the gate of $Q_5$ is actually $Q_5$'s drain, so $Q_5$ also forms a
current mirror with $Q_3$. Therefore we may redraw the diagram as follows:

\ctikzset{tripoles/mos style/arrows}
\begin{circuitikz}[american currents] \draw 
  % INSTANCES
  (0,0)     node[nmos] (q4) {}
  (q4.base) node[anchor=west] {$Q_4$}
  (q4.gate) node[anchor=east] {$V_{in}^{-}$}

  (q4.base) ++(3,0) node[nmos, xscale=-1] (q2) {}
  (q2.base) node[anchor=east] {$Q_2$}
  (q2.gate) node[anchor=west] {$V_{in}^{+}$}
  
  (1,-2) node[nmos, xscale=-1] (q1) {}
  (q1.base) node[anchor=east] {$Q_1$}

  (q4.drain) ++(0,1) node[pmos, xscale=-1] (q5) {}
  (q5.base) node[anchor=east] {$Q_5$}

  (1,4) node[pmos, xscale=-1] (q6) {}
  (q6.base) node[anchor=east] {$Q_6$}

  (q2.drain) ++(0,1) node[pmos] (q3) {}
  (q3.base) node[anchor=west] {$Q_3$}

  (q2.drain) -- ++(0,0) node[circ]{} -| ++(0.5,0) node[] (vout) {}
  (vout) node[anchor=west] {$V_{out}$}

  (6,-2)     node[nmos] (q8) {}
  (q8.base) node[anchor=west] {$Q_8$}

  (6,4) node[pmos] (q7) {}
  (q7.base) node[anchor=west] {$Q_7$}

  % CONNECTIONS
  (q4.source) -- (q2.source)
  (q4.source) ++(2,0) -| (q1.drain)
  (q4.source) to[short, -*] ++(1,0)
  (q4.drain)  -- (q5.drain)

  (q2.drain)  -- (q3.drain)

  (q5.gate)   -- (q3.gate)
  (q5.gate)   node[circ] {} |- (q5.drain) node[circ] {}
  (q5.source) -- (q3.source)
  (q5.source) ++(2,0) -| (q6.drain)
  (q5.source) to[short, -*] ++(1,0)

  (q2.drain)  -- (q3.drain)

  (q5.gate)   -- (q3.gate)
  (q5.gate)   node[circ] {} |- (q5.drain) node[circ] {}
  (q5.source) -- (q3.source)

  (q1.source) -- ++(0,0) node[sground] {}
  (q1.gate)   -- (q8.gate)

  (q8.source) -- ++(0,0) node[sground] {}
  (q8.gate)   node[circ] {} |- (q8.drain) node[circ] {}

  (q7.drain)  to[I, i=$I_b$] (q8.drain)
  (q7.source) -- ++(0,0) node[rground, yscale=-1] {}
  (q7.gate)   node[circ] {} |- (q7.drain) node[circ] {}

  (q6.source) -- ++(0,0) node[rground, yscale=-1] {}
  (q6.gate) -- (q7.gate)
;\end{circuitikz}
A fully differential input to this amplifier will cause no change in the 
current through $Q_1$, and thus no change in the voltage at the source of
the input stage transistors. Therefore this is a point of symmetry. As
$V_{in}^{+}$ increases by $\Delta v$ and $V_{in}^{-}$ decreases by 
$\Delta v$, the current through $Q_2$ increases by $g_m \Delta v$ and the
current through $Q_3$ (as reflected by the current mirror $Q_5$) decreases
by $g_m \Delta v$. Therefore the change in output current is 
$-g_m \Delta v - g_m \Delta v = -2 g_m \Delta v$. The output resistance is
the drain resistance $\frac{1}{g_d}$ of $Q_2$ in parallel with the drain
resistance $\frac{1}{g_d}$ of $Q_3$ and is thus about $\frac{1}{2g_d}$, so
the DC gain is
$$
A_v = \frac{-2 \frac{g_m}{g_d} \Delta v}{2 \Delta v} = -\frac{g_m}{g_d}.
$$ 
}
\item
{
}
\item
{
The current through $Q_6$ and $Q_7$ is $I_b$, so the power draw for the circuit
is $2 I_b V_{dd}$.
}
\end{enumerate}



\section*{Problem \#3}
\begin{enumerate}
\item{
If a fully differential input is applied, the currents through $Q_2$ and
$Q_3$ will both increase as the currents through $Q_4$ and $Q_5$ both 
decrease by an equal amount, so the drain of $Q_1$ is a point of symmetry.
Furthermore, the current through $Q_2$ will increase by the same amount 
that the current through $Q_4$ decreases, so there is no change in 
current through $Q_7$ or similarly through $Q_9$. Therefore the 
output current for the amplifier is $\Delta I_{out} = -g_m \Delta v$
so the DC gain is $-\frac{g_m}{4 g_d}$.
}
\item{
$Q_{11}$ and $Q_{12}$ form a current mirror since their source and gate are 
tied together, but the gate is floating so the mirror does not reliably 
operate the source follower $Q_6$.

}
\end{enumerate}

\section*{Problem \#4}


\section*{Problem \#5}



\end{document}
