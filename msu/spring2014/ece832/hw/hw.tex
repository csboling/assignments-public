\documentclass{article}

\title{ECE 832 - Homework}
\author{Sam Boling, PID A48788119}
\date{\today}

\usepackage{enumitem}
\usepackage{amsmath}
\usepackage{amsfonts}
\usepackage{amssymb}
\usepackage{graphicx}

\usepackage{tikz}
\usepackage{circuitikz}
\usepackage{siunitx}

\renewcommand*{\Re}{\operatorname{\mathfrak{Re}}}
\renewcommand*{\Im}{\operatorname{\mathfrak{Im}}}

\newcommand{\horline}
           {\begin{center}
              \noindent\rule{8cm}{0.4pt}
            \end{center}}

\newcommand\scalemath[2]{\scalebox{#1}{\mbox{\ensuremath{\displaystyle #2}}}}

\begin{document}

\maketitle

\section*{Problem \#1}

\begin{enumerate}
\item{ (Gain.)
Label the gate voltage of $Q_2$ as $V_l$ and of $Q_3$ as $V_r$, and the
voltage at the node fed by the current source as $V_b$. Observe that since
the gates of $Q_2$ and $Q_3$ are tied to their drains, $V_{gs} = V_{ds}$ for
these transistors, so the dependent current source in their small signal models
has a value of $g_m V_{gs} = g_m V_{ds}$. But a current source dependent on the
voltage applied across it is simply a resistor, so $Q_2$ and $Q_3$ can be 
represented as a resistor with value $\frac{1}{g_m}$ in parallel with a 
resistor with value $\frac{1}{g_d}$.

A heuristic solution for the gain is as follows. The left branch presents an
impedance of about $\frac{1}{g_m}$ to the node $V_{b}$ while the right branch
presents $\frac{1}{g_d}$, so the voltage at $V_{b}$ does not have much impact
on the current through $Q_5$. Since $Q_3$ acts as a resistor with a value of
about $\frac{1}{g_m}$, this means that in response to a change in $V_{in}^{-}$
of $-\Delta v$, the current through $Q_5$ is approximately given by
\begin{align*}
\Delta I_5 &= -g_m \Delta v - g_m \frac{\Delta I_5}{g_m} \\
\Delta I_5 &= -\frac{g_m}{2} \Delta v.
\end{align*}
Since $I_b$ is constant, this current and the current through the left branch
must sum to zero, so $\Delta I_4 = \frac{g_m}{2} \Delta v$. The voltage at
$V_b$ will rise to ensure that this happens and will have little impact on
the current through $Q_5$.

The output voltage then changes by 
$$
\Delta V_{out} = \left(\frac{g_m}{2} \Delta v 
                 - \left(-\frac{g_m}{2} \Delta v\right)\right)R_{out},
$$
where $R_{out}$ is the impedance of the cascode $Q_7$, $Q_8$ in parallel
with the impedance of the cascode $Q_9$, $Q_{10}$, that is
$$
R_{out} = g_m r_{ds}^2 \| g_m r_{ds}^2 
        = \frac{1}{2} \frac{g_m}{g_d^2},
$$
so
$$
A_v = \frac{\frac{1}{2}\left(\frac{g_m}{g_d}\right)^2 \Delta v}
           {\Delta v - (- \Delta v)} 
    = \frac{1}{4} \left(\frac{g_m}{g_d}\right)^2.
$$

A complete solution confirming this argument follows.

\begin{enumerate}
  \item{
       }
\end{enumerate}
}
\item{(Output swing.)
Since the current through the left and right branches of the input stage
is equal for perfectly matched transistors, the current through each of
these branches is $\frac{I_b}{2}$. Therefore the current through the output
stage is $\frac{I_b}{2}.$
Therefore the output is limited by
\begin{align*}
2V_{eff} &\leq \Delta V_{out} \leq V_{dd} - 2V_{eff} \\
2\frac{2 \left(\frac{I_b}{2}\right)}{g_m} 
  & \leq \Delta V_{out} 
    \leq V_{dd} - 2\frac{2 \left(\frac{I_b}{2}\right)}{g_m} \\
\frac{2 I_b}{g_m} &\leq \Delta V_{out} \leq V_{dd} - \frac{2 I_b}{g_m},
\end{align*}
noting that
$$
g_m \approx \frac{2 I_0}{V_{gs} - V_{th}}.
$$
}
\item{(Power.)
The current drawn by the middle $V_{dd}$ rail is $I_b$, and the current
drawn by each of the other rails is $\frac{I_b}{2}$, so the total power
dissipation is
$$
V_{dd}\left(I_b + 2\frac{I_b}{2}\right) = 2 I_b V_{dd}.
$$
}
\end{enumerate}

\pagebreak

\section*{Problem \#2}
\begin{enumerate}
\item{
Note that $Q_6$ is mirroring the current $I_b$ through $Q_7$ and $Q_1$ is
mirroring the same current through $Q_1$, so since current through $Q_5$ is
flowing towards the top of the circuit diagram, this means that the pin 
connected to the gate of $Q_5$ is actually $Q_5$'s drain, so $Q_5$ also forms a
current mirror with $Q_3$. Therefore we may redraw the diagram as follows:

\ctikzset{tripoles/mos style/arrows}
\begin{circuitikz}[american currents] \draw 
  % INSTANCES
  (0,0)     node[nmos] (q4) {}
  (q4.base) node[anchor=west] {$Q_4$}
  (q4.gate) node[anchor=east] {$V_{in}^{-}$}

  (q4.base) ++(3,0) node[nmos, xscale=-1] (q2) {}
  (q2.base) node[anchor=east] {$Q_2$}
  (q2.gate) node[anchor=west] {$V_{in}^{+}$}
  
  (1,-2) node[nmos, xscale=-1] (q1) {}
  (q1.base) node[anchor=east] {$Q_1$}

  (q4.drain) ++(0,1) node[pmos, xscale=-1] (q5) {}
  (q5.base) node[anchor=east] {$Q_5$}

  (1,4) node[pmos, xscale=-1] (q6) {}
  (q6.base) node[anchor=east] {$Q_6$}

  (q2.drain) ++(0,1) node[pmos] (q3) {}
  (q3.base) node[anchor=west] {$Q_3$}

  (q2.drain) -- ++(0,0) node[circ]{} -| ++(0.5,0) node[] (vout) {}
  (vout) node[anchor=west] {$V_{out}$}

  (6,-2)     node[nmos] (q8) {}
  (q8.base) node[anchor=west] {$Q_8$}

  (6,4) node[pmos] (q7) {}
  (q7.base) node[anchor=west] {$Q_7$}

  % CONNECTIONS
  (q4.source) -- (q2.source)
  (q4.source) ++(2,0) -| (q1.drain)
  (q4.source) to[short, -*] ++(1,0)
  (q4.drain)  -- (q5.drain)

  (q2.drain)  -- (q3.drain)

  (q5.gate)   -- (q3.gate)
  (q5.gate)   node[circ] {} |- (q5.drain) node[circ] {}
  (q5.source) -- (q3.source)
  (q5.source) ++(2,0) -| (q6.drain)
  (q5.source) to[short, -*] ++(1,0)

  (q2.drain)  -- (q3.drain)

  (q5.gate)   -- (q3.gate)
  (q5.gate)   node[circ] {} |- (q5.drain) node[circ] {}
  (q5.source) -- (q3.source)

  (q1.source) -- ++(0,0) node[sground] {}
  (q1.gate)   -- (q8.gate)

  (q8.source) -- ++(0,0) node[sground] {}
  (q8.gate)   node[circ] {} |- (q8.drain) node[circ] {}

  (q7.drain)  to[I, i=$I_b$] (q8.drain)
  (q7.source) -- ++(0,0) node[rground, yscale=-1] {}
  (q7.gate)   node[circ] {} |- (q7.drain) node[circ] {}

  (q6.source) -- ++(0,0) node[rground, yscale=-1] {}
  (q6.gate) -- (q7.gate)
;\end{circuitikz}

The current mirrors $Q_8$, $Q_1$ and $Q_7$, $Q_6$ provide non-ideal current
sources to the differential pair. Since the impedance of these current
sources is $\frac{1}{g_d}$ and since they are connected to the sources of
transistors which present an impedance of $\frac{1}{g_m}$, these current
sources are approximately ideal in this circuit. Therefore
a fully differential input to the amplifier will cause no change in the 
current through $Q_1$, and thus no change in the voltage at the source of
the input stage transistors. Therefore this is a point of symmetry. As
$V_{in}^{+}$ increases by $\Delta v$ and $V_{in}^{-}$ decreases by 
$\Delta v$, the current through $Q_2$ increases by $g_m \Delta v$ and the
current through $Q_3$ (as reflected by the current mirror $Q_5$) decreases
by $g_m \Delta v$. Therefore the change in output current is 
$-g_m \Delta v - g_m \Delta v = -2 g_m \Delta v$. The output resistance is
the drain resistance $\frac{1}{g_d}$ of $Q_2$ in parallel with the drain
resistance $\frac{1}{g_d}$ of $Q_3$ and is thus about $\frac{1}{2g_d}$, so
the DC gain is
$$
A_v = \frac{-2 \frac{g_m}{2 g_d} \Delta v}{2 \Delta v} = -\frac{g_m}{2 g_d}.
$$ 
}
\item
{
The output swing is 
\begin{align*}
V_{eff_2} + V_{eff_1} 
  &\leq \Delta V_{out} 
   \leq  V_{dd} - (V_{eff_3} + V_{eff_6}) \\
\frac{I_b}{g_m} + \frac{2 I_b}{g_m}
  &\leq \Delta V_{out}
   \leq  V_{dd} - \frac{I_b}{g_m} + \frac{2 I_b}{g_m} \\
\frac{3 I_b}{g_m} &\leq \Delta V_{out} \leq V_{dd} - \frac{3 I_b}{g_m}
\end{align*}
}
\item
{
The current through $Q_6$ and $Q_7$ is $I_b$, so the power draw for the circuit
is $2 I_b V_{dd}$.
}
\end{enumerate}



\section*{Problem \#3}
\begin{enumerate}
\item{(DC gain.)
If a fully differential input is applied, the currents through $Q_2$ and
$Q_3$ will both increase as the currents through $Q_4$ and $Q_5$ both 
decrease by an equal amount, so the drain of $Q_1$ is a point of symmetry.
Furthermore, the current through $Q_2$ will increase by the same amount 
that the current through $Q_4$ decreases, so there is no change in 
current through $Q_7$ or similarly through $Q_9$. Therefore the 
output current for the amplifier is $\Delta I_{out} = 0$
so the DC gain is $0$.
}
\item{(Output swing.)
No change in input will change the output of this circuit, so its output
swing is effectively 0.
}
\item{(Power consumption.)
$Q_{11}$ will draw $I_b$, while $Q_7$ and $Q_9$ each draw a current of 
$\frac{I_b}{2}$. $Q_8$ is shut off since its gate is floating and so it
draws no current. Therefore the power consumed is $2 V_{dd} I_b$.
}
\end{enumerate}

\section*{Problem \#4}
\begin{enumerate}
  \item{(Gain.)
    \begin{enumerate}
      \item{First, since the current mirror $M_{11}$ reflecting the reference
        current has an impedance of about $\frac{1}{g_d}$ and the four branches
        attached to it each present an impedance of $\frac{1}{g_m}$, the current
        source is approximately ideal and therefore the sources of the input stage
        transistors see a virtual ground.
      } 
      \item{Examining the middle two branches and writing a small signal model for the 
        transistors closest to $V_{dd}$ (while assuming that $g_d$ can be ignored
        compared to $g_m$) gives
        $$
        g_m \Delta v - g_m \Delta v_l = -g_m \Delta v 
          \Rightarrow \Delta v_l = 2 \Delta v 
          \Rightarrow \Delta I_l = -g_m \Delta v
        $$
        and
        $$
        -g_m \Delta v - g_m \Delta v_r = -g_m \Delta v 
          \Rightarrow \Delta v_r = -2 \Delta v 
          \Rightarrow \Delta I_r = g_m \Delta v,
        $$
        so these two branches cancel each other for a fully differential input
        and can be ignored.
      }
      \item{The remaining circuit is familiar: an increase by $\Delta v$ at 
            $V^{+}$ results in a current change of $g_m \Delta v$ that is 
            reflected to the output, and a decrease by $\Delta v$ at $V^{-}$
            results in a current change of $-g_m \Delta v$ that is reflected
            to the other side of the output. The output impedance is 
            $r_{ds} \| r_{ds}$ due to $M_7$ and $M_{10}$, so the DC gain is
            $$A_v = \frac{-2 g_m \Delta v}{2 g_d} \frac{1}{2 \Delta v} = -\frac{g_m}{2 g_d}.$$
           }
  \end{enumerate}
  }
  \item{(Output swing.)
        The output is limited by
        \begin{align*}
        V_{eff} &\leq \Delta V_{out} \leq V_{dd} - V_{eff} \\
        \frac{2 \left(\frac{I_b}{4}\right)}{g_m} 
        & \leq \Delta V_{out} 
          \leq V_{dd} - \frac{ \left(\frac{I_b}{4}\right)}{g_m} \\
        \frac{I_b}{2g_m} &\leq \Delta V_{out} \leq V_{dd} - \frac{I_b}{2g_m},
        \end{align*}
        since each of the middle branches draws an equal large-signal current.
       }
  \item{(Power consumption.)
        The branch connected to the reference current consumes a power of 
        $V_{dd} I_b$. Each of the six other branches requires 
        $V_{dd} \frac{I_b}{4}$, so the total power consumption is 
        $\frac{5}{2} V_{dd} I_b.$
       }
  \item{(Pole locations.)
       The current mirrors $M_5$, $M_8$ and $M_6$, $M_7$ each introduce a pole
       at about
       $$
       f_{p1} \approx f_{p2} \approx 
       \frac{1}{2 \pi} \frac{1}{\frac{1}{g_m} 2 C_{gs}} = \frac{g_m}{4 \pi C_{gs}}
       $$
       and the mirror $M_9$, $M_{10}$ produces another at about
       $$
       f_{p3} \approx \frac{1}{2 \pi}\frac{g_m}{2C_{gs}} = \frac{g_m}{4 \pi C_{gs}}.
       $$
       There is another pole at the output produced by the output resistance
       $r_{ds} \| r_{ds}$ and the output capacitance, so
       $$
       f_{p4} \approx \frac{1}{2\pi}\frac{g_d}{2C_{gd} + C_L},
       $$
       where $C_L$ is the load capacitance, if connected.
       The branches in the middle of the circuit are not in the signal path
       and thus do not contribute poles.
       }
  \item{(Zero locations.)
        First, find $I_1(s)$. The resistance at the gate of the current mirror
        $M_6$, $M_7$ is $\frac{1}{g_m}$ and the capacitance is $2 C_{gs}$, so 
        the voltage at this node is 
        $$
        V_1(s) = \frac{-R I_r(s)}{1 + RCs} 
               = \frac{\frac{1}{g_m} (g_m V(s))}{1 + \frac{2 C_{gs}}{g_m} s}
               = \frac{V(s)}{1 + \frac{2 C_{gs}}{g_m} s}
        $$
        so the current $I_1(s)$ is about
        $$
        I_1(s) = -g_m V_1(s) = \frac{-g_m V(s)}{1 + \frac{2 C_{gs}}{g_m} s}.
        $$
        This is also the current through $M_9$ but with opposite sign 
        (due to the equal and opposite input at $V^{+}$), which 
        charges the same resistance and capacitance as above, resulting in a
        voltage at the gate of
        $$
        V_2(s) = \frac{\frac{1}{g_m} (-I_1(s))}{1 + \frac{2 C_{gs}}{g_m} s}
        $$ 
        and thus a current through $M_{10}$ of
        $$
        I_2(s) = g_m V_2(s) = -\frac{I_1(s)}{1 + \frac{2 C_{gs}}{g_m} s}.
        $$
        These currents cancel each other when $I_1(s) = I_2(s)$, or when
        \begin{align*}
          1 + \frac{2 C_{gs}}{g_m} s &= -1 \\
          s &= -\frac{g_m}{C_{gs}},
        \end{align*}
        so there is a real zero in the left half-plane.
        The branches in the middle of the circuit are not in the signal path
        and thus do not contribute zeros.
       }
  \item{(Transfer function.)
       Using the currents derived in the previous part, we see that
       \begin{align*}
       V_{out}(s) &= \frac{(r_{ds} \| r_{ds})(I_1(s) - I_2(s))}
                          {1 + (r_{ds} \| r_{ds})(2C_{gd} + C_L) s)} \\
                  &= -\frac{\frac{1}{2g_d}
                      \left(1 + \frac{1}{1 + \frac{2C_{gs}}{g_m}s}\right)
                      \frac{g_m}{1 + \frac{2 C_{gs}}{g_m}s} V(s)}
                          {1 + \frac{2C_{gd} + C_L}{2g_d}s} \\
                  &= -\frac{g_m}{2g_d}
                      \frac{(2 + \frac{2C_{gs}}{g_m}s)V(s)}
                           {(1 + \frac{2C_{gd} + C_L}{2g_d}s)(1 + \frac{2C_{gs}}{g_m}s)^2} \\
                  &= -\frac{g_m}{g_d}
                      \frac{C_{gs}}{g_m}
                      \frac{2g_d}{2C_{gd} + C_L}
                      \left(\frac{2C_{gs}}{g_m}\right)^2
                     \frac{s + \frac{g_m}{C_{gs}}}
                          {(s + \frac{2g_d}{2C_{gd} + C_L})
                           (s + \frac{g_m}{2C_{gs}})^2
                          }
                     V(s) 
       \end{align*}
       }
\end{enumerate}


\section*{Problem \#5}
\begin{enumerate}
  \item{
    For the first circuit (with no cascode):
    \begin{enumerate}
      \item{
        The output range is bounded by
        \begin{align*}
        V_{eff} &\leq \Delta V_{out} 
                \leq V_{dd} - V_{eff} 
        \end{align*}
        so the size of the output range is
        $$
        V_{dd} - V_{eff_{p}} - V_{eff_{n}}.
        $$
        For the output to avoid saturation we thus require
        $$
        2|\Delta V_{out}| \leq 
        V_{dd} - V_{eff_p} - V_{eff_n}.
        $$

        The output impedance is $\frac{1}{g_{dn}} \| \frac{1}{g_{dp}}$ so the gain is 
        $-\frac{g_{mn}}{g_{dn} + g_{dp}}$, so the above condition gives
        $$
        |\Delta V_{in}| \leq \frac{g_{dn} + g_{dp}}{2g_{mn}} (
        V_{dd} - V_{eff_p} - V_{eff_n}),
        $$
        and since $V_{gs} > V_{th}$ is required for the input transistor
        to be in saturation, this gives the input bias point as
        \begin{align*}
        V_{bias} = V_{th_{n}} + |\Delta V_{in}|.
        \end{align*}

        For the ami06 MOS models, we have 
        \begin{align*}
        C_{ox} &= \frac{3.9 \dot 8.85 \times 10^{-14} ~\mathrm{F}~\mathrm{cm}^{-1}}
                      {1.41 \times 10^{-6} ~\mathrm{cm}} \\
               &= 2.45 \times 10^{-7} ~\mathrm{F}~\mathrm{cm}^{-2}, \\
        \mu_p  &= 202.5 ~\mathrm{cm}^2~\mathrm{V}^{-1}~\mathrm{s}^{-1}, \\
        \mu_n  &= 533.7 ~\mathrm{cm}^2~\mathrm{V}^{-1}~\mathrm{s}^{-1}.
        \end{align*}
        Operating points were determined from DC analysis simulations to find the
        input/output crossover point for each bias current, and these are recorded 
        in the "Input bias point" entries in the following table.
        Approximate $g_{d}$ values for each current are given as measured 
        from the saturation-region slope at the given bias currents, and these give
        reasonable predictions of the input swing around the simulated bias points.
        Parameters through Input Swing are estimated from the model equations, 
        though the discrepancies in gain from simulation show that these estimates
        carry considerable error.

        \begin{tabular}{c | c c c c | c}
$I_b$                    & 0.5 $\mu$A            & 1 $\mu$A              & 2 $\mu$A              & 4 $\mu$A              & 8 $\mu$A &  units \\
\hline                   & & & & &\\
$g_{mp}$                 & $1.22 \times 10^{-5}$ & $1.73 \times 10^{-5}$ & $2.44 \times 10^{-5}$ & $3.46 \times 10^{-5}$ &          & $\mathrm{A}~\mathrm{V}^{-1}$\\
$g_{mn}$                 & $1.98 \times 10^{-5}$ & $2.80 \times 10^{-5}$ & $3.97 \times 10^{-5}$ & $5.61 \times 10^{-5}$ &          & $\mathrm{A}~\mathrm{V}^{-1}$\\
$g_{dp}$                 & $3    \times 10^{-9}$ & $6    \times 10^{-9}$ & $1    \times 10^{-8}$ & $2.25 \times 10^{-8}$ &          & $\mathrm{A}~\mathrm{V}^{-1}$\\
$g_{dn}$                 & $1.5  \times 10^{-8}$ & $3    \times 10^{-8}$ & $5.6  \times 10^{-8}$ & $9.5  \times 10^{-8}$ &          & $\mathrm{A}~\mathrm{V}^{-1}$\\
Estimated DC gain        & $-1100$               & $-778$                & $-602$                & $-447$                &          & $\mathrm{V}$/$\mathrm{V}$   \\
$V_{eff_p}$              & $0.08$                & $0.11$                & $0.16$                & $0.23$                &          & $\mathrm{V}$ \\
$V_{eff_n}$              & $0.05$                & $0.07$                & $0.1$                 & $0.14$                &          & $\mathrm{V}$ \\
Output swing             & 2.87                  & 2.82                  & 2.74                  & 2.63                  &          & $\mathrm{V}_{pp}$ \\
Input swing              & 2.61                  & 3.62                  & 4.55                  & 5.88                  &          & $\mathrm{mV}_{pp}$ \\
\hline \\
Input bias point         & 706.78                & 765.2                 & 823.7                 & 882.4                 & 941.75   & $\mathrm{mV}$ \\
Measured DC gain         & 52.78                 & 6.75                  & 2                     & 2.98                  & 11.68    & dB \\
Bandwidth                & $2.55 \times 10^{-3}$ & $7.92 \times 10^{-3}$ & 2.397                 & 2.935                 & 1.65     & MHz (3 dB down) 
        \end{tabular}
      }
    \end{enumerate}
       }
  \item{ This circuit has a gain of about
         $$
         -g_{mn} \left[
          \left(\frac{g_{mn}}{g_{dn}^2}\right) \left\|
          \left(\frac{g_{mp}}{g_{dp}^2}\right)\right.\right]
         $$
         and an output swing of
         $$
         2V_{eff_n} \leq \Delta V_{out} \leq V_{dd} - 2V_{eff_p},
         $$
         so the following parameters change:

        \begin{tabular}{c | c c c c | c}
$I_b$                    & 0.5 $\mu$A            & 1 $\mu$A              & 2 $\mu$A              & 4 $\mu$A              & 8 $\mu$A & units \\
\hline                   & & & & &\\
DC gain                  & $-1.6 \times 10^{6}$  & $-8.2 \times 10^{5}$  & $-4.78 \times 10^{5}$  & $-3.2 \times 10^{5}$ & $\mathrm{V}$/$\mathrm{V}$ \\
Output swing             & 2.74                  & 2.64                  & 2.48                  & 2.36                  & $\mathrm{V}_{pp}$         \\
Input swing              & 1.71                  & 3.22                  & 5.19                  & 7.38                  & $\mathrm{\mu V}_{pp}$     \\
\hline \\
Input bias point         & 765.14                   & 0.7                   & 0.7                   & 0.7                   & $\mathrm{mV}$ \\
$V_b$                    & 1.53
Simulated DC gain        & &
Bandwidth                & \\
        \end{tabular}
        For each simulation $V_b$ was chosen as twice the operating point of 
        the input transistor to ensure that it remains in saturation.
       }
\end{enumerate}



\end{document}
