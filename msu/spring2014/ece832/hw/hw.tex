\documentclass{article}

\title{ECE 832 - Homework}
\author{Sam Boling}
\date{\today}

\usepackage{enumitem}
\usepackage{amsmath}
\usepackage{amsfonts}
\usepackage{amssymb}
\usepackage{graphicx}

\usepackage{tikz}
\usepackage{circuitikz}
\usepackage{siunitx}

\renewcommand*{\Re}{\operatorname{\mathfrak{Re}}}
\renewcommand*{\Im}{\operatorname{\mathfrak{Im}}}

\newcommand{\horline}
           {\begin{center}
              \noindent\rule{8cm}{0.4pt}
            \end{center}}

\newcommand\scalemath[2]{\scalebox{#1}{\mbox{\ensuremath{\displaystyle #2}}}}

\begin{document}

\maketitle

\section*{Problem \#1}

Label the gate voltage of $Q_2$ as $V_l$ and of $Q_3$ as $V_r$, and the
voltage at the node fed by the current source as $V_b$. Observe that since
the gates of $Q_2$ and $Q_3$ are tied to their drains, $V_{gs} = V_{ds}$ for
these transistors, so the dependent current source in their small signal models
has a value of $g_m V_{gs} = g_m V_{ds}$. But a current source dependent on the
voltage applied across it is simply a resistor, so $Q_2$ and $Q_3$ can be 
represented as a resistor with value $\frac{1}{g_m}$ in parallel with a 
resistor with value $\frac{1}{g_d}$.

\section*{Problem \#2}
\begin{enumerate}
\item{
Note that $Q_6$ is mirroring the current $I_b$ through $Q_7$ and $Q_1$ is
mirroring the same current through $Q_1$, so since current through $Q_5$ is
flowing towards the top of the circuit diagram, this means that the pin 
connected to the gate of $Q_5$ is actually $Q_5$'s drain, so $Q_5$ also forms a
current mirror with $Q_3$. Therefore we may redraw the diagram as follows:

\ctikzset{tripoles/mos style/arrows}
\begin{circuitikz}[american currents] \draw 
  % INSTANCES
  (0,0)     node[nmos] (q4) {}
  (q4.base) node[anchor=west] {$Q_4$}
  (q4.gate) node[anchor=east] {$V_{in}^{-}$}

  (q4.base) ++(3,0) node[nmos, xscale=-1] (q2) {}
  (q2.base) node[anchor=east] {$Q_2$}
  (q2.gate) node[anchor=west] {$V_{in}^{+}$}
  
  (1,-2) node[nmos, xscale=-1] (q1) {}
  (q1.base) node[anchor=east] {$Q_1$}

  (q4.drain) ++(0,1) node[pmos, xscale=-1] (q5) {}
  (q5.base) node[anchor=east] {$Q_5$}

  (1,4) node[pmos, xscale=-1] (q6) {}
  (q6.base) node[anchor=east] {$Q_6$}

  (q2.drain) ++(0,1) node[pmos] (q3) {}
  (q3.base) node[anchor=west] {$Q_3$}

  (q2.drain) -- ++(0,0) node[circ]{} -| ++(0.5,0) node[] (vout) {}
  (vout) node[anchor=west] {$V_{out}$}

  (6,-2)     node[nmos] (q8) {}
  (q8.base) node[anchor=west] {$Q_8$}

  (6,4) node[pmos] (q7) {}
  (q7.base) node[anchor=west] {$Q_7$}

  % CONNECTIONS
  (q4.source) -- (q2.source)
  (q4.source) ++(2,0) -| (q1.drain)
  (q4.source) to[short, -*] ++(1,0)
  (q4.drain)  -- (q5.drain)

  (q2.drain)  -- (q3.drain)

  (q5.gate)   -- (q3.gate)
  (q5.gate)   node[circ] {} |- (q5.drain) node[circ] {}
  (q5.source) -- (q3.source)
  (q5.source) ++(2,0) -| (q6.drain)
  (q5.source) to[short, -*] ++(1,0)

  (q2.drain)  -- (q3.drain)

  (q5.gate)   -- (q3.gate)
  (q5.gate)   node[circ] {} |- (q5.drain) node[circ] {}
  (q5.source) -- (q3.source)

  (q1.source) -- ++(0,0) node[sground] {}
  (q1.gate)   -- (q8.gate)

  (q8.source) -- ++(0,0) node[sground] {}
  (q8.gate)   node[circ] {} |- (q8.drain) node[circ] {}

  (q7.drain)  to[I, i=$I_b$] (q8.drain)
  (q7.source) -- ++(0,0) node[rground, yscale=-1] {}
  (q7.gate)   node[circ] {} |- (q7.drain) node[circ] {}

  (q6.source) -- ++(0,0) node[rground, yscale=-1] {}
  (q6.gate) -- (q7.gate)
;\end{circuitikz}

A fully differential input to this amplifier will cause no change in the 
current through $Q_1$, and thus no change in the voltage at the source of
the input stage transistors. Therefore this is a point of symmetry. As
$V_{in}^{+}$ increases by $\Delta v$ and $V_{in}^{-}$ decreases by 
$\Delta v$, the current through $Q_2$ increases by $g_m \Delta v$ and the
current through $Q_3$ (as reflected by the current mirror $Q_5$) decreases
by $g_m \Delta v$. Therefore the change in output current is 
$-g_m \Delta v - g_m \Delta v = -2 g_m \Delta v$. The output resistance is
the drain resistance $\frac{1}{g_d}$ of $Q_2$ in parallel with the drain
resistance $\frac{1}{g_d}$ of $Q_3$ and is thus about $\frac{1}{2g_d}$, so
the DC gain is
$$
A_v = \frac{-2 \frac{g_m}{g_d} \Delta v}{2 \Delta v} = -\frac{g_m}{g_d}.
$$ 
}
\item
{
}
\item
{
The current through $Q_6$ and $Q_7$ is $I_b$, so the power draw for the circuit
is $2 I_b V_{dd}$.
}
\end{enumerate}



\section*{Problem \#3}
\begin{enumerate}
\item{
If a fully differential input is applied, the currents through $Q_2$ and
$Q_3$ will both increase as the currents through $Q_4$ and $Q_5$ both 
decrease by an equal amount, so the drain of $Q_1$ is a point of symmetry.
Furthermore, the current through $Q_2$ will increase by the same amount 
that the current through $Q_4$ decreases, so there is no change in 
current through $Q_7$ or similarly through $Q_9$. Therefore the 
output current for the amplifier is $\Delta I_{out} = -g_m \Delta v$
so the DC gain is $-\frac{g_m}{4 g_d}$.
}
\item{
$Q_{11}$ and $Q_{12}$ form a current mirror since their source and gate are 
tied together, but the gate is floating so the mirror does not reliably 
operate the source follower $Q_6$.

}
\end{enumerate}

\section*{Problem \#4}


\section*{Problem \#5}



\end{document}
