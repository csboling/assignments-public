\documentclass{article}

\title{ECE 832 - Homework}
\author{Sam Boling, PID A48788119}
\date{\today}

\usepackage{enumitem}
\usepackage{amsmath}
\usepackage{amsfonts}
\usepackage{amssymb}

\usepackage{graphicx}
\usepackage{caption}
\usepackage{subcaption}
\usepackage{rotating}

\usepackage{tikz}
\usepackage{circuitikz}
\usepackage{siunitx}

\renewcommand*{\Re}{\operatorname{\mathfrak{Re}}}
\renewcommand*{\Im}{\operatorname{\mathfrak{Im}}}

\newcommand{\horline}
           {\begin{center}
              \noindent\rule{8cm}{0.4pt}
            \end{center}}

\newcommand\scalemath[2]{\scalebox{#1}{\mbox{\ensuremath{\displaystyle #2}}}}

\begin{document}

\maketitle

\section*{Problem \#1}

\begin{enumerate}
\item{ (Gain.)
Label the gate voltage of $Q_2$ as $V_l$ and of $Q_3$ as $V_r$, and the
voltage at the node fed by the current source as $V_b$. Observe that since
the gates of $Q_2$ and $Q_3$ are tied to their drains, $V_{gs} = V_{ds}$ for
these transistors, so the dependent current source in their small signal models
has a value of $g_m V_{gs} = g_m V_{ds}$. But a current source dependent on the
voltage applied across it is simply a resistor, so $Q_2$ and $Q_3$ can be 
represented as a resistor with value $\frac{1}{g_m}$ in parallel with a 
resistor with value $\frac{1}{g_d}$.

A heuristic solution for the gain is as follows. The left branch presents an
impedance of about $\frac{1}{g_m}$ to the node $V_{b}$ while the right branch
presents $\frac{1}{g_d}$, so the voltage at $V_{b}$ does not have much impact
on the current through $Q_5$. Since $Q_3$ acts as a resistor with a value of
about $\frac{1}{g_m}$, this means that in response to a change in $V_{in}^{-}$
of $-\Delta v$, the current through $Q_5$ is approximately given by
\begin{align*}
\Delta I_5 &= -g_m \Delta v - g_m \frac{\Delta I_5}{g_m} \\
\Delta I_5 &= -\frac{g_m}{2} \Delta v.
\end{align*}
Since $I_b$ is constant, this current and the current through the left branch
must sum to zero, so $\Delta I_4 = \frac{g_m}{2} \Delta v$. The voltage at
$V_b$ will rise to ensure that this happens and will have little impact on
the current through $Q_5$.

The output voltage then changes by 
$$
\Delta V_{out} = \left(\frac{g_m}{2} \Delta v 
                 - \left(-\frac{g_m}{2} \Delta v\right)\right)R_{out},
$$
where $R_{out}$ is the impedance of the cascode $Q_7$, $Q_8$ in parallel
with the impedance of the cascode $Q_9$, $Q_{10}$, that is
$$
R_{out} = g_m r_{ds}^2 \| g_m r_{ds}^2 
        = \frac{1}{2} \frac{g_m}{g_d^2},
$$
so
$$
A_v = \frac{\frac{1}{2}\left(\frac{g_m}{g_d}\right)^2 \Delta v}
           {\Delta v - (- \Delta v)} 
    = \frac{1}{4} \left(\frac{g_m}{g_d}\right)^2.
$$

A complete solution confirming this argument follows.

\begin{enumerate}
  \item{First, consider applying a change $\Delta v$ to $V_{in}^+$ while 
        holding $V_{in}^-$ constant. The impedance of $Q_3$ is
        $\frac{1}{g_m + g_d}$, so applying the small signal model at
        $Q_5$ gives
        $$
        \Delta I_r = g_d \Delta V_b - (g_m + g_d) \Delta V_r 
                   = g_d \Delta V_b - \Delta I_r,
        $$
        and thus the impedance looking into the right branch of the input
        stage is $\frac{2}{g_d}$. Then since $\Delta I_l + \Delta I_r = 0$,
        $$
        \Delta I_l + \frac{\Delta V_b}{\frac{2}{g_d}} = 0
        \Delta V_b = -\frac{2 \Delta I_l}{g_d},
        $$
        so
        \begin{align*}
        \Delta I_l &= -g_m (\Delta v + \frac{2 \Delta I_l}{g_d}) 
                      -g_d (\frac{\Delta I_l}{g_m + g_d} 
                         +  \frac{2 \Delta I_l}{g_d}) \\
                   &= -g_m \Delta v - \frac{2 g_m}{g_d} \Delta I_l
                      -\frac{g_d}{g_m + g_d} \Delta I_l - 2 \Delta I_l,
        \end{align*}
        so
        $$
        \Delta I_l (3 + \frac{2 g_m}{g_d} + \frac{g_d}{g_m + g_d}) 
         = -g_m \Delta v,
        $$
        so since $g_m \gg g_d$ this means
        $$
        \Delta I_l \approx -\frac{g_d}{2} \Delta v
        $$
        and thus
        $$
        \Delta I_r \approx \frac{g_d}{2} \Delta v
        $$
        in this case.
       }
       \item{
         Next consider holding $V_{in}^+$ constant and
         applying a change $\Delta v$ at $V_{in}^-$. We see that
         $$
         \Delta I_l = g_m \Delta V_b - \frac{g_d}{g_m + g_d} \Delta I_l + g_d \Delta V_b,
         $$
         so $\Delta I_l \approx g_m \Delta V_b$. But
         $$
         -2 g_m \Delta V_b \approx -g_m \Delta v + g_d \Delta V_b,
         -2 g_m \Delta V_b \approx -g_m \Delta v,
         \Delta V_b \approx \frac{\Delta v}{2},
         $$
         so
         $$
         \Delta I_l \approx \frac{g_m}{2} \Delta v, \Delta I_r \approx -\frac{g_m}{2} \Delta v
         $$
         in this case.
       }
       \item{
         Superposition of these two results gives
         $$
         \Delta I_{out} \approx (g_m - g_d) \Delta v \approx g_m \Delta v,
         $$
         and from the cascoded impedances in the output stage
         $$
         R_{out} = g_m r_{ds}^2 \| g_m r_{ds}^2 = \frac{g_m}{2 g_d},
         $$
         so
         $$
         A_v \approx \frac{g_m^2}{4g_d^2},
         $$
         confirming the heuristic analysis.
       }
\end{enumerate}
}
\item{(Output swing.)
Since the current through the left and right branches of the input stage
is equal for perfectly matched transistors, the current through each of
these branches is $\frac{I_b}{2}$. Therefore the current through the output
stage is $\frac{I_b}{2}.$
Therefore the output is limited by
\begin{align*}
2V_{eff} &\leq \Delta V_{out} \leq V_{dd} - 2V_{eff} \\
2\frac{2 \left(\frac{I_b}{2}\right)}{g_m} 
  & \leq \Delta V_{out} 
    \leq V_{dd} - 2\frac{2 \left(\frac{I_b}{2}\right)}{g_m} \\
\frac{2 I_b}{g_m} &\leq \Delta V_{out} \leq V_{dd} - \frac{2 I_b}{g_m},
\end{align*}
noting that
$$
g_m \approx \frac{2 I_0}{V_{gs} - V_{th}}.
$$
}
\item{(Power.)
The current drawn by the middle $V_{dd}$ rail is $I_b$, and the current
drawn by each of the other rails is $\frac{I_b}{2}$, so the total power
dissipation is
$$
V_{dd}\left(I_b + 2\frac{I_b}{2}\right) = 2 I_b V_{dd}.
$$
}
\end{enumerate}

\pagebreak

\section*{Problem \#2}
\begin{enumerate}
\item{
Note that $Q_6$ is mirroring the current $I_b$ through $Q_7$ and $Q_1$ is
mirroring the same current through $Q_1$, so since current through $Q_5$ is
flowing towards the top of the circuit diagram, this means that the pin 
connected to the gate of $Q_5$ is actually $Q_5$'s drain, so $Q_5$ also forms a
current mirror with $Q_3$. Therefore we may redraw the diagram as follows:

\ctikzset{tripoles/mos style/arrows}
\begin{circuitikz}[american currents] \draw 
  % INSTANCES
  (0,0)     node[nmos] (q4) {}
  (q4.base) node[anchor=west] {$Q_4$}
  (q4.gate) node[anchor=east] {$V_{in}^{-}$}

  (q4.base) ++(3,0) node[nmos, xscale=-1] (q2) {}
  (q2.base) node[anchor=east] {$Q_2$}
  (q2.gate) node[anchor=west] {$V_{in}^{+}$}
  
  (1,-2) node[nmos, xscale=-1] (q1) {}
  (q1.base) node[anchor=east] {$Q_1$}

  (q4.drain) ++(0,1) node[pmos, xscale=-1] (q5) {}
  (q5.base) node[anchor=east] {$Q_5$}

  (1,4) node[pmos, xscale=-1] (q6) {}
  (q6.base) node[anchor=east] {$Q_6$}

  (q2.drain) ++(0,1) node[pmos] (q3) {}
  (q3.base) node[anchor=west] {$Q_3$}

  (q2.drain) -- ++(0,0) node[circ]{} -| ++(0.5,0) node[] (vout) {}
  (vout) node[anchor=west] {$V_{out}$}

  (6,-2)     node[nmos] (q8) {}
  (q8.base) node[anchor=west] {$Q_8$}

  (6,4) node[pmos] (q7) {}
  (q7.base) node[anchor=west] {$Q_7$}

  % CONNECTIONS
  (q4.source) -- (q2.source)
  (q4.source) ++(2,0) -| (q1.drain)
  (q4.source) to[short, -*] ++(1,0)
  (q4.drain)  -- (q5.drain)

  (q2.drain)  -- (q3.drain)

  (q5.gate)   -- (q3.gate)
  (q5.gate)   node[circ] {} |- (q5.drain) node[circ] {}
  (q5.source) -- (q3.source)
  (q5.source) ++(2,0) -| (q6.drain)
  (q5.source) to[short, -*] ++(1,0)

  (q2.drain)  -- (q3.drain)

  (q5.gate)   -- (q3.gate)
  (q5.gate)   node[circ] {} |- (q5.drain) node[circ] {}
  (q5.source) -- (q3.source)

  (q1.source) -- ++(0,0) node[sground] {}
  (q1.gate)   -- (q8.gate)

  (q8.source) -- ++(0,0) node[sground] {}
  (q8.gate)   node[circ] {} |- (q8.drain) node[circ] {}

  (q7.drain)  to[I, i=$I_b$] (q8.drain)
  (q7.source) -- ++(0,0) node[rground, yscale=-1] {}
  (q7.gate)   node[circ] {} |- (q7.drain) node[circ] {}

  (q6.source) -- ++(0,0) node[rground, yscale=-1] {}
  (q6.gate) -- (q7.gate)
;\end{circuitikz}

The current mirrors $Q_8$, $Q_1$ and $Q_7$, $Q_6$ provide non-ideal current
sources to the differential pair. Since the impedance of these current
sources is $\frac{1}{g_d}$ and since they are connected to the sources of
transistors which present an impedance of $\frac{1}{g_m}$, these current
sources are approximately ideal in this circuit. Therefore
a fully differential input to the amplifier will cause no change in the 
current through $Q_1$, and thus no change in the voltage at the source of
the input stage transistors. Therefore this is a point of symmetry. As
$V_{in}^{+}$ increases by $\Delta v$ and $V_{in}^{-}$ decreases by 
$\Delta v$, the current through $Q_2$ increases by $g_m \Delta v$ and the
current through $Q_3$ (as reflected by the current mirror $Q_5$) decreases
by $g_m \Delta v$. Therefore the change in output current is 
$-g_m \Delta v - g_m \Delta v = -2 g_m \Delta v$. The output resistance is
the drain resistance $\frac{1}{g_d}$ of $Q_2$ in parallel with the drain
resistance $\frac{1}{g_d}$ of $Q_3$ and is thus about $\frac{1}{2g_d}$, so
the DC gain is
$$
A_v = \frac{-2 \frac{g_m}{2 g_d} \Delta v}{2 \Delta v} = -\frac{g_m}{2 g_d}.
$$ 

The complete analysis follows. Consider an input of $-\Delta v$ applied to
$V_{in}^{+}$ and of $\Delta v$ applied to $V_{in}^{-}$, and label the node
at the drain of $Q_4$ as $V_x$, the voltage at the source of $Q_6$ as $V_t$ 
and the voltage at the drain of $Q_1$ as $V_b$.  Approximating the
sources provided by the current mirrors to be ideal and writing the small 
signal models for $Q_4$, $Q_5$, $Q_3$ and $Q_2$, respectively, gives
\begin{align*}
\Delta I &= g_m \Delta v + g_d \Delta V_x - (g_m + g_d) \Delta V_b \\
\Delta I &= -g_m \Delta V_x - g_d \Delta V_x + (g_m + g_d) \Delta V_t \\
-\Delta I &= -g_m \Delta v + g_d \Delta V_{out} - (g_m + g_d) \Delta V_b \\
-\Delta I &= -g_m \Delta V_x - g_d \Delta V_{out} + (g_m + g_d) \Delta V_t.
\end{align*}
Approximations give
$$
g_m \Delta v - g_m \Delta V_b = -g_m \Delta V_x + g_m \Delta V_t
$$
or
$$
\Delta v + \Delta V_x = \Delta V_b + \Delta V_t
$$
and
$$
-\Delta v + 2\frac{g_d}{g_m} \Delta V_{out} + \Delta V_x = \Delta V_t + \Delta V_b,
$$
so subtracting the second of these equations from the first gives
$$
2 \Delta V - 2 \frac{g_d}{g_m} \Delta V_{out} = 0,
$$
or
$$
\Delta V_{out} = \frac{g_m}{g_d} \Delta v.
$$
Thus
$$
A_v = \frac{\Delta V_{out}}{\Delta V_{in}^+ - \Delta V_{in}^-}
    = \frac{\Delta V_{out}}{-2\Delta v} 
    = -\frac{g_m}{2 g_d}.
$$
}
\item
{
The output swing is 
\begin{align*}
V_{eff_2} + V_{eff_1} 
  &\leq \Delta V_{out} 
   \leq  V_{dd} - (V_{eff_3} + V_{eff_6}) \\
\frac{I_b}{g_m} + \frac{2 I_b}{g_m}
  &\leq \Delta V_{out}
   \leq  V_{dd} - \frac{I_b}{g_m} + \frac{2 I_b}{g_m} \\
\frac{3 I_b}{g_m} &\leq \Delta V_{out} \leq V_{dd} - \frac{3 I_b}{g_m}
\end{align*}
}
\item
{
The current through $Q_6$ and $Q_7$ is $I_b$, so the power draw for the circuit
is $2 I_b V_{dd}$.
}
\end{enumerate}



\section*{Problem \#3}
\begin{enumerate}
\item{(DC gain.)
The current source $Q_1$ mirroring $I_b$ is effectively ideal since it has
an impedance $\frac{1}{g_d}$ and the input transistors have a combined 
impedance of $\frac{1}{4g_m}$.
If a fully differential input is applied, the currents through $Q_2$ and
$Q_3$ will both increase as the currents through $Q_4$ and $Q_5$ both 
decrease by an equal amount, so there is no change in 
current through $Q_7$ or similarly through $Q_9$ (as long as $Q_{11}$'s 
gate is tied to its drain instead of floating, so the current mirror 
$Q_8$ is operational, and thus the source follower no current flows
through the output so $A_v = 0$.
}
\item{(Output swing.)
No change in input will change the output of this circuit, so its output
swing is effectively 0.
}
\item{(Power consumption.)
$Q_{11}$ and $Q_8$ will draw $I_b$, while $Q_7$ and $Q_9$ each draw a current of 
$\frac{I_b}{2}$. Therefore the power consumed is $3 V_{dd} I_b$.
}
\end{enumerate}

\section*{Problem \#4}
\begin{enumerate}
  \item{(Gain.)
    \begin{enumerate}
      \item{First, since the current mirror $M_{11}$ reflecting the reference
        current has an impedance of about $\frac{1}{g_d}$ and the four branches
        attached to it each present an impedance of $\frac{1}{g_m}$, the current
        source is approximately ideal and therefore the sources of the input stage
        transistors see a virtual ground.
      } 
      \item{Examining the middle two branches and writing a small signal model for the 
        transistors closest to $V_{dd}$ (while assuming that $g_d$ can be ignored
        compared to $g_m$) gives
        $$
        g_m \Delta v - g_m \Delta v_l = -g_m \Delta v 
          \Rightarrow \Delta v_l = 2 \Delta v 
          \Rightarrow \Delta I_l = -g_m \Delta v
        $$
        and
        $$
        -g_m \Delta v - g_m \Delta v_r = -g_m \Delta v 
          \Rightarrow \Delta v_r = -2 \Delta v 
          \Rightarrow \Delta I_r = g_m \Delta v,
        $$
        so these two branches cancel each other for a fully differential input
        and can be ignored.
      }
      \item{The remaining circuit is familiar: an increase by $\Delta v$ at 
            $V^{+}$ results in a current change of $g_m \Delta v$ that is 
            reflected to the output, and a decrease by $\Delta v$ at $V^{-}$
            results in a current change of $-g_m \Delta v$ that is reflected
            to the other side of the output. The output impedance is 
            $r_{ds} \| r_{ds}$ due to $M_7$ and $M_{10}$, so the DC gain is
            $$A_v = \frac{-2 g_m \Delta v}{2 g_d} \frac{1}{2 \Delta v} = -\frac{g_m}{2 g_d}.$$
           }
  \end{enumerate}
  }
  \item{(Output swing.)
        The output is limited by
        \begin{align*}
        V_{eff} &\leq \Delta V_{out} \leq V_{dd} - V_{eff} \\
        \frac{2 \left(\frac{I_b}{4}\right)}{g_m} 
        & \leq \Delta V_{out} 
          \leq V_{dd} - \frac{ \left(\frac{I_b}{4}\right)}{g_m} \\
        \frac{I_b}{2g_m} &\leq \Delta V_{out} \leq V_{dd} - \frac{I_b}{2g_m},
        \end{align*}
        since each of the middle branches draws an equal large-signal current.
       }
  \item{(Power consumption.)
        The branch connected to the reference current consumes a power of 
        $V_{dd} I_b$. Each of the six other branches requires 
        $V_{dd} \frac{I_b}{4}$, so the total power consumption is 
        $\frac{5}{2} V_{dd} I_b.$
       }
  \item{(Pole locations.)
       The current mirrors $M_5$, $M_8$ and $M_6$, $M_7$ each introduce a pole
       at about
       $$
       f_{p1} \approx f_{p2} \approx 
       \frac{1}{2 \pi} \frac{1}{\frac{1}{g_m} 2 C_{gs}} = \frac{g_m}{4 \pi C_{gs}}
       $$
       and the mirror $M_9$, $M_{10}$ produces another at about
       $$
       f_{p3} \approx \frac{1}{2 \pi}\frac{g_m}{2C_{gs}} = \frac{g_m}{4 \pi C_{gs}}.
       $$
       There is another pole at the output produced by the output resistance
       $r_{ds} \| r_{ds}$ and the output capacitance, so
       $$
       f_{p4} \approx \frac{1}{2\pi}\frac{2g_d}{2C_{gd} + C_L},
       $$
       where $C_L$ is the load capacitance, if connected.
       The branches in the middle of the circuit are not in the signal path
       and thus do not contribute poles.
       }
  \item{(Zero locations.)
        First, find $I_1(s)$. The resistance at the gate of the current mirror
        $M_6$, $M_7$ is $\frac{1}{g_m}$ and the capacitance is $2 C_{gs}$, so 
        the voltage at this node is 
        $$
        V_1(s) = \frac{-R I_r(s)}{1 + RCs} 
               = \frac{\frac{1}{g_m} (g_m V(s))}{1 + \frac{2 C_{gs}}{g_m} s}
               = \frac{V(s)}{1 + \frac{2 C_{gs}}{g_m} s}
        $$
        so the current $I_1(s)$ is about
        $$
        I_1(s) = -g_m V_1(s) = \frac{-g_m V(s)}{1 + \frac{2 C_{gs}}{g_m} s}.
        $$
        This is also the current through $M_9$ but with opposite sign 
        (due to the equal and opposite input at $V^{+}$), which 
        charges the same resistance and capacitance as above, resulting in a
        voltage at the gate of
        $$
        V_2(s) = \frac{\frac{1}{g_m} (-I_1(s))}{1 + \frac{2 C_{gs}}{g_m} s}
        $$ 
        and thus a current through $M_{10}$ of
        $$
        I_2(s) = g_m V_2(s) = -\frac{I_1(s)}{1 + \frac{2 C_{gs}}{g_m} s}.
        $$
        These currents cancel each other when $I_1(s) = I_2(s)$, or when
        \begin{align*}
          1 + \frac{2 C_{gs}}{g_m} s &= -1 \\
          s &= -\frac{g_m}{C_{gs}},
        \end{align*}
        so there is a real zero in the left half-plane.
        The branches in the middle of the circuit are not in the signal path
        and thus do not contribute zeros.
       }
  \item{(Transfer function.)
       Using the currents derived in the previous part, we see that
       (

       \begin{align*}
       V_{out}(s) &= \frac{\frac{1}{2g_d}}{1 + \frac{2 C_{gd} + C_L}{2g_d} s} 
                     (I_1(s) - I_2(s) \\
                  &= \frac{1}{2 C_{gd} + C_L}
                     \frac{1}{s + \frac{2 g_d}{2 C_{gd} + C_L}}
                     I_1(s)\left(1 + \frac{1}{1 + \frac{2 C_{gs}}{g_m} s}\right) \\
                  &= -\frac{g_m}{2 C_{gd} + C_L}
                      \frac{g_m}{2 C_{gs}}
                      \frac{V(s)}{(s + \frac{g_m}{2 C_{gs}})
                                  (s + \frac{2 g_d}{2 C_{gd} + C_L})}
                      \frac{2 + 2\frac{C_{gs}}{g_m} s}
                           {1 + \frac{2 C_{gs}}{g_m} s} \\
                  &= -\frac{g_m^2}{(2 C_{gd} + C_L)2C_{gs}}
                      \frac{g_m}{2 C_{gs}}
                      \frac{1 + \frac{C_{gs}}{g_m}s}
                           {(s + \frac{g_m}{2 C_gs})^2
                            (s + \frac{2g_d}{2 C_{gd} + C_L})} \\
                  &= -\frac{g_m^2}{(2 C_{gd} + C_L)2 C_{gs}}
                      \frac{s + \frac{g_m}{C_{gs}}}
                           {(s + \frac{g_m}{2 C_{gs}})^2
                            (s + \frac{2 g_d}{2 C_{gd} + C_L})},
       \end{align*}
       confirming the pole and zero locations estimated above.
       }
\end{enumerate}

\pagebreak

\section*{Problem \#5}
Selected simulation results are in the attached figures. The gain, bandwidth, 
voltage swings, and bias points are tabulated below. 

The input swing was chosen to be approximately the same for all bias currents
in the first circuit, since nonlinearities at the edges of the output swing 
were more pronounced for higher bias currents. The simulation used to select
an input range and bias point for the first circuit at a bias of 500 nA is
shown in figure \ref{fig:op-5a-500nA}. A similar simulation was repeated for
each bias.

\begin{figure}
\begin{tabular}{c | c c c c c | c}
Quantity          &       &       &       &       &       & Units          \\
$I_b$             & $0.5$ & $1$   & $2$   & $4$   & $8$   & $\mathrm{\mu A}$ \\
\hline \\
Input bias point  & 708.8 & 750.4 & 797.8 & 854.7 & 927.5 &$\mathrm{m V}$\\
Input swing       & 5     & 5     & 5     & 5     & 5     &$\mathrm{m V}$\\
Output swing      & 2.18  & 2.1   & 2     & 1.83  & 1.63  &$\mathrm{V}$ \\
Gain              & 52.96 & 52.66 & 52.18 & 51.45 & 50.43 & dB \\
Bandwidth         & 2.57  & 4.88  & 8.86  & 15.85 & 26.41 &$\mathrm{k Hz}$
\end{tabular}
\caption{Figures for the first circuit as determined from simulations.}
\end{figure}

\begin{figure}
\begin{tabular}{c | c c c c c | c}
Quantity          &         &         &         &         &         & Units          \\
$I_b$             & $0.5$   & $1$     & $2$     & $4$     & $8$     & $\mathrm{\mu A}$ \\
\hline \\
Selected $V_b$    & 1.3     & 1.2     & 1.3     & 1.5     & 1.7     &$\mathrm{V}$\\
Input bias point  & 710.250 & 752.249 & 799.644 & 856.342 & 928.945 &$\mathrm{m V}$\\
Input swing       & 275     & 285     & 280     & 275     & 272     &$\mathrm{\mu V}$\\
Output swing      & 1.78    & 1.9     & 1.68    & 1.33    & 0.9     &$\mathrm{V}$ \\
Gain              & 69.1    & 67.59   & 67.76   & 66.65   & 65.72   & dB \\
Bandwidth         & 70.84   & 149.8   & 265     & 501.2   & 890.8   &$\mathrm{Hz}$
\end{tabular}
\caption{Figures for the first circuit as determined from simulations.}
\end{figure}

In the second circuit, $V_b$ was chosen for each bias current to provide a 
symmetrical output swing around 1.5 V and to be linear over a broad range.
Multiple sweeps were performed for each bias current to select an appropriate
value of $V_b$, as shown in figure \ref{fig:op-5b-2uA}.
Note that the gain and bandwidth at 2 $\mathrm{\mu A}$ in this circuit
are both improved over the figures at 1 $\mathrm{\mu A}$. This is due to a 
fortunate choice of $V_b$ for this current. Comparing the AC response of
both circuits on the same plot (figure \ref{fig:allgains}) shows that 
despite the differences in frequency response, the second circuit's rolloff
after its dominant pole is as fast as the rolloff of the first.

For both amplifiers the bandwidth increases with bias current as the gain 
(generally) decreases. For the first circuit, the transfer function is given by
$$
\frac{V_{out}(s)}{V_{in}(s)} 
  \approx \frac{\frac{g_m}{g_d}}{1 + \frac{C_L + 2 C_{gd}}{2 g_d} s}
  = \frac{2 g_m}{C_L + 2C_{gd}}\frac{1}{s + \frac{2 g_d}{C_L + 2 C_{gd}}}.
$$
Since the current flowing through the output stage transistors which produce 
the output impedance $\frac{1}{2 g_d}$ is $I_b$, and since $g_d = \lambda I_b$, 
an increase in $I_b$ will cause the pole at 
$\omega \approx \frac{2 g_d}{C_L + 2 C_{gd}}$ to higher frequency, increasing
the bandwidth of the amplifier. At DC, $s = 0$, so the transfer function is
$$
\left.H(s)\right|_{\omega = 0} 
     = \frac{2 g_m}{C_L + 2 C_{gd}} \frac{1}{\frac{2 g_d}{C_L + 2 C_{gd}}}
     = \frac{g_m}{g_d},
$$
and since $g_m = \sqrt{2 \mu C_{ox} \frac{W}{L} I_b}$, this means
$$
\frac{g_m}{g_d} = \frac{\sqrt{2 \mu C_{ox} \frac{W}{L}}}{\lambda\sqrt{I_b}},
$$
so the DC gain decreases with increasing $I_b$.

Similarly for the second circuit the transfer function is
$$
H(s) \approx \frac{\frac{g_m^2}{g_d^2} V(s)}
                  {(1 + \frac{2 C_{gd}}{g_m} s)
                   (1 + \frac{2 C_{gd} + C_L}{2 g_d} s)},
$$
where the dominant pole and thus the bandwidth is still
the output pole with the same expression as before and the
DC gain is
$$
\left(\frac{g_m}{g_d}\right)^2 = \frac{2 \mu C_{ox} \frac{W}{L}}{\lambda^2 I_b}.
$$

\begin{sidewaysfigure}
  \centering
  \includegraphics[width=\textheight]{plot5a-op-500nA-crop}
  \caption{Finding a bias point and input/output voltage range for the 
           first circuit.  \label{fig:op-5a-500nA}}
\end{sidewaysfigure}

\begin{sidewaysfigure}
  \centering
  \includegraphics[width=\textheight]{plot5a-gains}
  \caption{Comparison of the frequency response of the first circuit
           at different bias currents. \label{fig:op-5a-gains}}
\end{sidewaysfigure}

\begin{sidewaysfigure}
  \centering
  \includegraphics[width=\textheight]{plot5b-op-2uA-cropped}
  \caption{Finding bias points and input/output voltage range for the 
           second circuit.  \label{fig:op-5b-2uA}}
\end{sidewaysfigure}

\begin{sidewaysfigure}
  \centering
  \includegraphics[width=\textheight]{plot5b-gains-cropped}
  \caption{Comparison of the frequency response of the second circuit
           at different bias currents. \label{fig:op-5b-gains}}
\end{sidewaysfigure}

\begin{sidewaysfigure}
  \centering
  \includegraphics[width=\textheight]{plot5-allgains-cropped}
  \caption{Comparison of the frequency responses for all circuits simulated
           in this problem. \label{fig:allgains}}
\end{sidewaysfigure}




\end{document}
