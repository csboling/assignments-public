\documentclass{article}

\title{ECE 875 - Homework \#5}
\author{Sam Boling}
\date{\today}

\usepackage{enumitem}
\usepackage{amsmath}
\usepackage{mathrsfs}
\usepackage{amsfonts}
\usepackage{amssymb}
\usepackage{graphicx}

\renewcommand*{\Re}{\operatorname{\mathfrak{Re}}}
\renewcommand*{\Im}{\operatorname{\mathfrak{Im}}}

\newcommand{\horline}
           {\begin{center}
              \noindent\rule{8cm}{0.4pt}
            \end{center}}

\newcommand\scalemath[2]{\scalebox{#1}{\mbox{\ensuremath{\displaystyle #2}}}}

\begin{document}

\maketitle

\section*{Problem \#2.2}
Note that
$$
\frac{d}{dV} \left(\frac{1}{C_D^2}\right) 
  = -\frac{2}{q \varepsilon_S N },
$$
so
\begin{align*}
\frac{1}{N} &= -\frac{q \varepsilon_s}
                {2}
             \frac{d}{dV}
             \left(\frac{1}{C_D^2}\right) \\ 
  &= -\frac{(1.6 \times 10^{-19} ~\mathrm{C})
           (11.9 \cdot 8.85 \times 10^{-14} ~\mathrm{F}~\mathrm{cm}^{-1})}
          {2}
      \frac{(-75 \times 10^{24} ~\mathrm{F}^{-2})}
           {(0.95 ~\mathrm{V})} \\
  &= 6.65 \times 10^{-6} ~\mathrm{cm}^{-1}
\end{align*}
so the donor concentration is
$$
N_D = \frac{N}{A^2} 
  = \frac{(1.5 \times 10^{-6} ~\mathrm{cm}^{-1})}
         {10^{-10} ~\mathrm{cm}^2}
  = 1.5 \times 10^{15} ~\mathrm{cm}^{-3}.
$$
Furthermore
$$
\psi_{bi} = V_0 + \frac{2 k T}{q},
$$
where $V_0$ is the point where the extension of the line $\frac{1}{C_D^2}$
crosses the $V$ axis. Since this is an abrupt junction and the $p$ side is
heavily doped, the majority of the depletion region is on the $n$ side, so
noting that
\begin{align*}
W_n &= W_D - W_{Dp} = \sqrt{\frac{2 \varepsilon_s (\psi_{bi} - V_{bias})}{q N_D}} - W_{Dp}\\
    &\approx \sqrt{\frac{2 \cdot (11.9 \cdot 8.85 \times 10^{-14} ~\mathrm{F}~\mathrm{cm}^{-1})
                         ((0.95 + \frac{2 \cdot 0.0259 ~\mathrm{eV}}{1 ~\mathrm{e}} - (-1))~\mathrm{V})}
                        {(1.6 \times 10^{-19} ~\mathrm{C})
                         (1.5 \times 10^{15} ~\mathrm{cm}^{-3})}} \\
    &- (0.07 \times 10^{-4} ~\mathrm{cm}) \\
    &\approx 1.26 \times 10^{-4} ~\mathrm{cm}.
\end{align*}

\section*{Problem \#2.3}
\begin{enumerate}
  \item{
         Observe first that the maximum electric field $\mathscr{E}_{max}$ will
         be the same as if the entire junction were linearly graded and thus had
         a depletion width of $2 W_{Dp}$, so
         \begin{align*}
         \mathscr{E}_{max} &= -\frac{q a (2W_{Dp})^2}{8 \varepsilon_s} \\
                           &= -\frac{q a W_{Dp}^2}{2 \varepsilon_s} \\
                           &= -\frac{(1.6 \times 10^{-19} ~\mathrm{C})
                                     (10^{19} ~\mathrm{cm}^{-4})
                                     (0.8 \times 10^{-4} ~\mathrm{cm})^2}
                                    {2 (11.9 \cdot 8.85 \times 10^{-14} ~\mathrm{C}~\mathrm{V}^{-1}~\mathrm{cm}^{-1})} \\
                           &= -4.86 \times 10^{3} ~\mathrm{V}~\mathrm{cm}^{-1}.
         \end{align*}
         Since the electric field at the boundary of the depletion region 
         is zero,
         $$ 
         \mathscr{E}(W_{Dn}) = -\varepsilon_{max} + \frac{q N_D W_{Dn}}{\varepsilon_s} = 0
         $$
         so
         \begin{align*}
         W_{Dn} &= \frac{|\mathscr{E}_{max}| \varepsilon_s}{N_D q} \\
                &= \frac{(4.86 \times 10^{3}  ~\mathrm{V}~\mathrm{cm}^{-1})
                         (11.9 \cdot 8.85 \times 10^{-14} ~\mathrm{C}~\mathrm{V}^{-1}~\mathrm{cm}^{-1})}
                        {(3 \times 10^{14} ~\mathrm{cm}^{-3})
                         (1.6 \times 10^{-19} ~\mathrm{C})} \\
                &= 1.07 \times 10^{-4} ~\mathrm{cm}.
         \end{align*}
         Then
         $$
         W_{D} = W_{Dp} + W_{Dn} = (0.8 \times 10^{-4} ~\mathrm{cm}) +(1.07 \times 10^{-4} ~\mathrm{cm}) = 1.87 \times 10^{-4} ~\mathrm{cm}
         $$
         and
         $$
         \psi_{bi} = \frac{|\mathscr{E}_{max}|}{2} W_D = 
                     \frac{(4.86 \times 10^{3} ~\mathrm{V}~\mathrm{cm}^{-1})}{2}
                     (1.87 \times 10^{-4} ~\mathrm{cm}) = 0.45 ~\mathrm{V}.
         $$
       }
  \item{\includegraphics[width=\textheight, angle=90]{fig2-3-b}}
\end{enumerate}

\end{document}
