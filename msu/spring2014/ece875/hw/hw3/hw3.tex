\documentclass{article}

\title{ECE 875 - Homework \#3}
\author{Sam Boling}
\date{\today}

\usepackage{enumitem}
\usepackage{amsmath}
\usepackage{amsfonts}
\usepackage{amssymb}
\usepackage{graphicx}

\renewcommand*{\Re}{\operatorname{\mathfrak{Re}}}
\renewcommand*{\Im}{\operatorname{\mathfrak{Im}}}

\newcommand{\horline}
           {\begin{center}
              \noindent\rule{8cm}{0.4pt}
            \end{center}}

\newcommand\scalemath[2]{\scalebox{#1}{\mbox{\ensuremath{\displaystyle #2}}}}

\begin{document}

\maketitle

\section*{Problem \#1.10}
Calculate the average kinetic energy of electrons in the conduction band of 
an $n$-type non-degenerate semiconductor. The density of states is given by
Eq. 14.
\horline
The average kinetic energy in the conduction band is defined as
$$
\frac{\int_{E_C}^{E_C + \Delta E} (E - E_C) ~dn(E)}
     {\int_{E_C}^{E_C + \Delta E} dn(E)},
$$
where
$$
n = \int_{E_C}^{E_C + \Delta E} dn(E) 
  = \int_{E_C}^{E_C + \Delta E} N(E) F(E) ~dE,
$$
so the kinetic energy $K$ is
\begin{align*}
K &= \frac{1}{n}\int_{E_C}^{E_C + \Delta E} (E - E_C) N(E) F(E) ~dE \\
  &= \frac{1}{n}\int_{E_C}^{E_C + \Delta E} 
             (E - E_C) 
             M_C \frac{\sqrt{2}}{\pi^2}
                 \frac{m_{de}^{3/2}(E - E_C)^{1/2}}{\hslash^3}
                 \frac{1}{1 + \exp[(E - E_F)/kT]} ~dE \\
  &= \frac{\sqrt{2} M_C m_{de}^{3/2}}{\pi^2 \hslash^3 n}
     \int_{E_C}^{E_C + \Delta E} 
        \frac{(E - E_C)^{3/2}}
             {1 + \exp[((E - E_C) + (E_C - E_F))/kT]} ~dE.
\end{align*}
Letting $\eta = \frac{E - E_C}{kT}$ gives $d\eta = \frac{dE}{kT}$ and
thus $dE = kT ~d\eta$, giving
\begin{align*}
K &=  \frac{\sqrt{2} M_C m_{de}^{3/2}}{\pi^2 \hslash^3 n}
   \int_{0}^{\frac{\Delta E}{kT}} 
     \frac{(kT)^{5/2}\eta^{3/2}}{1 + \exp[\eta - \eta_F]} ~d\eta
\end{align*}
where $\eta_F = \frac{E_F - E_C}{kT}$.
But the number of electrons $n$ in the conduction band is given by
\begin{align*}
n &= N_C \frac{2}{\sqrt{\pi}} F_{1/2}\left(\frac{E_F - E_C}{kT}\right) \\
  &= \frac{2 \cdot 2 M_C}{\sqrt{\pi}} 
    \left(\frac{2 \pi m_{de} kT}{h^2}\right)^{3/2} 
    F_{1/2}(\eta_F) \\
  &= M_C m_{de}^{3/2} (kT)^{3/2} \sqrt{2}2^3 \pi h^{-3}
    F_{1/2}(\eta_F)
\end{align*}
so
$$
\pi^2 \hslash^3 n = \sqrt{2} M_C m_{de}^{3/2} (kT)^{3/2} 
                    (2\pi\hslash)^3 (2 \pi \hslash)^{-3}
                    F_{1/2}(\eta_F)
                  = \sqrt{2} M_C m_{de}^{3/2} (kT)^{3/2}
                    F_{1/2}(\eta_F)
$$
so the integral above becomes
$$
K = \frac{kT}{F_{1/2}(\eta_F)}
   \int_{0}^{\frac{\Delta E}{kT}}
   \frac{\eta^{3/2}}{1 + \exp[\eta - \eta_F]}~d\eta
 = \frac{kT}{F_{1/2}(\eta_F)}
   \int_{0}^{\infty}
   \frac{\eta^{3/2}}{1 + \exp[\eta - \eta_F]}~d\eta
$$
when $\Delta E$ is large. But 
$\exp(\eta - \eta_F) = \exp\left(\frac{E - E_F}{kT}\right)$,
so in the hot approximation at 300K $\exp(\eta - \eta_F) \gg 1$.
Then 
\begin{align*}
K &= \frac{kT}{F_{1/2}(\eta_F)}
     \int_{0}^{\infty}
     \frac{\eta^{3/2}}{1 + \exp[\eta - \eta_F]}~d\eta  \\
  &= kT \frac{\int_{0}^{\infty}
     \frac{\eta^{3/2}}{1 + \exp[\eta - \eta_F]}~d\eta}
          {\int_{0}^{\infty}
            \frac{\eta^{1/2}}{1 + \exp[\eta - \eta_F]}~d\eta}  \\
  &\approx kT \frac{\int_{0}^{\infty}
     \frac{\eta^{3/2}}{\exp[\eta - \eta_F]}~d\eta}
          {\int_{0}^{\infty}
            \frac{\eta^{1/2}}{\exp[\eta - \eta_F]}~d\eta}  \\
  &= kT \frac{e^{\eta_F}\int_{0}^{\infty}
     \frac{\eta^{3/2}}{\exp[\eta]}~d\eta}
          {e^{\eta_F}\int_{0}^{\infty}
            \frac{\eta^{1/2}}{\exp[\eta]}~d\eta}  \\
  &= kT \frac{\Gamma\left(\frac{3}{2} + 1\right)}
             {\Gamma\left(\frac{1}{2} + 1\right)} \\
  &= kT \frac{\Gamma\left(\frac{1}{2} + 2\right)}
             {\Gamma\left(\frac{1}{2} + 1\right)}.
\end{align*}
But
$$
\Gamma\left(\frac{1}{2} + n\right) = \frac{(2n)!\sqrt{\pi}}{4^n n!}
$$
so
$$
\Gamma\left(\frac{1}{2} + 2\right) = \sqrt{\pi} \frac{4!}{4^2 2!} 
                                   = \sqrt{\pi} \frac{3}{4} \sqrt{\pi}
$$
and
$$
\Gamma\left(\frac{1}{2} + 1\right) = \sqrt{\pi} \frac{2!}{4^1 1!}
                                   = \sqrt{\pi} \frac{1}{2}
$$
so
$$
K = \frac{3}{2} kT.
$$

\section*{Problem \#1.11}
Show that
$$
N_{D}^{+} = N_D \left[1 + 2 \exp\left(\frac{E_F - E_D}{kT}\right)\right]^{-1}
$$
[Hint: The probability of occupancy is
$$
F(E) = \left[1 + \frac{h}{g} \exp\left(\frac{E - E_F}{kT}\right)\right]^{-1}
$$
where $h$ is the number of electrons that can physically occupy the level 
$E$, and $g$ is the number of electrons that can be accepted by the level,
also called the ground-state degeneracy of the donor impurity level ($g=2$).]
\horline
The local energy level near a donor atom can be occupied by one electron, but
may accept an electron of either spin, so the coefficient 
$\frac{h}{g} = \frac{1}{2}$. The probability that a donor is ionized is equal
to the probability that this energy level is unoccupied, or $1 - F(E_D)$. 
Thus the ionized concentration for donors is
\begin{align*}
  N_D^+ = N_D (1 - F(E_D)) 
  &= N_D\left[1 - 
     \frac{1}
          {1 + \frac{1}{2}\exp\left(\frac{E_D - E_F}{kT}\right)}\right] \\
  &= N_D\left[1 - \frac{2 \exp\left(\frac{E_F - E_D}{kT}\right)}
                       {2 \exp\left(\frac{E_F - E_D}{kT}\right) + 1}\right] \\
  &= N_D\left[\frac{2 \exp\left(\frac{E_F - E_D}{kT}\right) + 1 
                  - 2 \exp\left(\frac{E_F - E_D}{kT}\right)}
                   {1 + 2 \exp \left(\frac{E_F - E_D}{kT}\right)}\right] \\
  &= N_D\left[1 + 2 \exp \left(\frac{E_F - E_D}{kT}\right)\right]^{-1}.
\end{align*}


\section*{Problem \#1.16}
Gold in Si has two energy levels in the bandgap: $E_C - E_A = 0.54$ eV,
$E_D - E_V = 0.29$ eV. Assume the third level $E_D - E_V = 0.35$ eV is
inactive. (a) What will be the state of charge of the gold levels in Si
doped with high concentration of boron atoms? Why? (b) What is the effect
of gold on electron and hole concentrations?
\horline
\begin{enumerate}[label=(\Roman*)]
  \item{The acceptor level in boron will receive an electron from the donor
        level in gold at 0.29 eV, placing it at -1, so the donor gold level 
        will be at +1 and the acceptor gold level will be at 0 to preserve
        charge neutrality. This happens because the valence electron from the 
        donor level in gold can enter the nearby, lower-energy state in boron.
       }
  \item{Each gold atom will produce one positively ionized gold atom that can 
        accept an electron, so the hole concentration will increase.
       }
\end{enumerate}

\section*{Problem \#1.18}
For an $n$-type silicon sample doped with $2.86 \times 10^{16}$ cm$^{-3}$
phosphorous atoms, find the ratio of the neutral to ionized donors at 300 K.
($E_C - E_D) = 0.045$ eV.
\horline
Given that
$$
N_D^+ = \frac{N_D}{1 + 2 \exp \left(\frac{E_F - E_D}{kT}\right)}
$$
we see that
$$
\frac{N_D}{N_D^+} = 1 + 2 \exp \left(\frac{E_F - E_D}{kT}\right) 
                  = 1 + 2 \exp \left(\frac{(E_C - E_D) - (E_C - E_F)}{kT}\right), 
$$
where
$$
E_C - E_F = kT \ln \left(\frac{N_C}{n}\right)
$$
so
$$
\frac{N_D}{N_D^+} = 1 + 2 \exp \left(\frac{E_C - E_D}{kT}\right) 
                          \exp \left(-\ln\left(\frac{N_C}{n}\right)\right)
                  = 1 + \frac{2 n}{N_C} \exp \left(\frac{E_C - E_D}{kT}\right)
$$
where
$$
\frac{2n}{N_C} = \frac{2 \cdot 2.86 \times 10^{16}}{2.8 \times 10^{19}} 
  \approx 2.04 \times 10^{-3}
$$
and
$$
\frac{E_C - E_D}{kT} = \frac{0.045}{300 \cdot 1.38 \times 10^{-23}} 
  \approx 1.09 \times 10^{19}
$$
so
$$
\frac{N_D}{N_D^+} \approx 2.22 \times 10^16.
$$

\pagebreak
\section*{Problem \#1.25}
The recombination rate is given by Eq. 92. Under low injection condition, $U$
can be expressed as $(p_n - p_{no})/\tau_r$, where $\tau_r$ is the 
recombination lifetime. If $\sigma_n = \sigma_p = \sigma_o$, 
$n_{no} = 10^{15}$ cm$^{-3}$, and 
$\tau_{ro} \equiv (v_{th} \sigma_o N_t)^{-1}$, find the values of 
$(E_t - E_i)$ at which the recombination lifetime $\tau_r$ becomes
$2 \tau_{ro}$.
\horline

Equation 92 gives
$$
U = \frac{\sigma_n \sigma_p v_{th} N_t (pn - n_i^2)}
         {\sigma_n [n + n_i \exp \left(\frac{E_t - E_i}{kT}\right)]
        + \sigma_p [p + n_i \exp \left(\frac{E_i - E_t}{kT}\right)]}
$$
and under the given assumptions $U = \frac{p_n - p_{no}}{\tau_r}$ so
\begin{align*}
\frac{p_n - p_{no}}{\tau_r} 
  &= \frac{\sigma_o v_{th} N_t (pn - n_i^2)}
         {n + p + \exp\left(\frac{E_t - E_i}{kT}\right)
                + \exp\left(\frac{-(E_t - E_i)}{kT}\right)} \\
  &= \frac{\tau_{ro}^{-1} (pn - n_i^2)}
          {n + p + 2n_i \cosh \left(\frac{E_t - E_i}{kT}\right)}
\end{align*}
when $\frac{\tau_{ro}}{\tau_{r}} = 2$, so
$$
p - p_{no} = \Delta p 
           = \frac{2 (pn - n_i^2)}
                  {n + p + 2n_i \cosh \left(\frac{E_t - E_i}{kT}\right)}.
$$
Assuming the semiconductor in question is silicon or gallium arsenide,
$n_{no} \gg n_i \geq 10^{10}$, and in the low injection condition 
$n \approx n_{no} \gg \Delta p \gg p_{no}$, so since 
$p = p_{no} + \Delta p$ (the doping concentration of holes on the n-type
side of the junction plus the concentration of holes moving from the
p-type material) and $p_{no} = \frac{n_i^2}{n_{no}}$, we see that
\begin{align*}
pn - n_i^2 &= n_{no} (p_{no} + \Delta p) - n_i^2 \\
           &= n_{no} (\frac{n_i^2}{n_{no}} + \Delta p) - n_i^2 \\
           &= n_{no} \Delta p.
\end{align*}

Then we have
$$
\Delta p \approx \frac{2 n_{no} \Delta p}{n_{no} + p + 2n_i \cosh \left(\frac{E_t - E_i}{kT}\right)}
$$
or
$$
p + 2n_i \cosh \left(\frac{E_t - E_i}{kT}\right) = n_{no}
$$
and since $p = p_{no} + \Delta p$ and $\Delta p \gg p_{no}$, $p \approx \Delta p \ll n_{no}$ so
\begin{align*}
2 n_i \cosh\left(\frac{E_t - E_i}{kT}\right) &= n_{no} \\
\cosh\left(\frac{E_t - E_i}{kT}\right) &= \frac{n_{no}}{2 n_i} \\
E_t - E-i &= kT \cosh^{-1}\left(\frac{n_{no}}{2 n_i}\right) \\
  &\approx 300(1.38 \times 10^{-23})\cosh^{-1}\left(\frac{10^{15}}{2 \cdot 9.65 \times 10^9}\right) \\
  &\approx 4.78 \times 10^{-20} \mathrm{J} \approx 0.30 \mathrm{eV}
\end{align*}
in silicon at 300K.

\end{document}
