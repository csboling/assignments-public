\documentclass{article}

\title{ECE 875 - Midterm Exam}
\author{Sam Boling}
\date{\today}

\usepackage{enumitem}

\usepackage{amsmath}
\usepackage{mathrsfs}
\usepackage{amsfonts}
\usepackage{amssymb}

\usepackage{graphicx}
\usepackage{subcaption}
\usepackage{rotating}

\renewcommand*{\Re}{\operatorname{\mathfrak{Re}}}
\renewcommand*{\Im}{\operatorname{\mathfrak{Im}}}

\newcommand{\horline}
           {\begin{center}
              \noindent\rule{8cm}{0.4pt}
            \end{center}}

\newcommand\scalemath[2]{\scalebox{#1}{\mbox{\ensuremath{\displaystyle #2}}}}

\begin{document}

\maketitle

\section*{Problem \#1}
\begin{enumerate}
  \item{Show that the reciprocal lattice of a face centered cubic
        (fcc) lattice with a lattice constant $a$ is a body
        centered cubic (bcc) lattice with the side of the cubic
        cell equal to $\frac{4 \pi}{a}$.
       }
  \item{Find the volume of the bcc primitive unit cell.}
\end{enumerate}
\horline
\begin{enumerate}
  \item{The basis vectors for a face-centered cubic primitive
        cell with lattice constant $a$ are given by
        \begin{align*}
          \mathbf{a} &= \frac{a}{2} (\mathbf{y} + \mathbf{z}), \\
          \mathbf{b} &= \frac{a}{2} (\mathbf{x} + \mathbf{z}), \\
          \mathbf{c} &= \frac{a}{2} (\mathbf{x} + \mathbf{y}), \\
        \end{align*}
        where $\mathbf{x}$, $\mathbf{y}$, and $\mathbf{z}$ are orthogonal unit vectors. Then
        \begin{align*}
          \mathbf{b} \times \mathbf{c} &=
          \left|\begin{array}{c c c}
            \mathbf{x}   & \mathbf{y}  & \mathbf{z}  \\
             \frac{a}{2} & 0           & \frac{a}{2} \\
             \frac{a}{2} & \frac{a}{2} & 0 
          \end{array}\right| \\
          &= \left(0 - \frac{a^2}{4}\right) \mathbf{x} 
           - \left(0 - \frac{a^2}{4}\right) \mathbf{y} 
           + \left(\frac{a^2}{4} - 0\right) \mathbf{z} \\
          &= \frac{a^2}{4} (\mathbf{y} + \mathbf{z} - \mathbf{x}), \\
          \mathbf{c} \times \mathbf{a} &=
          \left|\begin{array}{c c c}
            \mathbf{x}   & \mathbf{y}  & \mathbf{z}  \\
             \frac{a}{2} & \frac{a}{2} & 0           \\
             0           & \frac{a}{2} & \frac{a}{2} \\
          \end{array}\right| \\
          &= \left(\frac{a^2}{4} - 0\right) \mathbf{x} 
           - \left(\frac{a^2}{4} - 0\right) \mathbf{y} 
           + \left(\frac{a^2}{4} - 0\right) \mathbf{z} \\
          &= \frac{a^2}{4} (\mathbf{x} + \mathbf{z} - \mathbf{y}), \\
          \mathbf{a} \times \mathbf{b} &=
          \left|\begin{array}{c c c}
            \mathbf{x}   & \mathbf{y}  & \mathbf{z}  \\
             0           & \frac{a}{2} & \frac{a}{2} \\
             \frac{a}{2} & 0           & \frac{a}{2} \\
          \end{array}\right| \\
          &= \left(\frac{a^2}{4} - 0\right) \mathbf{x} 
           - \left(0 - \frac{a^2}{4}\right) \mathbf{y} 
           + \left(0 - \frac{a^2}{4}\right) \mathbf{z} \\
          &= \frac{a^2}{4} (\mathbf{x} + \mathbf{y} - \mathbf{z}), \\
        \end{align*}
        and furthermore
        \begin{align*}
        \mathbf{a} \bullet \mathbf{b} \times \mathbf{c} &=
          \left(\frac{a}{2}(\mathbf{y} + \mathbf{z})\right) \bullet
          \left(\frac{a^2}{4}(\mathbf{y} + \mathbf{z} - \mathbf{x})\right) \\
        &= \frac{a^3}{8}(1 + 1) = \frac{a^3}{4}
        \end{align*}
        so
        $$
        \frac{2\pi}{\mathbf{a} \bullet \mathbf{b} \times \mathbf{c}} = \frac{8 \pi}{a^3}.
        $$
        Therefore
        \begin{align*}
          \mathbf{a}^{\ast} &= \frac{2 \pi}
                                    {\mathbf{a} \bullet \mathbf{b} \times \mathbf{c}}
                               \mathbf{b} \times \mathbf{c} \\
                            &= \frac{8 \pi}{a^3} \frac{a^2}{4} 
                               (\mathbf{y} + \mathbf{z} - \mathbf{x}) \\
                            &= \frac{2 \pi}{a}(\mathbf{y} + \mathbf{z} - \mathbf{x}), \\
          \mathbf{b}^{\ast} &= \frac{2 \pi}
                                    {\mathbf{a} \bullet \mathbf{b} \times \mathbf{c}}
                               \mathbf{c} \times \mathbf{a} \\
                            &= \frac{8 \pi}{a^3} \frac{a^2}{4} 
                               (\mathbf{x} + \mathbf{z} - \mathbf{y}) \\
                            &= \frac{2 \pi}{a}(\mathbf{x} + \mathbf{z} - \mathbf{y}), \\
          \mathbf{c}^{\ast} &= \frac{2 \pi}
                                    {\mathbf{a} \bullet \mathbf{b} \times \mathbf{c}}
                               \mathbf{a} \times \mathbf{b} \\
                            &= \frac{8 \pi}{a^3} \frac{a^2}{4} 
                               (\mathbf{x} + \mathbf{y} - \mathbf{z}) \\
                            &= \frac{2 \pi}{a}(\mathbf{x} + \mathbf{y} - \mathbf{z}),
        \end{align*}
        and since $\frac{2\pi}{a} = \frac{\frac{4\pi}{a}}{2}$, these vectors
        are the basis vectors of a bcc primitive cell with ``lattice constant"
        $\frac{4\pi}{a}$.
       }
       \item{
         The volume of the bcc primitive cell computed in part (a) is given by
         the scalar triple product
         \begin{align*}
         \mathbf{a}^{\ast} \bullet \mathbf{b}^{\ast} \times \mathbf{c}^{\ast} \\
         &= \left(\frac{2\pi}{a}(\mathbf{y} + \mathbf{z} - \mathbf{x})\right) \bullet 
            \left|\begin{array}{r r r}
            \mathbf{x}     &  \mathbf{y}     &  \mathbf{z}     \\
            \frac{2\pi}{a} & -\frac{2\pi}{a} &  \frac{2\pi}{a} \\
            \frac{2\pi}{a} &  \frac{2\pi}{a} & -\frac{2\pi}{a} \\
            \end{array}\right| \\
         &= \left(\frac{2\pi}{a}(\mathbf{y} + \mathbf{z} - \mathbf{x})\right) \bullet
            \left[\left(\frac{4\pi^2}{a^2} - \frac{4 \pi^2}{a^2}\right)\mathbf{x}
                - \left(-\frac{4\pi^2}{a^2} - \frac{4 \pi^2}{a^2}\right)\mathbf{y}
                + \left(\frac{4\pi^2}{a^2} + \frac{4 \pi^2}{a^2}\right)\mathbf{z}\right] \\
         &= \frac{8 \pi^{2}}{a^3}(\mathbf{y} + \mathbf{z} - \mathbf{x}) \bullet
                            (\mathbf{y} + \mathbf{z}) \\
         &= \frac{16 \pi^{2}}{a^3}.
         \end{align*}
       }
\end{enumerate}

\pagebreak 

\section*{Problem \#2}
Prove that the concentration of holes in neutral dopant acceptor
states is given by
$$
N_{A}^{0} = \frac{N_A}
                 {1 + \frac{h}{g} 
                      \exp\left(\frac{E_F - E_A}{kT}\right)}.
$$
\horline
We note from equation 35 that where $h=1$ and $g_A$ is the ground state 
degeneracy for acceptors, the concentration of ionized acceptors is given by
$$
N_{A}^{-} = \frac{N_A}{1 + \frac{g_{A}}{h}\exp\left(\frac{E_A - E_F}{kT}\right)}
          =  N_A ~\mathrm{Pr(an~acceptor~is~ionized)}.
$$
To find the concentration of holes in neutral dopant acceptor states, note 
that the concentration of such holes is equal to the concentration of neutral
acceptors (i.e., those that have "trapped" a hole), and that
\begin{align*}
\mathrm{Pr(an~acceptor~is~neutral)} &= 1 - \mathrm{Pr(an~acceptor~is~ionized)} \\
  &= 1 - \frac{1}{1 + \frac{g_A}{h} \exp\left(\frac{E_A - E_F}{kT}\right)} \\
  &= 1 - \frac{\frac{h}{g_A} \exp \left(\frac{E_F - E_A}{kT}\right)}
              {\frac{h}{g_A} \exp \left(\frac{E_F - E_A}{kT}\right) + 1} \\
  &= \frac{\frac{h}{g_A} \exp \left(\frac{E_F - E_A}{kT}\right) + 1 
         - \frac{h}{g_A} \exp \left(\frac{E_F - E_A}{kT}\right)}
          { \frac{h}{g_A} \exp \left(\frac{E_F - E_A}{kT}\right) + 1} \\
  &= \frac{1}{1 + \frac{h}{g_A} \exp\left(\frac{E_F - E_A}{kT}\right)},
\end{align*}
and thus
$$
N_A^0 = N_A ~\mathrm{Pr(an~acceptor~is~neutral)}
      = \frac{N_A}{1 + \frac{h}{g} \exp\left(\frac{E_F - E_A}{kT}\right)}
$$
as desired.

%Neutral acceptors are those which have not trapped a free electron in their
%empty local energy level $E_A$, so the probability that an acceptor is 
%neutral is, by definition, the probability that its local energy level is 
%unoccupied, or $1 - F(E_A)$. But
%\begin{align*}
%1 - F(E) &= 1 - \frac{1}{1 + \frac{h}{g}\exp\left(\frac{E - E_F}{kT}\right)} \\
%         &= 1 - \frac{\frac{g}{h}\exp\left(\frac{E_F - E}{kT}\right)}
%                     {\frac{g}{h}\exp\left(\frac{E_F - E}{kT}\right) + 1} \\
%         &= \frac{\frac{g}{h}\exp\left(\frac{E_F - E}{kT}\right) + 1
%                - \frac{g}{h}\exp\left(\frac{E_F - E}{kT}\right)}
%                 {\frac{g}{h}\exp\left(\frac{E_F - E}{kT}\right) + 1} \\
%         &= \frac{1}
%                 {1 + \frac{g}{h}\exp\left(\frac{E_F - E}{kT}\right)} 
%\end{align*}
%
%
%Thus the concentration of neutral acceptors, 
%which must be equal to the concentration of holes in neutral dopant acceptor 
%states, since an acceptor with an empty local energy level can be thought to
%have trapped a hole, is
%%\begin{align*}
%N_{A}^{0} &= N_A (1 - F(E_A)) = \frac{N_A}{1 + \frac{h}{g} \exp\left(\frac{E
%\end{align*}

\pagebreak

\section*{Problem \#3}
For a single level recombination process, find the average time that takes 
place between each recombination process in a region of a silicon
sample at 300K where $n = p = 10^{13} ~\mathrm{cm}^{-3}$,
$\sigma_n = \sigma_p = 2 \times 10^{-16} ~\mathrm{cm}^2$, 
$v_{th} = 10^{7} ~\mathrm{cm} ~\mathrm{s}^{-1}$, 
$N_t = 10^{16} ~\mathrm{cm}^{-3}$ and $(E_t - E_i) = 5 kT$.
\horline
The transition rate is given by
\begin{align*}
  U &= \frac{\sigma_n \sigma_p v_{th} N_t(pn - n_i^2)}
           {\sigma_n \left[n + n_i \exp\left(\frac{E_t - E_i}{kT}\right)\right]
          + \sigma_p \left[p + n_i \exp\left(\frac{E_i - E_t}{kT}\right)\right]} \\
    &= \frac{(2 \times 10^{-16} ~\mathrm{cm}^2)^2 
             (10^7 ~\mathrm{cm}~\mathrm{s}^{-1})
             (10^{16} ~\mathrm{cm}^{-3})
             ((10^{13} ~\mathrm{cm}^{-3})^2 - (9.65 \times 10^9 ~\mathrm{cm}^{-3})^2)}
            {(2 \times 10^{-16} ~\mathrm{cm}^2)
             \left[2(10^{13} ~\mathrm{cm}^{-3}) 
           + 2(9.65 \times 10^9 ~\mathrm{cm}^{-3})\cosh(5)\right]} \\
          &= \frac{4 \times 10^{17} 
                    ~\mathrm{cm}^{-4} ~\mathrm{s}^{-1}}
                  {4.29 \times 10^{-3} ~\mathrm{cm}^{-1}} \\
          &= 9.32 \times 10^{19} ~\mathrm{cm}^{-3} ~\mathrm{s}^{-1},
\end{align*}
which is positive since $pn \gg n_i^2$, indicating recombination. 

\pagebreak

\section*{Problem \#4}
The given measurements are taken from an abrupt $p^{+}n$ junction and a
symmetric linearly graded $pn$ junction. 
\begin{enumerate}
  \item{Which is which?}
  \item{Evaluate $\psi_{bi}$, $a$ and $\mathscr{E}_{m}$ for the
linearly graded $pn$ junction.}
  \item{Evaluate $\psi_{bi}$, $N$ and $W_D$ for the abrupt junction.}
  \item{A metal contact is made to the $n$-side of the linearly graded $pn$
  junction well outside of the junction depletion region. What is the width of
  the Schottky barrier depletion region $W_D$ that develops around the 
  contact?}
\end{enumerate}

\horline
\begin{enumerate}
  \item{$\frac{1}{C^2}$ curves are plotted from the given data in figure 
        \ref{fig:capplots}. Junction \#1 is nonlinear while junction \#2 is 
        linear, so junction \#2 is the abrupt junction and junction \#1 has 
        non-constant doping on either side of the junction.
       }
  \item{From equation 38 in Sze, 
        \begin{align*}
          C_D &= \left[\frac{q a \varepsilon_s^2}{12 (\psi_{bi} - V)}\right] \\
          \frac{1}{C_D^3} &= \frac{12(\psi_{bi} - V)}{q a \varepsilon_s^2} \\
          \frac{d}{dV}\left(\frac{1}{C_D^3}\right) &= -\frac{12}{q a \varepsilon_s^2} \\
          a &= -\frac{12}{q \varepsilon_s^2 \frac{d}{dV}\left(\frac{1}{C_D^3}\right)},
        \end{align*}
        so the doping gradient $a$ is given by
        \begin{align*}
        a &= -\frac{12}{(1.6 \times 10^{-19} ~\mathrm{C})
                       (11.9 \cdot 8.85 \times 10^{-14} ~\mathrm{F}~\mathrm{cm}^{-1})^2
                       (-5.67 \times 10^{-5} ~\mathrm{(cm^2 / nF)}^3 ~\mathrm{V}^{-1})
                      } \\
          &= 1.26 \times 10^{9} ~\mathrm{cm}^{-4}.
        \end{align*}
        Then from equation 37,
        \begin{align*}
          \psi_{bi} = 
        \end{align*}
       }
  \item{The $V$-intercept from the linear extrapolations of the curve is
        $0.6 ~\mathrm{V}$ for junction \#2, the abrupt junction,
        so $\psi_{bi} = 0.6 ~\mathrm{V}$. The slope of the line is 
        $-1.2 \times 10^{-4} ~(\mathrm{cm} / \mathrm{nF})^2 ~\mathrm{V}^{-1}=
         -1.2 \times 10^{14} ~\mathrm{cm} ~\mathrm{F}^{-1} ~\mathrm{V}^{-1}$ so the doping 
        concentration on the $n$ side is
        \begin{align*}
          N_D &= -\frac{2}{q \varepsilon_s}\left[-\frac{1}{\frac{d}{dV}\left(\frac{1}{C^2}\right)} \\
              &= \frac{2}{1.6 \times 10^{-19} ~\mathrm{C})(11.9 \cdot 8.85 \times 10^{-14} ~\mathrm{F}~\mathrm{cm}^{-1})}
                 \frac{1}{2.3 \times 10^{14} ~\mathrm{cm}^{4} ~\mathrm{F}^{-2} ~\mathrm{V}^{-1} \\
              &= 
        \end{align*}
        and therefore
        \begin{align*}
          W_D &= \sqrt{\frac{2 \varepsilon_s}{q N_D}\left(\psi_{bi} - \frac{kT}{q}\right) \\
              &= \sqrt{\frac{2 (11.9 \cdot 8.85 \times 10^{-14} ~\mathrm{F} ~\mathrm{cm}^{-1})}
                            {(1.6 \times 10^{-19} ~\mathrm{C})()}
                       (0.6 - 0.0259 ~\mathrm{V})} 
        \end{align*}
       }
\end{enumerate}

\begin{figure}[ht!]
  \centering
  \begin{subfigure}[b]{\textwidth}
    \includegraphics[width=\linewidth]{junction1}
  \end{subfigure} 

  \begin{subfigure}[b]{\textwidth}
    \includegraphics[width=\linewidth]{junction2}
  \end{subfigure}
  \caption{The $\frac{1}{C^2}$ plots for Problem 4. \label{fig:capplots}}
\end{figure}

\pagebreak

\section*{Problem \#5}
For the given device structure fabricated in Si, find the depletion width
at thermal equilibrium at 300K.



\end{document}
