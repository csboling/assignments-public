\documentclass{article}

\title{ECE 875 - Final Exam}
\author{Sam Boling}
\date{\today}

\usepackage{enumitem}

\usepackage{amsmath}
\usepackage{mathrsfs}
\usepackage{amsfonts}
\usepackage{amssymb}

\usepackage{graphicx}
\usepackage{subcaption}
\usepackage{rotating}
\usepackage{hyperref}

\renewcommand*{\Re}{\operatorname{\mathfrak{Re}}}
\renewcommand*{\Im}{\operatorname{\mathfrak{Im}}}

\newcommand{\horline}
           {\begin{center}
              \noindent\rule{8cm}{0.4pt}
            \end{center}}

\newcommand\scalemath[2]{\scalebox{#1}{\mbox{\ensuremath{\displaystyle #2}}}}

\begin{document}

\maketitle

\section*{Problem \#1}
For a metal-n-Si contact at 300 K, the effective barrier height obtained by
photoelectric measurements is 0.995 eV while the voltage intercept obtained 
from $C$-$V$ measurement is 0.75 V.

\begin{enumerate}
  \item{Find the doping of the uniformly doped Si substrate.
       }
  \item{Express the $C$-$V$ measurement as the linear relationship
        $$
        \frac{1}{C^2} = f(V) = A - BV.
        $$
        Assume that the basic unit of $C$ is $\mathrm{F}/\mathrm{cm}^{2}$.
        \begin{enumerate}
          \item{Evaluate $A$ with correct units.}
          \item{Evaluate $B$ with correct units.}
        \end{enumerate}
       }
\end{enumerate}
\horline
\begin{enumerate}
\item{
The photoelectric measurement gives
$$
\phi_{Bn} = 0.995 ~\mathrm{V}
$$
while the C-V line intercept gives
$$
\psi_{bi} = V_{\mathrm{intercept}} + \frac{kT}{q} =  0.75 + 0.0259  ~\mathrm{V} = 0.7759 ~\mathrm{V}.
$$
But from the C-V measurement approach,
\begin{align*}
\phi_{Bn} &= \psi_{bi} + \phi_n + \frac{kT}{q} - \Delta \phi \\
          &= \psi_{bi} 
           + \frac{kT}{q}\ln\frac{N_C}{N_D}
           + \frac{kT}{q}
           - \Delta \phi \\
\phi_{Bn} - \psi_{bi} + \Delta \phi &= \frac{kT}{q}\left(\ln\frac{N_C}{N_D} + 1\right) \\
\frac{N_C}{N_D} &= \exp\left(\frac{\phi_{Bn} - \psi_{bi} + \Delta \phi}
                                  {\frac{kT}{q}}\right) \\
N_D &= N_C \exp\left(-\frac{\phi_{Bn} - \psi_{bi} + \Delta \phi}{\frac{kT}{q}}\right).
\end{align*}

Neglecting the image force lowering term $\Delta \phi$ at first,
the doping concentration can then be found as
\begin{align*} 
N_D &\approx (2.8 \times 10^{19} ~\mathrm{cm}^{-3}) 
             \exp\left(-\frac{(0.995 - 0.7759) ~\mathrm{V}}{0.0259 ~\mathrm{V}}\right) \\
    &=       (2.8 \times 10^{19} ~\mathrm{cm}^{-3}) 
             \exp\left(-\frac{0.2191 ~\mathrm{V}}{0.0259 ~\mathrm{V}}\right) \\
    &= 5.93 \times 10^{15} ~\mathrm{cm}^{-3}
\end{align*}
which gives a maximum electric field of
\begin{align*}
E_m &= \sqrt{\frac{2 q N_D}{\varepsilon_s}\left(\psi_{bi} - \frac{kT}{q}\right)} \\
    &= \sqrt{\frac{2 (1.6 \times 10^{-19} ~\mathrm{C})
                     (5.93 \times 10^{15} ~\mathrm{cm}^{-3})}
                  {(11.9 \cdot 8.85 \times 10^{-14} ~\mathrm{F}~\mathrm{cm}^{-1})}
             \left(0.7759 - 0.0259 ~\mathrm{V}\right)} \\
    &= 3.68 \times 10^{4} ~\mathrm{V}~\mathrm{cm}^{-1}
\end{align*}
and thus an image charge lowering of
\begin{align*}
\Delta \phi &= \sqrt{\frac{q E_m}{4 \pi \varepsilon_s}} \\
            &= \sqrt{\frac{(1.6 \times 10^{-19} ~\mathrm{C})
                           (3.68 \times 10^{4} ~\mathrm{V}~\mathrm{cm}^{-1})}
                    {4\pi(11.9 \cdot 8.85 \times 10^{-14})}} \\
            &= 21 ~\mathrm{mV}. 
\end{align*}
Repeating the computation with this $\Delta \phi$ value, and iterating this
process until $N_D$ changes by no more than $1\%$ between iterations,
we get the following values:

\begin{tabular}{c | c c c}
iteration & $\Delta \phi ~(\mathrm{mV})$ & $N_D (\mathrm{cm}^{-3}$ & $E_m (\mathrm{V}~\mathrm{cm}^{-1}$ \\
1         & 0                           & $5.93 \times 10^{15}$   & $3.68 \times 10^4$ \\
2         & 21.1                        & $2.63 \times 10^{15}$   & $2.45 \times 10^4$ \\
3         & 17.2                        & $3.05 \times 10^{15}$   & $2.64 \times 10^4$ \\
4         & 17.9                        & $2.98 \times 10^{15}$   & $2.60 \times 10^4$ \\
5         & 17.7                        & $2.99 \times 10^{15}$   & $2.61 \times 10^4$ \\
6         & 17.7                        & $2.99 \times 10^{15}$   & $2.61 \times 10^4$
\end{tabular}

Therefore the doping concentration is about $2.99 \times 10^{15} ~\mathrm{cm}^{-3}.$
}

\item{ 
  Observing that
  \begin{align*}
    C &= \frac{\varepsilon_s}{W_D} \\
      &= \sqrt{\frac{q \varepsilon_s N_D}
                      {2[\psi_{bi} - V - \frac{kT}{q}}} 
  \end{align*}
  we see that
  \begin{align*}
    \frac{1}{C^2} &= \frac{2\left[\psi_{bi} - V - \frac{kT}{q}\right]}
                          {q \varepsilon_s N_D} \\
                  &= \frac{2\left[\psi_{bi} - \frac{kT}{q}\right]}
                          {q \varepsilon_s N_D}
                   - \frac{2}{q \varepsilon_s N_D} V
  \end{align*}
  so that
  \begin{align*}
    A &= \frac{2\left[\psi_{bi} - \frac{kT}{q}\right]}
              {q \varepsilon_s N_D}
  \end{align*}
  and
  \begin{align*}
    B &= \frac{2}{q \varepsilon_s N_D}.
  \end{align*}
  \begin{enumerate}
    \item{We have that 
          \begin{align*}
          A &= \frac{2\left[\psi_{bi} - \frac{kT}{q}\right]}
                    {q \varepsilon_s N_D} \\
            &= \frac{2(0.7759 - 0.0259 ~\mathrm{V})}
                    {(1.6 \times 10^{-19} ~\mathrm{C})
                     (11.9 \cdot 8.85 \times 10^{-14} ~\mathrm{F}~\mathrm{cm}^{-1})
                     (2.99 \times 10^{15} ~\mathrm{cm}^{-3})} \\
            &= 2.98 \times 10^{15} ~\mathrm{cm}^4 ~\mathrm{F}^{-2}. 
          \end{align*}
         }
    \item{We have that
          \begin{align*}
          B &= \frac{2}{q \varepsilon_s N_D} \\
            &= \frac{2}{(1.6 \times 10^{-19} ~\mathrm{C})
                        (11.9 \cdot 8.85 \times 10^{-14} ~\mathrm{F}~\mathrm{cm}^{-1})
                        (2.99 \times 10^{15} ~\mathrm{cm}^{-3})} \\
            &= 3.97 \times 10^{15}~\mathrm{cm}^4 ~\mathrm{V} ~\mathrm{C}^{-2}.
          \end{align*}
         }
  \end{enumerate}
}

\end{enumerate}

\pagebreak

\section*{Problem \#2}
Find the charge per unit area $Q_n$ in the channel region of a $p$-type MOSFET
which has the following fabrication parameters:
\begin{itemize}
  \item{$N_D = 1 \times 10^{16} ~\mathrm{cm}^{-3}$,}
  \item{$d = 10 ~\mathrm{nm}$,}
  \item{gate material: $n^+$ polysilicon,}
  \item{From tests, an oxide charge 
        $$
        \frac{Q_f}{q} = 5 \times 10^{10} ~\mathrm{cm}^{-2},
        $$
       }
  \item{Operating parameters: $T = 300 ~\mathrm{K}$, $V_G = 1.5 V_T$ 
        (past strong inversion).}
\end{itemize}
\horline
Under strong inversion the charge per unit area in the channel 
is given by 
\begin{align*}
Q_n &= Q_s - Q_B \\
    &= \frac{\sqrt{2} \varepsilon_s}{\beta L_D}F - qN_D W_D,
\end{align*}
where $L_D$ is the Debye length
\begin{align*}
L_D &= \sqrt{\frac{\varepsilon_s kT}{q^2 N_D}} \\
    &= \sqrt{\frac{(11.9 \cdot 8.85 \times 10^{-14} ~\mathrm{F} ~\mathrm{cm}^{-1})
                   (0.0259 ~\mathrm{V})}
                  {(1.6 \times 10^{-19} ~\mathrm{C})
                   (10^{16} ~\mathrm{cm}^{-3})}} \\
    &= 4.129 \times 10^{-6} ~\mathrm{cm}
\end{align*}
and $W_D$ is the width of the depletion region in the channel
\begin{align*}
W_{D} &= \sqrt{\frac{2\varepsilon_s}{q} 
               \left(\frac{1}{N_A} + \frac{1}{N_D}
               \right)\psi_s }.
\end{align*}
Under strong inversion we have 
\begin{align*}
\psi_s &\approx \frac{2kT}{q} \ln \frac{N_D}{n_i} \\
       &= 2(0.0259 ~\mathrm{V}) \ln \frac{10^{16}}{9.65 \times 10^9} \\
       &= 0.7175 ~\mathrm{V}
\end{align*}
so
\begin{align*}
W_D &= \sqrt{\frac{2 \varepsilon_s \psi_s}
                    {q N_D} } \\
    &= \sqrt{\frac{2(11.9 \cdot 8.85 \times 10^{-14} ~\mathrm{F}~\mathrm{cm}^{-1})
                    (0.7175 ~\mathrm{V})}
                  {(1.6 \times 10^{-19} ~\mathrm{C})
                   (10^{16} ~\mathrm{cm}^{-3})}} \\
    &= 3.073 \times 10^{-5} ~\mathrm{cm},
\end{align*}
and furthermore we can make the approximation
\begin{align*}
F &\approx \exp \left(\frac{\beta|\psi_s|}{2}\right) \\
  &= \exp \left(\frac{(0.7175 ~\mathrm{V})}
                     {2(0.0259 ~\mathrm{V})}\right) \\
  &= 1.037 \times 10^{6} 
\end{align*}
Therefore
\begin{align*}
|Q_s| &= \frac{\sqrt{2}\varepsilon_s}{\beta L_D} F \\
      &= \frac{\sqrt{2}(0.0259 ~\mathrm{V})
                       (11.9 \cdot 8.85 \times 10^{-14} ~\mathrm{F}~\mathrm{cm}^{-1})}
              {4.129 \times 10^{-6} ~\mathrm{cm}}
         (1.037 \times 10^{6}) \\
      &= 9.68 \times 10^{-3} ~\mathrm{C}~\mathrm{cm}^{-2}
\end{align*}
and
\begin{align*}
Q_B &= qN_D W_D \\
    &= (1.6 \times 10^{-19} ~\mathrm{C})
       (10^{16} ~\mathrm{cm}^{-3})
       (3.073 \times 10^{-5} ~\mathrm{cm}) \\
    &= 4.8 \times 10^{-8} ~\mathrm{C}~\mathrm{cm}^{-2}
\end{align*}

so
\begin{align*}
Q_n &= Q_s - Q_B \\
    &= 9.68 \times 10^{-3} ~\mathrm{C}~\mathrm{cm}^{-2}.
\end{align*}

\pagebreak

\section*{Problem \#3}
A $p$-channel MOSFET is designed to have a threshold voltage of 
$V_T = -3 ~\mathrm{V}$ and a gate oxide thickness of 10 nm. Find the channel
doping to give the desired $V_T$ if $n^+$ polysilicon is used as the gate
material and the oxide charge is 
$\frac{Q_f}{q} = 5 \times 10^{10} ~\mathrm{cm}^{-3}$.
\horline
The threshold voltage for a $p$-channel device is given by
\begin{align*}
V_{T} &= -V_{FB} - 2|\psi_B| - \frac{\sqrt{4\varepsilon_s q N_D |\psi_B|}}{C_{ox}} \\
\end{align*}
We find that
\begin{align*}
|\psi_B| &= \frac{kT}{q} \ln \frac{N_D}{n_i} \\
         &= (0.0259 ~\mathrm{V})\ln \frac{N_D}{9.65 \times 10^{9}}
\end{align*}
and
\begin{align*}
C_{ox} &= \frac{\varepsilon_i}{d} \\
       &= \frac{3.9 \cdot 8.85 \times 10^{-14} ~\mathrm{F}~\mathrm{cm^{-1}}}
               {10 \times 10^{-7} ~\mathrm{cm}} \\
       &= 3.45 \times 10^{-7} ~\mathrm{F}
\end{align*}
so
\begin{align*}
V_{FB} &= \phi_{ms} - \frac{Q_f}{C_{ox}} \\
       &= \phi_{ms} - \frac{q\frac{Q_f}{q}}{C_{ox}} \\
       &= \phi_{ms} - \frac{(1.6 \times 10^{-19} ~\mathrm{C})
                            (5 \times 10^{10} ~\mathrm{cm}^{-3})}
                           {3.45 \times 10^{-7} ~\mathrm{F}} \\
       &= \phi_{ms} - 0.023 ~\mathrm{V}
\end{align*}
and
\begin{align*}
\frac{\sqrt{4 \varepsilon_s q N_D \psi_B}}{C_{ox}}
  &= \frac{\sqrt{4 \varepsilon_s q}}{C_{ox}} \sqrt{N_D \psi_B} \\
  &= \frac{2\sqrt{(11.9 \cdot 8.85 \times 10^{-14} ~\mathrm{F}~\mathrm{cm}^{-1})
                  (1.6 \times 10^{-19} ~\mathrm{C})}}
          {3.45 \times 10^{-7}} 
     \sqrt{(0.0259 ~\mathrm{V})} 
     \sqrt{N_D \ln \frac{N_D}{9.65 \times 10^9}} \\
  &= 3.83 \times 10^{-10} \sqrt{N_D \ln \frac{N_D}{9.65 \times 10^9}}
\end{align*}
so that
\begin{align*}
V_{T} &= -V_{FB} - 2|\psi_B| - \frac{\sqrt{4\varepsilon_s q N_D |\psi_B|}}{C_{ox}} \\
      &= \phi_{ms} - 0.023 
         - 0.0259\ln\frac{N_D}{9.65 \times 10^9} 
         - 3.83 \times 10^{-10} \sqrt{N_D \ln \frac{N_D}{9.65 \times 10^9}}.
\end{align*}

An appropriate doping concentration $N_D$ can then be found iteratively as
follows:

\begin{tabular}{c | c c c }
iteration & $N_D ~(\mathrm{cm}^{-3})$ & $\phi_{ms} ~(\mathrm{V})$ & $V_T ~(\mathrm{V})$ \\
\hline
1         & $10^{16}$                 & -0.2                      & -0.724 \\
2         & $10^{17}$                 & -0.15                     & -1.08  \\
3         & $10^{18}$                 & -0.1                      & -2.25  \\
4         & $10^{19}$                 & -0.05                     & -6.13  \\
5         & $10^{18.5}$               & -0.075                    & -3.62  \\
6         & $10^{18.25}$              & -0.0875                   & -2.83  \\
7         & $10^{18.375}$             & -0.08125                  & -3.20  \\
8         & $10^{18.3125}$            & -0.078125                 & -3.00 
\end{tabular}

Thus a doping concentration of 
$$
N_D = 10^{18.3125} ~\mathrm{cm}^{-3} = 2.054 \times 10^{18} ~\mathrm{cm}^{-3}
$$
can be used.

\pagebreak

\section*{Problem \#4}
The goal is to achieve a low $I_{DS} = 10^{-14} ~\mathrm{A}$ t $V_G = 0$ 
for a $p$-channel MOSFET with the following fabrication parameters:
\begin{itemize}
  \item{$N_D = 1 \times 10^{16} ~\mathrm{cm}^{-3}$,}
  \item{$d = 50 ~\mathrm{nm}$,}
  \item{$I_{SD} = 1 ~\mathrm{mA}$ at $V_T$}
  \item{gate material: $n^+$ polysilicon,}
  \item{an oxide charge of $\frac{Q_f}{q} = 5 \times 10^{10} ~\mathrm{cm}^{-2}$.}
\end{itemize}

\begin{enumerate}
  \item{Evaluate the subthreshold swing $S$ for this device.}
  \item{Evaluate the threshold voltage $V_T$.}
  \item{Create an $I_{DS}$ versus $V_G$ plot to investigate the raw device
        performance and evaluate $I_{DS}$ at $V_G = 0 ~\mathrm{V}$.}
  \item{State how you can achieve the goal of $I_{DS} = 10^{-14} ~\mathrm{A}$
        at $V_G = 0$.}
  \item{Draw a MOSFET architecture diagram with exact or reasonable values 
        for all batteries labeled on the diagram.}
\end{enumerate}
\horline
\begin{enumerate}
  \item{The subthreshold swing is given by
        \begin{align*}
          S &= (\ln 10) \frac{kT}{q} \frac{C_{ox} + C_D}{C_{ox}} 
        \end{align*}
        where
        \begin{align*}
          C_{ox} &= \frac{\varepsilon_i}{d} \\
                 &= \frac{3.9 \cdot 8.85 \times 10^{-14} ~\mathrm{F} ~\mathrm{cm}^{-1}}
                         {50 ~\mathrm{nm}} \\
                 &= 6.9 \times 10^{-8} ~\mathrm{F}
        \end{align*}
        and
        \begin{align*}
        \psi_s &= \frac{kT}{q} \ln \frac{N_D}{n_i} \\
               &= (0.0259 ~\mathrm{V})\ln\frac{10^{16} ~\mathrm{cm}^{-3}}
                                           {9.65 \times 10^9 ~\mathrm{cm}^{-3}} \\
               &= 0.3587 ~\mathrm{V}
        \end{align*}
        since the device is in the subthreshold version and thus in weak inversion. Then
        \begin{align*}
          W_{D} &= \sqrt{\frac{2\varepsilon_s}{q} 
                         \left(\frac{1}{N_A} + \frac{1}{N_D}
                         \right)\psi_s } \\
                &= \sqrt{\frac{2 \varepsilon_s \frac{kT}{q} \ln \frac{N_D}{n_i}}
                         {q N_D} } \\
                &= \sqrt{\frac{2 (11.9 \cdot 8.85 \times 10^{-14} ~\mathrm{F}~\mathrm{cm}^{-1})
                                 (0.3587 ~\mathrm{V})}
                              {(1.6 \times 10^{-19} ~\mathrm{C})
                               (10^{16} ~\mathrm{cm}^{-3})}} \\
                &= 2.17 \times 10^{-5} ~\mathrm{cm}
        \end{align*}
        so
        \begin{align*}
          C_{D} &= \frac{\varepsilon_s}
                        {W_{D}} \\
                &= \frac{(11.9 \cdot 8.85 \times 10^{-14} ~\mathrm{F}~\mathrm{cm}^{-1})}
                        {2.17 \times 10^{-5}} \\
                &= 4.85 \times 10^{-8} ~\mathrm{F}
        \end{align*}
        and thus
        \begin{align*}
          S &= (\ln 10) \frac{kT}{q} \frac{C_{ox} + C_D}{C_{ox}} \\
            &= (\ln 10) (0.0259 ~\mathrm{V})
               \frac{6.9 \times 10^{-8} + 4.85 \times 10^{-8}}
                    {6.9 \times 10^{-8}} \\
            &= 0.102,
        \end{align*}
        so the subthreshold swing is 102 mV per decade.
  }
  \item{The threshold voltage for a $p$-MOSFET is given by
        \begin{align*}
          V_{T} &= V_{FB} - 2\psi_B - \frac{\sqrt{4\varepsilon_s q N_D \psi_B}}{C_{ox}} \\
                &= \phi_{ms} - \frac{Q_f}{C_{ox}} - 2 |\psi_B| - \frac{\sqrt{4\varepsilon_s q N_D |\psi_B|}}{C_{ox}} \\
                &= -0.2 \\
                &- \frac{(1.6 \times 10^{-19} ~\mathrm{C})
                         (5 \times 10^{10} ~\mathrm{cm}^{-2})}
                        {6.9 \times 10^{-8} ~\mathrm{F}} \\
                &- 2(0.3587 ~\mathrm{V}) \\
                &- 2\frac{\sqrt{(11.9 \cdot 8.85 \times 10^{-14} ~\mathrm{F}~\mathrm{cm}^{-1})
                                (1.6 \times 10^{-19} ~\mathrm{C})
                                (10^{16} ~\mathrm{cm}^{-3})
                                (0.3587 ~\mathrm{V})}}
                         {6.9 \times 10^{-8}} \\
                &= -0.2 - 0.116 - 0.717 - 0.713 ~\mathrm{V}\\
                &= -1.746 ~\mathrm{V}.
        \end{align*} 
       }
   \item{For a gate bias less than the threshold voltage, the current falls 
         off exponentially at a rate of 102 mV per decade, so the current can 
         be plotted versus gate voltage on a semilog plot as in figure 
         \ref{fig:iv_curve}. When $V_G = 0 ~\mathrm{V}$, we see 
         $I_{SD} = 7.62 \times 10^{-21} ~\mathrm{A}$.
         \begin{sidewaysfigure}
           \includegraphics[width=\textheight]{iv}
           \caption{Drain current versus gate bias for problem \#4. \label{fig:iv_curve}}
         \end{sidewaysfigure}}
   \item{The current at $V_G = 0$ can be increased to the desired value by 
         applying a bulk bias to change the threshold voltage $V_T$. We wish to
         increase the current by 
         $\log_{10}\left(\frac{10^{-14}}{7.62 \times 10^{-21}}\right) = 6.12$ decades,
         which requires decreasing the magnitude of the threshold voltage by 
         $6.12 \cdot 102 ~\mathrm{mV} = 624 ~\mathrm{mV}$. The change in $V_T$ for a
         given bulk bias is given by
         $$
         \Delta V_{T} = \frac{\sqrt{2 \varepsilon_s q N_D}}
                               {C_{ox}}
                          (\sqrt{2 \psi_B - V_{BS}} - \sqrt{2 \psi_B}) 
         $$
         so
         \begin{align*}
         V_{BS} &= 2\psi_B - \left(\Delta V_{T}\left(\frac{\sqrt{2 \varepsilon_s q N_D}}{C_{ox}}\right)^{-1} + \sqrt{2 \psi_B}\right)^2 \\
                &= 0.717 - \left(\frac{-0.624}{0.504} - \sqrt{0.717}\right)^2 \\
                &= -3.63 ~\mathrm{V}.
         \end{align*}
        }
        \item{ A MOSFET connected as described is diagrammed in figure \ref{fig:mosfet}.
               \begin{figure}
                 \includegraphics[width=\textwidth]{mosfet}
                 \caption{A MOSFET connected as described in problem \#4. \label{fig:mosfet}}
               \end{figure}
             }
\end{enumerate}

\pagebreak

\section*{Problem \#5}
Consider an $n$-channel MOSFET built in Si and operated at 300 K, with 
$V_{DS}$ biased such that the carriers are travelling at their saturated 
average velocity. If constant field scaling is applied to the device, show
how the drain current $I_{DS}$ will scale.
\horline
Under velocity saturation we have initially
\begin{align*}
(I_{D_{sat}})_0 &= Z (V_G - V_T)C_{ox} v_s \\
                &= \frac{Z}{d} (V_G - V_T) \varepsilon_s v_s.
\end{align*}
Under constant field scaling, geometric parameters of the device are scaled
while simultaneously scaling the gate and threshold voltages in order to keep
the electric field $\mathscr{E}_c$ constant. For a scaling factor $\kappa$, we
see
\begin{align*}
(I_{D_{sat}})_{\kappa} &= \frac{Z}{\frac{d}{\kappa}} 
                          \left(\frac{V_G}{\kappa} - \frac{V_T}{\kappa}\right) 
                          \varepsilon_s v_s \\
                       &= \kappa 
                          \frac{Z}{d}
                          \frac{1}{\kappa}\left(V_G - V_T\right) 
                          \varepsilon_s v_s \\
                       &= (I_{D_{sat}})_0,
\end{align*}
so the current does not change.

\pagebreak

\section*{Problem \#6}
You wish to measure the strain $S = \frac{\Delta l}{l}$ of a bar of 
$n$-type Si, shown below, during an electro-MEMS application. unfortunately,
further device constraints require the doping to be 
$N_D = 5 \times 10^{18} ~\mathrm{cm}^{-3}$.

\begin{enumerate}
  \item{Evaluate the strain $S$ under the following temperature conditions and
        plot $S$ versus $T$.
        \begin{enumerate}
          \item{$T = 27^{\circ}~\mathrm{C}$,}
          \item{$T = 40^{\circ}~\mathrm{C}$,}
          \item{$T = 100^{\circ}~\mathrm{C}$,}
        \end{enumerate}
        You may assume that
        \begin{itemize}
          \item{the Young's modulus does not change appreciably from its
                room temperature value}
          \item{the Poisson's ratio $\nu$ does not change appreciably as a
                function of doping.}
        \end{itemize}
       }
       \item{Is the bar becoming more strained or less strained as a function 
             of increasing temperature?}
       \item{Perform a literature search and provide one reference that tests
             the assumption that the Young's modulus does not change as a 
             function of temperature. Print the reference out and attach it
             to your exam.}
\end{enumerate}
\horline
\begin{enumerate}
  \item{From Figure 5 in Chapter 14 of Sze, the gauge factor has a variation 
        with temperature. For a given $T$ we can find the change $\Delta T$ 
        from the reference temperature, and use $\alpha$ and the
        temperature-dependent gauge factors to compute the strain.

        \begin{tabular}{c | c c c c}
        $T ~(^\circ \mathrm{C})$ & Gauge factor $G$ & $\Delta T ~(^\circ \mathrm{C})$ & $\frac{\Delta R}{R}$ & Strain $S$ \\
        \hline
        27                       & 130              & 7                               & -0.49                & $-3.77 \times 10^{-3}$ \\
        40                       & 120              & 20                              & -1.4                 & $-1.17 \times 10^{-2}$ \\
        100                      & 100              & 80                              & -5.6                 & $-5.60 \times 10^{-2}$
        \end{tabular}

        \begin{sidewaysfigure}
          \includegraphics[width=\textheight]{strain}
          \caption{Strain versus temperature. \label{fig:strain}}
        \end{sidewaysfigure}
       }
  \item{ The plot (figure \ref{fig:strain}) indicates that the strain increases (in magnitude)
         with temperature, so that at higher temperatures more strain is 
         required to produce the same change in resistance.
       }
   \item{According to the attached publication, the Young's modulus changes by
         less than 1\% over the indicated temperature range even in intrinsic
         silicon, and the plot given in Sze indicates that 
         temperature-dependence of the gauge factor diminishes with increased
         doping.
        }
\end{enumerate}

\end{document}
