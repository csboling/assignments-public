\documentclass{paper}

\usepackage{natbib}
\usepackage{graphicx}

\title{Progress Report -- Simulating and characterizing rate distortion of a compressed sensing architecture for multichannel neural recordings}
\author{Sam Boling}
\date{11/25/2013}

\begin{document}

\maketitle

With the motivation of reducing the bandwidth requirements of multielectrode
neural recording systems, I chose to examine the rate-distortion performance
of an analog compressed sensing architecture by performing proof-of-concept 
simulations. Preliminary development and simulations are complete, and have
given a better indication of what directions should be pursued for 
analysis. Next, a rate distortion curve will be plotted that shows the
relationship between the number of compressive measurements and the 
mean-squared distortion suffered by action potentials as they pass through
the system. Additionally, the probability of sparse reconstruction 
erroneously recovering a coefficient will be computed, and these error 
probabilities will be used as input to the Blahut-Arimoto algorithm in an
attempt to establish what coefficients should be preferred or discarded in
the reconstruction step.

Compressed sensing relies on the observation (and rigorous demonstration in
\cite{candes2006}) that a high-dimensional signal which is known to be 
"sparse" can be represented with a smaller number of measurements than its 
dimension alone would indicate. More precisely, a dimension $N$ signal 
$x$ with a representation in basis $\{\psi_i\}_{i=1}^N = \mathbf{\Psi}$ that
has only $K$ nonzero coefficients, i.e. $x = \mathbf{\Psi}\alpha$ 
and $K = \|\alpha\|_0 \ll N$, is said to be sparse in $\mathbf{\Psi}$. 
Compressed sensing theory guarantees that such a signal can be reconstructed
from projections $y = \mathbf{\Phi}x$ provided that the projective 
transformation $\mathbf{\Phi}$ meets certain requirements. In particular,
a sufficiently random choice of $\Phi$ guarantees that signals will not be
sparse in $\mathbf{\Phi}{\Psi}$, and in this case it is sufficient for the
dimensionality of the projected measurements $y$ to be 
$M > O\left(K \log \frac{N}{K}\right)$ regardless of the sparsity basis
$\mathbf{\Psi}$ \cite{candes2010}. For a specified sparsity basis it is 
possible to recover the coefficients $\alpha$ by solving an optimization
problem that attempts to minimize 
$\|y - \mathbf{\Phi}\mathbf{\Psi}\alpha\|_2$ 
while penalizing candidate solutions for a lack of sparsity. 

The ability of compressed sensing to greatly reduce the dimensionality of
measurements is a promising approach for reducing the bandwidth and 
computational load of a data acquisition platform, thereby helping wireless 
biocompatible systems such as neural recording devices meet their strict
power constraints. Brain tissue in particular is highly susceptible to
temperature change, which can affect function and cause long-term damage 
\cite{cais2008}, so power dissipation for implantable brain-computer 
interfaces must be carefully controlled. Reconstruction by sparse 
optimization, however, is generally a computationally intensive procedure 
that is not well-suited for real-time implementation, especially at low 
power. This would seem to limit the viability of a compressed sensing 
approach in closed-loop clinical applications. However, recent theoretical 
work has demonstrated that reconstruction is not necessary for many control
and machine learning algorithms since appropriately designed systems can 
operate entirely in the compressed domain \cite{ganguli2012}. These 
developments provide some encouragement that the problem of wireless
low-power multielectrode recording and control may lend itself well to
compressed sensing. Some authors have even suggested that compressed 
sensing theory can provide insight into the storage and computation
capabilities of the brain itself \cite{ganguli2012}, and this should 
further motivate the investigation of the technique's applicability to
neural signals.

Although much of the attention paid to compressed sensing is due to its
promise to enable measurement systems that achieve effectively sub-Nyquist
rate sampling for certain classes of signals, most existing implementations
of the approach use ordinary analog-to-digital converters and perform 
random projections digitally, essentially using compressed sensing as a 
conventional, if unique, signal compression method. Such approaches have
included work with extracellular neural recordings 
\cite{charbiwala2011}, relying on the sparsity of neural action potentials 
in the wavelet domain. Some researchers, however, have proposed circuits 
they describe as analog-to-information  converters since they project input 
signals by analog means as part of the conversion process 
\cite{kirolos2006}. Such an approach saves on the power costs of both 
transmission bandwidth and digital signal processing, magnifying the 
benefits of this sampling scheme even at the expense of increased 
reconstruction costs.

Information theoretic bounds \cite{aeron2010, sarvotham2006} and empirical 
rate distortion results for compressed sensing \cite{rambeloarison2012} have
been published, but much of the available research seems limited to a 
one-channel data stream. Expansion to multiple channels poses an interesting
challenge to reconstruction discussed in \cite{romberg2009}, since events of
interest, which represent the nonzero elements of the sparse coefficient 
vector $\alpha$, often occur simultaneously across several channels. Neural
action potentials have a high probability of occurring synchronously across
multiple channels, and indeed synchronous neural activity is often of 
particular interest to researchers. Nonetheless, increasing spatial 
resolution of neural recordings to desired levels will require novel
approaches for controlling system bandwidth, and architectures which can
afford multiple channels on a single analog-to-digital converter at a fixed
rate are therefore highly attractive technologies for this problem.

Despite some theoretical developments regarding multichannel systems,
relatively few architectures have been proposed for performing compressive
sampling of numerous channels in analog. The most significant is the
so-called compressive multiplexer described in \cite{slavinsky2011}. In 
this approach, $J$ channels each with a bandwidth of $W/2$ Hz are sampled
in parallel. A linear-feedback shift register (LFSR) with taps chosen for
a maximum period of repetition is used to generate chipping sequences that 
take values of $\pm 1$ with fairly uniform probability, where new chips are
generated at the rate $W$. Each channel is multiplied in analog with its 
chipping sequence, then all such modulated channels are summed and sampled 
at rate $W$ with a conventional ADC. These chipping sequences effectively 
produce a projection basis $\mathbf{\Phi}$ incoherent with any sparsity 
basis, and the use of a pseudorandom deterministic source allows a receiver
using the same pseudorandom seed to reconstruct signals acquired in this 
fashion. Here the projection matrix $\mathbf{\Phi}$ consists of a union of
$J$ matrices with the chip sequences along the diagonal. By multiplying the 
resulting stream by the chip sequence for a 
given channel, a receiver can reproduce the input for that channel plus a
noise term with a magnitude determined by the sparsity of all other 
channels. This is the trivial reconstruction mode of the system. The 
authors show substantially improved reconstruction and even denoising of
the original input when the block-coordinate relaxation algorithm described 
by \cite{sylvain2000} is used. This procedure involves fitting coordinates 
for each channel in sequence to a residual signal derived by subtracting out
the most current estimate of the contribution from all other channels. The
main advantage of the compressive multiplexer is that the number of channels
may be increased, up to some limit, without affecting the output bandwidth 
of the converter.

The compressive multiplexer architecture described in \cite{slavinsky2011} 
has been implemented in Python along with both trivial and block-coordinate
relaxation (BCR) \cite{sylvain2000} reconstruction methods. The latter 
algorithm required some alterations from the procedure given by the authors 
of \cite{slavinsky2011} since neural action potentials are sparse in a wavelet basis and not in the Fourier domain \cite{oweiss2006}. The Python 
implementation is structured in a flexible coroutine-based pipeline that 
allows the software to flow much like a hardware system and produce output
on a sample-by-sample basis. NumPy and SciPy are used for data management,
linear algebra and signal processing, and the PyWavelets and sklearn 
libraries are used for wavelet decomposition and sparse optimization 
solving, respectively. Some tuning of parameters and experimentation with
wavelet bases and sparsification techniques has been performed and 
development will now focus on information theoretic analysis. When comparing
reconstructions with inputs it is important to consider only the windows
immediately surrounding spike events, since for most analysis and control
applications all data that does not contain action potentials is discarded
as noise. 

At present, reasonably good reconstruction performance for action potentials
has been obtained from both trivial and BCR methods (see figure), although
parameters could still be tuned. Addition of a learned union of supports
as in \cite{charbiwala2011} might help reconstruction quality but is outside
the scope of this project. In this implementation we have $N$ equal to the 
window size of the discrete wavelet transform times the number of channels 
and $M$ equal to the number of wavelet coefficients, which depends on choice
of basis and decomposition level. A rate distortion curve for the system can
be plotted as in \cite{schulz2009} with the rate measured as 
$\frac{M}{N} H_y$, where $H_y$ is the sample entropy. In order to keep the 
test samples the same, we must vary the parameter $M$ for a fixed choice of 
wavelet basis, since wavelet basis affects sparsity and other relevant 
variables. By varying the number of decomposition levels and the number of 
coefficients to discard it should be possible to characterize the system's 
performance tradeoffs in this way. Additionally, we note that the 
reconstruction from multi-channel activity often suffers from coefficients 
with large magnitude where the input has nearly-zero coefficients. By 
quantitatively analyzing the probability that such an error will occur, 
using the finite alphabet of 1 for a positive coefficient with a magnitude 
above a certain threshold, -1 for a negative coefficient with a magnitude 
above the threshold, and 0 for any coefficient with a magnitude below the 
threshold, it should be possible to use the Blahut-Arimoto algorithm to find
an optimal probability distribution for the "channel" that the compressed 
sensing architecture implements. Locating such a probability distribution 
may provide some insight into which coefficients should be preferred or 
discarded during sparse optimization.

\begin{figure}
  \includegraphics[width=\linewidth]{prelim-results.png}
\end{figure}

\bibliographystyle{plain}
\bibliography{references}

\end{document}
